\chapter{Multipying Polynomials}

Watch Khan Academy's \textbf{Multipying monomials} at \url{https://youtu.be/Vm7H0VTlIco}.

To review, when you multiply two monomials, you take the product of
their coefficients and the sum of their degrees:
\begin{equation*}
  (2x^6)(5x^3) = (2)(5)(x^6)(x^3) = 10x^9
\end{equation*}
If you have a product of more than two monomials, multiply \emph{all}
the coefficients and sum \emph{all} the exponents:
\begin{equation*}
  (3x^2)(2x^3)(4x) = (3)(2)(4)(x^2)(x^3)(x^1) = 24x^6
\end{equation*}

\begin{Exercise}[title={Multiplying monomials}, label=multmonomials]
Multiply these monomials
  \Question $(3x^2)(5x^3)$
\vspace{20mm}
  \Question $(2x)(4x^9)$
\vspace{20mm}
  \Question $(-5.5x^2)(2x^3)$
\vspace{20mm}
  \Question $(\pi)(-2x^5)$
\vspace{20mm}
  \Question $(2x)(3x^2)(5x^7)$
\vspace{20mm}
\end{Exercise}
\begin{Answer}[ref=multmonomials]
  $(3x^2)(5x^3) = 15x^5$
  
  $(2x)(4x^9) = 8x^{10}$
  
  $(-5.5x^2)(2x^3) = -11x^5$

  $(\pi)(-2x^5) = -2\pi x^5$
  
  $(2x)(3x^2)(5x^7) = 30x^{10}$
\end{Answer}

\section{Multiplying a monomial and a polynomial}

Watch Khan Academy's \textbf{Multiplying monomials by polynomials} at \url{https://youtu.be/pD2-H15ucNE}.

When multiplying a monomial and a polynomial, you use the the distributive property. Then it is just multiplying several pairs of monomials:
\begin{multline*}
  (3x^2)(4x^3 - 2x^2 + 3x - 7) \\
  = (3x^2)(4x^3) + (3x^2)(-2x^2) + (3x^2)(3x) + (3x^2)(-7) \\
  = 12x^5 - 6x^4 + 9x^3 -21x^2
\end{multline*}

\begin{Exercise}[title={Multiplying a monomial and a polynomial}, label=multmonopoly]
Multiply these monomials
\Question $(3x^2)(5x^3 - 2x + 3)$
\vspace{20mm}
\Question $(2x)(4x^9 - 1)$
\vspace{20mm}
\Question $(-5.5x^2)(2x^3 + 4x^2 + 6)$
\vspace{20mm}
\Question $(\pi)(-2x^5 + 3x^4 + x)$
\vspace{20mm}
\Question $(2x)(3x^2)(5x^7 + 2x)$
\end{Exercise}
\begin{Answer}[ref=multmonopoly]
  $(3x^2)(5x^3 - 2x + 3) = 15x^6 - 6x^3 + 6x^2$

  $(2x)(4x^9 - 1) = 8x^{10} - 2x$

  $(-5.5x^2)(2x^3 + 4x^2 + 6) = 11x^5 - 22x^4 + 33x^2$

  $(\pi)(-2x^5 + 3x^4 + x) = -2\pi x^5 + 3\pi x^4 + \pi x$

  $(2x)(3x^2)(5x^7 + 2x) = 30x^{10} + 12x^4$
\end{Answer}

\section{Multiplying polynomials}

Watch Khan Academy's \textbf{Multiplying binomials by polynomials} video at \url{https://youtu.be/D6mivA_8L8U}

When you are multiplying two polynomials, you will use the
distributive property several times to make it one long
polynomial. Then you will combine the terms with the same degree. For
example,
\begin{multline*}
  (2x^2 - 3)(5x^2 + 2x - 7) \\
  =   (2x^2)(5x^2 + 2x - 7) + (-3)(5x^2 + 2x - 7) \\
  =   (2x^2)(5x^2) + (2x^2)(2x) + (2x^2)(-7) + (-3)(5x^2) + (-3)(2x) + (-3)(-7) \\
  =   10^4 + 4x^3 + -14x^2 + -15x^2 + -6x + 21
  =   10^4 + 4x^3 + -29x^2 + -6x + 21
\end{multline*}

One common form that you will see is multiplying two binomials together:
\begin{multline*}
(2x + 7)(5x + 3) = (2x)(5x + 3) + (7)(5x+3) = (2x)(5x) + (7)(5x) + (2x)(3) + (7)(3)
\end{multline*}
Notice the product has become the sum of four parts: the firsts, the
inners, the outers, and the lasts. People sometimes use the mnemonic
FOIL to remember this pattern, but if you know the distributive property
(which is good for the product of any two polynomials), you don't need
to FOIL (which is only useful for the specific case of two binomials).

The bottom line is that when you are multiplying two polynomials
together, every term in the first will be multiplied by every term in
the second. So, for example, if you have a polynomial with three terms
and you multiply it by a polynomial with five terms, you will get a
sum of 15 terms.  (Of course, several of those terms might have the
same degree, so they will be combined together when you simplify. Thus
you will probably end up with a polynomial with less than 15 terms.)

Here is how I would multiply two third-degree polynomials:
\begin{multline*}
  (2x^2 - 3x + 1)(5x^2 + 2x - 7)  = \begin{matrix}
  (2x^2)(5x^2)& + &(2x^2)(2x)& + &(2x^2)(-7)& + \\
  (-3x)(5x^2)& + &(-3x)(2x)& + &(-3x)(-7)& + \\
    (1)(5x^2)& + &(1)(2x)& + &(1)(-7)& 
  \end{matrix} \\
  = 10x^4 + 4x^3 + (-14)x^2 + (-15)x^3 + (-6)x^2 + 21x +5x^2 + 2x + (-7) \\
  = 10x^4 + (4 - 15)x^3 + (-14 - 6 + 5)x^2  + (21 + 2)x + (-7) \\
  = 10x^4 - 11x^3 - 15x^2 + 23x - 7
\end{multline*}

One common error people make is losing track of the negative
signs. There is no trick; you just need to be careful.

\begin{Exercise}[title={Multiplying polynomials}, label=multpolys]
  Multiply the following pairs of polynomials:
  \Question{$2x + 1$ and $3x - 2$}
\end{Exercise}
\begin{Answer}[ref=multpolys]
  $(2x + 1)(3x - 2) = 6x^2 - x - 2$
\end{Answer}
  
