\chapter{Solving Quadratics}

A quadratic function has three terms: $ax^2 + bx + c$. $a$, $b$, and
$c$ are known as the \textit{coefficents}. The coefficients can be any
constant, except that $a$ can never be zero. (If $a$ is zero, it is a linear
function, not a quadratic.)

When you have an equation with a quadratic function on one side and a
zero on the other, you have a quadratic equation. For example:

$$72x^2 - 12x + 1.2 = 0$$

How can you find the values of $x$ that will make this equation true?

You can always reduce a quadratic equation so that the first
coefficient is 1, so that your equation looks like this:

$$x^2 +bx + c = 0$$

For example, if you are asked to solve $4x^2 + 8x - 19 = -2x^2 - 7$
\begin{multline*}
  4x^2 + 8x - 19 = -2x^2 - 7 \\
  6x^2 + 8x -12 = 0 \\
  x^2 + \frac{4}{3}x - 2 = 0
\end{multline*}
Here, $b = \frac{4}{3}$ and $c = -2$.

\begin{mdframed}[style=important]
$x^2 + bx + c = 0$ when
\begin{equation*}
x = -\frac{b}{2} \pm \frac{\sqrt{b^2 - 4c}}{2}  
\end{equation*}
\end{mdframed}

What does this mean?

For any $b$ and $c$, the graph of $x^2 + bx + c$ is a parabola
that goes up on each end. Its low point is at $x = -\frac{b}{2}$.

If there are no real roots ($b^2 - 4c < 0$), that means the
parabola never gets low enough to cross the $x$-axis:

\begin{tikzpicture}
    \begin{axis}[
        xmin=-0.5,xmax=2.75,
        ymin=-1,ymax=5,
        axis x line=middle,
        axis y line=middle,
        axis line style=<->,
        xlabel={$x$},
        ylabel={$y$},
      ]
      \addplot[no marks,sdkblue] expression[domain=-0.25:2.5,samples=100]{x^2 - 2* x + 3} node[above, xshift=-1cm] {$x^2 - 2x + 3$};
      \addplot[dashed,gray] coordinates {(1,-1)(1,3)};
    \end{axis}
\end{tikzpicture}

If there is one real root ($b^2 - 4c = 0$), it means that the parabola just touches the x-axis.

\begin{tikzpicture}
    \begin{axis}[
        xmin=-0.5,xmax=3.75,
        ymin=-0.5,ymax=5,
        axis x line=middle,
        axis y line=middle,
        axis line style=<->,
        xlabel={$x$},
        ylabel={$y$},
      ]
      \addplot[no marks,sdkblue] expression[domain=-0.25:3.5,samples=100]{x^2 - 4*x + 4} node[above, xshift=-1cm] {$x^2 - 4x + 4$};
      \addplot[dashed,gray] coordinates {(2,-0.5)(2,1)};
    \end{axis}
\end{tikzpicture}

If there are two real roots ($b^2 - 4c > 0$), it means that the parabola crosses the x-axis twice as it dips below and then returns:

\begin{tikzpicture}
    \begin{axis}[
        xmin=-0.5,xmax=3.75,
        ymin=-2,ymax=5,
        axis x line=middle,
        axis y line=middle,
        axis line style=<->,
        xlabel={$x$},
        ylabel={$y$},
      ]
      \addplot[no marks,sdkblue] expression[domain=-0.25:3.5,samples=100]{x^2 - 3*x + 1} node[above, xshift=-1cm] {$x^2 - 4x + 4$};
      \addplot[dashed,gray] coordinates {(1.5,-2)(1.5,1)};
    \end{axis}
\end{tikzpicture}

\begin{Exercise}[title={Roots of a Quadratic}, label=solve_quadratic]

  In the last chapter, you found that the function for the height of your flying hammer is:

  $$p = -\frac{1}{2}9.8 t^2 + 12t + 2$$

  At what time will the hammer hit the ground?

  
\end{Exercise}
\begin{Answer}[ref=solve_quadratic]

  For what $t$ is  $-4.9 t^2 + 12t + 2 = 0$?  Start by dividing both sides of the equation by -4.9.

  $$t^2 - 2.45 t - 0.408 = 0$$

  The roots of this are at

  $$x = -\frac{b}{2} \pm \frac{\sqrt{b^2 - 4c}}{2} = -\frac{-2.45}{2} \pm \frac{\sqrt{(-2.45)^2 - 4(-0.408)}}{2} = 1.22 \pm 1.36$$

  We only care about the root after we release the hammer ($t > 0$).

  $1.22 + 1.36 = 2.58$ seconds after releasing the hammer, it will hit the ground.

  
\end{Answer}


\section{The Traditional Quadratic Formula}

I like the approach I just showed you. I find it easier to remember
and easier to prove than the traditional quadratic fomula, but you
should probably know the traditional quadratic formula.

\begin{mdframed}[style=important, frametitle={The Quadratic Formula}]

$ax^2 + bx + c = 0$ when
\begin{equation*}
  x = \frac{-b \pm \sqrt{b^2 - 4ac}}{2a}
\end{equation*}

\end{mdframed}
