\chapter{Introduction}

This book is designed to walk you down the path to being a modern
problem solver. Along the path you will learn how to use the tools of
math, computers, and science. It will be a long journey, and not an
easy one.

Why should you bother? There are big problems in this world that will
require expert problem solvers. Those people will make the world a
better place while enjoying interesting and lucrative careers. We are
talking about engineers, scientists, doctors, computer programmers,
architects, actuaries, and mathematicians. Right now, those represent
about 6\% of all the jobs in the United States. In the near future
that number is expected to rise above 10\%.  On average, the people in
that 10\% of the population are expected to have salaries twice that
of the other 90\%.

Solving problems is difficult. At some point on this journey, you will
see people who are better at solving problems than you are. You, like
every other person who has gone on this journey, will think ``I have
worked so hard on this, but that person is obviously better at it than
I am. I should quit.'' Don't.

First, solving problems is like a muscle. The more you do, the better
you get at it.  It is OK to say ``I am not good at this yet.'' That
just means you need more practice.

Second, you don't need to be the best in the world. 10 million people
your age can be better at solving problems than you, \textit{and you
  can still be in the 10\% of the world}. If you complete this
journey, there will be problems for you to solve, and a job where your
problem solving skills will be appreciated.

So where do we start?

\section{Atoms}

The famous physicist Richard Feynman once asked this question: ``If,
in some cataclysm, all of scientific knowledge were to be destroyed,
and only one sentence was passed on to the next generation of
creatures, what statement would contain the most information in the
fewest words?''

His answer was ``All things are made of atoms—little particles that move around in
perpetual motion, attracting each other when they are a little
distance apart, but repelling upon being squeezed into one another.''

That seems like a good place to start.

All things (including the air around you) are made of atoms. Atoms are
very tiny -- there are more atoms in a drop of water than there are
drops of water in all the oceans.

Every atom has a nucleus which contains protons and neutrons. There is
a cloud of electrons flying around the nucleus. The mass of the atom
comes mainly from the protons and neutrons, which are much heavier
than electrons.

We classify atoms by the number of protons they have. An atom with one
proton is a hydrogen atom. We say that hydrogen is an
\textit{element}. All atoms with eight protons are oxygen atoms. We
give each element an atomic symbol. Hydrogen gets $H$.  Oxygen
gets $O$. Carbon, with six protons, gets $C$.

Often two hydrogen atoms will attach to an oxygen atom.  The result is
a water molecule. Why do they cluster together? They are sharing the
electrons in their clouds.

The molecule is sometimes described by the elements it contains.
Water, for example, is $H_2O$.

Sometimes two hydrogen atoms form a molecule ($H_2$). Sometimes two
oxygen atoms form a molecule ($O_2$). If you mix a lot of these
together and light a match, they will rearrange themselves into water
molecules. This is called a \textit{chemical reaction}.  In any
chemical reaction, the atoms are rearranged into new molecules.

Some chemical reactions (like the burning of hydrogen gas described
above) are \textit{exothermic} -- that is, they give off energy.
Burning hydrogen gas happens quickly and gives off a lot of energy. If
you have enough, it will make quite an explosion.

Other chemical reactions are \textit{endothermic} -- that is they consume
energy.  Photosynthesis, the process by which plants consume energy
from the sun to make sugar from $CO_2$ and $H_2O$, requires an endothermic
chemical reaction.

\section{Mass and Acceleration}

Each atom has a mass, thus everything that is made up of atoms has a
mass.  We measure mass in grams.  A paper clip is about 1 gram of
steel. An adult human can weigh 70,000 grams, so for larger things we
often talk about kilograms. A kilogram is 1000 grams.

The first interesting thing about mass is that objects with more mass
require more force to accelerate. For example, pushing a bicycle so
that it accelerates from a standstill to jogging speed in 2 seconds
requires a lot less force than pushing a train so that it accelerates
at the same rate.

Newton's second law of motion says that force necessary to accelerate
an object of mass $m$ is given by:

$$F = m a$$

That is the force is equal to the mass times the acceleration.

What are the units here? We already know that mass is measured in
kilograms. We can measure velocity in meters per second, but that is
different from acceleration. Acceleration is the rate of change of
velocity. So if we want to go from 0 to 5 meters per second (that's
jogging speed) in two seconds. That is a change in velocity of 2.5
meters per second every second. We would say this acceleration is $2.5
m/s^2$.

What about measuring force? Newton decided to name the unit after
himself: The force necessary to accelerate one kilogram at $1 m/s^2$
is known as a Newton.

\section{Mass and Gravity}

The second interesting thing about mass is that things with mass are
attracted to each other by the force we call \textit{gravity}. The
force of attraction between two objects is proportional to the product
of their masses. As objects get farther away, the force decreases.
That is why you are more attracted to the earth than you are to
distant stars, which have much more mass than the earth.

Newton is famous for coming up with the law of universal gravitation
which says that two masses ($m_1$ and $m_2$) that are a distance of
$r$ from each other, are attracted toward each other with a force of
magnitude:

$$F = G\frac{m_1 m_2}{r^2}$$

where $G$ is the universal gravitational constant. If you measure the
mass in kilograms and the distance in meters. $G$ is about $6.674
\times 10^{-11}$.  That will get you the force of the attraction in
Newtons.

