\chapter{Introduction to Electricity}

What happens when you turn on a flashlight?  The battery in the
flashlight acts as an electron pump. The electrons flow through the
wires to the lightbulb (or LED).  As the electrons pass through the
lightbulb, they excite the molecules there, which give off light and
heat. (LEDs also give off light and heat, but they give off a lot less
heat.) Then the electrons return to the battery to be pumped around
again.

When electricity is flowing through a copper wire, the protons and
neutrons of the copper stay put while the electrons jump between the
atoms on their way from the battery to the lightbulb and back again.

\section{Units}

Electrons are very small, so to study them, scientists came up with a
unit that represents \textit{a lot} of electrons. 1 \textit{coulomb}
is about 6,241,509,074,460,762,608 electrons.  When 5 coulombs pass
through a wire every second, we say ``This wire is carrying 5 ampere of
current.''\index{coulombs}

(Truthfully, we usually shorten ampere to just ``amp''.  This is
sometimes a little awkward because we often shorten the word
``amplifier'' to ``amp''. You should be able to tell which is which
from the context.)\index{amp or ampere}

If you look at the circuit breakers or fuses for your home's
electrical system, you'll see that each one is rated in amps.  For
example, maybe the circuit that supplies power to your kitchen has 10
amp circuit breaker. If for some reason, more than 10 amps tries to
pass through that wire, the circuit breaker will turn off the whole
circuit.

When it is on, your flashlight pushes something like 1 amp of current
through the lightbulb. (When it is off, there is no current in the
lightbulb.)

The lightbulb creates \textit{resistance} that the current pushes
through.\index{resistance} You can think of plumbing: The current is the amount of water
passing through a pipe. The resistence is something that tries to stop
the current -- like a ball of hair.  The battery is what creates the
pressure that pushes the current through the resistance. We call that
pressure \textit{voltage}.\index{voltage}

Resistance is measured in \textit{ohms}.  Voltage is measured in
\textit{volts}.\index{ohms}\index{volts}

\begin{mdframed}[style=important, frametitle={Ohm's Law}]\index{Ohm's law}
  Whenever a voltage $V$ is pushing a current $I$ through a resistance of $I$, the following is true:

  $$V = IR$$

  where $V$ is in volts, $I$ is in amps, and $R$ is in ohms.
\end{mdframed}

As mentioned in the introduction to work and energy, the amount of
energy that a resistance consumes is proportional to the square of the
current passing through the resistance.

\begin{mdframed}[style=important, frametitle={Joule's Law}]\index{Joule's law}

  When a current $I$ is passing through a resistance $R$, the power consumed is
  
  $$W = I^2 R$$

  where $W$ is in watts, $I$ is in amps, and $R$ is in ohms.
\end{mdframed}

Of course $V = IR$, so we can extend this to:

$$W = I^2 R = I V = \frac{V^2}{R}$$

Your flashlight's batteries probably provide about 3 volts.  How much
battery power is the flashlight using when it is on? The power (in
watts) produced by the battery is the product of the voltage (in
volts) and the current (in amps). So your flashlight giving off $3
volts \times 1 amp = 3 watts$ of power. Some of that power is given
off as light, some as heat.\index{watts}








