\chapter{Charge}

If you rub a balloon against your hair, it will stick to a wall. We
say that it has gotten an \textit{electrical charge}. It stole some
electrons from your hair, and now the ballon has slightly more
electrons than protons.  We say that it has a negative electrical
charge.

Objects with slight more protons than electrons have a positive charge.

The charge is measured in coulombs. The charge of a single proton is
about $1.6 \times 10^{-19}$ coulombs.

An object with a negative charge and an object with a positive charge
will be attracted to each other. Two objects with the same charge will
be repelled by each other.

\begin{mdframed}[style=important, frametitle={Coulomb's Law}]\index{Coulomb's law}

  If two objects with charge $q_1$ and $q_2$ (in coulombs) are $r$ meters from each other, the force of attraction or repulsion is given by

  $$F = K\frac{\lvert q_1 q_2 \rvert}{r^2}$$

    where $F$ is in newtons and $K$ is Coulomb's constant: about $8.988 \times 10^9$.
  
\end{mdframed}


\begin{Exercise}[title={Coulomb's Law}, label=charged_balloons]

Two balloons are charged with an identical quantity and type of
charge: $-5 \times 10^{-9}$ coulombs. They are held apart at a
separation distance of 12 cm. Determine the magnitude of the
electrical force of repulsion between them. 
  
\end{Exercise}
\begin{Answer}[ref=charged_balloons]

  $$F = K\frac{\lvert q_1 q_2 \rvert}{r^2} = (8.988 \times 10^9) \frac{(-5 \times 10^{-9})(-5 \times 10^{-9})}{0.12^2} = \frac{224.7 \times 10^{-9}}{0.0144} = 15.6 \times 10^{-6}$$

  15.6 micronewtons.
  
\end{Answer}

\section{Lightning}

A cloud is a cluster of water droplets and ice particles. These
droplets and ice particles are always moving up and down through the
cloud. In this process, electrons get stripped off and end up on the
water droplets at the bottom of the cloud. The air between the
droplets is a pretty good insulator, and the electrons are reluctant
to jump anywhere. However, eventually the charge gets so strong that
even the insulating properties of the air is not enought to prevent
the jump.

A lot of lightning moves within a cloud or between clouds. However, a
few jump to the earth. These bolts of lightning vary in the amount of
electrons they carry, but the average is about 15 coulombs.

The electrons heat the air they pass through. The air expands
suddenly, and the resulting shockwave is thunder.

\section{But...}

This idea that opposite charges attract creates some heavy questions
that you do not yet have the tools to work with. So the answer is
basically ``Don't ask that question now!''

However, you probably have these questions, so I will wave my hands in
the direction of the answers.

The first is ``In any atom bigger than hydrogen, there are multiple
protons in the nucleus. Why don't the protons push each other out of
the nucleus?''

We aren't ready talk about it, but there is a force called \textit{the
  nuclear force} which pulls the protons and neutrons in the nucleus
of the atom together.

Another question is ``Why do the electrons whiz around in a cloud so
far from the nucleus of the atom? Negatively charged electrons should
cling to the protons in the center, right?''

We aren't ready to talk about it, but quantum mechanics tells us that
electrons like to live a certain specific energy levels. Cuddling with
a proton isn't one of those levels.
