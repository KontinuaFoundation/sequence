\chapter{Units and Conversions}

At this point, you are working with a lot of units: grams for weight,
joules for energy, newtons for force, meters for distance, seconds for
time, etc. For each type of measurement, there are several different
units; for example, distance can be measured in feet, miles,
and light-years.

For your reference, here is a table of equivalencies:

\begin{tabular}{r | l}
  \hline
  \multicolumn{2}{c}{\textbf{Distance}}\\
  1 mile & 1.6093 kilometers \\
  1 foot & 0.3048 meters \\
  1 inch & 2.54 centimeters \\
  1 light-year & $9.461 \times 10^{12}$ kilometers\\
  \hline
  \multicolumn{2}{c}{\textbf{Volume}}\\
  1 milliliter & 1 cubic centimeter \\
  1 quart & 0.9461 liters \\
  1 gallon & 3.7854 liters \\
  1 fluid ounce & 29.6 milliliters \\
  \hline
  \multicolumn{2}{c}{\textbf{Mass}}\\
  1 pound & 0.4535924 kilograms\\
  1 ounce & 0.4535924 grams\\
  1 metric ton & 1000 kilograms \\
  \hline
  \multicolumn{2}{c}{\textbf{Force}}\\
  1 newton & 1 kilogram meter per sec$^2$\\
  \hline
  \multicolumn{2}{c}{\textbf{Pressure}}\\
  1 pascal & 1 newton per square meter \\
  1 bar & 0.98692 atmosphere \\
  1 pound per square inch & 6897 pascals \\
  \hline
  \multicolumn{2}{c}{\textbf{Energy}}\\
  1 joule & 1 newton meter \\
  1 calorie & 4.184 joules \\
  1 kilowatt-hour & $3.6 \times 10^{6}$ joules  \\
\end{tabular}\index{units table}

(You don't need to memorize these! Just remember that this page is here.)

In the metric system, prefixes are often used to express a multiple. Here are the common prefixes:\index{metric system!prefixes}

\begin{tabular}{r | l}
giga  & $\times 10^{9}$\\
mega  & $\times 10^{6}$\\
kilo  & $\times 10^{3}$\\
milli  & $\div 10^{3}$\\
micro  & $\div 10^{6}$\\
nano  & $\div 10^{9}$\\
\end{tabular}

(These are worth memorizing.)

\section{Conversion Factors}

Here is a really handy trick to remembering how to do conversions
between units.\index{conversion factors}

Often, you will a table like the one above, and someone will ask you
``How many miles are in 0.23 light-years?''  You know that 1 mile = 1.6093
kilometers and that 1 light-year is $9.461 \times 10^{12}$ kilometers.
How do you do the conversion?

The trick is to treat the two parts of the equality as a fraction that equals 1.  That is, you think:

$$\frac{1 \text{ miles}}{1.6093 \text{ km}} = \frac{1.6093 \text{ km}}{1 \text{ miles}} = 1$$

and

$$\frac{1 \text{ light-years}}{9.461 \times 10^{12} \text{ km}} = \frac{9.461 \times 10^{12} \text{ km}}{1 \text{ light-years}} = 1$$

We call these fractions \textit{conversion factors}.

Now, your problem is

$$0.23 \text{ light-years} \times \textit{ Some conversion factors} = ? \text{ miles}$$

Note that when you multiply fractions together, things in the numerators can cancel with things in the denominator:

$$\left( \frac{31\pi}{47} \right) \left( \frac{11}{37\pi}\right) = \left(\frac{31\cancel{\pi}}{47}\right) \left( \frac{11}{37\cancel{\pi}}\right) = \left(\frac{31}{47} \right) \left( \frac{11}{37} \right)$$

When working with conversion factors, you will do the same with the units:

\begin{multline*}
  0.23 \text{ light-years} \left( \frac{9.461 \times 10^{12} \text{ km}}{1 \text{ light-years}} \right) \left( \frac{1 \text{ miles}}{1.6093 \text{ km}} \right) = \\
  0.23 \text{ \cancel{light-years}} \left( \times \frac{9.461 \times 10^{12} \text{ \cancel{km}}}{1 \text{ \cancel{light-years}}} \right) \left( \frac{1 \text{ miles}}{1.6093 \text{ \cancel{km}}}\right) = \frac{(0.23)(9.461 \times 10^{12})}{1.6093} \text{ miles}$$
\end{multline*}

\begin{Exercise}[title={Simple Conversion Factors}, label=simple_conversion_factors]

  How many calories are in 4.5 killowatt-hours?
  
\end{Exercise}
\begin{Answer}[ref=simple_conversion_factors]

  $$4.5 \text{ \cancel{kWh}} \left( \frac{3.6 \times 10^{6} \text{ \cancel{joules}}}{1 \text{ \cancel{kWh}}} \right) \left( \frac{1 \text{ calories}}{4.184 \text{ \cancel{joules}}}\right) = \frac{(4.5)(3.6 \times 10^6)}{4.184} = 1.08 \times 10^6 \text {calories}$$
  
\end{Answer}

\section{Conversion Factors and Ratios}

Conversion factors also work on ratios.  For example, if you are told
that a bug is moving 0.5 feet every 120 milliseconds. What is that in
meters per second?

The problem, then is

$$\frac{0.5 \text{ feet}}{120 \text{ milliseconds}} = \frac{\text{? m}}{second}$$

So you will need conversion factors to replace the ``feet'' with ``meters'' and to replace ``milliseconds'' with ``seconds'':

\begin{multline*}
\left(\frac{0.5 \text{ \cancel{feet}}}{120 \text{ \cancel{milliseconds}}}\right) \left( \frac{0.3048 \text{ meters}}{1 \text{ \cancel{feet}}} \right) \left( \frac{ 1000 \text{ \cancel{milliseconds}}} {1 \text{ second}}\right) = \frac{(0.5)(0.3048)(1000)}{120}\text{ m/second}
\end{multline*}

\begin{Exercise}[title={Conversion Factors}, label=conversion_factors]

The hole in the bottom of the boat lets in 0.1 gallons every 2 minutes.  How many milliliters per second is that?
  
\end{Exercise}
\begin{Answer}[ref=onversion_factors]

  \begin{multline*}
    \frac{0.1 \text{ \cancel{gallons}}}{2 \text{ \cancel{minutes}}}
  \left( \frac{3.7854 \text{ \cancel{liters}}}{1 \text{ \cancel{gallons}}} \right)
  \left( \frac{1000 \text{ milliliters}}{1\text{ \cancel{liters}}}\right)
  \left( \frac{1 \text{ \cancel{minutes}}}{60 \text{ seconds}} \right) = \\
  \frac{(0.1)(3.7854)(1000)}{(2)(60)} \text{ ml/second} = 3.1545 \text{ ml/second}
  \end{multline*}
  
\end{Answer}

\section{When Conversion Factors Don't Work}

Conversion factors only work when the units being converted are
proportional to each other. Gallons and liters, for example are
proportional to each other: If you have $n$ gallons, you have $n
\times 3.7854$ liters.

Degrees celsius and degrees farenheit are \textit{not} proportional to
each other.  If your food is $n$ degrees celsius, it is $n \times
\frac{9}{5} + 32$ degrees farenheit.  You can't use conversion factors
to convert celsius to farenheit.
