\chapter{Permutations and Sorting}

In the last chapter, we talked about permutations. If you have a list
of three letters, like $[a, b, c, d]$, you can rearrange them in $4!$
ways:\index{permutations}

\begin{tabular}{c c c c c c}
  a,b,c,d & a,b,d,c & a, d, b, c & a, d, c, b & a, c, b, d & a, c, d, b \\
  b,a,c,d & b,a,d,c & b, d, a, c & b, d, c, a & b, c, a, d & b, c, d, a \\
  c,b,a,d & c,b,d,a & c, d, b, a & c, d, a, b & c, a, b, d & c, a, d, b \\
  d,b,c,a & d,b,a,c & d, a, b, c & d, a, c, b & d, c, b, a & d, c, a, b
\end{tabular}

You can make Python generate all the permutations for you:

\begin{Verbatim}
from itertools import permutations
all_permutations = permutations(('a', 'b', 'c', 'd'))
for p in all_permutations:
    print(p)
\end{Verbatim}

\section{Notation}

How do we define or write down a single permutation? You could say
something like ``Swap the first and second items and swap the third
and fourth items.'' However, that gets pretty difficult to read. So we
usually write a permutation as two lines: the first line is before
permutation and the second line is after.  Like this:

$$\begin{pmatrix}
  1 & 2 & 3 & 4 \\
  2 & 1 & 4 & 3
\end{pmatrix}$$

And can we assign permutations to variables. For example, if we wanted the
variable $A$ to represent ``swapping the first and second item'', we
would write this:

$$A = \begin{pmatrix}
  1 & 2 & 3 & 4 \\
  2 & 1 & 3 & 4
\end{pmatrix}$$

And if we wanted $B$ to represent ``swapping the third and fourth item'', we would write:

$$B = \begin{pmatrix}                                                                                                                             
  1 & 2 & 3 & 4 \\                                                                                                                                
  1 & 2 & 4 & 3
\end{pmatrix}$$

Now, we can \textit{compose} permutations together. For example, we might say:\index{permutations!composing}

$$B \circ A = \begin{pmatrix}
  1 & 2 & 3 & 4 \\
  2 & 1 & 4 & 3
\end{pmatrix}$$

That is, if we have the list $[a, b, c, d]$ and we apply permutation $A$ and then permutation $B$, we get $[b, a, d, c]$.

\textbf{Important:} Note that permutations are applied from right to
left.  $B \circ A$ means ``Applying $A$ and then $B$.''  Who cares?
Permutations are not necessarily commutative. That is, if you have two
permutations $S$ and $T$, $S \circ T$ is not always the same as $T
\circ S$.

Note that ``don't change anything'' is also a permutation. We call it
\textit{the identity permutation}. If you have four items, the identity
permutation would be written:\index{permutations!identity permutation}

$$I = \begin{pmatrix}
  1 & 2 & 3 & 4 \\
  1 & 2 & 3 & 4
\end{pmatrix}$$

(We use capital ``I'' for the identity.)

\subsection{Challenge} Find an example of two permutations $S$ and $T$ such that $S \circ T$ does not equal $T \circ S$.

\section{Sorting in Python}

One of the common forms of permutation in software is sorting.
Sorting is putting data in a particular order. For example, in Python,
if you had a list of numbers, you can sort it in ascending order like
this:\index{sorting}

\begin{Verbatim}
my_grades = [92, 87, 76, 99, 91, 93]
grades_worst_to_best = sorted(my_grades)
\end{Verbatim}

You want to sort backwards?

\begin{Verbatim}
my_grades = [92, 87, 76, 99, 91, 93]
grades_best_to_worst = sorted(my_grades, reverse=True)
\end{Verbatim}

Note that \pyfunction{sorted} makes a new list with the correct
order. If you want to sort the array in place, you can use the
\pyfunction{sort} method:

\begin{Verbatim}
my_grades = [92, 87, 76, 99, 91, 93]
my_grades.sort(reverse=True)
\end{Verbatim}

\section{Inverses}

Think for a second about this permutation:

$$S = \begin{pmatrix}
  1 & 2 & 3 & 4 \\
  3 & 4 & 2 & 1
\end{pmatrix}$$

You could say this permutation shuffles a list a bit.  What is its
inverse? That is, what is the permutation that unshuffles the items
back to where they were originally?\index{permutations!inverses}

$$S^{-1} = \begin{pmatrix}
  1 & 2 & 3 & 4 \\
  4 & 3 & 1 & 2
\end{pmatrix}$$

That is, the original moved an item in the first spot to the third
spot. The inverse must move what ever was in the third spot back to
the first spot.

(Notation note: Because in multiplication, $b \times b^{-1} = 1$, we
use ``to the negative one'' to indicate inverses in lots of places.)

Mechanically, how do you find the inverse? Flip the rows, and then sort the columns using the top number:

$$\begin{pmatrix}
  1 & 2 & 3 & 4 \\
  3 & 4 & 2 & 1
\end{pmatrix}
\text{ flip }\rightarrow
\begin{pmatrix}
  3 & 4 & 2 & 1 \\
  1 & 2 & 3 & 4
\end{pmatrix}
\text{ sort }\rightarrow
\begin{pmatrix}
  1 & 2 & 3 & 4 \\
  4 & 3 & 1 & 2
\end{pmatrix}
$$    

Let's say you have two permutations $A$ and $B$.  Permuting by $B$ and then $A$ would look like this:

$$C = A \circ B$$

If you know $A^{-1}$ and $B^{-1}$, what is $C^{-1}$?  You would un-$A$ and then un-$B$, so

$$C^{-1} = B^{-1} \circ A^{-1}$$

\section{Cycles}

Here is a permutation:

$$\begin{pmatrix}
  1 & 2 & 3 & 4 & 5 \\
  2 & 4 & 5 & 1 & 3
\end{pmatrix}$$

When this is applied, whatever is at 1 gets moved to 2, 2 gets moved
to 4, and 4 gets moved to 1.  That is a \textit{cycle}: $1 \rightarrow
2 \rightarrow 4$ and then it goes back to 1. It involves three locations, so we say
it is a \textit{3-cycle}.\index{permutations!cycles}

There is another cycle in this permutation: $3 \rightarrow 5$ and then it goes back to 3.

Because these cycles share no members, we say the cycles are \textit{disjoint}.

Every permutation can be broken down into a collection of disjoint cycles.

$$T = \begin{pmatrix}
  1 & 2 & 3 & 4 & 5 \\
  2 & 4 & 5 & 1 & 3
\end{pmatrix} = (1 \rightarrow 2 \rightarrow 4)(3 \rightarrow 5)$$

The first handy thing about this notation is that it makes it easy for
us to describe the inverse: we just run the cycles backwards:

$$T^{-1} = (4 \rightarrow 2 \rightarrow 1)(5 \rightarrow 3)$$

Starting with the list $[a, b, c, d, e]$, lets repeatedly apply the permutation $T$

\begin{tabular}{r | l}
  Initial & {\color{red} a, b,} {\color{blue} c,} {\color{red} d,} {\color{blue} e} \\ 
  $T$ applied & {\color{red} d, a,}  {\color{blue} e,} {\color{red} b,} {\color{blue} c}\\
  $T \circ T$ applied & {\color{red} b, d,} {\color{blue} c,} {\color{red} a,}  {\color{blue} e} \\
  $T \circ T \circ T$ applied & {\color{red} a, b,}  {\color{blue} e,} {\color{red} d,} {\color{blue} c} \\
  $T \circ T \circ T \circ T$ applied & {\color{red} d, a,} {\color{blue} c,} {\color{red}b,}  {\color{blue} e} \\
  $T \circ T \circ T \circ T \circ T$ applied & {\color{red} b, d,}  {\color{blue} e,} {\color{red} a,} {\color{blue} c} \\
  $T \circ T \circ T \circ T \circ T \circ T$ applied & {\color{red} a, b,} {\color{blue} c,} {\color{red} d,}  {\color{blue} e}\\
\end{tabular}

This permutation, results in six combinations, and then it loops back
on itself. The number of combinations is the least common multiple of
all the cycles.  In this case, there is a 3-cycle and a 2-cycle. The
least common multiple of 2 and 3 is 6.

