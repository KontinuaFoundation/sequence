\chapter{Introduction to Discrete Probability}

First, let's take care of the word \emph{discrete} vs \emph{discreet}.
They sound exactly same, but ``discrete'' means ``individually
separate and distinct'' and ``discreet'' means ``careful about what
other people know''.  So you might say, ``You can think of light as a
continuous wave or as a blast of discrete particles.'' And you might say,
``Please go get the box of doughnuts from the kitchen. Oh, and there
are a lot of hungry people in the house, so be discreet.''

When we are talking about probabilities, there are problems that deal
with discrete quantities like ``What is the probability that I will
throw these three dice and the numbers that come up sum to 9?''. There
are also problems that deal with continuous properties like ``What is
the probability that the next bird to fly over my house will weigh
between 97.2 and 98.1 grams ?'' In this module, we are going to focus
on the probability problems that deal with discrete quantities.

Watch Khan Academy's Introduction to Probability at \url{https://youtu.be/uzkc-qNVoOk}.

Let's say that I have a cloth sac filled with 100 marbles; 99 are red
and 1 is white. If I ask you to reach in without looking and pull out
one marble, you will probably pull out a red one. We say that ``There
is a 1 in 100 chance that you would pull out a white marble.'' Or we
can use percentages and say ``There is a 1\% chance that you will pull
out a white marble.'' Or we can use decimals and say ``There is a 0.01
probability that you will pull out a white marble.''

In probability, we often talk about the probability of certain
events. ``Pulling out a white marble'' is an event, and we can give it
a symbol like $W$.  Then, in equations we use $p$ to mean ``the
probability of''.  Thus, we can say ``There is a 0.01 probability that
you will pull out a white marble'' becomes the equation
\begin{equation*}
  p(W) = 0.01
\end{equation*}


\section{The Probability of All Possibilities is 1.0}

We know that you are going to pull out a red marble or a white marble,
so the probability of a white marble being pulled and the probability
of a red marble being pulled must add up to be 100\%.  So the odds of
pulling out a red marble must be 99\% or 0.99. If we let the event ``Pull out a red marble'' be given by the symbol $R$, we can say:
\begin{equation*}
  p(R) = 1.0 - P(W) = 1.0 - 0.01 = 0.99
\end{equation*}

Now, let's say that I make you take a marble from the bag and then
toss a coin. What is the probability that you will pull a white marble
and then get heads on the coin? It is the product of the two
probabilities: $0.01 \times 0.5 = 0.005$, so one half of one percent
chance.  Do the probabilities still sum to 1?
\begin{itemize}
\item White and Heads = $0.01 \times 0.5 = 0.005$
\item White and Tails = $0.01 \times 0.5 = 0.005$
\item Red and Heads = $0.99 \times 0.5 = 0.495$
\item Red and Tails = $0.99 \times 0.5 = 0.495$
\end{itemize}
Yes, the probabilitites of all the possibilities still add to 1.

\begin{Exercise}[title={Rolling Dice}, label=rolling-dice]
  If I give you three dice to roll, what is the
  probability that you will roll a 5 on all three dice?
\end{Exercise}
\begin{Answer}[ref=rolling-dice]
  probability of all 5's $ = \frac{1}{6}\times\frac{1}{6}\times\frac{1}{6} = \left(\frac{1}{6}\right)^3 = \frac{1}{216} \approx 0.0046$
  \end{Answer}
    
\begin{Exercise}[title={Flipping Coins}, label=flipping-coins]
  If I give you five coins to flip, what is the
  probability that at least one coin will come up heads?
\end{Exercise}
\begin{Answer}[ref=rolling-dice]
  probability of at least one heads = 1.0 - probability of all tails $ = 1.0 - \left(\frac{1}{2}\right)^5 =1.0 - \frac{1}{32} = \frac{31}{32} = \approx 0.97$ 
  \end{Answer}
    
\section{Why 7 is the mostly likely sum of two dice}

If you roll two dice, the sum will can be 2 or 12 or any number in
between. It is very tempting to assume that the likelihood of any of
those numbers is the same. In fact, the probability of a 2 is
$\frac{1}{36} \approx 3\%$ and the probability of a 7 is $\frac{1}{6}
\approx 17\%$. A 7 is six times more likely than a 12! Why?

When you roll the first die, there are six possibilities with equal
probability. When you roll the second die, there are six possibilities
with equal probability. so there are a total of 36 possible events
with equal probabilities: 1 then 1, 1 then 2, 2 then 1, 1 then 3, 3
then 1, etc. Only one of these (1 then 1) adds to 2.  But six of these
sum to 7: 1 then 6, 6 then 1, 2 then 5, 5 then 2, 3 then 4, 4 then
3. So a 7 is six times more likely than a 2.


