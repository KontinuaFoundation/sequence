\chapter{Trignometric Functions}

As mentioned earlier, in a right triangle where one angle is $\theta$,
the sine of $\theta$ is the length of the side opposite $\theta$
divided by the length of the hypotenuse.

The sine function is defined for any real number. We treat that real number
$\theta$ as an angle, we draw a ray from the origin out to the unit
circle.  The $y$ value of that point is the sine. So, for example,
the $\sin(\frac{4\pi}{3})$ is $-\sqrt{3}/2$

\begin{tikzpicture}[declare function={angle=240;},bullet/.style={inner
    sep=1pt,fill,draw,circle,solid}, scale=3]
    % Axis
    \draw[thick,-stealth,black] (-1.2,0)--(1.2,0) node[right] {$x$}; % x axis
    \draw[thick,-stealth,black] (0,-1.2)--(0,1.2) node[left] {$y$}; % y axis
    % Rest
    \draw (0,0) circle (1);
    \draw[thick] (0,0) -- (angle:1.0) node [midway, right] {1};
    \draw[sdkblue] (-0.1, 0.32) node[above] {$\theta = \frac{4\pi}{3}\text{ radians} = 240^\circ$};
    \draw[-stealth,sdkblue] (0.3,0) arc (0:angle:0.3);
    \draw[dashed, black] (-0.7, -0.866) -- (0.05, -0.866) node[right] {$\sin(\theta) = -\sqrt{3}/2$}; % horizontal
    \filldraw[black] (angle:1.0) circle(1pt);
\end{tikzpicture}

(Note that in this section, we will be using radians instead of
degrees unless otherwise noted. While degrees is more familiar to most
people, engineers and mathematicians nearly always use radians when
solving problems. Your calculator should have a radians mode and a
degrees mode. You want to be in radians mode.)

Similarly, we define cosine using the unit circle: to find the cosine
of $\theta$, we draw a ray from the origin at the angle $\theta$. The
$x$ component of the point where the ray intersects the unit circle is
the cosine of $\theta$.

\begin{tikzpicture}[declare function={angle=240;},bullet/.style={inner
    sep=1pt,fill,draw,circle,solid}, scale=3]
    % Axis
    \draw[thick,-stealth,black] (-1.2,0)--(1.2,0) node[right] {$x$}; % x axis
    \draw[thick,-stealth,black] (0,-1.2)--(0,1.2) node[left] {$y$}; % y axis
    % Rest
    \draw (0,0) circle (1);
    \draw[thick] (0,0) -- (angle:1.0) node [midway, right] {1};
    \draw[sdkblue] (0.1, 0.32) node[above] {$\theta = \frac{4\pi}{3}\text{ radians} = 240^\circ$};
    \draw[-stealth,sdkblue] (0.3,0) arc (0:angle:0.3);
    \draw[dashed, black]  (-0.5, -0.95) -- (-0.5, 0.05) node[left, above] {$\cos(\theta) = -0.5$}; % horizontal
    \filldraw[black] (angle:1.0) circle(1pt);
\end{tikzpicture}

From this description, it is easy to see why $\sin(\theta)^2 +
\cos(\theta)^2 = 1$.  They are the legs of a right triangle with a
hypotenuse of length 1.

It should also be easy to see why $\sin(\theta) = \sin(\theta +
2\pi)$: Each time you go around the circle, you come back to where
you started.

Can you see why $\cos(\theta) = \sin(\theta + \pi/2)$? Turn the picture sideways.

\section{Graphs of sine and cosine}

Here is a graph of $y = \sin(x)$:

\begin{tikzpicture}[
tl/.style = {% tick labels
    fill=white, inner sep=1pt, font=\scriptsize,
            },                        ]
% grid
\draw[sdkblue, very thin, xstep=0.5235, ystep=0.5] (-6.6,-1.2) grid (6.6,1.2);

% y tick label
\foreach \y in {-1, -1/2, 1/2, 1}{\node[tl,left=1mm] at (0,\y) {$\y$};}
% x tick label
\foreach \x [count=\xx from -4] in 
       {-2\pi,
        -\frac{3\pi}{2},
        -\pi,           
        -\frac{\pi}{2}, 
        { },
         \frac{\pi}{2},
         \pi, 
         \frac{3\pi}{2}, 
         2\pi
        }{\node[tl,below=1mm] at (3*0.5235*\xx,0) {$\x$};}
% axes
    \draw[->,thick] (-6.5,0) -- (6.5,0) node[right] {$x$};
    \draw[->,thick] (0,-1.25) -- (0, 1.25) node[above] {$y$};
% curve
\draw[<->,thick,draw=black,
      domain=-6.5:6.5,samples=300,variable=\x] 
      plot (\x,{sin(deg{\x})});
\end{tikzpicture}

It looks like waves, right? It goes forever to the left and
right. Remembering that $\cos(\theta) = \sin(\theta + \pi/2)$, we can
guess what the graph of $y = \cos(x)$ looks like:
    
\begin{tikzpicture}[
tl/.style = {% tick labels
    fill=white, inner sep=1pt, font=\scriptsize,
            },                        ]
% grid
\draw[sdkblue, very thin, xstep=0.5235, ystep=0.5] (-6.6,-1.2) grid (6.6,1.2);

% y tick label
\foreach \y in {-1, -1/2, 1/2, 1}{\node[tl,left=1mm] at (0,\y) {$\y$};}
% x tick label
\foreach \x [count=\xx from -4] in 
       {-2\pi,
        -\frac{3\pi}{2},
        -\pi,           
        -\frac{\pi}{2}, 
        { },
         \frac{\pi}{2},
         \pi, 
         \frac{3\pi}{2}, 
         2\pi
        }{\node[tl,below=1mm] at (3*0.5235*\xx,0) {$\x$};}
% axes
    \draw[->,thick] (-6.5,0) -- (6.5,0) node[right] {$x$};
    \draw[->,thick] (0,-1.25) -- (0, 1.25) node[above] {$y$};
% curve
\draw[<->,thick,draw=black,
      domain=-6.5:6.5,samples=300,variable=\x] plot (\x,{cos(deg{\x})});
\end{tikzpicture}

Here is a wonderful property of sine and cosine functions:
\begin{itemize}
\item At any point $\theta$, the slope of the sine graph at $\theta$ equals $cos(\theta)$.
\item At any point $\theta$, the area under the cosine graph from zero to $\theta$ equals $sin(\theta)$.
\end{itemize}

For example, we know that $\sin(4\pi/3) = -(1/2)\sqrt{3}$ and
$\cos(4\pi/3) = -1/2$. If we drew a line tangent to the sine curve at
this point, it would have a slope of -1/2:

\begin{tikzpicture}[
tl/.style = {% tick labels
    fill=white, inner sep=1pt, font=\scriptsize,
            },                        ]
% grid
\draw[sdkblue, very thin, xstep=0.5235, ystep=0.5] (-1.25,-1.7) grid (6.6,1.2);

% y tick label
\foreach \y in {-3/2, -1, -1/2, 1/2, 1}{\node[tl,left=1mm] at (0,\y) {$\y$};}
% x tick label
\foreach \x [count=\xx from -1] in 
       {-\frac{\pi}{2}, 
        { },
         \frac{\pi}{2},
         \pi, 
         \frac{3\pi}{2}, 
         2\pi
        }{\node[tl,below=1mm] at (3*0.5235*\xx,0) {$\x$};}
% axes
    \draw[->,thick] (-1.25,0) -- (6.5,0) node[right] {$x$};
    \draw[->,thick] (0,-1.5) -- (0, 1.25) node[above] {$y$};
% curve
\draw[<->,thick,draw=black,
      domain=-1.75:6.5,samples=300,variable=\x] 
      plot (\x,{sin(deg{\x})});
\filldraw[black] (4.188790204786391,-0.866025403784439) circle(2pt);
\draw[->, thick, draw=red] (4.188790204786391,-0.866025403784439) -- (5.188790204786391,-1.366025403784439) node [right] {slope = -1/2} ;
\end{tikzpicture}

We say ``The derivative of the sine function is the cosine function.''

If we take the area between the graph and the $x$ axis of the cosine
function (and if the function is below the $x$ axis, it counts as
negative area), from 0 to $4\pi/3$, we find that it is equal to
$-(1/2)\sqrt{3}$

\begin{tikzpicture}[
tl/.style = {% tick labels                                                                                               
    fill=white, inner sep=1pt, font=\scriptsize,
            },                        ]

% y tick label                                                                                                           
\foreach \y in {-1, -1/2, 1/2, 1}{\node[tl,left=1mm] at (0,\y) {$\y$};}
% x tick label                                                                                                           
\foreach \x [count=\xx from -1] in
       {-\frac{\pi}{2},
        { },
         \frac{\pi}{2},
         \pi,
         \frac{3\pi}{2},
         2\pi
        }{\node[tl,below=1mm] at (3*0.5235*\xx,0) {$\x$};}
       % axes
       \draw[->,thick] (-1.25,0) -- (6.5,0) node[right] {$x$};
       \draw[->,thick] (0,-1.25) -- (0, 1.25) node[above] {$y$};
       % curve
       \draw[<->,thick,draw=black, domain=-1.75:6.5,samples=300,variable=\x] plot (\x,{cos(deg{\x})});
       \fill[sdkblue, domain=0:1.57,samples=100, variable=\b]
       (0, 1)
       -- plot (\b,{cos(deg(\b))})
       -- (0, 0)
       -- cycle;
       \fill[red, domain=1.57:4.188790204786391,samples=100, variable=\b]
       (1.57, 0)
       -- plot (\b,{cos(deg(\b))})
       -- (4.188790204786391, 0)
       -- cycle;
       \draw[thick, draw=black] (4.188790204786391, 1) -- (4.188790204786391,-1) node [right]{area=$-(1/2)\sqrt{3}$};
\end{tikzpicture}

We say ``The integral of the cosine function is the sine function.''

Can you guess the derivative of the cosine function? For any $\theta$, the slope of the graph of the $\cos(\theta)$ is $-\sin(\theta)$.

\section{A weight on a spring}

If a weight is sitting still on a stiff spring, and you try to push it
down, it will push back against you, as if trying to return to its
equilibrium.  The farther you push it, the harder it will push
back. That is, the magnitude of force is proportional to the magnitude
of the displacement vector. However the two vectors point in opposite
directions.

If you release the weight, it will accelerate back toward its
equilibrium. At any moment of that journey, its acceleration is
proportional to its displacement.  The acceleration will be zero as
the weight passes its equilibrium point, however, its velocity will
not be zero. So it will overshoot its equilibrium point.

If you plotted the displacement of the weight, it would look like a
sine wave. (However, friction and wind resistance would gradually
absorb the energy, so the wave would get smaller and eventually the
weight would come to rest back at its equilibrium point.)

In these sorts of systems (and there are a lot of them), we know the
acceleration is proportional to the displacement, but in the opposite
direction. In math symbols:

$$a \propto -1 * p$$

where $a$ is acceleration and $p$ is the displacement from equilibrum.

Remember that if you take the derivative of the displacement, you get
the velocity.  And if you take the derivative of that, you get
acceleration.  So, we are looking for a function $f$ such that

$$f(t) \propto -1 * f''(t)$$

Remember that the derivative of the $\sin(\theta)$ is $\cos(\theta)$.

And the derivative of the $\cos(\theta)$ is $- \sin(\theta)$

Thus these sorts of waves have an almost-magical power: their
acceleration is proportional to -1 times their displacement.

Thus sine waves of various magnitudes and frequencies are ubiquitous
in nature and technology.








