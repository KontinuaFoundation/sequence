\chapter{Exponents}

Let's quickly review exponents. Ancient scientists started coming up
with a lot of formulas that involved multiplying the same number
several times.  For example, if they knew that if a sphere was $r$
centimeters in radius, its volume in milliliters was

$$V = \frac{4}{3} \times \pi \times r \times r \times r$$

They did two things to make the notation less messy.  First, they
decided that if two numbers were written next to each other, the
reader would assume that meant ``multiply them''.  Second, they came
up with the exponent, a little number that was lifted off the
baseline of the text, that meant ``multiply it by itself''.  For
example $5^3$ was the same as $5 \times 5 \times 5$.\index{exponents}

Now the formula for the volume of a sphere is written

$$V = \frac{4}{3} \pi r^3$$

Tidy, right? In an exponent expression like this, we say that $r$ is
\textit{the base} and $3$ is \textit{the exponent}.

\section{Identities for Exponents}

What about exponents of exponents?  What is $\left(5^3\right)^2$?

$$\left(5^3\right)^2 = (5 \times 5 \times 5)^2 = (5 \times 5 \times 5)(5 \times 5 \times 5) = 5^6$$

In general, for any $a$, $b$, and $c$:

$$\left(a^b\right)^c = a^{(bc)}$$

If you have $\left( 5^3 \right) \left(5^4 \right)$ that is just $5 \times 5 \times 5 \times 5 \times 5 \times 5 \times 5$ or $5^7$

The general rule is, for any $a$, $b$, and $c$

$$\left(a^b\right)\left(a^c\right) = a^{(b + c)}$$

Mathematicians \textit{love} this rule, so we keep extending the idea
of exponents to keep this rule true. For example, at some point,
someone asked ``What about $5^0$?'' According to the rule, $5^{2}$
must equal $5^{(2 + 0)}$ which must equal
$\left(5^2\right)\left(5^0\right)$.  Thus, $5^2$ must be 1. So
mathematicians declared ``Anything to the power of 0 is 1''.\index{exponents!zero}

We don't typically assume that $0^0 = 1$. It is just too
weird. So we say, that for any $a$ not equal to zero,

$$a^0 = 1$$

What about $5^{(-2)}$?  By our beloved rule, we know that
$\left(5^{-2}\right)\left(5^5\right)$ must be equal to $5^3$, right?
So $5^{-2}$ must be equal to $\frac{1}{5^2}$.\index{exponents!negative}

We say, for any $a$ not equal to zero and any $b$,

$$a^{-b} = \frac{1}{a^{b}}$$

This makes dividing one exponential expression by another (with the same base) easy:

$$\frac{a^b}{a^c} = a^{(b-c)}$$

We often say ``cancel out'' for this.  Here I can ``cancel out'' $x^2$:

$$\frac{x^5}{x^2} = x^3$$

What about $5^{\frac{1}{3}}$? By the beloved rule, we know that $5^{\frac{1}{3}}5^{\frac{1}{3}}5^{\frac{1}{3}}$ must equal $5^1$. Thus $5^{\frac{1}{3}} = \sqrt[3]{5}$.\index{exponents!fractions}

We say, for any $a$ and $b$ not equal to zero and any $c$ greater than zero,

$$a^{\frac{b}{c}} = a^b \sqrt[c]{a}$$

Before you go on to the exercises, note that the beloved rule demands a common base.
\begin{itemize}
\item We can combine these: $\left(5^2\right)\left(5^4\right) = 5^6$
\item We cannot combine: $\left(5^2\right)\left(3^5\right)$
\end{itemize}

With that said, we note for any $a$,$b$, and $c$:

$$\left(ab\right)^c = \left(a^c\right) \left(b^c\right)$$

So, for example, if I were asked to simplify
$\left(3^4\right)\left(6^2\right)$, I would note that $6 = 2 \times
3$, so

$$\left(3^4\right)\left(6^2\right) = \left(3^4\right)\left(3^2\right)\left(2^2\right)  = \left(3^6\right)\left(2^2\right)$$


If these ideas are new to you (or maybe they have been forgotten),
watch the Khan Academy's \textbf{Intro to rational exponents} video at
\url{https://youtu.be/lZfXc4nHooo}.

%https://www.pinterest.com/pin/800796377464592621/
