\chapter{Introduction to Triangles}

Connecting any three points with three line segments will get you a
triangle. Here is the triangle $ABC$ created by connecting three points $A$, $B$, and $C$:

\begin{tikzpicture}[scale=1.5]
  \coordinate [circle, fill, inner sep=2pt] (a) at (0,0) ;
  \coordinate [circle, fill, inner sep=2pt] (b) at (1, 2) ;
  \coordinate [circle, fill, inner sep=2pt] (c) at (3,0) ;
  \draw (a)--(b) node [outer sep=3pt, above]{$B$};
  \draw (b)--(c) node[outer sep=3pt, right]{$C$};
  \draw (c)--(a) node[outer sep=3pt, left]{$A$};
\end{tikzpicture}

If all three sides of the triangle are the same length, we say it is an \emph{equilateral triangle}:

\begin{tikzpicture}[scale=1.5]
  \coordinate [circle, fill, inner sep=2pt] (a) at (0,0) ;
  \coordinate [circle, fill, inner sep=2pt] (b) at (1.5, 2.6) ;
  \coordinate [circle, fill, inner sep=2pt] (c) at (3,0) ;
  \draw (a)--(b) node [outer sep=3pt, above]{$B$};
  \draw (b)--(c) node[outer sep=3pt, right]{$C$};
  \draw (c)--(a) node[outer sep=3pt, left]{$A$};
  \tkzMarkSegment[pos=.5,mark=||](a,b)
  \tkzMarkSegment[pos=.5,mark=||](b,c)
  \tkzMarkSegment[pos=.5,mark=||](c,a)
\end{tikzpicture}

If only two sides of the triangle are the same length, we say it is an \emph{isosceles triangle}:

\begin{tikzpicture}[scale=1.3]
  \coordinate [circle, fill, inner sep=2pt] (a) at (0,0) ;
  \coordinate [circle, fill, inner sep=2pt] (b) at (1.5, 4) ;
  \coordinate [circle, fill, inner sep=2pt] (c) at (3,0) ;
  \draw (a)--(b) node [outer sep=3pt, above]{$B$};
  \draw (b)--(c) node[outer sep=3pt, right]{$C$};
  \draw (c)--(a) node[outer sep=3pt, left]{$A$};
  \tkzMarkSegment[pos=.5,mark=||](a,b)
  \tkzMarkSegment[pos=.5,mark=||](c,b)
\end{tikzpicture}

We talk a lot about the interior angles of a triangle:

\begin{tikzpicture}[scale=2]
  \coordinate [circle, fill, inner sep=2pt] (a) at (0,0) ;
  \coordinate [circle, fill, inner sep=2pt] (b) at (1, 2) ;
  \coordinate [circle, fill, inner sep=2pt] (c) at (3,0) ;
  \draw (a)--(b) node [outer sep=3pt, above]{$B$};
  \draw (b)--(c) node[outer sep=3pt, right]{$C$};
  \draw (c)--(a) node[outer sep=3pt, left]{$A$};
  \pic [draw, <->, "$a$", angle eccentricity=1.5] {angle = c--a--b};
  \pic [draw, <->, "$b$", angle eccentricity=1.5] {angle = a--b--c};
  \pic [draw, <->, "$c$", angle eccentricity=1.5] {angle = b--c--a};
\end{tikzpicture}

You have probably heard of angles. For example, you know that these line segments meet at a  \emph{right angle}:

\begin{tikzpicture}[scale=2]
  \coordinate (a) at (0,1) ;
  \coordinate [circle, fill, inner sep=2pt] (b) at (0,0) ;
  \coordinate (c) at (1,0) ;
  \draw (a)--(b);
  \draw (b)--(c);
  \pic [draw, <->, "90$^\circ$", angle eccentricity=1.9] {angle = c--b--a};  
\end {tikzpicture}
A right angle is $90^\circ$.

A triangle where one of the interior angles is a right angle is said to be a \emph{right triangle}:

\begin{tikzpicture}[scale=1.5]
  \coordinate [circle, fill, inner sep=2pt] (a) at (0,0) ;
  \coordinate [circle, fill, inner sep=2pt] (b) at (0,4) ;
  \coordinate [circle, fill, inner sep=2pt] (c) at (3,0) ;
  \draw (a)--(b) node [outer sep=3pt, above]{$B$};
  \draw (b)--(c) node[outer sep=3pt, right]{$C$};
  \draw (c)--(a) node[outer sep=3pt, left]{$A$};
  \pic [draw, <->, "$90^\circ$", angle eccentricity=1.5] {angle = c--a--b};
\end{tikzpicture}


A right angle can be divided into two $45^\circ$ angles:

\begin{tikzpicture}[scale=2]
  \coordinate (a) at (0,1) ;
  \coordinate [circle, fill, inner sep=2pt] (b) at (0,0) ;
  \coordinate (d) at (0.7, 0.7);
  \coordinate (c) at (1,0);
  \draw (b)--(a);
  \draw (b)--(c);
  \draw (b)--(d);
  \pic [draw, <->, "45$^\circ$", angle eccentricity=1.9] {angle = c--b--d};  
  \pic [draw, <->, "45$^\circ$", angle eccentricity=1.9] {angle = d--b--a};  
\end {tikzpicture}

An angle that is less than $90^\circ$ is said to be \emph{acute}.

If you add $45^\circ$ to a $90^\circ$ angle, you get a $135^\circ$ angle:

\begin{tikzpicture}[scale=2]
  \coordinate (a) at (-0.8,0.8) ;
  \coordinate [circle, fill, inner sep=2pt] (b) at (0,0) ;
  \coordinate (d) at (0, 0.6);
  \coordinate (c) at (1,0);
  \draw (b)--(a);
  \draw (b)--(c);
  \draw (b)--(d);
  \pic [draw, <->, "45$^\circ$", angle eccentricity=1.5] {angle = d--b--a};
  \pic [draw, <->, "90$^\circ$", angle eccentricity=1.7] {angle = c--b--d};
  \pic [draw, <->, "135$^\circ$", angle eccentricity=1.25, angle radius=1.5cm] {angle = c--b--a};
\end {tikzpicture}

Angles that are over $90^\circ$ are said to be \emph{obtuse}.
