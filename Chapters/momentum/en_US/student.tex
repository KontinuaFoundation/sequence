\chapter{Momentum}

Let's say a 2 kg block of putty is flying through space at 5 meters
per second, and it collides with a larger 3 kg block of putty that is not
moving at all. When the two blocks deform and stick to each other, how
fast will the resultant big block be moving?
\begin{center}
  
  \includegraphics[width=0.4\textwidth]{putty1.png}
  \includegraphics[width=0.4\textwidth]{putty2.png}
\end{center}
\index{momentum}


\begin{mdframed}[style=important, frametitle={Formula for Momentum}]
Every object has \newterm{momentum}. The momentum is a vector
quantity --- it points in the direction that the object is moving and has
a magnitude equal to its mass times its speed.
\begin{equation}
  p=mv
\end{equation}
\end{mdframed}

Given a set of objects that are interacting, we can sum all their
momentum vectors to get the total momentum. In such a set, the total
momentum will stay constant.
\begin{equation}
p = p_1 + p_2 + \dots + p_n = m_1v_1+m_2v_2 + \dots + m_nv_n
\end{equation}

In our example, one object has a momentum vector of magnitude of
10 kg m/s, the other has a momentum of magnitude 0.  Once they have
merged, they have a combined mass of 5 kg.  This means the velocity vector
must have magnitude 2 m/s and pointing in the same direction that the
first mass was moving.

\begin{Exercise}[title={Cars on Ice}, label=cars_on_ice]
A car weighing 1000 kg is going north at 12 m/s.  Another car weighing
1500 kg is going east at 16 m/s.  They both hit a patch of ice (with
zero friction) and collide.  Steel is bent, and the two objects become
one.  How what is the velocity vector (direction and magnitude) of the
new object sliding across the ice?
\includegraphics[width=0.4\textwidth]{icecar.png}
\includegraphics[width=0.4\textwidth]{icecar2.png}

\end{Exercise}
\begin{Answer}[ref=cars_on_ice]
  The momentum of the first car is 12,000 kg m/s in the north direction.

  The momentum of the second car is 24,000 kg m/s in the east direction.

  The new object will be moving northeast. What is the angle compared with the east?

  $$\theta = \arctan{\frac{12,000}{24,000}} \approx 0.4636 \text{ radians } \approx 26.565\text{ degrees north of east}$$

  The magnitude of the momentum of the new object is $\sqrt{12,000^2 + 24,000^2} \approx 26,833\text{ kg m/s}$

  Its new mass is 2,5000 kg.  So the speed will be $26,833/2,500 = 10.73$ m/s.
\end{Answer}


Note that kinetic energy ($1/2 m v^2$) is \emph{not} conserved
here.  Before the collision, the moving putty block has $(1/2)(2)(5^2) = 25$
joules of kinetic energy.  Afterward, the big block has $(1/2)(5)(2^2)
= 10$ joules of kinetic energy.  What happened to the energy that was
lost? It was used up deforming the putty.

What if the blocks were marble instead of putty?  Then there would be
very little deforming, so kinetic energy \emph{and} momentum would be
conserved. The two blocks would end up having different velocity
vectors.

Let's assume for a moment that they strike each other straight on, so
there is motion in only one direction, both before and after the
collision.  Can we solve for the speeds of the first block ($v_1$) and
the second block ($v_2$)?

We end up with two equations. Conservation of momentum says:

$$2 v_1 + 3 v_2 = 10$$

Conservation of kinetic energy says:

$$(1/2)(2)(v_1^2) + (1/2)(3)(v_2^2) = 25$$

Using the first equation, we can solve for $v_1$ in terms of $v_2$:

$$v_1 = \frac{10 - 3 v_2}{2}$$

Substituting this into the second equation, we get:

$$\left(\frac{10 - 3 v_2}{2}\right)^2 + \frac{3 v_2^2}{2} = 25$$

Simplifying, we get:

$$v_2^2 - 4 v_2 + 0 = 0$$

This quadratic has two solutions: $v_2 = 0$ and $v_2 = 4$.  $v_2 = 0$
represents the situation before the collision.  Substituting in $v_2 = 4$:

$$v_1 = \frac{10 - 3(4)}{2} = -1$$

Thus, if the blocks are hard enough that kinetic energy is conserved,
after the collision, the smaller block will be heading in the opposite
direction at 1 m/s. The larger block will be moving at 4 m/s in the
direction of the original motion.

\begin{Exercise}[title={Billiard Balls}, label=billiards]
  
A billiard ball weighing 0.4 kg and traveling at 3 m/s hits a billiard
ball (same weight) at rest. It strikes obliquely (neither perpendicular nor parallel), so that the ball at rest starts to
move at a 45 degree angle from the path of the ball that hit it.

Assuming all kinetic energy is conserved, what is the velocity
vector of each ball after the collision?

\includegraphics[width=.75\textwidth]{poolball.png}

\end{Exercise}
\begin{Answer}[ref=billiards]

  The original forward momentum was 1.2 kg m/s.  The original kinetic energy is $(1/2)(0.4)(3^2)$ = 1.8 joules. 

  Let $s$ be the post-collision speed of the ball that had been at
  rest.  Let $x$ and $y$ be the forward and sideways speeds
  (post-collision) of the other ball. Conservation of kinetic energy says

  $$(1/2)(0.4)(s^2) + (1/2)(0.4)(x^2+y^2) = 1.8$$

  Forward momentum is conserved:

  $$0.4\frac{s}{\sqrt{2}} + 0.4 x = 1.2$$

  Which can be rewritten:

  $$x = 3 - \frac{s}{\sqrt{2}}$$
  
  Sideways momentum stays zero:

  $$(0.4)\frac{s}{\sqrt{2}} - 0.4 y = 0.0$$

  Which can be rewritten:

  $$y = \frac{s}{\sqrt{2}}$$

  Substituting into to the conservation of kinetic energy equation above:

  $$(1/2)(0.4)(s^2) + (1/2)(0.4)(\left(3 - \frac{s}{\sqrt{2}}\right)^2+\left(\frac{s}{\sqrt{2}}\right)^2 = 1.8$$

  Which can be rewritten:

  $$s^2 - \frac{3}{\sqrt{2}} s + 0 = 0$$

  There are two solutions to this quadratic: $s = 0$ (before collision) and $s = \frac{3}{\sqrt{2}}$. Thus,

  $$y = \frac{3}{2}$$

  and

  $$x = 3 - \frac{3}{2} = \frac{3}{2}$$

  So, both balls careen off at $45^\circ$ angles at the exact same speed. 

  
\end{Answer}
\section{Impulse}
We can talk about a \emph{change in Momentum} as what we refer to as \newterm{Impulse}. When an object has a change in momentum, it is said to have been given an Impulse. Since momentum is a vector quantity, impulse is as well. 

\begin{mdframed}[style=important, frametitle={Formula for Impulse}]
Impulse, $\textbf{J}$ said to be the change in momentum, is given by:
\begin{equation}
  \textbf{J} = \Delta p
\end{equation}
Equivallently, it can be given as the following equation, in terms of force:
\begin{equation}
  \textbf{J} = F\Delta t \label{eq:impulse-momentum1}
\end{equation}
If the force varies with time, we use integration to find the impulse:
\begin{equation}
  \textbf{J} = \int_{t_0}^{t_1} F(t) dt \label{eq:impulse-momentum2}
\end{equation}
Both Equations~\eqref{eq:impulse-momentum1} and \eqref{eq:impulse-momentum2} are referred to as the \textbf{Impulse-Momentum Theorem}. 
\end{mdframed}
By the impulse momentum theorem, a large force for a short time \textit{or} a small force for a long time can produce the same impulse.

\subsection{Golf Swings}
The best example of impulse in action is a golf swing. Let's analyse this using equations:

The force of your swing will theoretically always be the same, as the maximum force you can apply is limited by human ability. You cannot change the human-provided force, however, you can increase the \emph{contact time} of your club on the ball.
\[
\textbf{J} = F\Delta t
\]

Since $\text{J}$ is equivalent to $\Delta p$, $\Delta t$, the contact time of the swing, has a proportional relationship with the momentum of the ball, which starts of as $0$. Thus, it can be said:
\[
\Delta p \propto \Delta t
\]
And, holding $F$, $m$, and assuming the golf ball is initially at rest, $p_i=0$, we can say
\[
v_f \propto \Delta t
\]

So the longer the contact time of a golf swing (or any swing-based sport, really), the greater the velocity.
\section{Collisions}
\index{collisions}
When two (or more) objects collide, we can classify their collision as one of three main categories. These classifications tell us when we can apply the conservation of momentum, the consevation of kinetic energy, both, or neither. Momentum is only conserved when the system is isolated, meaning no external forces act on the system.

\subsection{Elastic Collision}
Recall our billiard ball problem Exercise~\ref{billiards}. That problem provides both balls with velocity after the collision. In an ideal and more realistic billiards scenario, one ball transfers all of its velocity onto the other ball (in this case, the red ball stops and the blue continues with close to same velocity that the red ball had initially). This would be a \newterm{elastic collision}\index{collisions!elastic}. 

In an elastic collision, \emph{both momentum and kinetic energy are conserved}. There is minimal to zero loss of energy in the collision. Although no collision is ever \emph{truly} elastic (due to deformation of objects, sound, a transfer of heat through molecular changes), we can think of a collision like this as perfectly elastic. The sound of billiard balls colliding is obsolete (especially compared to a car crash), so very little energy is lost to sound. 

FIXME Diagram

Another example to consider here is Newton's Cradle. There are two videos demonstrating momentum in Newton's Cradle on your digital resources.
The bottom line here to consider is, in Elastic Collisions, 
\[\text{total }\textbf{p}_{\text{before collision}} = \text{total }\textbf{p}_{\text{after collision}} \implies m_1v_{i,1} + m_2v_{i,2} + \cdots  = m_1v_{f,1} + m_2v_{f,2} + \cdots\]
\emph{and}
\[\text{total }\textbf{KE}_{\text{before collision}} = \text{total }\textbf{KE}_{\text{after collision}}\]

FIXME practice problem elastic collision
\subsection{Inelastic Collisions}
Inelastic collisions, then, are ones in which the total kinetic energy after the collision is \emph{different} than before the collision, in other words, there is a change in kinetic energy, usually lost due to friction, heat, or deformation. 

However, momentum \emph{is} conserved, meaning we can apply conservation of momentum principles.

\[\text{total }\textbf{p}_{\text{before collision}} = \text{total }\textbf{p}_{\text{after collision}} \implies m_1v_{i,1} + m_2v_{i,2} + \cdots  = m_1v_{f,1} + m_2v_{f,2} + \cdots\]

A good example of an elastic collision is dropping a basketball on a floor from an initial height. It will never return to its initial height after ``colliding'' with the floor, as kinetic energy is `lost' to sound and deformation of the ball. The missing height can be equated to the lost energy.

FIXME in-book example

\subsection{Perfectly Inelastic}
A collision is \emph{perfectly inelastic} when the masses stick together after the collision.\index{collisions!perfectly inelastic}In this kind of collision, the \emph{maximum} kinetic energy is lost (crumpling, bending, embedding of objects) and share the same final velocity. 

The most common example of a perfectly inelastic collision is a car crash, especially high speed T-Bones. In this case, a car at a high speed collides with a car at a lower speed, resulting in a combined final speed.


FIXME in book example finding v2 and v1 final given lost energy

\section{Center of mass explosion idea}
% should we mention elastic and inellastic collisions? those are on the ap test
% FIXME they are in the topics list docs
% FIXME momentum = p/t