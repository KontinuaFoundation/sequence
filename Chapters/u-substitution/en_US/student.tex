\chapter{u-Substitution}

\index{u-substitution}
\index{u-sub|see{u-substitution}}
U-Substitution, also known as the method of substitution, is a technique used 
to simplify the process of finding antiderivatives and integrals of 
complicated functions. The method is similar to the chain rule for 
differentiation in reverse.

Suppose we have an integral of the form:

\begin{equation}
\int f(g(x)) \cdot g'(x) \, dx
\end{equation}

The u-substitution method suggests letting a new variable $u$ be equal to the inside function $g(x)$, i.e., 

\begin{equation}
u = g(x)
\end{equation}

Next, the differential of $u$, $du$, is given by:

\begin{equation}
du = g'(x) \, dx
\end{equation}

Substituting $u$ and $du$ back into the integral gives us a simpler integral:

\begin{equation}
\int f(u) \, du
\end{equation}

This new integral can often be simpler to evaluate. Once the antiderivative of 
$f(u)$ is found, we can substitute $u=g(x)$ back into the antiderivative to 
get the antiderivative of the original function in terms of $x$.

The method of u-substitution is a powerful tool for evaluating integrals, 
especially when combined with other techniques like integration by parts, 
partial fractions, and trigonometric substitutions.

\textbf{Example}: Find $\int 2 x^2 \cos{ \left( x^3 - 3 \right)}\,dx$.

\textbf{Solution}: Integrating $\cos{ \left( x^3 - 3 \right)}$ isn't so 
straightforward, so let's try the substitution $u = x^3 - 3$. Then:
$$du = 3 x^2 \text{ }dx$$

We don't have $3 x^2$ in the integral, but we do have $2 x^2$:
$$\frac{2}{3} du = 2 x^2 \text{ } dx$$

Substituting:
$$\int 2 x^2 \cos{ \left( x^3 - 3 \right)} \,dx = \int \frac{2}{3} cos{ \left( 
u \right)}\,du$$
$$= \frac{2}{3} \sin{ \left( u \right)} + C$$

Now that we have the antiderivative of $f(u)$, we can back-substitute in for 
$u$:
$$ \frac{2}{3} \sin{ \left( u \right)} + C = \frac{2}{3} \sin{ \left( x^3 - 3 
\right)} + C$$

We can check our answer by taking its derivative: we should get the original 
integrand back:
$$\frac{d}{dx} \left[ \frac{2}{3} \sin{ \left( x^3 - 3 \right)} + C \right] = 
\frac{2}{3} \cos{ \left( x^3 - 3 \right)}  \cdot \left[ \frac{d}{dx} \left( x^3 
- 3 \right) \right]$$
$$= \frac{2}{3} \cos{ \left( x^3 - 3 \right)} \cdot \left( 3 x^2 \right) = 2 
x^2 \cos{ \left( x^3 - 3 \right)}$$

Sometimes, the right substitution takes a little thinking. Consider the 
following example:

\textbf{Example}: Find $\int \sqrt{x^2 - 1} \text{ }x^5\,dx$.

\textbf{Solution}: We can guess that $u = x^2 - 1$ could be an appropriate 
substitution, as that is what is under the square root. What to do with $x^5$? 
First, let's look at how the $u$-substitution for $x^2 - 1$ works out:
$$u = x^2 - 1$$
$$du = 2x\text{ }dx$$
$$\frac{du}{2} = x\text{ }dx$$

Then we will need to use one of the $x$'s in $x^5$ for the square root 
$u$-substitution. What can we do with the remaining $x^4$? Well, we see that 
if $u = x^2 - 1$, then $u + 1 = x^2$ and therefore $\left( u + 1 \right)^2 = 
x^4$. Substituting this all in:
$$\int \sqrt{x^2 - 1} \text{ }x^5\,dx = \int x^4 \sqrt{x^2 - 1}\text{ }x\,dx$$
$$= \frac{1}{2} \int \left( u + 1 \right)^2 \sqrt{u}\,du$$

We can expand this to find the antiderivative:
$$= \frac{1}{2} \int \left( u^2 + 2u + 1 \right) u^{1/2}\,du = \frac{1}{2} 
\int u^{5/2} + 2u^{3/2} + u^{1/2}\,du$$
$$= \frac{1}{2} \left[ \frac{2}{7}u^{7/2} + \frac{4}{5}u^{5/2} + \frac{2}{3}
u^{3/2} \right] + C$$
$$= \frac{1}{7}u^{7/2} + \frac{2}{5}u^{5/2} + \frac{1}{3}u^{3/2} + C$$
$$= \frac{1}{7} \left( x^2 - 1 \right)^{7/2} + \frac{2}{5} \left( x^2 - 1 
\right)^{5/2} + \frac{1}{3} \left( x^2 - 1 \right)^{3/2} + C$$

\begin{Exercise}[title = {Indefinite Integrals and u-substitution}, 
label = u_indef]
Use $u$-substitution to evaluate the following indefinite integrals. Confirm 
your answer by taking the derivative of the result. 
\begin{enumerate}
    \item $\int \sin{x} \sqrt{1 + \cos{x}}\,dx$
    \item $\int \frac{\cos{ \left( \pi / x \right)}}{x^2}\,dx$
    \item $\int 2x^2 \left( 9 - x^3 \right)^{2/3}\,dx$
    \item $\int 3x^2 \sqrt{1 + x}\,dx$
    \item $\int \frac{3x^2}{x^3 - 1}\,dx$
\end{enumerate}
\vspace{75mm}
\end{Exercise}

\begin{Answer}[ref = u_indef]
\begin{enumerate}
    \item Let $u = 1 + \cos{x}$. Then $du = -\sin{x} \text{ } dx$ and $-du = 
    \sin{x} \text{ } dx$. Substituting:
    $$\int \sin{x} \sqrt{1 + \cos{x}}\,dx = \int -\sqrt{u}\,du = -\frac{2}{3} 
    u^{3/2} + C$$
    $$= -\frac{2}{3} \left( 1 + \cos{x} \right)^{3/2} + C$$

    Taking the derivative:
    $$\frac{d}{dx} \left[ -\frac{2}{3} \left( 1 + \cos{x} \right)^{3/2} + C 
    \right] = -\frac{2}{3} \left[ \frac{d}{dx} \left( 1 + \cos{x} 
    \right)^{3/2} + \frac{d}{dx}C \right]$$
    $$= -\frac{2}{3} \left[ \frac{3}{2} \left(1 + \cos{x} \right)^{1/2} \cdot 
    \frac{d}{dx} \left( 1 + \cos{x} \right) \right] = -1 \sqrt{1 + \cos{x}} 
    \cdot \left( - \sin{x} \right) = \sin{x} \sqrt{1 + \cos{x}}$$

    \item Let $u = \pi / x$. Then $du = (- \pi / x^2)dx$ and $-du / \pi = (1 / 
    x^2)dx$. Substituting:
    $$\int \frac{\cos{ \left( \pi / x \right)}}{x^2}\,dx = -\frac{1}{\pi} \int 
    \cos{u}\,du = -\frac{1}{\pi} \sin{u} + C = -\frac{1}{\pi} \sin{ \left( \pi 
    / x \right)} + C$$

    Taking the derivative:
    $$\frac{d}{dx} \left[ - \frac{1}{\pi} \sin{ \left( \pi / x \right)} + C 
    \right] = -\frac{1}{\pi} \left[ \frac{d}{dx} \sin{ \left( \pi / x \right)} 
    + \frac{d}{dx} C \right]$$
    $$= -\frac{1}{\pi} \left[ \cos{ \left( \pi / x \right)} \cdot \frac{d}{dx} 
    \left( \frac{\pi}{x} \right) \right] = -\frac{1}{\pi} \left[ \cos{ \left( 
    \pi / x \right)} \cdot \left( \frac{-\pi}{x^2} \right) \right]$$
    $$= \frac{\cos{ \left( \pi / x \right)}}{x^2}$$

    \item Let $u = 9 - x^3$. Then $du = -3x^2 \text{ } dx$ and $-\frac{2}{3}du 
    = 2x^2 \text{ } dx$. Substituting:
    $$\int 2x^2 \left( 9 - x^3 \right)^{2/3}\,dx = -\frac{2}{3} \int \left( u 
    \right)^{2/3}\,du$$
    $$= -\frac{2}{3} \left[ \frac{3}{5}u^{5/3} + C \right] = -\frac{2}{5} 
    \left( 9 - x^3 \right)^{5/3} + C$$

    Taking the derivative:
    $$\frac{d}{dx} \left[ -\frac{2}{5} \left( 9 - x^3 \right)^{5/3} + C 
    \right] = -\frac{2}{5} \left[ \frac{d}{dx} \left( 9 - x^3 \right)^{5/3} 
    \right]+ \frac{d}{dx}C$$
    $$= -\frac{2}{5} \left[ \frac{5}{3} \left( 9 - x^3 \right)^{2/3} \cdot 
    \frac{d}{dx} \left( 9 - x^3 \right) \right] = -\frac{2}{3} \left( 9 - x^3 
    \right)^{2/3} \left( -3x^2 \right) = 2x^2 \left( 9 - x^3 \right)^{2/3}$$

    \item Let $u = 1 + x$. Then $du = dx$. Additionally, $u - 1 = x$ and $x^2 
    = \left( u - 1 \right)^2$. Substituting:
    $$\int 3x^2 \sqrt{1 + x}\,dx = \int 3 \left(u - 1 \right)^2 \sqrt{u}\,du = 
    3 \int \left( u^2 - 2u + 1 \right) \sqrt{u}\,du$$
    $$= 3 \int u^{5/2} - 2u^{3/2} + u^{1/2}\,du = 3 \left[ \frac{2}{7}u^{7/2} 
    - \frac{2 \cdot 2}{5}u^{5/2} + \frac{2}{3}u^{3/2} + C \right]$$
    $$= \frac{6}{7}u^{7/2} - \frac{12}{5}u^{5/2} + 2u^{3/2} + C$$
    $$= \frac{6}{7} \left( 1 + x \right)^{7/2} - \frac{12}{5} \left( 1 + x 
    \right)^{5/2} + 2 \left( 1 + x \right)^{3/2} + C$$

    Taking the derivative:
    $$\frac{d}{dx} \left[ \frac{6}{7} \left( 1 + x \right)^{7/2} - 
    \frac{12}{5} \left( 1 + x \right)^{5/2} + 2 \left( 1 + x \right)^{3/2} + C 
    \right]$$
    $$= \frac{6}{7} \left( \frac{7}{2} \left( 1 + x \right)^{5/2} \right) - 
    \frac{12}{5} \left( \frac{5}{2} \left( 1 + x \right)^{3/2} \right) + 2 
    \left( \frac{3}{2} \left( 1 + x \right)^{1/2} \right)$$
    $$= 3 \left( 1 + x \right)^{5/2} - 6 \left( 1 + x \right)^{3/2} + 3 \left( 
    1 + x \right)^{1/2}$$
    $$= \left[ 3 \left( 1 + x \right)^2 - 6 \left( 1 + x \right) + 3 \right] 
    \left( 1 + x \right)^{1/2}$$
    $$= \left[ 3 \left( 1 + 2x + x^2 \right) - 6 - 6x + 3 \right] \sqrt{1 + x}$$
    $$= \left[ 3 + 6x + 3x^2 - 6 - 6x + 3 \right] \sqrt{1 + x} = 3x^2 
    \sqrt{1 + x}$$

    \item Let $u = x^3 - 1$. Then $du = 3x^2 \text{ } dx$. Substituting:
    $$\int \frac{3x^2}{x^3 - 1}\,dx = \int \frac{1}{u}\,du = \ln{u} + C$$
    $$= \ln{ \left( x^3 - 1 \right)} + C$$

    Taking the derivative:
    $$\frac{d}{dx} \left[ \ln{ \left( x^3 - 1 \right)} + C \right] = 
    \frac{d}{dx} \ln{ \left( x^3 - 1 \right)} + \frac{d}{dx}C$$
    $$= \frac{1}{x^3 -1} \cdot \left[ \frac{d}{dx} \left( x^3 - 1 \right) 
    \right] = \frac{3x^2}{x^3 - 1}$$
    
\end{enumerate}
\end{Answer}

\section{The Substitution Rule for Definite Integrals}
How to we use $u$-substitution for definite integrals? We will apply the 
fundamental theorem of calculus to answer this question. We define $f$ and $F$ 
such that $F$ is the antiderivative of $f$. Then:
$$\int_a^b f(g(x)) \cdot g'(x)\,dx = F(g(x))|_{a}^{b} = F(g(b)) - F(g(a))$$

This represents the method of finding the indefinite antiderivative and 
evaluating from the original limits of integration. 

We can also see that:
$$F(g(b)) - F(g(a)) = F(u)|_{g(a)}^{g(b)} = \int_{g(a)}^{g(b)} f(u)\,du$$

Therefore, if $g'$ is continuous on $\left[ a, b \right]$ and $f$ is continuous 
on the range of $u = g(x)$, then:
$$\int_a^b f(g(x)) \cdot g'(x)\,dx = \int_{g(a)}^{g(b)} f(u)\,du$$

\index{u-substitution!changing the bounds}
This represents \emph{changing the limits of integration} into the new variable, $u$, 
then evaluating the integral. While the second method is preferable, both 
methods yield the same answer. 

\textbf{Example}: Evaluate $\int_0^{5} \sqrt{3x + 1}\,dx$ using both methods 
outlined above. 

\textbf{Solution}: We start with the first method. We will use the substitution 
$u = 3x + 1$, and therefore $du/3 = dx$:
$$\int_0^5 \sqrt{3x + 3}\,dx = \frac{1}{3} \int_{x = 0}^{x = 5} \sqrt{u}\,du$$

(We write the limits as $x = \cdots$ to remind us the limits are for $x$, not 
$u$.)
$$\frac{1}{3} \int_{x = 0}^{x = 5} \sqrt{u}\,du = \frac{1}{3} \left[ \frac{2}{3} 
u^{3/2} \right]_{x = 0}^{x = 5}$$

Now we substitute back in for $u$ and evaluate:
$$\frac{2}{9} \left( 3x + 1 \right)^{3/2}|_{0}^{5} = \frac{2}{9} \left( 16 
\right)^{3/2} - \frac{2}{9} \left( 1 \right)^{3/2}$$
$$= \frac{2}{9} \left( 64 - 1 \right) = \frac{2 \cdot 63}{9} = 14$$

Let's compare this to the second, preferred method. We already know the 
$u$-substitution we'll make, so next we need to find $g(0)$ and $g(5)$ (recall 
that we choose $u$ such that $u = g(x)$):
$$g(x) = 3x + 1$$
$$g(0) = 1$$
$$g(5) = 16$$

Now we can make our substitution \textit{and} change the limits of integration:
$$\int_0^5 \sqrt{3x + 1}\,dx = \frac{1}{3} \int_1^{16} \sqrt{u}\,du$$
$$= \frac{1}{3} \left[ \frac{2}{3} u^{3/2} \right]_1^{16} = \frac{2}{9} \left[ 
16^{3/2} - 1^{3/2} \right]$$
$$= \frac{2}{9} \left(64 - 1 \right) = 14$$

With the second method, we get the same answer in fewer steps. 

\begin{Exercise}[title = {Definite Integrals and u-substitution}, label = u_def]
Use $u$-substitution to evaluate the following definite integrals. 
\begin{enumerate}
\item $\int_0^{\pi / 2} \cos{x} \sin{ \left( \sin{ x } \right) }\,dx$
\item $\int_0^{13} \frac{1}{\sqrt[3]{\left( 1 + 2x \right)^2}}\,dx$
\item $\int_1^2 \frac{e^{1/x}}{x^2}\,dx$
\item $\int_0^{\pi / 6} \frac{\sin{x}}{\cos^2{x}}\,dx$
\item $\int_0^4 \frac{x}{\sqrt{1 + 2x}}\,dx$
\end{enumerate}
\vspace{50mm}
\end{Exercise}

\begin{Answer}[ref = u_def]
\begin{enumerate}
    \item Let $g(x) = u = \sin{x}$. Then $du = \cos{x} \text{ }dx$. 
    Additionally, $g(0) = \sin{0} = 0$ and $g(\pi/2) = \sin{\left( \pi / 2 
    \right)} = 1$. Substituting and changing the limits of integration:
    $$\int_0^{\pi / 2} \cos{x} \sin{ \left( \sin{x} \right) }\,dx = \int_0^1 
    \sin{u}\,du = -\cos{u}|_0^1 = \cos{0} - \cos{1} = 1 - \cos{1}$$

    \item Let $g(x) = u = 1 + 2x$. Then $du = 2 \text{ }dx$ and $\frac{du}{2} 
    = dx$. Additionally, $g(0) = 1$ and $g(13) = 1 + 2(13) = 27$. Substituting 
    and changing the limits of integration:
    $$\int_0^{13} \frac{1}{\sqrt[3]{\left( 1 + 2x \right)^2}}\,dx = \frac{1}{2}
    \int_1^{27} \frac{1}{\sqrt[3]{u^2}}\,du = \frac{1}{2} \int_1^{27} u^{-2/3}
    \,du$$
    $$= \frac{1}{2} \left[ 3u^{1/3} \right]_1^{27} = \frac{3}{2} \left[ 
    \sqrt[3]{27} - \sqrt[3]{1} \right] = \frac{3}{2} \left( 3 - 1 \right) = 3$$

    \item Let $g(x) = u = 1 / x$. Then $du = (-1 / x^2) dx$ and $-du = dx / 
    x^2$. Additionally, $g(1) = 1$ and $g(2) = 1 / 2$. Substituting and 
    changing the limits of integration:
    $$\int_1^2 \frac{e^{1/x}}{x^2}\,dx = -\int_1^{1/2} e^u\,du = \int_{1/2}^1 
    e^u\,du$$
    $$= e^u|_{1/2}^1 = e - \sqrt{e}$$

    \item Let $g(x) = u = \cos{x}$. Then $du = -\sin{x} \text{ } dx$ and $-du 
    = \sin{x} \text{ }dx$. Additionally, $g(0) = \cos{0} = 1$ and $g(\pi / 6) 
    = \cos{ \frac{\pi}{6}} = \frac{\sqrt{3}}{2}$. Substituting and changing 
    the limits of integration:
    $$\int_0^{\pi / 6} \frac{\sin{x}}{\cos^2{x}}\,dx = -\int_{1}^{\sqrt{3} 
    / 2} \frac{1}{u^2}\,du = \int_{\sqrt{3} / 2}^1 \frac{1}{u^2}\,du$$
    $$= -\frac{1}{u}|_{\sqrt{3} / 2}^1 = \frac{2}{\sqrt{3}} - 1 = \frac{2
    \sqrt{3} - 3}{3}$$

    \item Let $g(x) = u = 1 + 2x$. Then $du = 2 \text{ }dx$ and $\frac{du}{2} 
    = dx$. And if $u = 1 + 2x$, then $x = \frac{u - 1}{2}$. Additionally, 
    $g(0) = 1$ and $g(4) = 9$. Substituting and changing the limits of 
    integration:
    $$\int_0^4 \frac{x}{\sqrt{1 + 2x}}\,dx = \frac{1}{2} \int_1^9 \frac{
    \frac{u-1}{2}}{\sqrt{u}}\,du = \frac{1}{4} \int_1^9 \frac{u - 1}{\sqrt{u}}
    \,du$$
    $$= \frac{1}{4} \int_1^9 \sqrt{u} - \frac{1}{\sqrt{u}}\,du = \frac{1}{4} 
    \left[ \frac{2}{3}u^{3/2} - 2u^{1/2} \right]_1^9$$
    $$= \frac{1}{4} \left[ \frac{2}{3} \left( 9^{3/2} - 1^{3/2} \right) - 2 
    \left( \sqrt{9} - \sqrt{1} \right) \right] = \frac{1}{4} \left[ \frac{2}{3} 
    \left( 26 \right) - 2 \left( 2 \right) \right]$$
    $$= \frac{1}{4} \left[ \frac{52}{3} - 4 \right] = \frac{1}{4} \left[ 
    \frac{52 - 12}{3} \right] = \frac{1}{4} \left[ \frac{40}{3} \right] = 
    \frac{10}{3}$$
\end{enumerate}
\end{Answer}

