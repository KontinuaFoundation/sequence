\chapter{The Piston Engine}

Most cars,  airplanes,  and chainsaws get their power from burning hydrocarbons in a 
combustion chamber.   We say they have \newterm{internal combustion engines}.   There are many types of internal combustion engines: jet engines, rotary engines, diesel engines, etc.  In this chapter, we are 
going to explain how piston engines work. 

Most piston engines burn gasoline,  which is a blend of liquid hydrocarbons.  Hydrocarbons are molecules made of hydrogen and carbon (and maybe a little oxygen).  In the presence of oxygen and heat,  hydrocarbons burn -- the carbon combines with oxygen to become $CO^2$ and the hydrogen combines with oxygen to become $H_2O$.  In the process,  a lot of energy is released.

For example,  ethanol (or alcohol) is one of the components usually present in gasoline.  The Ford Model T was the first popular car, and it ran on pure ethanol, (which could be produced by fermenting grains) not gasoline.  To simplify things, let's talk about a piston engine running on pure ethanol.

A molecule of ethanol has 2 carbon atoms, 6 hydrogen atoms, and 1 oxygen atom.  The oxygen in the atmosphere is $O_2$.   When one molecule of ethanol combines with three molecules of $O_2$,  2 molecules of $CO_2$ and 3 molecules of $H_2O$ are created.  Also, a lot of heat is created: 1330 kilojoules for every mole burned.

\section{Parts of the Engine}

The engine block is a big hunk of metal.  There are cylindrical holes bored into the engine block.  A piston can slide up and down the cylinder.  There are two valves in the wall of the cylinder:  
\begin{itemize}
\item Before the burn, one valve opens to let ethanol and air into the cylinder.
\item After the burn,  the other valve opens to let the exhaust out.
\end{itemize}

There is also a spark plug,   which creates the spark that triggers the burn.

As you give the engine more gas,  the cylinder does more frequent burns.  When the engine is just idling,  the cylinder fires about 9 times per second.  When you depress the gas pedal all the way down,  it is more like 40 times per second.

The cylinder has a rod that connects it to the crank shaft.   As the cylinder moves back and forth,  the crank shaft turns around and around.  Sometimes the cylinder is pushing the crankshaft, and sometimes the crank shaft is pushing or pulling the piston.  All the cylinders share one crank shaft.

How many cylinders does a car have?  Nearly all car models have between 3 and 8 cylinders.    The opening of the valve and the firing of the spark plugs are timed so that cylinders all do their burns 
at different times.   This makes the total power delivered to the crank shaft smoother.

\section{The Four-Stroke Process}

Cars have four-stroke engines -- this means for every two rotations of the crank shaft,  each cylinder fires once.  Smaller engines, like those in chainsaws,  are often two-stroke engines -- every cylinder fires every time the crank shaft rotates.  For now,  let's focus on four-stroke engines.

Here is the cycle of a single cylinder:
\begin{itemize}
\item As the drive shaft turns,  it pulls the piston down.  The intake valve opens and lets the gas/air mixture into the combustion chamber.
\item As the piston reaches the bottom of the stroke,  the intake valve closes.
\item Now the crank shaft starts to push the piston up, compressing the gas and oxygen.
\item As the piston reaches the top of its stroke,  the spark plug creates a spark.  The fuel and oxygen burn quickly.  The cool liquid fuel becomes hot carbon dioxide and water vapor.
\item Now there is very high pressure inside the cylinder.  It pushes hard on the piston which pushes  the crank shaft.
\item When the piston reaches the bottom of this stroke,  the exhaust valve opens.
\item As the crank shaft pushes the piston up,  the carbon dioxide and water vapor is pushed out.
\item When the piston reaches the top of this stroke,  the exhaust valve is closed.
\end{itemize}

\section{Dealing with Heat}

Burning fuel inside a block of metal generates a lot of heat.  If there is too much heat,  parts of the engine will start to melt.  So modern car engines are liquid cooled -- there are arteries in the engine block carrying a liquid (called "coolant").  The hot coolant is pumped through the radiator (where the air passing through takes away the heat) and then back into the engine.

\section{Dealing with Friction}


From the description,  it is clear that there is a lot of metal sliding against metal,  which would grind the engine up quickly if there were no lubrication.  In a modern car, the moving parts in the engine are constantly bathed in oil.  There is an oil pump that causes it to get sprayed on the crankshaft, the connecting rod, and in the cylinder under the piston. (That is, not on the combustion side.)  

The oil eventually falls through the oil into a pan at the bottom of the engine.   The oil pump sucks the oil up,  pushes it through a filter (so bits of metal are not pumped back into the engine),  and then is sprayed on the moving parts again.

\section{Challenges}

With an engine, there are a lot of things that can go wrong.  Let's enumerate a few:

\begin{itemize}

\item \textit{The seal around the piston leaks}.  Mechanics say "We aren't getting any compression."  The cylinder doesn't get much power to the drive train.

\item \textit{The valves open or the spark plug fires at the wrong time.}  This is known as a timing problem. 

\item \textit{The spark plug doesn't make a spark}.   The spark plug has two prongs of metal and electrons jump from one to the other.  For a good spark,  the prongs need to be a very precise distance apart.  Sometimes you need to bend one of the prongs to get the right gap.  This is known as \newterm{gapping}.

\item \textit{The mix of fuel and oxygen is wrong.}  If there is too much fuel and not enough oxygen (so not all the fuel burns),  we say the mix is too rich.  If there is not enough fuel (so the pressure created by the burn is as high as possible),   we say the mix is too lean.

\end{itemize}
