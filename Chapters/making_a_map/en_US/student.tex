\chapter{Making a Map}


Plotly is an open-source data visualization library for Python, R, and
JavaScript. It allows for interactive plots, including geographical
maps. In this brief example, we will learn how to create a simple
annotated map using Plotly in Python.\index{plotly}

To begin, you need to install Plotly. In Python, you can do this via pip:

\begin{lstlisting}[language=Python]
pip install plotly
\end{lstlisting}

Once installed, you can create a map with annotations as follows:


\begin{verbatim}
import plotly.graph_objects as go

fig = go.Figure(data=go.Scattergeo(
    lon = [-75.789], 
    lat = [45.4215],
    text = ['Ottawa'],
    mode = 'text',
))

fig.update_layout(
    title_text = 'Annotated Map with Plotly',
    showlegend = False,
    geo = dict(
        scope = 'world',
        projection_type = 'azimuthal equal area',
        showland = True,
        landcolor = 'rgb(243, 243, 243)',
        countrycolor = 'rgb(204, 204, 204)',
    ),
)

fig.show()


\end{verbatim}
This code creates a world map and places a text annotation at the
geographic coordinates for Ottawa. Then, the plotly is opened in your browser using a localhost IP address. 
The `go.Scattergeo` function is
used to define the geographical scatter plot (i.e., the annotation),
while the `update\_layout` function is used to define the appearance
and the properties of the map itself.

In this example, you can replace the latitude, longitude, and text
with the values corresponding to your desired location.
