\chapter{Implicit Differentiation}

Implicit differentiation is a technique in calculus for finding the
derivative of a relation defined implicitly (that is, a relation
between variables $x$ and $y$ that is not explicitly solved for one
variable in terms of the other).\index{implicit differentiation}

\section{Implicit Differentiation Procedure}

Consider an equation that defines a relationship between $x$ and $y$:

\[
F(x, y) = 0
\]

To find the derivative of $y$ with respect to $x$, we differentiate
both sides of this equation with respect to $x$, treating $y$ as an
implicit function of $x$:

$$\frac{d}{dx} F(x, y) = \frac{d}{dx} 0$$

Applying the chain rule during the differentiation on the left side of
the equation gives:

$$\frac{\partial F}{\partial x} + \frac{\partial F}{\partial y} 
\frac{dy}{dx} = 0$$

Finally, we solve for $\frac{dy}{dx}$ to find the derivative of $y$
with respect to $x$:

$$\frac{dy}{dx} = 
-\frac{\frac{\partial F}{\partial x}}{\frac{\partial F}{\partial y}}$$

This result is obtained using the implicit differentiation method.

\section{Example}

Consider the equation of a circle with radius $r$:

$$x^2 + y^2 = r^2$$

First, we will find $\frac{dy}{dx}$ without implicit 
differentiation. Newxt, we will apply implicit differentiation to get 
the same result. 

\subsection{Without Implicit Differentiation}
First, we need to rearrange the equation to solve for y: $$y^2 = r^2 
- x^2$$
$$y = \pm \sqrt{r^2 - x^2}$$

We take the derivative of $y$ by applying the Chain Rule:
$$\frac{dy}{dx}=\frac{1}{2 \pm \sqrt{r^2 - x^2}} \cdot (- 2x) = 
\frac{-x}{\pm \sqrt{r^2 - x^2}}$$ 

Notice the denominator of this fraction is the same as the solution 
we found for $y$, $y=\pm \sqrt{r^2-x^2}$. So, we can also represent 
this as: $$\frac{dy}{dx}=\frac{-x}{y}$$

\subsection{With Implicit Differentiation}
With implicit differentiation, we assume $y$ is a function of $x$ and 
apply the Chain Rule. $$\frac{d}{dx}[x^2+y^2]=\frac{d}{dx}[r^2]$$ For 
$x^2$ and $r^2$, we take the derivative as we normally would. For 
$y^2$, we apply the Chain Rule, as outlined above. 
$$2x+2y\frac{dy}{dx}=0$$\\ Solving for $\frac{dy}{dx}$, we find 
$$\frac{dy}{dx}=\frac{-x}{y}$$ \\, which is the same result as we found 
without implicit differentiation. 

\section{Folium of Descartes}
It was relatively easy to rearrange the equation for a circle to 
solve for y, but that is not always the case. Consider the equation 
for the folium of Descartes (yes, that Descartes!): $$x^3+y^3=3xy$$ 
It is much more difficult to isolate $y$ in this equation. In fact, 
were we to do so, we would need three separate equations to completely 
describe the original equation. 

\subsection{Example: Tangent to Folium of Descartes}

In this example, we will use implicit differentiation to easily find 
the tangent line at a point on the folium. 

(a) Find $\frac{dy}{dx} \text{ if } x^3+y^3 = 6xy$

(b) Find the tangent to the folium $x^3+y^3=3xy$ at the point $(2, 2)$

(c) Is there any place in the first quadrant where the tangent line 
is horizontal? If so, state the point(s). 

Solution:

(a) $\frac{d}{dx}[x^3+y^3]=\frac{d}{dx}[3xy]$
$$3x^2+3y^2\frac{dy}{dx}=3x\frac{dy}{dx}+3y$$
$$x^2+y^2\frac{dy}{dx}=x\frac{dy}{dx}+y$$
Rearranging to solve for $\frac{dy}{dx}$:
$$\frac{dy}{dx}(y^2-x)=y-x^2$$
$$\frac{dy}{dx}=\frac{y-x^2}{y^2-x}$$

(b) We already have the coordinate point, $(2, 2)$, so to write an 
equation for the tangent line, all we need is the slope. Substituting 
$x=2 \text{ and }y=2$ into our result from part (a):
$$\frac{dy}{dx}=\frac{2-2^2}{2^2-2}=\frac{-2}{2}=-1$$
This is the slope, $m$. Using the point-slope form of a line, our 
tangent line is $y-2=-(x-2)$. 

(c) Recall that in the first quadrant, $x > 0$ and $y > 0$. We will 
set our solution for $\frac{dy}{dx}$  equal to 0:
$$\frac{y-x^2}{y+2-x} = 0$$ \\
which implies that $$y - x^2 = 0$$ \\
Substituting $y = x^2$ into the original equation: 
$$x^3 + (x^2)^3 = 3(x)(x^2)$$ 
$$x^3 + x^6 = 3x^3$$ \\
Which simplifies to 
$$x^6 = 2x^3$$\\
Since we have excluded $x = 0$ by restricting our search to the first 
quadrant, we can divide both sides by $x^3$: 
$$x^3 = 2$$ 
$$x = \sqrt[3]{2} \approx 1.26$$\\
Substituting $x \approx 1.26$ into our equation for $y$: 
$$y \approx 1.26^2 = 1.59$$\\ 
Therefore, the folium has a horizontal tangent line at the point 
$(1.26, 1.59)$.

\section{Practice}
\begin{Exercise}[label=implicit1]
	[This problem was originally presented as a no-calculator, 
	multiple-choice question on the 2012 AP Calculus BC Exam.] If 
	$\arcsin{x} = \ln{y}$, what is $\frac{dy}{dx}$?
\end{Exercise}

\begin{Answer}[ref=implicit1]
	Using implicit differentiation, we see that:
	$$\frac{d}{dx}\arcsin{x} = \frac{d}{dx}\ln{y}$$
	$$\frac{1}{\sqrt{1-x^2}} = \frac{1}{y}\frac{dy}{dx}$$
	Multiplying both sides by $y$ to isolate $\frac{dy}{dx}$, we find that:
	$$\frac{dy}{dx} = \frac{y}{\sqrt{1-x^2}}$$
\end{Answer}

\begin{Exercise}[label = implicit2]
	[This problem was originally presented as a no-calculator, 
	multiple-choice question on the 2012 AP Calculus BC Exam.] The points 
	$(-1, -1)$ and $(1, -5)$ are on the graph of a function $y = f(x)$ 
	that satisfies the differential equation $\frac{dy}{dx} = x^2 + y$. 
	Use implicit differentiation to find $\frac{d^2y}{dx^2}$. Determine 
	if each point is a local minimum, local maximum, or inflection point 
	by substituting the $x$ and $y$ values of the coordinates into 
	$\frac{dy}{dx}$ and $\frac{d^2y}{dx^2}$. 
\end{Exercise}

\begin{Answer}[ref=implicit2]
	First, we need to find $\frac{d^y}{dx^2}$:
	$$\frac{d}{dx}\frac{dy}{dx} = \frac{d}{dx}x^2 + \frac{d}{dx}y$$
	$$= 2x + \frac{dy}{dx} = 2x + x^2 + y$$
	At $(-1,-1)$, $\frac{dy}{dx} = (-1)^2 + (-1) = 0$ and 
	$\frac{d^2y}{dx^2} = 2(-1) + (-1)^2 + (-1) = -2 < 0$. Since the 
	slope of y is zero and the graph of y is concave down, $(-1,-1)$ is a 
	local maximum. At $(1, -5)$, $\frac{dy}{dx} = 1^2 + -5 = -4 \neq 0$ 
	and $\frac{d^2y}{dx^2} = 2(1) + 1^2 + (-5) = -2 \neq 0$. Since 
	neither the first nor second derivative of $y$ are zero, $(1, -5)$ 
	is neither a local extrema nor an inflection point. 
\end{Answer}






 
 
 
 
 
 
 
 