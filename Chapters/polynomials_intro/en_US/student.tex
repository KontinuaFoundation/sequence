\chapter{Introduction to Polynomials}

Watch Khan Academy's \textbf{Polynomials intro} video at \url{https://youtu.be/Vm7H0VTlIco}\index{polynomial}

A \emph{monomial}\index{monomial} is the product of a number and a variable raised to a non-negative (but possibly zero) integer power. Here are some examples of monomials:
\begin{multicols}{4}
  \begin{equation*}
    3 x^2
  \end{equation*}

  \begin{equation*}
    -2 x^{15}
  \end{equation*}

  \begin{equation*}
    \pi x^2
  \end{equation*}

  \begin{equation*}
    (3.33)x^{100}
  \end{equation*}

  \begin{equation*}
    7x
  \end{equation*}

  \begin{equation*}
    3
  \end{equation*}

  \begin{equation*}
    -\frac{2}{3}x^{12}
  \end{equation*}

  \begin{equation*}
    0
  \end{equation*}

  
\end{multicols}

The exponent is called the \emph{degree} of the monomial\index{monomial!degree}. For example, $3x^{17}$
has degree 17, $-7x$ has degree 1, and $3.2$ has degree 0 (because you can think of it as $(3.2)x^0$).\index{degree!polynomial}. The exponent cannot be $x$ (or any non-constant variable), as that becomes an exponential function rather than polynomial.\footnote{A quadratic is a polynomial with a maximum degree of 2}

The number in the product is called the \emph{coefficient}\index{monomial!coefficient}.  Example: $3x^{17}$ has a coefficient of 3, $-2x$ has a coefficient of -2, and $(3.4)x^{1000}$ has a coefficient of 3.4.\index{coefficient!polynomial}

A \emph{polynomial} \index{polynomial!definition of} is the sum of one or more monomials.  Here are some polynomials:
\begin{multicols}{3}
  \begin{equation*}
    4 x^2 + 9x + 3.9
  \end{equation*}

  \begin{equation*}
    -2 x^{10} + (3.4)x - 45x^{900} - 1
  \end{equation*}

  \begin{equation*}
    \pi x^2 + \pi x + \pi
  \end{equation*}

  \begin{equation*}
    3.3
  \end{equation*}

  \begin{equation*}
   7x + 2
  \end{equation*}

  \begin{equation*}
    3x^{20}
  \end{equation*}
\end{multicols}
We say that each monomial is a \emph{term} of the polynomial.

$x^{-5} + 12$ is \emph{not} a polynomial because the first term has a negative exponent.

$x^{2} - 32x^{\frac{1}{2}} + x$ is \emph{not} a polynomial because the second term has a non-integer exponent.

$\frac{x + 2}{x^2 + x + 5}$ is \emph{not} a polynomial because it is not just a sum of monomials.

\begin{Exercise}[title={Identifying Polynomials}, label=findpolynomials]
    Circle only the polynomials.
\begin{multicols}{3}
  \begin{equation*}
    -2 x^3 + \frac{1}{2}x + 3.9
  \end{equation*}

  \begin{equation*}
    2 x^{-10} + 4x - 1
  \end{equation*}
  
  \begin{equation*}
    (4.5)x^2 + \pi x
  \end{equation*}
  
  \begin{equation*}
    x^{\frac{2}{3}}
  \end{equation*}
  
  \begin{equation*}
   7
  \end{equation*}

  \begin{equation*}
    3x^{20} + 2x^{19} -5 x^{18}
  \end{equation*}
\end{multicols}
\end{Exercise}

\begin{Answer}[ref=findpolynomials]
\begin{multicols}{3}
  \begin{equation*}
    \boxed{-2 x^3 + \frac{1}{2}x + 3.9}
  \end{equation*}

  \begin{equation*}
    2 x^{-10} + 4x - 1
  \end{equation*}

  \begin{equation*}
    \boxed{(4.5)x^2 + \pi x}
  \end{equation*}

  \begin{equation*}
    x^{\frac{2}{3}}
  \end{equation*}

  \begin{equation*}
   \boxed{7}
  \end{equation*}

  \begin{equation*}
    \boxed{3x^{20} + 2x^{19} -5 x^{18}}
  \end{equation*}
\end{multicols}

\end{Answer}

We typically write a polynomial starting at the term with the highest
degree and proceed in decreasing order to the term with the lowest
degree:
\begin{equation*}
2 x^9 - 3x^7 + \frac{3}{4}x^3 + x^2 + \pi x -9.3
\end{equation*}
This is known as \emph{the standard form}.  The first term of the
standard form is called \emph{the leading term}, and we often call the
coefficient of the leading term \emph{the leading coefficient}.  We
sometimes speak of the degree of the polynomial, which is just the
degree of the leading term.\index{standard form!polynomial}

\begin{Exercise}[title={Standard of a Polynomial}, label=polynomialstandardform]
  Write $21x^2 - x^3 + \pi - 1000x$ in standard form. What is the degree of this polynomial? What is its leading coefficient?
\end{Exercise}
\begin{Answer}[ref=polynomialstandardform]
  Standard form would be $-x^3 + 21x^2 - 1000x + \pi$. The degree is 3. The leading coefficient is $-1$
\end{Answer}

\begin{Exercise}[title={Evaluate a Polynomial}, label=evaluatepolynomial]
  Let $y = x^3 - 3x^2 + 10x - 12$. What is $y$ when $x$ is $4$?
\end{Exercise}
\begin{Answer}[ref=evaluatepolynomial]
  $4^3 - (3)(4^2) + (10)(4) - 12 = 64 - 48 + 40 - 12$. So $y = 44$ 
\end{Answer}

We wouild be remiss in our duties if we didn't mention one more thing
about polynomials: Mathematicians have defined a polynomial to be a sum
of a \emph{finite} number of monomials.

It is certainly possible to have a sum of an infinite number of monomials
like this:
\begin{equation*}
1 + \frac{1}{2}x + \frac{1}{4}x^2 + \frac{1}{8}x^3 + \frac{1}{16}x^4 + \ldots
\end{equation*}
This is an example of an \emph{infinite series}, which we don't consider
polynomials. Infinite series are interesting and useful, but we 
will not discuss them in detail until later in the course.
