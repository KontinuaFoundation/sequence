\chapter{Methods of Integration}

\section{u-substitution}
Sometimes a function's antiderivative isn't obvious. Take this 
integral for example: 

$$\int 4x \sqrt{1 + 2x^2}\, dx$$

We can solve this integral using \textit{u-substitution}\index{u-substitution}. 
Recall from implicit differentiation that if $u = f(x)$, then we can also say 
$du = f'(x) dx$. Let's set $u$ so that it is equal to the statement under the 
square root sign: 

$$u = 1 + 2x^2$$

Taking the derivative of both sides, we see that 

$$du = (4x) dx$$ 

How does this help us evaluate the integral? First, let's rearrange the 
integrand a bit: 

$$\int 4x \sqrt{1 + 2x^2}\,dx = \int \sqrt{1 + 2x^2} 4x\,dx$$

We can substitute $u = 1 + 2x^2$ and $du = 4x dx$ to get: 

$$= \int \sqrt{u}\,du$$

That is a much nicer integral! We can evaluate this integral using the Power 
Rule: 

$$\int \sqrt{u}\,du = \frac{2}{3}u^{3/2}$$

We can now substitute $u = 1 + 2x^2$ back into our solution to yield: 

$$= \frac{2}{3}(1 + 2x^2)^{3/2}$$

Feel free to double-check this answer by taking the derivative using the 
Chain Rule. You should get the original integrand, $4x \sqrt{1 + 2x^2}$, back. 

As you may have guessed, u-substitution is a method to help us "undo" the 
Chain Rule. Recall that the Chain Rule states: 

$$\frac{d}{dx}f(g(x)) = f'(g(x))g'(x)$$

If we integrate both sides we see that: 

$$f(g(x)) = \int f'(g(x))g'(x)\,dx$$ 

Which leads us to the formal definition of the u-substitution method:

If u = g(x) is a differentiable function whose range is an interval 
$I$ and $f$ is continuous on $I$, then $\int f(g(x))g'(x)\,dx = \int 
f(u)\,du$

Let's apply $u$-substitution to a definite integral:

\textbf{Example}: Evaluate $\int_e^{e^4} \frac{1}{x\sqrt{\ln{x}}}\,dx$.

\textbf{Solution}: Recall that $\frac{d}{dx}\ln{x} = \frac{1}{x}$. Letting 
$\ln{x} = u$, it follows that $\frac{dx}{x} = du$. Rearranging the integral 
and substituting:

$$\int_e^{e^4} \frac{1}{x\sqrt{\ln{x}}}\,dx = \int_e^{e^4} \frac{1}{\sqrt{\ln{
x}}} \frac{dx}{x}$$
$$=\int_{x=e}^{x=e^4} \frac{1}{\sqrt{u}}\,du$$

Proceeding from here, there are two options: you can find the value of $u$ at 
$x = e$ and $x = e^4$ and change the limits of the integral OR you can 
evaluate the integral, resubstitute back for $x$ and then evaluate the result 
with the original limits. We will show both to demonstrate each method and 
show they have the same result. 

\textit{Method 1: change the limits of integration}
When $x = e$, $u = \ln{e} = 1$. And when $x = e^4$, $u = \ln{e^4} = 4$. 
Therefore, we can change the limits of the integral to:
$$\int_1^4 \frac{1}{\sqrt{u}}\,du = 2\sqrt{u}|_1^4 = 2 \left[ \sqrt{4} - 
\sqrt{1} \right] = 2(2-1) = 2$$

\textit{Method 2: keep the limits of integration and resubstitute for $u$}:
$$\int_{x=e}^{x=e^4} \frac{1}{\sqrt{u}}\,du = 2\sqrt{u}|_{x=e}^{x=e^4} = 2\sqrt{
\ln{x}}|_e^{e^4}$$
$$= 2 \left[ \sqrt{\ln{e^4}} - \sqrt{\ln{e}} \right] = 2(\sqrt{4} - \sqrt{1}) 
= 2(2-1) = 2$$
Which is the same result as method 1. When done correctly, either method will 
yield the correct result. Choose the method you prefer. 

\begin{Exercise}[label=int_meth1]
Using the substitution $u = x^2 - 3$, re-write $\int_{-1}^4 x(x^2 - 3)
^5\.dx$ in terms of $u$.
\vspace{20mm}
\end{Exercise}

\begin{Answer}[ref=int_meth1]
If $u = x^2 - 3$, then $du = 2x dx$ and $x(x^2 - 3)^5 dx = \frac{1}{2}
u^5 du$. When $x = -1$, $u = -2$ and when $x = 4$, $u = 13$. Putting 
it all together, we find an equivalent integral is $\frac{1}{2}\int
_{-2}^{13} u^5\,du$. 
\end{Answer}

\begin{Exercise}[label=int_meth4]
Evaluate $\int_1^{\infty}xe^{-x^2}\,dx$. 
\vspace{30mm}
\end{Exercise}

\begin{Answer}[ref=int_meth4]
Letting $u = -x^2$, then $du = -2x dx$ and $x dx = \frac{-1}{2}du$. 
Substituting $u$ and $du$ into the integral, we have $\int_{x = 1}^{x 
= \infty} \frac{-1}{2}e^u\,du$, which equals $\frac{-1}{2}e^u = 
\frac{-1}{2}e^{-x^2}|_1^{\infty}$. Evaluating the statement, we get 
$\frac{-1}{2}(e^{-\infty} - e^{-1}) = \frac{-1}{2}(0-\frac{1}{e}) = 
\frac{1}{2e}$
\end{Answer}


\section{Partial Fractions}
We can integrate rational functions by using partial fractions to decompose a 
complex rational function into simpler ones\index{partial fractions and 
integration}. Suppose we wanted to integrate $f(x) = \frac{4x + 5}{x^2 + x - 2}$:
$$\int \frac{4x + 5}{x^2 + x - 2}\,dx = \int \left( \frac{3}{x - 1} + \frac{1}{
x + 2} \right)\,dx$$
$$= 3\ln{|x - 1|} + \ln{|x + 2|} + C$$

\textbf{Example}: Find $\int \frac{x^2 + x + 1}{(x + 1)^2 (x + 2)}\,dx$

\textbf{Solution}: We start by defining:
$$\frac{x^2 + x + 1}{(x + 1)^2 (x + 2)} = \frac{A}{x + 1} + \frac{B}{(x + 1)^
2} + \frac{C}{x + 2}$$
Multiplying both sides by $(x + 1)^2 (x + 2)$:
$$x^2 + x + 1 = A(x + 1)(x + 2) + B(x + 2) + C(x + 1)^2$$
Since there are only 2 roots to $(x + 1)^2 (x + 2)$, we will equate the 
coefficients to find $A$, $B$, and $C$. 
$$x^2 + x + 1 = A(x^2 + 3x + 2) + B(x + 2) + C(x^2 + 2x + 1)$$
$$x^2 + x + 1 = Ax^2 + 3Ax + 2A + Bx + 2B + Cx^2 + 2Cx + C$$
$$x^2 + x + 1 = (A + C)x^2 + (3A + B + 2C)x + (2A + 2B + C)$$
For this equation to be true, we know that:
$$A + C = 1$$
$$3A + B + 2C = 1$$
$$2A + 2B + C = 1$$
Solving for each, you should find that:
$$A = -2$$
$$B = 1$$
$$C = 3$$
And therefore, 
$$\frac{x^2 + x + 1}{(x + 1)^2 (x + 2)} = \frac{-2}{x + 1} + \frac{1}{(x + 1)^
2} + \frac{3}{x + 2}$$
Substituting this into our integral, 
$$\int \frac{x^2 + x + 1}{(x + 1)^2 (x + 2)}\,dx = \int \left[ \frac{-2}{x + 1} 
+ \frac{1}{(x + 1)^2} + \frac{3}{x + 2} \right]\,dx$$
$$ = -2\ln{|x + 1|} + \frac{-1}{x + 1} + 3\ln{|x + 2|} + C = \ln{ \left| \frac{
(x + 2)^3}{(x + 1)^2} \right| } - \frac{1}{x + 1} + C$$


\textbf{Example}: Evaluate $\int \frac{2x^2 - x + 4}{x^3 + 4x}\,dx$

\textbf{Solution}: We begin by factoring the denominator:
$$x^3 + 4x = x(x^2 + 4)$$
Which cannot be factored further. Therefore, we define:
$$\frac{2x^2 - x + 4}{x(x^2 + 4)} = \frac{A}{x} + \frac{Bx + C}{x^2 + 4}$$
$$2x^2 - x + 4 = A(x^2 + 4) + (Bx + C)x$$
$$2x^2 - x + 4 = Ax^2 + 4A + Bx^2 + Cx$$
Which implies that:
$$2 = A + B$$
$$C = -1$$
$$4A = 4$$
Therefore, $A = 1$, $B = 1$, and $C = -1$ and we can say that:
$$\int \frac{2x^2 - x + 4}{x^3 + 4x}\,dx = \int	\left[ \frac{1}{x} + \frac{x - 
1}{x^2 + 4} \right] \,dx$$
$$= \int \left[ \frac{1}{x} + \frac{x}{x^2 + 4} - \frac{1}{x^2 + 4} \right]\,
dx$$
$$= \ln{|x|} + \frac{1}{2}\ln{(x^2 + 4)} - \frac{1}{2}\arctan{\left( \frac{x}{2} 
\right) } + C$$

A useful identity that we used here is 
$$\int \frac{1}{x^2 + a^2}\,dx = \frac{1}{a} \arctan{\left( \frac{x}{a} \right)} 
+ C$$

\begin{Exercise}[label = int_meth2]
	Evaluate $\int_0^1 \frac{5x + 8}{x^2 + 3x + 2}\,dx$ without a 
	calculator. 
	\vspace{30mm}
\end{Exercise}

\begin{Answer}[ref=int_meth2]
	We cannot use u-substitution because $\frac{d}{dx}(x^2 + 3x + 2) \neq 
	n(5x + 8)$. We will use partial fractions to simplify the integrand. 
	Settig up: $\frac{5x + 8}{(x + 1)(x + 2)} = \frac{A}{x + 1} + 
	\frac{B}{x + 2}$. Rearranging, we find $5x + 8 = A(x + 2) + B(x + 1)$. 
	Letting $x = -2$, we find that $B = 2$. And taking $x = -1$, we find 
	$A = 3$. Therefore, $\int_0^1 \frac{5x + 8}{x^2 + 3x + 2}\,dx = \int
	_0^1 \frac{3}{x + 1}\,dx + \int_0^1 \frac{2}{x + 2}\,dx$. Evaluating 
	the integrals, we get $3\ln{(x + 1)}|_0^1 + 2\ln{(x + 2)}|_0^1 = 3(
	\ln{2} - \ln{1}) + 2(\ln{3} - \ln{2}) = 3\ln{2} + 2\ln{\frac{3}{2}} 
	= \ln{8} + \ln{\frac{9}{4}} = \ln{\frac{8 \cdot 9}{4}} = \ln{18}$. 
\end{Answer}

\begin{Exercise}[label = partfrac]
Use the method of partial fractions to evaluate the following integrals:
\begin{enumerate}
\item $\int \frac{4x}{x^3 + x^2 + x + 1}\,dx$
\item $\int_{-1}^0 \frac{x^3 - 4x + 1}{x^2 - 3x + 2}\,dx$
\item $\int \frac{x^3 + 2x}{x^4 + 4x^2 + 3}\,dx$
\end{enumerate}
\vspace{50mm}
\end{Exercise}

\begin{Answer}[ref=partfrac]
\begin{enumerate}
\item Let $\frac{4x}{x^3 + x^2 + x + 1} = \frac{A}{x + 1} + \frac{Bx + C}{x^2 
+ 1}$. Rearranging, we see that $4x = A(x^2 + 1) + (Bx + C)(x + 1)$. Which 
means that $4x = Ax^2 + A + Bx^2 + Bx + Cx + C$, which implies that $A + B = 
0$ and $B + C = 4$ and $A + C = 0$. Solving this system of equations, we see 
that $A = -2$, $B = 2$, and $C = 2$. So we can say that $\int \frac{4x}{x^3 + 
x^2 + x + 1}\,dx = \int \left[ \frac{-2}{x + 1} + \frac{2x}{x^2 + 1} + \frac{2
}{x^2 + 1} \right]\,dx$. Which evaluates to $-2\ln{|x + 1|} + \ln{|x^2 + 1|} 
+ 2\arctan{(x)} + K$, where $K$ is the constant of integration.  
\item Since the order of x is greater in the numerator, first we divide and 
see that $\frac{x^3 - 4x + 1}{x^2 - 3x + 2} = (x + 3) + \frac{3x - 5}{x^2 - 3x 
+ 2}$. Now let $\frac{3x-5}{x^2 - 3x + 2} = \frac{A}{x-2} + \frac{B}{x - 1}$, 
which means that $3x - 5 = A(x - 1) + B(x - 2)$. Solving, we find that $A = 1$ 
and $B = 2$. Therefore, $\int_{-1}^0 \frac{x^3 - 4x + 1}{x^2 - 3x + 2}\,dx = 
\int_{-1}^0 \left[ x + 3 + \frac{1}{x-2} + \frac{2}{x-1} \right]\,dx$ which 
evaluates to $\frac{1}{2}x^2 + 3x + \ln{|x - 2|} + \ln{|x - 1|}|_{x = -1}^{x = 
0} = \frac{5}{2} - \ln{(3)}$. 
\item Note that $\frac{x^3 + 2x}{x^4 + 4x^2 + 3} = \frac{x^3 + 2x}{(x^2 + 1)(x^
2 + 3)}$. Then let $\frac{x^3 + 2x}{(x^2 + 1)(x^2 + 3)} = \frac{Ax + B}{x^2 + 1
} + \frac{Cx + D}{x^2 + 3}$. Then $x^3 + 2x = (A + C)x^3 + (B + D)x^2 + (3A + C
)x + (3B + D)$ which implies that $A + C = 1$, $B + D = 0$, $3A + C = 2$, and $
3b + D = 0$. Solving this system of equations, we see that $A = C = \frac{1}{2}
$ and $B = D = 0$, which means that $\frac{x^3 + 2x}{x^4 + 4x^2 + 3} = \frac{x
}{2(x^2 + 1)} + \frac{x}{2(x^3 + 3}$. And therefore, $int \frac{x^3 + 2x}{x^4 
+ 4x^2 + 3}\,dx = \int \left[ \frac{x}{2(x^2 + 1)} + \frac{x}{2(x^3 + 3} 
\right]\,dx = \frac{1}{4}\ln{|x^2 + 1|} + \frac{1}{4}\ln{|x^2 + 3|} + K$, 
where $K$ is the constant of integration. 
\end{enumerate}
\end{Answer}

\section{Integration by Parts}

Recall the Product Rule for derivatives:

$$\frac{d}{dx} \left[ f(x) \cdot g(x) \right] = f(x) \cdot g'(x) + f'(x) \cdot g(x)$$

If we integrate both sides, we find that:

$$f(x) \cdot g(x) = \int \left[ f(x) \cdot g'(x) + f'(x) \cdot g(x) \right]\,dx$$
$$f(x) \cdot g(x) = \int f(x)g'(x)\,dx + \int f'(x)g(x)\,dx$$

Rearranging, 

$$\int f(x)g'(x)\,dx = f(x)g(x) - \int f'(x)g(x)\,dx$$

This identity allows us to perform \textbf{integration by parts}
\index{integration by parts}, a powerful method that allows us to evaluate 
integrals of complex functions. 

\textbf{Example}: Evaluate $\int x \cos{x} \,dx$.

\textbf{Solution}: We may be tempted to try $u$-substitution, but that won't 
work because $\frac{d}{dx} \cos{x}$ is not proportional to $x$ and $\frac{d}{dx} 
x$ is not proportional to $\cos{x}$. Let us define $f(x) = x$ and $g'(x) = 
\cos{x}$. This implies $f'(x) = 1$ and $g(x) = \sin{x}$. Then we can say that:

$$\int x \cos{x} \,dx = \int f(x)g'(x)\,dx$$

Using the identity $\int f(x)g'(x)\,dx = f(x)g(x) - \int f'(x)g(x)\,dx$ and 
substituting for $f(x)$, $f'(x)$, $g(x)$, and $g'(x)$, we see that:

$$\int x \cos{x} \,dx = \left[ x \sin{x} \right] - \int 1 \cdot \sin{x}\,dx$$
$$= x \sin{x} - \int \sin{x}\,dx$$
$$= x \sin{x} - \left( -\cos{x}  + C \right) = x \sin{x} + \cos{x} + C$$

(recall that C is the integration constant). You can check your results by 
taking the derivative: you should get the original integrand back. Let's check 
our result in this case:
$$\frac{d}{dx} \left[ x \sin{x} + \cos{x} + C \right] = \frac{d}{dx} \left[x 
\sin{x} \right] + \frac{d}{dx} \cos{x}$$
$$= x \frac{d}{dx} \left( \sin{x} \right) + \sin{x} \frac{d}{dx} (x) - \sin{x}$$
$$= x \cos{x} +|SIN{x} - \sin{x} = x \cos{x}$$

How did we choose that $f(x)$ should be $x$ and $g(x)$ should be $\sin{x}$ in 
the example above? In general, you want to choose such that the resulting 
integral is simpler than the one we started with. This means you want to choose 
$f$ such that $f'$ is \textit{less complex} or a \textit{lower order} than $f$. 

To illustrate this, let's re-evalute the example above, but this time let $f(x) 
= \cos{x}$ and $g'(x) = x$. Then we can say that $f'(x) = -\sin{x}$ and $g(x) 
= \frac{1}{2}x^2$. Substituting this into the integration by parts identity, 
we find that:
$$\int x \cos{x} \,dx = \frac{1}{2}x^2 \cos{x} - \int -\frac{1}{2}x^2 \sin{x}\,dx$$

Now the integral on the right side is more complex than the one we started 
with (on the left)! A good general rule for integration by parts is that 
\textit{if} the two functions in the original integral are a polynomial and a 
sine or cosine function, set the polynomial to be $g(x)$ and the trigonometric 
function to be $f'(x)$. The polynomial will be differentiated and become 
\textit{less} complex, while integrating the trigonometric function won't make 
it \textit{more} complex. 

Integration by parts is valid for definite integrals as well. Mathematically, 
this means:

$$\int_a^b f(x)g'(x)\,dx = f(x)g(x)|_a^b - \int_a^b f'(x)g(x)\,dx$$

Which is the same as:

$$\int_a^b f(x)g'(x)\,dx = \left( f(b) g(b) \right) - \left( f(a) g(a) \right) 
- \int_a^b f'(x)g(x)\,dx$$

Let's see one more example that incorporates both $u$-substitution and 
integration by parts.

\textbf{Example}: Evaluate $\int \frac{\arcsin{\ln{x}}}{x}\,dx$

\textbf{Solution}: First, we notice that $\ln{x}$ and $\frac{1}{x}$ both 
appear in the integrand. Let us define $u = \ln{x}$. Then $du = \frac{dx}{x}$:

$$\int \arcsin{\ln{x}} \frac{dx}{x} = \int \arcsin{u}\,du$$

For integration by parts, if we let $\arcsin{u} = f(u)$ and $du = g'(u)$, it 
follows that $f'(u) = \frac{1}{\sqrt{1-u^2}}$ and $g(u) = u$. Then we can say 
that:
$$\int \arcsin{u}\,du = \arcsin{u} \cdot u - \int \frac{u}{\sqrt{1-u^2}}\,du$$

We can use $u$-substitution again to evaluate the second integral (we will use 
$v$, since we have already said that $u = \ln{x}$). Let $v = 1-u^2$, which 
means that $\frac{dv}{2} = (-u)du$. Substituting:

$$= u \cdot \arcsin{u} + \int \frac{1}{2\sqrt{v}}\,dv = u \cdot \arcsin{u} + 
\sqrt{v}$$

Substituting back for $v$:

$$= u \cdot \arcsin{u} + \sqrt{1-u^2}$$

And substituting back for $u$:

$$= \ln{x} \cdot \arcsin{\ln{x}} + \sqrt{1 - \ln^2{x}}$$

\begin{Exercise}[label=int_meth3]
Let $f$ be a function such that $\int f(x) \sin{x}\,dx = -f(x)\cos{x} 
+ \int 4x^3 \sin{x}\,dx$. Give a possible expression for $f(x)$. 
\end{Exercise}

\begin{Answer}[ref=int_meth3]
This question takes the form of integration by parts. That is, $\int 
f(x)g'(x)\,dx = f(x)g(x) - \int g(x)f'(x)\,dx$. If we let $g(x) = 
-\cos{x}$, then $g'(x) =\sin{x}$. The structure of the equation 
implies that $f'(x) = 4x^3$ and therefore that $f$ could be $f(x) = 
x^4$. 
\end{Answer}

\begin{Exercise}[label = int_meth5]
Evaluate the following integrals using integration by parts:
\begin{enumerate}
\item $\int_0^{1} x \sin{\frac{\pi}{2} x}\,dx$
\item $\int e^{\theta} \cos{\theta}\,d\theta$
\item $\int (1-t)^2 \cos{\beta t}\,dt$
\end{enumerate}
\vspace{60mm}
\end{Exercise}

\begin{Answer}[ref = int_meth5]
\begin{enumerate}
\item $\frac{4}{\pi^2}$. Let $f = x$ and $g' = \sin{\frac{\pi}{2}x}dx$. Then 
$f' = dx$ and $g = -\frac{2}{\pi}\cos{\frac{\pi}{2}x}$. Which implies that 
$\int_0^{1} x \sin{\frac{\pi}{2} x}\,dx = \left[ \frac{-2}{\pi}\cos{\frac{\pi}{
2}x} \right]_{x=0}^{x=1} - \int_0^1 \frac{-2}{\pi}\cos{\frac{\pi}{2}x}\,dx$. 
Evaluating $\left[ \frac{-2x}{\pi}\cos{\frac{\pi}{2}x} \right]_{x=0}^{x=1} = 
\left( \frac{-2}{\pi}\cos{\frac{\pi}{2}} \right) - \left(0 \cos{0} \right) = 0 
- 0 = 0$. Therefore, $\int_0^{1} x \sin{\frac{\pi}{2} x}\,dx = \int_0^1 \frac{2
}{\pi}\cos{\frac{\pi}{2}x}\,dx = \frac{2}{\pi} \left[\frac{2}{\pi}\sin{\frac{
\pi}{2}x} \right]_0^1 = \frac{4}{\pi^2} \left[ \sin{\frac{\pi}{2}} - \sin{0} 
\right] = \frac{4}{\pi^2}$. 
\item $\frac{e^{\theta}}{2} \left( \sin{\theta} + \cos{\theta} \right)$. Let 
$f = e^{\theta}$ and $g' = \cos{\theta} d\theta$. Then $f' = e^{\theta} 
d\theta$ and $g = \sin{\theta}$ and according to integration be parts $\int 
e^{\theta} \cos{\theta}\,d\theta = e^{\theta} \sin{\theta} - \int e^{\theta} 
\sin{\theta}\,d\theta$. We can also evaluate $\int e^{\theta} \sin{\theta}\,
d\theta$ using integration by parts. Let $f = e^{\theta}$ and $g' = \sin{
\theta} d\theta$. Then $f' = e^{\theta} d\theta$ and $g = -\cos{\theta}$ and 
according to integration by parts $\int e^{\theta} \cos{\theta}\,d\theta = e^{
\theta} \sin{\theta} - \left[ -e^{\theta} \cos{\theta} - \int -e^{\theta} \cos{
\theta}\,d\theta \right] = e^{\theta} \sin{\theta} + e^{\theta} \cos{\theta} - 
\int e^{\theta} \cos{\theta}\,d\theta$. We can rearrange this to solve for 
$\int e^{\theta} \cos{\theta}\,d\theta$: $2\int e^{\theta} \cos{\theta}\,
d\theta = e^{\theta} \sin{\theta} + e^{\theta} \cos{\theta} \rightarrow \int 
e^{\theta} \cos{\theta}\,d\theta = \frac{e^{\theta}}{2} \left(\sin{\theta} + 
\cos{\theta} \right)$. 
\item $\frac{(1 - t)^2}{\beta} \sin{\beta t} + \frac{2(t - 1)}{\beta ^2} \cos{
\beta t} - \frac{2}{\beta ^3} \sin{\beta t}$. Let $f = (1 - t)^2$ and $g' = 
\cos{\beta t}dt$. Then $f' = -2(1-t)dt$ and $g = \frac{1}{\beta} \sin{\beta t
}$. Then using integration by parts $\int (1-t)^2 \cos{\beta t}\,dt = \frac{(1 
- t)^2}{\beta} \sin{\beta t} - \int \frac{(-2)(1-t)}{\beta} \sin{\beta t}\,dt 
= \frac{(1 - t)^2}{\beta} \sin{\beta t} + \frac{2}{\beta} \int (1 - t) \sin{
\beta t}\,dt$. We use integration by parts again to evaluate $\int (1 - t) 
\sin{\beta t}\,dt$. Let $f = 1 - t$ and $g' = \sin{\beta t}dt$. Then $f' = 
-dt$ and $g = -\frac{1}{\beta} \cos{\beta t}$. Then $\int (1 - t) \sin{\beta t}
\,dt = (1 - t) \left( -\frac{1}{\beta} \right) \cos{\beta t} - \int \left( -
\frac{1}{\beta} \right) \cos{\beta t}\,-dt = \frac{t - 1}{\beta} \cos{\beta t} 
- \int \frac{\cos{\beta t}}{\beta}\,dt = \frac{t - 1}{\beta} \cos{\beta t} - 
\frac{1}{\beta ^2} \sin{\beta t}$. Substituting this back in for $\int (1 - t)
\sin{\beta t}\,dt$, we see that $\int (1 - t)^2 \cos{\beta t}\,dt = \frac{(1-t)
^2}{\beta} \sin{\beta t} + \frac{2}{\beta} \left[ \frac{t - 1}{\beta} \cos{
\beta t} - \frac{1}{\beta ^2} \sin{\beta t} \right] = \frac{(1 - t)^2}{\beta} 
\sin{\beta t} + \frac{2(t - 1)}{\beta ^2} \cos{\beta t} - \frac{2}{\beta ^3} 
\sin{\beta t}$.
\end{enumerate}
\end{Answer}