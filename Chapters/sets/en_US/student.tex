\chapter{Sets and Logic}

The study of math can be basically broken into two purposes:
\begin{itemize}
\item \textit{Developing mathematical tools that let us make better predictions.}  This is how engineers and scientists use math.  It is usually referred to as \newterm{applied math}.
\item \textit{Creating interesting statements and proving them to be true or false.}  This is known as \newterm{pure math}.
\end{itemize}

A lot of mathematical ideas start out as pure math, and eventually
become useful.  For example, the field of number theory is devoted to
proving things about prime numbers.  The mathematicians who created
number theory were certain that it could never be used for any
practical purpose.  After a century or two, number theory was used as
the basis most cryptography systems.

Some ideas start out as a "rule of thumb" that engineers use and are
eventually rigorously defined and proven.

This course tends to emphasize applied math, but you should know
something about the tools of pure math.

All of mathematics is built on a very small and simple foundation:\index{axiom}
\begin{itemize}
\item The idea of a set.
\item A short list of axioms defined in terms of sets. (An \newterm{axiom} is statement that we just accept as true.)
\item A few rules of logic.
\end{itemize}

There have been several efforts to codify a small but complete
axiomatic system. The most popular one is known as \newterm{ZFC}.  "Z"
is for the Ernst Zermelo, who did most of the work.  "F" is for
Abraham Fraenkel, who tidied up a couple of things. "C" is for The
Axiom of Choice.  As a community, mathematicians debate whether the
Axiom of Choice should be an axiom; we get a couple of strange results
if we include it the system.  If we don't, there are a few obviously
useful ideas that we can't prove true.

ZFC has 10 axioms.  We simply accept these 10 statements as true, and
all the proofs of modern mathematics can be extrapolated from them.

\section{Sets}

A set is a collection.  For example, you might talk about the set of
odd numbers greater than 5.  Or the set of all protons in the
universe.  \index{set}

We have a notation for sets.  For example, here is how define $A$ to
be the set containing 1, 2, and 3:

$$S = \{1, 2, 3\}$$

We say that 1, 2, and 3, are \newterm{elements} of the set $S$.
(Sometimes we will also use the word "member")

If you want to say "2 is an element of the set $S$" in mathematical
notation, it is done like this:

$$ 2 \in S$$

If you want to say "5 is \textit{not} an element of the set $S$" it
looks like this:

$$ 5 \notin S$$

We have notation for a few sets that we use all the time:

\begin{tabular}{c|c}
Set & Symbol \\
\hline
The empty set & $\emptyset$ \\
Natural numbers & $\mathbb{N}$ \\
Integers & $\mathbb{Z}$ \\
Rational numbers & $\mathbb{Q}$ \\
Real numbers & $\mathbb{R}$ \\
Complex numbers & $\mathbb{C}$ \\
\end{tabular}

The empty set is the set that contains nothing.  It is also sometimes
called \newterm{the null set}.

Often when we define a set, we start with a big set and say "The set
I'm talking about is the members of the set for which this statement
is true".  For example, if you wanted to talk about all the integers
greater than or equal to -5, you could do it like this:

$$A = \{ x \in \mathbb{Z} \mid x \geq -5 \}$$

When you read this aloud, you say "$A$ is the set of integers $x$
where $x$ is greater than or equal to negative 5."

\subsection{And and Or}


Sometimes you need the members to satisfy two conditions; for this we
use "and": \index{and}

$$A = \{ x \in \mathbb{Z} \mid x > -5 \text{ and }  x < 100\}$$

This is the set of integers that are greater than -5 \textit{and} less
than 100.  In this book, we usually just write "and," but if you do a
lot of set and logic work, you will use the symbol $\land$:

$$A = \{ x \in \mathbb{Z} \mid (x > -5) \land (x < 100)\}$$

Sometimes you want a set that satisfies at least one of two
conditions.  For this you use "or":\index{or}

$$A = \{ x \in \mathbb{Z} \mid x < -5 \text{ or } x > 100\}$$

These are the number that are less then -5 or greater than 100.  Once
again, there is a symbol for this:

$$A = \{ x \in \mathbb{Z} \mid (x < -5) \lor (x > 100)\}$$

\subsection{How simple are sets?}

Sets are so simple that some questions just don't make any sense:
\begin{itemize}
\item "What is the first item in the set?" makes no sense to a mathematician.  Sets have no order.
\item "How many times does the number 6 appear in the set?"  makes no sense.   6 is a member, or it isn't.   
\end{itemize}

\subsection{Subsets}

If every member of set $A$ is also in set $B$, we say that "$A$ is a
subset of $B$". \index{subset}

For example, if $A = \{1,4,5\}$ and $B = \{1,2,3,4,5,6\}$, then $A$ is
a subset of $B$.  There is a symbol for this:

$$ A \subseteq B$$

Remember the table of commonly used sets?  We can arrange them as
subsets of each other:

$$\emptyset \subseteq \mathbb{N} \subseteq \mathbb{Z} \subseteq \mathbb{Q} \subseteq \mathbb{R} \subseteq \mathbb{C}$$

Note that that subsets have the transitive property: $\mathbb{N}
\subseteq \mathbb{Z} \subseteq \mathbb{Q}$ thus $\mathbb{N} \subseteq
\mathbb{Q}$

Note that if $A$ and $B$ have the same elements, $A \subseteq B$
\textit{ and } $B \subseteq A$.  We say that the two sets are equal.

We also have a symbol for "is not a subset of": $A \not\subset B$

\subsection{Union and Intersection of Sets}

If you have two sets $A$ and $B$, you might want to say "Let $C$ be
the set containing element that are in \textit{either} $A$ or $B$."
We say that $C$ is the \newterm{union} of $A$ and $B$.  There is
notation for this too:

$$C = A \cup B$$

For example,  if $A = \{1,3,4,9\}$ and $B = \{3,4,5,6,7,8\}$  then $A \cup B =  \{1,3,4,5,6,7,8,9\}$.

You also want to say "Let $C$ be the set containing elements that are
in \textit{both} $A$ and $B$."  We say that $C$ is the
\newterm{intersection} of $A$ and $B$.  There is notation for this
too:

$$C = A \cap B$$

For example,  if $A = \{1,3,4,9\}$ and $B = \{3,4,5,6,7,8\}$  then $A \cap B =  \{3,4\}$.


