\chapter{Projectile Motion}

A projectile is an object that, once thrown or dropped, continues to move only 
under the influence of gravity. Throwing a baseball, shooting a cannon, and 
diving off a high diving board are all examples. NASA flight planners use 
projectile motion to plan flight paths for space vehicles, such as sending 
rovers to Mars. You've already learned how to describe and model 
one-dimensional projectile motion in the Falling Bodies chapter. Now, we 
consider projectiles that also have horizontal motion, and therefore are 
moving in two dimensions. 

First, we will compare the motion of projectiles that are dropped versus 
horizontally launched from the same height. This will frame our discussion of 
the important concept of independence of motion: the vertical and horizontal 
motions of a projectile can be considered and described independent from each 
other. This will allow you to predict how far horizontally launched objects 
will travel before hitting the ground. Next, you'll learn to describe the 
motion of projectiles launched at an angle (like some heavy ground artillery). 
Finally, you'll use what you've learned to create a model of any projectile motion. 

\section{Comparing Projectiles}
This video was mentioned at the end of the kinematics chapter: \url{https://www.youtube.com/watch?v=zMF4CD7i3hg}.
From the video, we can see that the addition of horizontal motion does not effect how fast an object is acted upon by gravity. Both objects hit the ground at the same time, regardless of whether horizontal motion was added or not.


%FIXME complete entire chapter
\section{Independence of Motion}
we can describe the x and y motion separately. 
you already know how to describe the y motion from falling bodies
show 2D kinematics equations
graphs comparing x motions and y motions

\section{Horizontally-launched Projectiles}
example

exercise

exercise - Newton's cannon

\section{Projectiles launched at an Angle}
\subsection{From the Ground}
separating vertical and horizontal components of initial motion with trigonometry
example

exercise - how far does the object travel?

exercise - at what angle should you launch for an object to go the furthest given a maximum launch velocity?

exercise - I have a target x-meters away, I must launch at v-miles per hour, what angle will allow me to hit my target, if any?

\section{Simulating Projectile Motion}
% python script?
