\chapter{Projectile Motion}

A projectile is an object that, once thrown or dropped, continues to move only 
under the influence of gravity. Throwing a baseball, shooting a cannon, and 
diving off a high diving board are all examples. NASA flight planners use 
projectile motion to plan flight paths for space vehicles, such as sending 
rovers to Mars. You've already learned how to describe and model 
one-dimensional projectile motion in the Falling Bodies chapter. Now, we 
consider projectiles that also have horizontal motion, and therefore are 
moving in two dimensions. 

First, we will compare the motion of projectiles that are dropped versus 
horizontally launched from the same height. This will frame our discussion of 
the important concept of independence of motion: the vertical and horizontal 
motions of a projectile can be considered and described independent from each 
other. This will allow you to predict how far horizontally launched objects 
will travel before hitting the ground. Next, you'll learn to describe the 
motion of projectiles launched at an angle (like some heavy ground artillery). 
Finally, you'll use what you've learned to create a model of any projectile motion. 

\section{Comparing Projectiles}
This video was mentioned at the end of the kinematics chapter: \url{https://www.youtube.com/watch?v=zMF4CD7i3hg}.
From the video, we can see that the addition of horizontal motion does not effect how fast an object is acted upon by gravity. Both objects hit the ground at the same time, regardless of whether horizontal motion was added or not. This is because of a concept called \emph{independence of motion}.


\section{Independence of Motion}
In projectile motion, such as a ball being thrown off of a cliff, the horizontal and vertical components of motion are independent of each other. This means that horizontal motion (and forces in the horizontal direction) do not affect vertical motion (and forces in the vertical direction), and vice versa. This is because gravity only acts in the vertical direction.\footnote{This holds true for objects on sloped surfaces. Gravitational motion is just separated into components parallel and perpendicular to the slope. This will be covered in a future chapter.}

If you recall the falling bodies chapter, you know that the vertical motion of an object in free fall is described by the equations of unformly accelerated motion, with a constant acceleration of $9.8 \frac{m}{s^2}$ downward (this may simplified to $10 \frac{m}{s^2}$ for simplicity in many calculations).

In the horizontal direction, if we ignore air resistance (a common practice for elementary physics), there are no forces acting on the object. This means that the object will continue to move at a constant velocity in the horizontal direction.

Because the horizontal and vertical motions are independent of each other, we can use different equations to describe the motion in each direction.

\begin{center}
    \begin{table}[]
    \begin{tabular}{|c| c|}
        \underline{Horizontal Motion}       & \underline{Vertical Motion}  \\
        \hline
    
    $\Delta x = v_{0x} t$      & $\Delta y = v_{0y} t + \frac{1}{2} (-g) t^2$  \\
    $v_x = v_{0x}$ (constant!) & $v_y = v_{0y} + (-g) t$   \\
    $a_x = 0$                & $a_y = -g$ 
    
    \end{tabular}
    \end{table}
    
\end{center}

Note that the velocity in the horizontal direction is \emph{\textbf{constant}}, while the velocity in the vertical direction is \emph{\textbf{not constant}} due to the acceleration of gravity ($9.8 \frac{m}{s^2}$ downward).

% FIXME
graphs comparing x motions and y motions

\section{Horizontally-launched Projectiles}
Imagine a ball is thrown horizontally off of a cliff. The ball has an initial horizontal velocity of $20$ m/s, but no initial vertical velocity. The ball will continue to move horizontally at a constant velocity, while simultaneously accelerating downward due to gravity. The ball falls for 5 seconds before hitting the ground. We will use $-10 \frac{m}{s^2}$ for the acceleration due to gravity to make calculations easier.

\begin{tikzpicture}
    %FIXME
\end{tikzpicture}

Let's calculate the change in $y$. 

The vertical motion can be described by the equation:

\[
\Delta y = v_{0y} t + \frac{1}{2} (-g) t^2
\]

Since the initial vertical velocity $v_{0y} = 0$, this simplifies to:

\[
\Delta y = \frac{1}{2} (-g) t^2
\]

Substituting in the values for $g$ and $t$:

\[
\Delta y = \frac{1}{2} (-10 \frac{m}{s^2}) (5 s)^2
\]

Calculating this gives:

\[
\Delta y = -125 m
\]

So the ball falls a vertical distance of $122.5$ m from its origin before hitting the ground.

Now let's calculate the change in $x$.
The horizontal motion can be described by the equation:

\[
\Delta x = v_{0x} t
\]

Substituting in the values for $v_{0x}$ and $t$:
\[
\Delta x = (20 \frac{\text{m}}{\text{s}}) (5 \text{s}) = 100 \text{m}
\]

%FIXME
exercise
% FIXME 
exercise - Newton's cannon % https://en.wikipedia.org/wiki/Newton's_cannonball 

\section{Projectiles launched at an Angle}
\subsection{From the Ground}

separating vertical and horizontal components of initial motion with trigonometry
example

exercise - how far does the object travel?

exercise - at what angle should you launch for an object to go the furthest given a maximum launch velocity?

exercise - I have a target x-meters away, I must launch at v-miles per hour, what angle will allow me to hit my target, if any?

\section{Simulating Projectile Motion}
% python script?
