\chapter{Linear Combinations and Span}

In the introductory linear algebra chapter, you learned that vectors and 
matrices can be rotated, inverted, and added. In this chapter, we will explore 
linear combinations of vectors and the span of group of vectors. The 
\textbf{span}\index{span} of a group of vectors is the set of vectors that can 
be made with linear combinations of the original group of vectors. We will 
offer mathematical and visual explanations later in the chapter. First, let's 
examine linear combinations. 

\section{Linear Combinations of Vectors}
A linear combination is simply the addition of vectors with leading scalar 
multipliers. For example, $3 \left[ 2, -1 \right] + 2 \left[ 3, 5 \right]$ is 
a linear combination of the vectors $\left[ 2, -1 \right]$ and $\left[ 3, 5 
\right]$. Another way to say this is:

\begin{mdframed}[style = important, frametitle = {Linear Combination of 
Vectors}]
A linear combination of a list of $n$ vectors, $v_1, v_2, \cdots, v_n$ takes 
the form:
$$a_1 v_1 + a_2 v_2 + \cdots + a_n v_n$$

where $a_1, a_2, \cdots, a_n \in \mathbb{R}$
\end{mdframed}

\textbf{Example}: Find a linear combination of $\left[ 2, 1, -3 \right]$ and 
$\left[ 1, -2, 4 \right]$ that gives the vector $\left[ 17, -4, 2 \right]$.

\textbf{Solution}: We are looking for $a_1$ and $a_2$ such that:
$$a_1 \left[ 2, 1, -3 \right] + a_2 \left[ 1, -2, 4 \right] = \left[17, -4, 2 
\right]$$

Looking at each dimension separately, we get the system of equations:
$$2 a_1 + 1 a_2 = 17$$
$$1 a_1 - 2 a_2 = -4$$
$$-3 a_1 + 4 a_2 = 2$$

If we can solve this system of equations, we will find $a_1$ and $a_2$. Let's 
multiply the first equation by 2 and add it to the second equation:
$$2 \left[ 2 a_1 + a_2 \right] + \left[ a_1 - 2 a_2 \right] = 2 \left( 17 
\right) + -4$$
$$4 a_1 + 2 a_2 + a_1 - 2 a_2 = 34 - 4$$
$$5 a_1 = 30$$
$$a_1 = 6$$

Now we can take $a_1$ and substitute it back into any equation in our system 
to find $a_2$. Let's use the third equation:
$$-3 \left( 6 \right) + 4 a_2 = 2$$
$$ -18 + 4 a_2 = 2$$
$$4 a_2 = 20$$
$$a_2 = 5$$

Since we used all 3 equations, we know $a_1 = 6$ and $a_2 = 5$ are solutions 
to all 3 equations. If we had only used the first two equations to find $a_1$ 
and $a_2$, we would want to substitute our values back into the third equation 
to make sure our solution holds for that equation also. 

Therefore, $6 \left[ 2, 1, -3 \right] + 5 \left[ 1, -2, 4 \right] = \left[ 17, 
-4, 2 \right]$.

\begin{Exercise}[title = {Linear Combinations}, label = combo]
Find a linear combination of the first two vectors that yields the third 
vector. 
\begin{enumerate}
\item $\left[1, 2 \right]$, $\left[ -3, 1 \right]$, $\left[ 4, 5 \right]$
\item $\left[ 9, 4 \right]$, $\left[ 0, 1 \right]$, $\left[ -5, 3 \right]$
\item $\left[ 7, -2 \right]$, $\left[ -8, 4 \right]$, $\left[ 6, -2 \right]$
\end{enumerate}
\vspace{50mm}
\end{Exercise}

\begin{Answer}[ref = combo]
\begin{enumerate}
    \item We are looking for $a_1$ and $a_2$ such that:
    $$a_1 \left[ 1, 2 \right] + a_2 \left[ -3, 1 \right] = \left[ 4, 5 
    \right]$$
    Which creates the system of equations:
    $$a_1 - 3 a_2 = 4$$
    $$2 a_1 + a_2 = 5$$

    We can multiply the first equation by $-2$ and add it to the second to 
    solve for $a_2$:
    $$-2 \left( a_1 - 3 a_2 \right) + 2 a_1 + a_2 = -2 \left( 4 \right) + 5$$
    $$6 a_2 + a_2 = -8 + 5$$
    $$7 a_2 = -3$$
    $$a_2 = -\frac{3}{7}$$

    Substituting $a_2$ back into an equation and solving for $a_1$:
    $$a_1 - 3 \left( -\frac{3}{7} \right) = 4$$
    $$a_1 + \frac{9}{7} = 4$$
    $$a_1 = \frac{19}{7}$$

    Therefore, $\frac{19}{7} \left[ 1, 2 \right] - \frac{3}{7} \left[ -3, 1 
    \right] = \left[ 4, 5 \right]$.

    \item We are looking for $a_1$ and $a_2$ such that:
    $$a_1 \left[ 9, 4 \right] + a_2 \left[ 0, 1 \right] = \left[ -5, 3 \right]$$
    Which creates the system of equations:
    $$9 a_1 = -5$$
    $$4 a_1 + a_2 = 3$$

    We can find $a_1$ from the first equation:
    $$a_1 = -\frac{5}{9}$$

    Substituting for $a_1$ back into the second equation and solving for $a_2$:
    $$4 \left( -\frac{5}{9} \right) + a_2 = 3$$
    $$a_2 - \frac{20}{9} = 3$$
    $$a_2 = \frac{47}{9}$$

    Therefore, $-\frac{5}{9} \left[ 9, 4 \right] + \frac{47}{9} \left[ 0, 1 
    \right] = \left[ -5, 3 \right]$.

    \item We are looking for $a_1$ and $a_2$ such that:
    $$a_1 \left[ 7, -2 \right] + a_2 \left[ -8, 4 \right] = \left[ 6, -2 
    \right]$$

    Which yields the system of equations:
    $$7 a_1 - 8 a_2 = 6$$
    $$-2 a_1 + 4 a_2 = -2$$

    Doubling the second equation and adding it to the first:
    $$7 a_1 - 8 a_2 + 2 \left( -2 a_1 + 4 a_2 \right) = 6 + 2 \left( -2 
    \right)$$
    $$7 a_1 - 8 a_2 - 4 a_1 + 8 a_2 = 6 - 4$$
    $$3 a_1 = 2$$
    $$a_1 = \frac{2}{3}$$

    Substituting for $a_1$ back into the second equation and solving for $a_2$:
    $$-2 \left( \frac{2}{3} \right) + 4 a_2 = -2$$
    $$-\frac{4}{3} + 4 a_2 = -2$$
    $$4 a_2 = -\frac{2}{3}$$
    $$a_2 = -\frac{1}{6}$$

    Therefore, $\frac{2}{3} \left[ 7, -2 \right] - \frac{1}{6} \left[ -8, 4 
    \right] = \left[ 6, -2 \right]$
\end{enumerate}
\end{Answer} 

Sometimes, a set of vectors cannot be combined to make a specific vector. Take 
the pair of vectors we have looked at before: $\left[ 2, 1, -3 \right]$ and 
$\left[ 1, -2, 4 \right]$. Can we find a combination to make vector $\left[ 
17, -4, 5 \right]$? Let's try. We define $a_1$ and $a_2$ such that:
$$a_1 \left[ 2, 1, -3 \right] + a_2 \left[ 1, -2, 4 \right] = \left[ 17, -4, 5 
\right]$$

Which creates the system of equations:
$$2 a_1 + a_2 = 17$$
$$a_1 - 2 a_2 = -4$$
$$-3 a_1 + 4 a_2 = 5$$

We have two variables ($a_1$ and $a_2$) and three equations. Let's use the 
first two to find $a_1$ and $a_2$, then check our answers by substituting our 
solutions into the third equation. First, we'll multiply the second equation 
by $-2$ and add that to the first equation:
$$2 a_1 + a_2 + \left( -2 \right) \left( a_1 - 2 a_2 \right) = 17 + \left( -2 
\right) \left( -4 \right)$$
$$2 a_1 + a_2 - 2 a_1 + 4 a_2 = 17 + 8$$
$$5 a_2 = 25$$
$$a_2 = 5$$

Substituting for $a_2$ back into the first equation and solving for $a_1$:
$$2 a_1 + 5 = 17$$
$$2 a_1 = 12$$
$$a_1 = 6$$

Now, let's check if $a_1 = 6$, $a_2 = 5$ is a solution to the third equation:
$$-3 \left( 6 \right) + 4 \left( 5 \right) = 5$$
$$-18 + 20 = 2 \neq 5$$

Therefore, there is no linear combination of the vectors $\left[ 2, 1, -3 
\right]$ and $\left[ 1, -2, 4 \right]$ that yields $\left[ 17, -4, 5 \right]$.

\section{Span}
The span of a list of vectors is all the vectors than can be made from a 
linear combination of those vectors. Above, we saw that $\left[ 17, -4, 2 
\right]$ is in the span of $\left[ 2, 1, -3 \right]$ and $\left[1, -2, 4 
\right]$ while $\left[ 17, -4, 5 \right]$ is not. 

\begin{mdframed}[style = important, frametitle = {Span of a List of Vectors}]
The set of all linear combinations of a list of vectors, $v_1, v_2, \cdots, 
v_n$ is call the span of $v_1, v_2, \cdots, v_n$ and is denoted by $\text{span}
(v_1, v_2, \cdots, v_n)$. Mathematically, 
$$\text{span}(v_1, v_2, \cdots, v_n) = \{a_1 v_1 + a_2 v_2 + \cdots + a_n v_n 
| a_1, a_2, \cdots, a_n \in \mathbb{R} \}$$

The span of an empty list of vectors is $\{0\}$. 
\end{mdframed}



\section{Linear Independence}