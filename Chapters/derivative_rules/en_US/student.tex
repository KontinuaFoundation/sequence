\chapter{Rules for Finding Derivatives}


Derivatives play a key role in calculus, providing us with a means of
calculating rates of change and the slopes of curves. In this chapter, we present
some common rules used to calculate derivatives.

\section{Constant Rule}

The derivative of a constant is zero. If $c$ is a constant and $x$ is
a variable, then:\index{constant rule}

\begin{equation}
\frac{d}{dx}c = 0
\end{equation}

\section{Power Rule}

For any real number $n$, the derivative of $x^n$ is:\index{power rule}

\begin{equation}
\frac{d}{dx}x^n = nx^{n-1}
\end{equation}

\section{Product Rule}

The derivative of the product of two functions is:\index{product rule}

\begin{equation}
\frac{d}{dx}(fg) = f'g + fg'
\end{equation}

where $f'$ and $g'$ denote the derivatives of $f$ and $g$,
respectively.

\section{Quotient Rule}

The derivative of the quotient of two functions is:\index{quotient rule}

\begin{equation}
\frac{d}{dx}\left(\frac{f}{g}\right) = \frac{f'g - fg'}{g^2}
\end{equation}

\section{Chain Rule}

The derivative of a composition of functions is:\index{chain rule}

\begin{equation}
\frac{d}{dx}(f(g(x))) = f'(g(x)) \cdot g'(x)
\end{equation}

\section{Practice}
\begin{Exercise}[label=deriv_rules1]
If $f$ is the function given, find $f'$. 
	\begin{enumerate}
		\item$f(x) = x\sin{x}$
		\item$f(x)=(x^3-\cos{x})^5$
		\item$f(x) =\sin^3{x}$
	\end{enumerate}
\end{Exercise}

\begin{Answer}[ref=deriv_rules1]
	\begin{enumerate}
		\item $\frac{dy}{dx} = \frac{d}{dx}[x\sin{x}] = x\frac{d}{dx}\sin{x} 
		+ \sin{x}\frac{d}{dx}x = x(-\cos{x}) + \sin{x}(1) = \sin{x} - 
		x\cos{x}$
    	\item By the chain rule, $f'(x) = 5(x^3 - \cos{x})^4 \cdot 
    	\frac{d}{dx}(x^3 - \cos{x}) = 5(x^3 - \cos{x})^4 \cdot (3x^2 + 
    	\sin{x})$
    	\item By the chain rule, $f'(x) = \frac{d}{d(\sin{x})}[\sin^3{x}] 
    	\times \frac{d}{dx}\sin{x} = 3\sin^2{x} \cdot \cos{x}$
	\end{enumerate}
    
\end{Answer}

\begin{Exercise}[label=deriv_rules2]
    Let $f(x)=7x-3+\ln{x}$. Find $f'(x)$ and $f'(1)$
\end{Exercise}
\begin{Answer}[ref=deriv_rules2]
    $f'(x) = \frac{d}{dx}(7x) - \frac{d}{dx}(3) + \frac{d}{dx}(\ln{x})
    = 7 - 0 + \frac{1}{x} = 7 - \frac{1}{x}$
    and
    $f'(1) = 7-\frac{1}{1} = 6$
\end{Answer}
	
\begin{Exercise}[label=deriv_rules3]
	[This question was originally presented as a multiple-choice, 
	no-calculator question on the 2012 AP Calculus BC exam.] The position 
	of a particle in the $xy$-plane is given by the parametric equations 
	$x(t) = t^3-3t^2$ and $y(t) = 12t-3t^2$. State a coordinate point 
	$(x,y)$ at which the particle is at rest. 
\end{Exercise}
	
\begin{Answer}[ref=deriv_rules3]
	The particle is at rest when $x'(t) = y'(t) = 0$. First, we find each
	of the derivatives:\\
	$$x'(t) = 3t^2 - 6t$$
	$$y'(t) = 12 - 6t$$
	We can solve $y' = 0$ for $t$ and find that the $y$-velocity is $0$ 
	when $t=2$. Substituting $t=2$ into our expression for $x'$, we find 
	$x'(2) = 3(2)^2 - 6(2) = 0$. Therefore, the particle is at rest when 
	$t=0$. To find the $xy$-coordinate, we substitute $t=2$ into $x(t)$ 
	and $y(t)$:\\
	$$x(2) = (2)^3 - 3(2)^2 = 8 - 12 = -4$$
	$$y(2) = 12(2) - 6(2) = 24 - 12 = 12$$
	Therefore, the particle is at rest when it is located at $(-4, 12)$. 
\end{Answer}

\begin{Exercise}
    [label=deriv_rules4]
    Let $f(x) = \sqrt{x^2-4}$ and $g(x) = 3x-2$. Find the derivative 
    of $f(g(x))$ at $x=3$.
\end{Exercise}
\begin{Answer}
    [ref=deriv_rules4]
    $f(g(x)) = \sqrt{(3x - 2)^2 - 4} = \sqrt{9x^2 - 12x}$ and 
    $\frac{d}{dx}f(g(x)) = \frac{18x - 12}{2\sqrt{9x^2 - 12x}}$. 
    Substituting $x = 3$, we find $f'(g(x)) = \frac{18(3) - 12}{2
    \sqrt{9(3)^2 - 12(3)}} = \frac{42}{2\sqrt{45}} = 
    \frac{21}{3\sqrt{5}} = \frac{7}{\sqrt{5}}$
\end{Answer}

\begin{Exercise}
    [label=deriv_rules5]
    The particle's position on the x-axis is given by $x(t) = 
    (t - a)(t - b)$, where $a$ and $b$ are constants and $a \neq b$. 
    At what time(s) is the particle at rest?
\end{Exercise}
\begin{Answer}
    [ref=deriv_rules5]
    First, recall that the velocity of a particle is the derivative of 
    its position function. Therefore, $v(t) = x'(t) = 
    \frac{d}{dt}[(t - a)(t - b)]$. Applying the Product Rule for 
    derivatives, we see that $v(t) = (t - a)(1) + (t - b)(1) = 2t - a 
    - b$. To find the time(s) when the particle is at rest, we set 
    $v(t) = 0$ and solve for $t$. 
    $$0 = 2t - a - b$$ 
    $$2t = a + b$$ 
    $$t = \frac{a + b}{2}$$
\end{Answer}

\begin{Exercise}
    [label = deriv_rules6]
    [This question was originally presented as a multiple-choice, 
    no-calculator question on the 2012 AP Calculus BC exam.] Let 
    $f(x) = \frac{x}{x + 2}$. At what values of $x$ does $f$ have the 
    property that the line tangent to $f$ has a slope of $\frac{1}{2}$?
\end{Exercise}

\begin{Answer}
    [ref=deriv_rules6]
    The question is asking when the derivative of $f$ is $\frac{1}{2}$.
     We will take the derivative and set it equal to $\frac{1}{2}$. 
     $$f'(x) = \frac{(x + 2)(1) - x(1)}{(x + 2)^2}=\frac{2}{(x + 2)^2}$$
    $$\frac{2}{(x + 2)^2}=\frac{1}{2}$$
    $$4=(x + 2)^2$$
    $$\pm 2 = x + 2$$
    $$x= 2 - 2 = 0 \text{ and } x = -2 - 2 = -4$$
\end{Answer}

\begin{Exercise}
    [label=deriv_rules7]
    For $t \geq 0$, the position of a particle moving along the x-axis 
    is given by $x(t) = \sin{t} - \cos{t}$. (a) When does the velocity 
    first equal $0$? (b) What is the acceleration at the time when the 
    velocity first equals $0$?
\end{Exercise}

\begin{Answer}
    [ref=deriv_rules7]
    (a) Let $t_0$ be the time at which the particle is first at rest. 
    The velocity of the particle is given by $v(t) = x'(t) = \cos{t} 
    + \sin{t}$. Setting $v(t) = 0$, we find:
    $$\cos{t} = -\sin{t}$$ 
    which is true for $ t = \frac{3\pi + 4n}{4}$, where $n$ is an 
    integer. Therefore, the first time the velocity is $0$ is $t_0 = 
    \frac{3\pi}{4}$. \\
    (b) To find the acceleration at $t=\frac{3\pi}{4}$, we take the 
    derivative of the velocity function to yield the acceleration 
    function. $$a(t) = v'(t) = -\sin{t} + \cos{t}$$. Substituting $t =
    \frac{3\pi}{4}$, we find the acceleration is $-\sin{\frac{3\pi}{4}}
     + \cos{\frac{3\pi}{4}} = \frac{-\sqrt{2}}{2} - \frac{\sqrt{2}}{2} 
     = -\sqrt{2}$
\end{Answer}

\begin{Exercise}
    [label=deriv_rules8]
    The graph of $y=e^(\tan{x}) - 2$ crosses the x-axis at one point 
    on the interval $[0, 1]$. What is the slope of the graph at this 
    point?
\end{Exercise}

\begin{Answer}
    [ref=deriv_rules8]
    First, we find the $x$ such that $y = 0$ 
    $$0 = e^{\tan{x}} - 2$$ 
    $$2 = e^{\tan{x}}$$ $$\ln{2} = \tan{x}$$ 
    $$x=\arctan{(\ln{2})} = \arctan{0.693} \approx 0.606$$
    Then, we find the slope of the function at $x=0.606$ by finding 
    $y'(0.606)$ 
    $$y'=e^{\tan{x}}(\sec{x})^2  = \frac{e^{\tan{x}}}{(\cos{x})^2}$$
    $$y'(0.606) = \frac{e^{\tan{0.606}}}{(\cos{0.606})^2} = 2.961$$
\end{Answer}

\begin{Exercise}
    [label=deriv_rules9]
    The function $f$ is defined by $f(x) = \sqrt{25-x^2}$ for $-5 \leq 
    x \leq 5$. \\
    (a) Find $f'(x)$. \\
    (b) Write an equation for the line tangent to the graph at $x = -3$. 
\end{Exercise}
\begin{Answer}
    [ref=deriv_rules9]
    (a) Apply the chain rule to find $f'(x)$ 
    $$f'(x) = \frac{1}{2\sqrt{25 - x^2}} \cdot(-2x) = \frac{-x}{
    \sqrt{25-x^2}}$$. \\
    (b) First, substitute $x=-3$ into $f'(x)$ 
    $$f'(-3) = \frac{-(-3)}{\sqrt{25 - (-3)^2}} = \frac{3}{\sqrt{16}} 
    = \frac{3}{4}$$
    This is the slope of the line. To complete an equation for the 
    tangent line, we need a point. We know the tangent line touches 
    $f(x)$ at $x = -3$, so the tangent line must pass through the 
    point $(-3, f(-3))$. $$f(-3) = \sqrt{25-(-3)^2} = 4$$ We use $m = 
    \frac{3}{4}$ and the coordinate point $(x_1, y_1) = (-3, 16)$ to 
    complete the equation $y - y_1 = m(x - x_1)$ 
    $$y-16 = \frac{3}{4}(x+3)$$
\end{Answer}

\begin{Exercise}
    [label=deriv_rules10]
    For $0 \leq t \leq 12$, a particle moves along the x-axis. The 
    velocity of the particle at a time $t$ is given by $v(t) = 
    \cos{\frac{\pi}{6}t}$. What is the acceleration of the particle 
    at time $t = 4$?
\end{Exercise}
\begin{Answer}
    [ref=deriv_rules10]
    $$a(t) = v'(t) = -\frac{\pi}{6}\sin{\frac{\pi}{6}t}$$
    $$a(4) = -\frac{\pi}{6}\sin{\frac{2\pi}{3}} = 
    -\frac{\pi}{6}\cdot\frac{\sqrt{3}}{2}=-\frac{\pi \sqrt{3}}{12}$$
\end{Answer}

\begin{Exercise}[label=deriv_rules11]
	[This question was originally presented as a multiple-choice, 
	calculator-allowed question on the 2012 AP Calculus BC exam.]
	Let $f$ and $g$ be the functions given by $f(x) = e^{x}$ and $g(x) = 
	x^4$. On what intervals is the rate of change of $f(x)$ greater than 
	the rate of change of $g(x)$?
\end{Exercise}

\begin{Answer}[ref=deriv_rules11]
	Recall that the rate of change of a function is given by the 
	derivative of that function. Therefore, we are looking for the
	interval(s) where $f'(x) > g'(x)$. First, we find each derivative:
	$$f'(x) = e^x$$
	$$g'(x) = 4x^3$$
	We are looking for $x$-values, such that $e^x > 4x^3$. This inequality 
	can be restated as $e^x - 4x^3 > 0$. Using a calculator, you should 
	find that $e^x - 4x^3 = 0$ when $x \approx 0.831$ and $x \approx 
	7.384$. We will check values on either side of and in the interval 
	$x\in(0.831, 7.384)$ to determine the sign value of $e^x - 4x^3$. 
	We know that when $x = 0$, $e^x - 4x^3 > 0$, when $x = 5$, $e^x - 
	4x^3 < 0$, and when $x=10$, $e^x - 4x^3 > 0$. Therefore, $f'(x)$ is 
	greater than $g'(x)$ on the open intervals $x \in (-\infty, 0.831) 
	\cup (7.384, \infty)$.
\end{Answer}

\section{Conclusion}

These rules form the basis for calculating derivatives in
calculus. Many more complex rules and techniques are built upon these
fundamental rules.
