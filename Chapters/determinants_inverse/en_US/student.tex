\chapter{Determinants and Inverse Matrices}\label{chap:det_inv}

\section{Determinants}
We have established that matrices can be thought of as transformations.

There is a measurement of how much a matrix, when acting as a transformation, effects the area or volume of a matrix. This measurement is referred to as the \newterm{\textbf{determinant}}.

% this was copied from the span chapter
FIXME 
Checking the independence of multitudes of vectors may take an immens amount of time. What if you had a list 
of 5, 10, or even 100 vectors? The determinant of a matrix is a scalar value 
that also indicates whether the columns of a matrix are linearly independent. So, 
if you put all your vectors together in a matrix and take the determinant of 
that matrix, the result will tell you if all the vectors are independent or 
not. For a 2D matrix, the determinant is the area of the parallelogram defined 
by the column vectors. For a 3D matrix, the determinant is the volume of the 
parallelepiped (a six-dimensional figure formed by six parallelograms, such as 
a cube).\footnote{Note that determinants can only be found for square, $n \times n$ matrices.} \index{determinant}

Let's plot the parallelogram for this matrix (see figure \ref{fig:twos}):
$$
\begin{bmatrix}
2 & 0  \\
0 & 2 
\end{bmatrix}
$$

\begin{figure}[H]
    \centering
    \begin{tikzpicture}
        \begin{axis}[xmin = 0, xmax = 3, ymin = 0, ymax = 3, xlabel = {$x$}, 
        ylabel = {$y$}, axis lines = center, clip = false]
        	\draw[red!30, fill = red!30, opacity = 0.4] (0,0) -- (2, 0) -- (2, 2) 
        	-- (0, 2) -- cycle;
            \draw[red, thick, -latex] (0,0) -- (2, 0) node[above, black, 
            xshift = -1cm] {$\begin{bmatrix}
                2\\
                0
            \end{bmatrix}$};
            \draw[red, thick, -latex] (0,0) -- (0,2) node[right, black, 
            yshift = -1cm] {$\begin{bmatrix}
                0\\
                2
            \end{bmatrix}$};
            \draw[red, dashed] (2, 0) -- (2, 2);
            \draw[red, dashed] (0, 2) -- (2, 2);
            \draw[black, latex-] (1.2, 0) arc(180:90:0.25);
            \draw[black, latex-] (0, 1) arc(-90:0:0.2);
        \end{axis}
    \end{tikzpicture}
    \caption{A parallelogram constructed from vectors $\left[2, 0 \right]$ and 
    $\left[0, 2 \right]$}
    \label{fig:twos}
\end{figure}
 
\begin{mdframed}[frametitle = {2 by 2 Determinant}, style = important]
The formal definition for calculating the determinant of a 2 by 2 matrix $A$ is:
$$det(A) = (a\cdot d)-(b\cdot c)$$
where
$$A = 
\begin{bmatrix}
a & b  \\
c & d 
\end{bmatrix}
$$
\end{mdframed}\index{determinant!2 by 2}

For the matrix plotted above, the determinant is $(2*2)-(0*0)$. You can also 
see that 4.0 is the area, base (2) times height (2).

You can use the determinant to see what happens to a shape when it goes 
through a linear transformation. Let's scale the 2 by 2 matrix by 4:
$$
\begin{bmatrix}
8 & 0  \\
0 & 8 
\end{bmatrix}
$$
Plot it (see figure \ref{fig:scaled}):

\begin{figure}[H]
    \centering
    \begin{tikzpicture}
        \begin{axis}[xmin = 0, xmax = 9, ymin = 0, ymax = 9, xlabel = {$x$}, 
        ylabel = {$y$}, axis lines = center, clip = false]
            \draw[blue!30, fill = blue!30, opacity = 0.4] (2, 0) -- (8, 0) -- 
            (8, 8) -- (0, 8) -- (0, 2) -- (2, 2) -- cycle;
            \draw[blue, thick, -latex] (0,0) -- (8, 0) node[above, black, 
            font = \scriptsize, xshift = -1.25cm] {$\begin{bmatrix}
                8\\
                0
            \end{bmatrix}$};
            \draw[black, latex-] (7.5, 0) arc (0:90:0.75);
            \draw[blue, thick, -latex] (0,0) -- (0,8) node[right, black, 
            font = \scriptsize, yshift = -0.5cm] {$\begin{bmatrix}
                0\\
                8
            \end{bmatrix}$};
            \draw[black, latex-] (0,6) arc (-90:0:0.5);
            \draw[blue, dashed] (8, 0) -- (8, 8);
            \draw[blue, dashed] (8, 8) -- (0,8);
            
            
        \end{axis}
    \end{tikzpicture}
    \caption{Scaling the matrix also scales the parallelogram.}
    \label{fig:scaled}
\end{figure}

Find the determinant using $(8*8)-(0*0) = 64$

You can see that scaling the matrix scaled the area by the scaling factor 
squared (see figure \ref{fig:both}).

\begin{figure}[H]
    \centering
    \begin{tikzpicture}
        \begin{axis}[xmin = 0, xmax = 9, ymin = 0, ymax = 9, xlabel = {$x$}, 
        ylabel = {$y$}, axis lines = center, clip = false]
            \draw[blue!30, fill = blue!30, opacity = 0.4] (2, 0) -- (8, 0) -- 
            (8, 8) -- (0, 8) -- (0, 2) -- (2, 2) -- cycle;
            \draw[blue, thick, -latex] (0,0) -- (8, 0) node[above, black, 
            font = \scriptsize, xshift = -1.25cm] {$\begin{bmatrix}
                8\\
                0
            \end{bmatrix}$};
            \draw[black, latex-] (7.5, 0) arc (0:90:0.75);
            \draw[blue, thick, -latex] (0,0) -- (0,8) node[right, black, 
            font = \scriptsize, yshift = -0.5cm] {$\begin{bmatrix}
                0\\
                8
            \end{bmatrix}$};
            \draw[black, latex-] (0,6) arc (-90:0:0.5);
            \draw[blue, dashed] (8, 0) -- (8, 8);
            \draw[blue, dashed] (8, 8) -- (0,8);
            
            \draw[red!30, fill = red!30, opacity = 0.4] (0,0) -- (2, 0) -- 
            (2, 2) -- (0, 2) -- cycle;
            \draw[red, thick, -latex] (0,0) -- (2, 0) node[above, black, 
            font = \scriptsize, xshift = 0.25cm] {$\begin{bmatrix}
                2\\
                0
            \end{bmatrix}$};
            \draw[black, latex-] (1.2, 0) arc(180:90:0.75);
            \draw[red, thick, -latex] (0,0) -- (0,2) node[right, black, 
            font = \scriptsize, yshift = 0.4cm] {$\begin{bmatrix}
                0\\
                2
            \end{bmatrix}$};
            \draw[black, latex-] (0, 1.5) arc(-90:0:0.5);
            \draw[red, dashed] (2, 0) -- (2, 2);
            \draw[red, dashed] (0, 2) -- (2, 2);
            
            
        \end{axis}
    \end{tikzpicture}
    \caption{Scaling a matrix by a constant $c$ increases the area of the 
    parallelogram by a factor of $c^2$.}
    \label{fig:both}
\end{figure}

We can show why this is true mathematically. Suppose we have a 2 by 2 matrix A:
$$A = \begin{bmatrix}
w & x\\
y & z
\end{bmatrix}$$

Then $det(A) = wz - xy$. We can scale this matrix by a constant, $c$:
$$cA = c \cdot \begin{bmatrix}
w & x\\
y & z
\end{bmatrix} = \begin{bmatrix}
cw & cx\\
cy & cz
\end{bmatrix}$$

And we can take the determinant:
$$det(cA) = det \left( \begin{bmatrix}
cw & cx\\
cy & cz
\end{bmatrix} \right)= cw(cz) - cx(cy) = c^2 (wz - xy) = c^2 \cdot det(A)$$

Therefore, scaling a 2 by 2 matrix by a factor changes the determinant by that 
factor squared. What about higher dimensions? If each side of a cube were 
scaled by a factor of $c$, then the volume of the cube would change by a 
factor of $c^3$ (feel free to confirm this yourself). And if a tesseract (a 
four-dimensional cube) had each side scaled by a factor of $c$, then the 
hypervolume (four-dimensional volume) would be scaled by a factor of $c^4$. Do 
you notice a pattern?

In fact, scaling an $n \times n$ matrix by a constant factor, $c$, changes the 
determinant of that $n \times n$ matrix by a factor of $c^n$. 

What happens if the columns of a matrix are not independent? Let's plot this 
matrix (see figure \ref{fig:zero_det}):
$$
\begin{bmatrix}
2 & 1  \\
4 & 2 
\end{bmatrix}
$$

\begin{figure}[H]
    \centering
    \begin{tikzpicture}
        \begin{axis}[xmin = 0, ymin = 0, xmax = 4, ymax = 5, xlabel = {$x$}, 
        ylabel = {$y$}, axis lines = center]
            \draw[red, thick, -latex] (0,0) -- (2, 4) node[below, black, 
            font = \scriptsize, yshift = -0.5cm] {$\begin{bmatrix}
                2\\
                4
            \end{bmatrix}$};
            \draw[-latex] (2, 2.9) arc (0:-120:0.5);
            \draw[blue, thick, -latex] (0,0) -- (1, 2) node[above, black, 
            font = \scriptsize, xshift = -0.75cm, yshift = -0.75cm] {$\begin{bmatrix}
                1\\
                2
            \end{bmatrix}$};
            \draw[-latex] (0.4, 1.6) arc (90:280:0.3);
        \end{axis}
    \end{tikzpicture}
    \caption{The vectors $\begin{bmatrix} 1\\2 \end{bmatrix}$ and 
    $\begin{bmatrix} 2\\4 \end{bmatrix}$ are co-linear, so there is no area 
    between them and the determinant of $\begin{bmatrix} 2 & 1\\ 4 & 2 
    \end{bmatrix}$ is zero.}
    \label{fig:zero_det}
\end{figure}

One vector overwrites the other. As you can see, the area is 0 because there 
is no space between the vectors. Therefore, the columns of the matrix are 
linearly dependent.

\begin{Exercise}[title = {Finding the Determinate}, label = geo_det]
Plot the parallelogram represented by the columns of the matrix. What is the 
area of this parallelogram?
\begin{enumerate}
\item $\begin{bmatrix}
1 &4\\
-3 & 1
\end{bmatrix}$
\item $\begin{bmatrix}
5 & -5 \\
5 & -1
\end{bmatrix}$
\item $\begin{bmatrix}
0 & -5 \\
-2 & 0 
\end{bmatrix}$
\end{enumerate}
\end{Exercise}

\begin{Answer}[ref = geo_det]
\begin{enumerate}
    \item Our two vectors from the columns of the matrix are $\left[1, -3 
    \right]$ and $\left[4, 1 \right]$. Plotting:
    
    \begin{tikzpicture}
        \begin{axis}[xmin = -1, xmax = 5, ymin = -4, ymax = 2, xlabel = {$x$}, 
        ylabel = {$y$}, axis lines = center]
            \draw[blue, thick, -latex] (0,0) -- (1, -3);
            \draw[blue, thick, -latex] (0,0) -- (4, 1);
            \draw[blue, dashed] (1, -3) -- (5, -2);
            \draw[blue, dashed] (5, -2) -- (4, 1);
        \end{axis}
    \end{tikzpicture}

    The area of this parallelogram is the same as the determinant of the matrix:
    $$det\left( \begin{bmatrix}
        1 & 4\\
        -3 &1
    \end{bmatrix} \right) = 1 \cdot 1 - \left( 4 \cdot -3 \right) = 1 + 12 = 13$$

    \item Our two vectors from the columns of the matrix are $\left[5, 5 
    \right]$ and $\left[-5, -1 \right]$. Plotting:

    \begin{tikzpicture}
        \begin{axis}[xmin = -6, xmax = 6, ymin = -2, ymax = 6, xlabel = {$x$}, 
        ylabel = {$y$}, axis lines = center]
            \draw[blue, thick, -latex] (0,0) -- (5, 5);
            \draw[blue, thick, -latex] (0,0) -- (-5, -1);
            \draw[blue, dashed] (-5, -1) -- (0, 4);
            \draw[blue, dashed] (0, 4) -- (5, 5);
        \end{axis}
    \end{tikzpicture}

    The area of this parallelogram is the same as the determinant of the 
    matrix:
    $$\det \left( \begin{bmatrix}
        5 & -5\\
        5 & -1
    \end{bmatrix} \right) = 5 \cdot -1 - \left( -5 \cdot 5 \right) = -5 + 25 = 
    20$$

    \item Our two vectors from the columns of the matrix are $\left[ 0, -2 
    \right]$ and $\left[ -5, 0 \right]$. Plotting:

    \begin{tikzpicture}
        \begin{axis}[xmin = -6, xmax = 1, ymin = -2, ymax = 1, xlabel = {$x$}, 
        ylabel = {$y$}, axis lines = center]
            \draw[blue, thick, -latex] (0,0) -- (0, -2);
            \draw[blue, thick, -latex] (0,0) -- (-5, 0);
            \draw[blue, dashed] (-5, 0) -- (-5, -2);
            \draw[blue, dashed] (-5, -2) -- (0, -2);
        \end{axis}
    \end{tikzpicture}

    This is a rectangle, and we can see the area is $5 \cdot 2 = 10$. However, 
    the determinant is:
    $$det \left( \begin{bmatrix}
        0 & -5\\
        -2 & 0
    \end{bmatrix} \right) = 0 \cdot 0 - \left( -5 \cdot -2 \right) = 0 - 10 = 
    -10$$

    We will discuss this unusual response in a future chapter. 
\end{enumerate}
\end{Answer}

Calculating the determinant for a 2 by 2 matrix is easy. For a larger matrix, 
finding the determinant is more complex and requires breaking down the matrix 
into smaller matrices until you reach the 2x2 form. The process is called 
expansion by minors. 
For example, 
\begin{mdframed}[frametitle = {$3\times 3$ Determinant}, style = important]
The determinant of a 3 by 3 matrix is found by
\[
%FIXME color a b  c red and d-i blue
\begin{bmatrix}
a & b & c  \\
d & e & f \\
g & h & i
\end{bmatrix}
=
a \cdot \begin{bmatrix}
e & f \\
h & i
\end{bmatrix}
- b \cdot \begin{bmatrix}
d & f \\
g & i
\end{bmatrix}
+ c \cdot \begin{bmatrix}
d & e \\
g & h
\end{bmatrix}
\]
\end{mdframed}
As you can see, this involves a recursive process of breaking a larger matrix into a smaller $2 \times 2$ matrix.

For our purposes, we simply want to first check to see if 
a matrix contains linearly independent rows and columns before using our 
Python code to solve. 

\section{Determinants in Python}
Modify your code so that is uses the $np.linalg.det()$ function. If the 
determinant is not zero, then you can call the $np.linalg.solve()$ function. 
Your code should look like this:
\begin{minted}{python}
if (np.linalg.det(D) != 0):
    j = np.linalg.solve(D,e)
    print(j)
else:
    print("Rows and columns are not independent.")
\end{minted}

How does this work below the hood? Let's also write a recursive python function that finds our determinant:

There are two base cases:
\begin{itemize}
    \item The matrix is of size $1 \times 1$
    \item The matrix is of size $2 \times 2$
\end{itemize}

And further sizes can be simplified into one of the base cases:

