\chapter{Conditional Probability}

Let's say there is a virus going around, and there is a vaccine for it
that requires two shots. You are working at a school, and you are
wondering how effective the vaccines are. Some students are
unvaccinated, some have had one shot, and some have had two shots. One
day, you test all 644 students to see who has the virus. You end up
with the following table:
% Diagram, 
\begin{tabular}{c | c c c}
  & $V_0$ & $V_1$ & $V_2$ \\
  \hline
  $T_{+}$ & 88 students & 36 students & 96 students \\
  $T_{-}$ & 92 students & 76 students & 256 students \\
\end{tabular}

Here is what each symbol means:

\begin{itemize}
\item $V_0$: student has had zero vaccination shots
\item $V_1$: student has had one vaccination shot
\item $V_2$: student has had both vaccination shots
\item $T_{+}$: student tested positive for the virus
\item $T_{-}$: student tested negative for the virus
\end {itemize}

So, for example, your data indicates that 76 students who
had only one of the two shots and tested negative for the virus.

Your principal has a few questions. The first is, ``If I put five
randomly chosen students in a study group together, what is the
probabiltiy that one of them has the virus?''

The first thing you might do is make a new table that shows what is
the probability of a randomly chosen student being in any particular
group. You just divide each entry by 644 (the total number of
students).

\begin{tabular}{c | c c c}
  & $V_0$ & $V_1$ & $V_2$ \\
  \hline
  $T_{+}$ & $p(V_0 \text{ AND } T_{+}) = 13.7\%$ & $p(V_1 \text{ AND } T_{+}) = 5.6\%$ & $p(V_2 \text{ AND } T_{+}) = 14.9\%$\\
  $T_{-}$ & $p(V_0 \text{ AND } T_{-}) = 14.3\%$ & $p(V_1 \text{ AND } T_{-}) = 11.8\%$ & $p(V_2 \text{ AND } T_{-}) = 39.8\%$
\end{tabular}

(In this table, we expressed the number as a percentage with a decimal
point --- you had to round off the numbers. If you wanted exact answers, you
would have to keep each as a fraction: 36 students represents
$\frac{9}{161}$ of the student body.)

\section{Marginalization}

Now we can sum across the columns and rows.

\begin{tabular}{c | c c c | c}
  & $V_0$ & $V_1$ & $V_2$ & sum \\
  \hline
  $T_{+}$ & 0.137 & 0.056 & 0.149 & $p(T_{+}) = 0.342$\\
  $T_{-}$ & 0.143 & 0.118 & 0.398 & $p(T_{+}) = 0.547$\\
  \hline
  sum & $p(V_0) = 0.280$ & $p(V_1) = 0.174$ & $p(V_2) = 0.547$ & 
\end{tabular}

If a child is chosen randomly from the entire student body, there is
a 34.2\% that the student has tested positive for the virus, and there is
17.4\% chance that the student has one shot of the vaccine.

This summing of the probabilities across one dimension is known as
\textit{marginalizing}. Marginalization is just summing across all the
variables that you don't care about. If you don't care who has the virus,
just the probability that a student has not received even one shot of
the vaccine, you can simply marginalize all the vaccine statuses.

To answer the principal's question, the easy thing to do is find the
answer of the opposite: ``if I put five randomly chosen students in a
study group together, what is the probabiltiy that \textit{none} of
them has tested positive for the virus?''

The chance that a randomly chosen student doesn't have the virus
($p(T_{-}$) is 54.7\%.  This means the chance that 5 randomly chosen
students don't have the virus is $0.547 \times 0.547 \times 0.547
\times 0.547 \times 0.547 = 0.0489$ Thus, the probability of the
opposite is $1.0 - 0.0489 = 0.951$

The answer, then, is ``If you put 5 kids in a study group together,
there is a 95.1 \% probability that at least one of them has the
virus.''
% KA: https://www.khanacademy.org/math/ap-statistics/probability-ap/stats-conditional-probability/v/testing-independence-from-experimental-data

\section{Conditional Probability}

Now the principal asks you, ``What if I make a group of five kids who
have had both shots of the vaccine? What are the odds that one of them
has tested positive for the virus?''

This involves the idea of \textit{Conditional probability}.  You want
to know the odds that a student doesn't have the virus, given that
the student has had both shots of the vaccine.

There is a mathematical notation for this:

$$p(T_{-} | V_{2})$$

That is the probability that a student who has had both vaccination
shots will test negative for the virus.

How would you calculate this? You would count all the students who had
a positive test \textit{and} both vaccination shots, which you would
divide by the total number of students who had both vaccination shots.

$$p(T_{-} | V_{2}) = \frac{256}{96 + 256} = \frac{8}{11} \approx 72.7\%$$

If we are working from the probabilities, you can get the same result
this way. Divide the probability that a randomly chosen student had a
positive test \textit{and} both vaccination shots by the probabiltiy
that a student had both vaccination shots:

$$p(T_{-} | V_{2}) = \frac{p(T_{-} \text{ AND } V_{2})}{p(T_{-})} =  \frac{0.398}{0.547} \approx 72.7\%$$

Notice that this is different from $p( V_{2} | T_{-})$, which is the
probability that a student has had both vaccinations, given they
tested negative for the virus.

Back to the principal's question: ``If you have five students who have had
both vaccinations, what is the probability that all of them tested
negative for the virus?'' The probability that one student is virus-free
is $\frac{8}{11}$, so the probability that five students are virus-free
is $\frac{8}{11}^5 \approx 0.203$.  So, there is a $79.6\%$ chance
that at least one of the five has the virus.
% KA: https://www.khanacademy.org/math/statistics-probability/probability-library/conditional-probability-independence/v/calculating-conditional-probability

\section{Chain Rule for Probability}

You just used this equality: For any events $A$ and $B$

$$p(A | B) = \frac{p(A \text{ AND } B)}{p(B)}$$

This is more commonly written like this:

$$p(A \text{ AND } B) = \frac{p(A | B)}{p(B)}$$

This is an abstract way of writing the idea, but the idea itself
is pretty intuitive. The probability that you are going to buy a ticket
and win the lottery is equal to the probability that you buy a ticket
times the probability that you win, given that you have bought a ticket.
(Here $A$ is ``win the lottery'' and $B$ is ``buy a ticket''.)

This is known as \textit{The Chain Rule of Probability}.  We can
chain together as many events as we want: The probability that you are
going to die in the car that you bought with your winnings from the
lottery ticket you bought is:

$$p(W \text{ AND } X \text{ AND } Y \text{ AND } Z) = p( W | X \text{ AND } Y \text{ AND } Z) p( X |  Y \text{ AND } Z) p (Y | Z) p(Z)$$

where

\begin{itemize}
\item $W =$ Dying in car accident
\item $X =$ Buying a car with lottery winnings
\item $Y =$ Winning the lottery
\item $Z =$ Buying a lottery ticket
\end{itemize}

In English, the equation says:

``The probability that you will die in a car accident, buy a car with
lottery winnings, win the lottery, and buy a lottery ticket is equal
to the probability that you buy a lottery ticket times the probability
that you win the lottery (given that you have bought a ticket) times
the probability that buy a car with those lottery winnings (given that
bought a ticket and won) times the probability that you crash that car
(given that you have bought the car, won the lottery, and bought a
ticket).''

% KA: https://www.khanacademy.org/math/ap-calculus-ab/ab-differentiation-2-new/ab-3-1a/v/chain-rule-introduction



  
