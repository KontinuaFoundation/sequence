\chapter{Multiplying Polynomials}

Watch Khan Academy's \textbf{Multiplying monomials} at \url{https://www.youtube.com/watch?v=7EplN4Ch98Q}.

To review, when you multiply two monomials, you take the product of
their coefficients and the sum of their degrees:
\begin{equation*}
  (2x^6)(5x^3) = (2)(5)(x^6)(x^3) = 10x^9
\end{equation*}
If you have a product of more than two monomials, multiply \emph{all}
the coefficients and sum \emph{all} the exponents:
\begin{equation*}
  (3x^2)(2x^3)(4x) = (3)(2)(4)(x^2)(x^3)(x^1) = 24x^6
\end{equation*}\index{multiplication!polynomials}

\begin{Exercise}[title={Multiplying monomials}, label=multmonomials]
Multiply these monomials:
  \Question $(3x^2)(5x^3)$
\vspace{20mm}
  \Question $(2x)(4x^9)$
\vspace{20mm}
  \Question $(-5.5x^2)(2x^3)$
\vspace{20mm}
  \Question $(\pi)(-2x^5)$
\vspace{20mm}
  \Question $(2x)(3x^2)(5x^7)$
\vspace{20mm}
\end{Exercise}
\begin{Answer}[ref=multmonomials]
  $(3x^2)(5x^3) = 15x^5$
  
  $(2x)(4x^9) = 8x^{10}$
  
  $(-5.5x^2)(2x^3) = -11x^5$

  $(\pi)(-2x^5) = -2\pi x^5$
  
  $(2x)(3x^2)(5x^7) = 30x^{10}$
\end{Answer}

\section{Multiplying a monomial and a polynomial}

Watch Khan Academy's \textbf{Multiplying monomials by polynomials} at \url{https://youtu.be/pD2-H15ucNE}.

When multiplying a monomial and a polynomial, you use the the distributive property. After that, it is just multiplying several pairs of monomials:
\begin{multline*}
  (3x^2)(4x^3 - 2x^2 + 3x - 7) \\
  = (3x^2)(4x^3) + (3x^2)(-2x^2) + (3x^2)(3x) + (3x^2)(-7) \\
  = 12x^5 - 6x^4 + 9x^3 -21x^2
\end{multline*}

\begin{Exercise}[title={Multiplying a monomial and a polynomial}, label=multmonopoly]
Multiply these monomials:
\Question $(3x^2)(5x^3 - 2x + 3)$
\vspace{20mm}
\Question $(2x)(4x^9 - 1)$
\vspace{20mm}
\Question $(-5.5x^2)(2x^3 + 4x^2 + 6)$
\vspace{20mm}
\Question $(\pi)(-2x^5 + 3x^4 + x)$
\vspace{20mm}
\Question $(2x)(3x^2)(5x^7 + 2x)$
\end{Exercise}
\begin{Answer}[ref=multmonopoly]
  $(3x^2)(5x^3 - 2x + 3) = 15x^6 - 6x^3 + 6x^2$

  $(2x)(4x^9 - 1) = 8x^{10} - 2x$

  $(-5.5x^2)(2x^3 + 4x^2 + 6) = 11x^5 - 22x^4 + 33x^2$

  $(\pi)(-2x^5 + 3x^4 + x) = -2\pi x^5 + 3\pi x^4 + \pi x$

  $(2x)(3x^2)(5x^7 + 2x) = 30x^{10} + 12x^4$
\end{Answer}

\section{Multiplying polynomials}

Watch Khan Academy's \textbf{Multiplying binomials by polynomials} video at \url{https://youtu.be/D6mivA_8L8U}

When you are multiplying two polynomials, you will use the
distributive property several times to make it one long
polynomial. You will then combine the terms with the same degree. For
example,
\begin{multline*}
  (2x^2 - 3)(5x^2 + 2x - 7) \\
  =   (2x^2)(5x^2 + 2x - 7) + (-3)(5x^2 + 2x - 7) \\
  =   (2x^2)(5x^2) + (2x^2)(2x) + (2x^2)(-7) + (-3)(5x^2) + (-3)(2x) + (-3)(-7) \\
  =   10^4 + 4x^3 + -14x^2 + -15x^2 + -6x + 21
  =   10^4 + 4x^3 + -29x^2 + -6x + 21
\end{multline*}

One common form that you will see is multiplying two binomials together:
\begin{multline*}
(2x + 7)(5x + 3) = (2x)(5x + 3) + (7)(5x+3) = (2x)(5x) + (7)(5x) + (2x)(3) + (7)(3)
\end{multline*}
Notice the product has become the sum of four parts: the firsts, the
inners, the outers, and the lasts. People sometimes use the mnemonic
FOIL to remember this pattern, but there is a general rule that works
for all product of polynomials, not just binomials: Every
term in the first will be multiplied by every term in the second, and
then just add them together.

So, for example, if you have a polynomial $s$ with three terms and you
multiply it by a polynomial $t$ with five terms, you will get a sum of
15 terms --- each term is a product of two monomials, one from $s$ and
one from $t$.  (Of course, several of those terms might have the same
degree, so they will be combined together when you simplify. Thus you
typically end up with a polynomial with less than 15 terms.)

Using this rule, here is how you would multiply $2x^2 - 3x + 1$ and
$5x^2 + 2x - 7$:
\begin{multline*}
  (2x^2 - 3x + 1)(5x^2 + 2x - 7)  = \begin{matrix}
  (2x^2)(5x^2)& + &(2x^2)(2x)& + &(2x^2)(-7)& + \\
  (-3x)(5x^2)& + &(-3x)(2x)& + &(-3x)(-7)& + \\
    (1)(5x^2)& + &(1)(2x)& + &(1)(-7)& 
  \end{matrix} \\
  = 10x^4 + 4x^3 + (-14)x^2 + (-15)x^3 + (-6)x^2 + 21x +5x^2 + 2x + (-7) \\
  = 10x^4 + (4 - 15)x^3 + (-14 - 6 + 5)x^2  + (21 + 2)x + (-7) \\
  = 10x^4 - 11x^3 - 15x^2 + 23x - 7
\end{multline*}
Note that the product (before combining terms with the same degree) has
$3 \times 3 = 9$ terms --- every possible combination of a term from the
first polynomial and a term from the second polynomial.

One common source of error is losing track of the negative
signs. You will need to be really careful. We have found that it
helps to use + between all terms, and use negative coefficients to
express subtraction. For example, if the problem says $4x^2 - 5x - 3$,
you should work with that as $4x^2 + (-5)x + (-3)$

\begin{Exercise}[title={Multiplying polynomials}, label=multpolys]
  Multiply the following pairs of polynomials:
  \Question{$2x + 1$ and $3x - 2$}
  \vspace{15mm}
  \Question{$-3x^2 + 5$ and $4x -2$}
  \vspace{15mm}
  \Question{$-2x - 1$ and $-3x - \pi$}
  \vspace{15mm}
  \Question{$-2x^5 + 5x$ and $3x^5 + 2x$}
  \vspace{15mm}

\end{Exercise}
\begin{Answer}[ref=multpolys]
  $(2x + 1)(3x - 2) = 6x^2 - x - 2$

  $(-3x^2 + 5)(4x - 2) = -12x^3 + 6x^2 + 20x - 10$

  $(-2x - 1)(-3x - \pi) = 6x^2 + (4 + 2\pi)x + \pi$ 

  $(-2x^5 + 5x)(3x^5 + 2x) = -6x^{10} + 12x^6 + 10x^2$
\end{Answer}

\begin{Exercise}[title={Observations}, label=obsmultpoly]
  Let's say you have two polynomials, $p_1$ and $p_2$.  $p_1$ has degree
  23.  $p_2$ has degree 12.  What is the degree of their product?
\end{Exercise}
\begin{Answer}[ref=obsmultpoly]
  The degree of the product is determined by the term that is the
  product of the highest degree term in $p_1$ and the highest degree
  term in $p_2$. Thus, the product of a degree 23 polynomial and a
  degree 12 polynomial has degree 35.
\end{Answer}
  
