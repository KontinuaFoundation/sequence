\chapter{Subspaces}\label{chap:subspaces}
Recall that, in Chapter~\ref{chap:syseqma}, we established that all linear systems can be represented in matrix form as $A\vec{x}=\vec{b}$. In this chapter, we will explore the concept of subspaces, which are fundamental to understanding the structure of solutions to linear systems.

Quickly, let's establish some vocabulary. 
The zero vector, denoted as $\vec{0}$, is the vector where all components are zero. It will be a column vector of appropriate size $n\times 1$.

If $A\vec{x} = \vec{0}$, where $\vec{b} = \vec{0}$, then the system is called \textbf{homogeneous}. If $A\vec{x} = \vec{b}$ where $\vec{b} \neq \vec{0}$, then the system is called \textbf{non-homogeneous}.

\section{What is a Subspace?}
A subspace $V$ of $\mathbb{R}^n$ is a subset $V$ of $\mathbb{R}^n$ satisfying the following properties:
\begin{itemize}
    \item The zero vector $\vec{0}$ is in the subspace.
    \item If $\vec{u}$ and $\vec{v}$ are in the subspace, then their sum $\vec{u} + \vec{v}$ is also in the subspace.
    \item If $\vec{u}$ is in the subspace and $c$ is a scalar, then the scalar multiple $c\vec{u}$ is also in the subspace.
\end{itemize}

In short, a subspace is exactly the set of all linear combinations of some collection of vectors. Additionally, every subpace is a span. 
\begin{mdframed}[frametitle = {Subspace Span}, style = important]
If $V$ is a subspace of $\mathbb{R}^n$, then there exists a set of vectors
$$\{\vec{v}_1,\vec{v}_2,\ldots,\vec{v}_k\} \subseteq \mathbb{R}^n$$
such that
$$V = \text{span}\{\vec{v}_1,\vec{v}_2,\ldots,\vec{v}_k\}$$
That is, every vector in $V$ can be written as a linear combination of vectors in $V$.
\end{mdframed}

Because $V$ is closed under:

\begin{itemize}
    \item vector addition, and
    \item scalar multiplication,
\end{itemize}

any linear combination of vectors in $V$ must also lie in $V$.

So if $\vec{u},\vec{v} \in V$ and $a,b \in \mathbb{R}$, then
$$a\vec{u} + b\vec{v} \in V$$

The converse is also true: any span is a subspace.
$$\text{span}(V) \text{ is a subspace of } \mathbb{R}^n$$

For example, vectors $\vec{v_{1}}= [1, 0]$ and $\vec{v_{2}}= [0, 1]$ span all of $\mathbb{R}^2$, since they can be scaled to fit all of $\mathbb{R}^2$.    
\begin{itemize}
    \item the span $v_1$ is a line through $\vec{0}$
    \item the span $v_{1},v_{2}$ is a line or plane through $\vec{0}$
    \item the span of $v_{1}, v_{2}, v_{3}$ is a line or plane through $\vec{0}$, or all of $\mathbb{R}^3$
\end{itemize}


% https://textbooks.math.gatech.edu/ila/1553/subspaces.html

FIXME maybe explain basis and dimension here
\section{Nullspace}

We now examine a fundamental example of a subspace: the \textbf{nullspace} of a matrix. \index{nullspace}
Recall that a subspace is a subset of $\mathbb{R}^n$ that is closed under vector addition and scalar multiplication and contains the zero vector.

\medskip

The nullspace of a matrix $A$, denoted $\text{Null}(A)$, is the set of all vectors $\vec{x}$ such that
\[
A\vec{x} = \vec{0}.
\]
That is, the nullspace is precisely the solution set of the homogeneous system $A\vec{x}=\vec{0}$.

\begin{mdframed}[frametitle = {Nullspace}, style = important]
The nullspace of a matrix $A$ is defined as
\[
\text{Null}(A) = \{\vec{x} \in \mathbb{R}^n : A\vec{x} = \vec{0}\}.
\]
\end{mdframed}

Because the equation $A\vec{x}=\vec{0}$ is homogeneous, its solution set always contains the zero vector. Moreover, if $\vec{x}_1$ and $\vec{x}_2$ are solutions, then any linear combination of the form
\[
a\vec{x}_1 + b\vec{x}_2
\]
is also a solution. For this reason, the nullspace of a matrix is always a subspace of $\mathbb{R}^n$.

\subsection{Linear Combinations and Span}

Recall that a \textbf{linear combination} of vectors $\vec{v}_1,\vec{v}_2,\ldots,\vec{v}_n$ is any vector of the form
\[
a_1\vec{v}_1 + a_2\vec{v}_2 + \cdots + a_n\vec{v}_n,
\]
where $a_1,a_2,\ldots,a_n \in \mathbb{R}$ are scalars.

The set of all possible linear combinations of a collection of vectors is called their \textbf{span}. If a subspace can be written as the span of one or more vectors, those vectors describe all possible directions within the subspace.

\subsubsection*{Example}

Consider the matrix
\[
A =
\begin{bmatrix}
1 & 2 \\
2 & 4
\end{bmatrix}.
\]

To find the nullspace of $A$, we solve the homogeneous system $A\vec{x}=\vec{0}$:
\[
\begin{bmatrix}
1 & 2 \\
2 & 4
\end{bmatrix}
\begin{bmatrix}
x_1 \\ x_2
\end{bmatrix}
=
\begin{bmatrix}
0 \\ 0
\end{bmatrix}.
\]

This corresponds to the system of equations
\[
x_1 + 2x_2 = 0,
\qquad
2x_1 + 4x_2 = 0.
\]

Solving for $x_1$ in terms of $x_2$ gives
\[
x_1 = -2x_2.
\]

Notice that $x_1$ can be written in terms of $x_2$. This implies \emph{linear dependence}, an important idea we will talk about later. Thus, every vector in the nullspace can be written in the form
\[
\vec{x} = x_2
\begin{bmatrix}
-2 \\ 1
\end{bmatrix}.
\]

Therefore, the nullspace of $A$ is
\[
\text{Null}(A) = \text{span}\left\{
\begin{bmatrix}
-2 \\ 1
\end{bmatrix}
\right\}.
\]

Geometrically, this nullspace is a line through the origin in $\mathbb{R}^2$, which is a one-dimensional subspace of $\mathbb{R}^2$.
