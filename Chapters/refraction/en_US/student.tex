\chapter{Refraction}


Refraction of light is the phenomenon where light changes its direction when it passes from one medium to another. The change in direction is due to a change in the speed of light as it moves from one medium to another. 

This phenomenon is explained by Snell's law, which states:

\begin{equation}
n_1 \cdot \sin(\theta_1) = n_2 \cdot \sin(\theta_2)
\end{equation}

where:
\begin{itemize}
\item $n_1$ and $n_2$ are the indices of refraction for the first and second media, respectively. The index of refraction is the ratio of the speed of light in a vacuum to the speed of light in the medium. It is a dimensionless quantity.
\item $\theta_1$ and $\theta_2$ are the angles of incidence and refraction, respectively. These angles are measured from the normal (perpendicular line) to the surface at the point where light hits the boundary.
\end{itemize}

The angle of incidence ($\theta_1$) is the angle between the incident ray and the normal to the interface at the point of incidence. Similarly, the angle of refraction ($\theta_2$) is the angle between the refracted ray and the normal.

When light travels from a medium with a lower refractive index to a medium with a higher refractive index, it bends towards the normal. Conversely, when light travels from a medium with a higher refractive index to one with a lower refractive index, it bends away from the normal.
