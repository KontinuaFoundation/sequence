\chapter{Introduction to the Kontinua Sequence}

This book will start you on the long and difficult trek to becoming a modern
problem solver. Along the path, you will learn how to use the tools of
math, computers, and science.

Why should you bother? There are big problems in this world that will
require expert problem solvers. Those people will make the world a
better place while enjoying interesting and lucrative careers. We are
talking about engineers, scientists, doctors, computer programmers,
architects, actuaries, and mathematicians. Right now, those occupations represent
about 6\% of all the jobs in the United States. Soon,
that number is expected to rise above 10\%.  On average, people in
that 10\% of the population are expected to have salaries twice that
of their non-technical counterparts.\index{career}

Solving problems is difficult. At some point on this journey, you will
see people who are better at solving problems than you are. You, like
every other person who has gone on this journey, will think ``I have
worked so hard on this, but that person is better at it than
I am. I should quit.'' Don't.\index{quitting}

First, solving problems is like a muscle. The more you do, the better
you get at it.  It is OK to say ``I am not good at this yet.'' That
just means you need more practice.

Second, you don't need to be the best in the world. 10 million people
your age can be better at solving problems than you, \textit{and you
  can still be in the top 10\% of the world}. If you complete this
journey, there will be problems for you to solve and a job where your
problem-solving skills will be appreciated.

\emph{Where do we start?}

The famous physicist Richard Feynman once asked this question: ``If,
in some cataclysm, all of scientific knowledge were to be destroyed,
and only one sentence was passed on to the next generation of
creatures, what statement would contain the most information in the
fewest words?''

His answer was ``All things are made of atoms—little particles that move around in
perpetual motion, attracting each other when they are a little
distance apart, but repelling upon being squeezed into one another.''

\emph{That} seems like a good place to start.
