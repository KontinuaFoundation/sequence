\chapter{Orbits}

Gravity is the force that attracts two bodies toward each other. It is responsible for the behavior of orbital motion.
Gravity is described by Newton's Law of Universal Gravitation, which states that every point mass attracts every other 
point mass with a force that is directly proportional to the product of their masses and inversely proportional to the 
square of the distance between them. The formula for gravitational force is:

\[
F_g = \frac{G m_1 m_2}{r^2}
\]

Where \(F_g\) is the gravitational force between two objects, \(G\) is the 
gravitational constant, \(m_1\) and \(m_2\) are the masses of the objects, 
and \(r\) is the distance between the centers of the two objects. \(G\) is 
a constant with a value of approximately \(6.674 \times 10^{-11} \, \text{N m}^2/\text{kg}^2\).

If we think about acceleration due to gravity where one mass is significantly larger than the other, we can rewrite the equation as:

\[
A_g = \frac{G m_1 m_2}{r^2 m_2} = \frac{G m_1}{r^2}
\]

Where \(m_1\) is the mass of the more massive object and \(m_2\) is the mass of the significantly less massive object. 
This cancelling of the \(m_2\) mass is why the acceleration due to gravity is independent of the mass of the object in free fall.

A satellite stays in orbit around the planet because the pull of the
planet's gravity causes it to accelerate toward the center of the
planet. 

\begin{figure}[H]
    \centering
    \includegraphics[width=.75\textwidth]{satellite.png}
    \caption{A satellite's centripetal force is gravity.}
    \label{fig:satellite}
\end{figure}


The satellite must be moving at a very particular speed to keep a
constant distance from the planet --- to travel in a circular orbit.
If it is moving too slowly, it will get closer to the planet.  If it
is going too fast, it will get farther from the planet.
\begin{figure}[H]
    \centering
    \includegraphics[width=.75\textwidth]{orbitSpeeds.png}
    \caption{A diagram showing the required speeds for entering orbit.}
    \label{fig:orbitSpeeds}
\end{figure}

The radius of a satellite in a low orbit is typically about 2 million meters above the
ground. At that distance, the acceleration due to gravity is more like
$6.8 m/s^2$, instead of the $9.8 m/s^2$ that we experience on the
surface of the planet.

How fast does the satellite need to be moving in a circle with a
radius of 8.37 million meters to have an acceleration of $6.8 m/s^2$? Real fast.

Recall that the acceleration vector is

$$a = \frac{v^2}{r}$$

Thus the velocity $v$ needs to be:

$$v = \sqrt{a r} = \sqrt{6.8(8.37 x 10^6)} = 7,544 \text{ m/s}$$

(That's 16,875 miles per hour.)

When a satellite falls out of orbit, it enters the atmosphere at that
7,544 m/s.  The air rushing by generates so much friction that the
satellite gets very, very hot, and usually disintegrates.

\section{Astronauts are \emph{not} weightless}

Some people see astronauts floating inside an orbiting spacecraft and
think there is no gravity: that the astronauts are so far away that
the gravity of the planet doesn't affect them. This is incorrect.  The
gravity might be slightly less (maybe 6 newtons per kg instead of 9.8
newtons per kg), but the weightlessness they experience is because they
and the spacecraft are in free fall.  They are moving so fast (in
a direction perpendicular to gravity) that they don't collide with the
planet.

% \includegraphics[width=1\textwidth]{satellite.png}


\begin{Exercise}[title={Mars Orbit}, label=mars_orbit]
  
  The radius of Mars is 3.39 million meters. The atmosphere goes up
  another 11 km.  Let's say you want to put a satellite in a circular
  orbit around Mars with a radius of 3.4 million meters.

  The acceleration due to gravity on the surface of Mars is $3.721
  m/s^2$. We can safely assume that it is approximately the same 11 km
  above the surface.

  How fast does the satellite need to be traveling in its orbit?  How
  long will each orbit take?

\end{Exercise}
\begin{Answer}[ref=circular]
  $$v = \sqrt{3.721(3.4 \times 10^6)} = 3,557\text{ m/s}$$

  The circular orbit is $2\pi(3.4 \times 10^6) = 21.4 \times 10^6$ meters in circumference.

  The period of the orbit is $(21.4 \times 10^6)/3,557 \approx 6,000$ seconds.
\end{Answer}

\section{Geosynchronous Orbits}
\index{geosynchronous orbits}
The planet Earth rotates once a day.  Satellites in low orbits circle
the Earth many times a day. Satellites in very high orbits circle
less than once per day. There is a radius at which a satellite orbits
exactly once per day.  Satellites at this radius are known as
``geosynchronous'' or ``geostationary'', because they are always
directly over a place on the planet.
\begin{figure}[H]
    \centering
    \includegraphics[width=.75\textwidth]{geoSync.png}
    \caption{A satellite in geosynchronous orbit.}
    \label{fig:geoSync}
\end{figure}

The radius of a circular geosynchronous orbit is 42.164 million
meters. (About 36 km above the surface of the earth.)

A geosynchronous satellite travels at a speed of 3,070 m/s.

Geosynchronous satellites are used for the Global Positioning
Satellite system, weather monitoring system, and communications
system.
FIXME: Add text for escape velocity
\begin{figure}[H]
    \centering
    \includegraphics[width=.75\textwidth]{escape.png}
    \caption{A satellite can reach a speed at which it "escapes" earth's orbit (centripetal force).}
    \label{fig:escape}
\end{figure}
\section{Kepler's Laws} 
% required by AP standards
There are three laws that describe the motion of planets and their ellipical properties.
\begin{enumerate}
  \item Every planet moves along an ellipse with the sun at one focus. FIXME diagram here
  \item As a planet moves around its orbit, it sweeps out equal areas in equal times. This means that a planet moves faster when it is closer to the sun and slower when it is farther from the sun. FIXME diagram here
  \item If $T$ is the period of a planet's orbit, and $a$ is the length of the semi-major axis of the ellipse, then $\frac{T^2}{a^3}$ is the same for all planets orbiting the same star. 
\end{enumerate}

%FIXME this could be expanded with exercises
\section{Escape velocity}
\index{escape velocity}
The escape velocity is the minimum speed needed for an object to ``break free'' from the gravitational attraction of a massive body or planet, without further propulsion. Gravity will never pull it back, because we assume no further forces act on the object. This is most commonly associated with a spaceship or satellite escaping from a planet's gravitational pull.

We need to understand that escape velocity relies on energy, rather than a force or acceleration. An object must have enough kinetic energy to overcome the gravitational potential energy pulling it back toward the planet.
To escape from an orbit, the object's mechanical energy (the sum of its kinetic and potential energy) must be greater than or equal to zero.

We can derive its initial energy as: $E = K + U = \frac{1}{2}mv^2 - \frac{GMm}{R}$, where $m$ is the mass of the object, $M$ is the mass of the planet, $R$ is the distance from the center of the planet to the object, and $G$ is the gravitational constant.

After its escape, the object will be infinitely far away from the planet, and its potential energy will be zero. If we assume it just barely escapes, its kinetic energy will also be zero. Therefore, we set the initial energy equal to zero:
\begin{align}
  \frac{1}{2}mv^2 - \frac{GMm}{R} = 0 \\
  v_{\text{esc}} = \sqrt{\frac{2GM}{R}}
  \label{eq:escveleq}
\end{align}


Notice that the mass of the object $m$ cancels out, meaning that the escape velocity is independent of the mass of the escaping object. It only depends on the mass of the planet and the distance from its center.

Further, \textbf{notice that the escape velocity is $\sqrt{2}$ times the orbital velocity}, $v_{orb} = \sqrt{GM/R}$ at the same distance from the planet's center. This means that to escape from a circular orbit, an object must increase its speed by about 41.4\%.
For Earth, the escape velocity at the surface is approximately 11.2 km/s (about 25,000 mph). This means that any object, regardless of its mass, must reach this speed to escape Earth's gravitational pull without further propulsion.

\begin{Exercise}[title={Escape Velocity from Mars}, label=escape_mars]
  
  The radius of Mars is $3.4 \times 10^6$ meters. The mass of Mars is
  $6.42 \times 10^{23}$ kg. A spaceship is sitting on the surface of Mars.


  \begin{enumerate}
    \item Find the escape velocity from the surface of Mars.
    \item Suppose the spaceship is in a circular orbit 11 km above the surface of Mars. What is the escape velocity from that orbit?
  \end{enumerate}
\end{Exercise}
\begin{Answer}[ref=escape_mars]
Using equation~\ref{eq:escveleq}:
\begin{enumerate}
  \item 
  \begin{align*}
      v_{\text{esc}} &= \sqrt{\frac{2GM}{R}} \\
      &= \sqrt{\frac{2(6.674 \times 10^{-11} \text{ N m}^2/\text{kg}^2)(6.42 \times 10^{23} \text{ kg})}{3.4 \times 10^6 \text{ m}}} \\
      &\approx 5,027 \text{ m/s} 
\end{align*}
\item 
Now the radius is $3.4 \times 10^6 + 11,000 = 3.411 \times 10^6$ meters, because the satellite is 11 km above the surface. Recalculating:
\begin{align*}
    v_{\text{esc}} &= \sqrt{\frac{2GM}{R}} \\
    &= \sqrt{\frac{2(6.674 \times 10^{-11} \text{ N m}^2/\text{kg}^2)(6.42 \times 10^{23} \text{ kg})}{3.411 \times 10^6 \text{ m}}} \\
    &\approx 5,013 \text{ m/s} 
\end{align*}
\end{enumerate}
\end{Answer}

Note that giving an object an extra height $h$ causes the escape velocity to decrease (as escape velocity is inversely proportional to the square root of the radius plus height). This is due to gravitation potential energy being less negative at larger radii. However, the effect is small compared to the overall radius of the planet. Problems may give you a radius plus height, or just the radius of the planet, or a total distance from the center, depending on the context, so read each problem carefully.