\chapter{Antiderivatives}

In your study of calculus, you have learned about derivatives, which
allow us to find the rate of change of a function at any given
point. Derivatives are powerful tools that help us analyze the
behavior of functions. Now, we will explore another concept called
antiderivatives, which are closely related to derivatives.\index{Antiderivatives}

An antiderivative, also known as an integral or primitive, is the
reverse process of differentiation. It involves finding a function
whose derivative is equal to a given function. In simple terms, if you
have a function and you want to find another function that, when
differentiated, gives you the original function back, you are looking
for its antiderivative.

The symbol used to represent an antiderivative is $\int$. It is called
the integral sign. For example, if $f(x)$ is a function, then the
antiderivative of $f(x)$ with respect to $x$ is denoted as $\int f(x)
, dx$. The $dx$ at the end indicates that we are integrating with
respect to $x$.

Finding antiderivatives requires using specific techniques and
rules. Some common antiderivative rules include:

\begin{itemize}
\item The power rule: If $f(x) = x^n$, where $n$ is any real number
  except $-1$, then the antiderivative of $f(x)$ is given by $\int
  f(x) , dx = \frac{1}{n+1}x^{n+1} + C$, where $C$ is the constant of
  integration.

\item The constant rule: The antiderivative of a constant function is
  equal to the constant times $x$. For example, if $f(x) = 5$, then
  $\int f(x) , dx = 5x + C$.

\item The sum and difference rule: If $f(x)$ and $g(x)$ are functions,
  then $\int (f(x) + g(x)) , dx = \int f(x) , dx + \int g(x) ,
  dx$. Similarly, $\int (f(x) - g(x)) , dx = \int f(x) , dx - \int
  g(x) , dx$.
\end{itemize}

Antiderivatives have various applications in mathematics and
science. They allow us to calculate the total accumulation of a
quantity over a given interval, compute areas under curves, and solve
differential equations, among other things.

It is important to note that an antiderivative is not a unique
function. Since the derivative of a constant is zero, any constant
added to an antiderivative will still be an antiderivative of the
original function. This is why we include the constant of integration,
denoted by $C$, in the antiderivative expression.

In summary, antiderivatives are the reverse process of
differentiation. They help us find functions whose derivatives match a
given function. Understanding antiderivatives is crucial for various
advanced calculus concepts and real-world applications.

Now, let's explore different techniques and methods for finding
antiderivatives and discover how they can be applied in solving
problems.
