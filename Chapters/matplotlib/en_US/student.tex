\chapter{Making Plots with matplotlib}

Matplotlib is a comprehensive library for creating static, animated,
and interactive visualizations in Python. It is highly useful for
presenting data in a more intuitive and easy-to-understand manner.\index{matplotlib}

In order to use Matplotlib, you must first import it, typically using
the following line of code:

\begin{lstlisting}[language=Python]
import matplotlib.pyplot as plt
\end{lstlisting}

Let's create a simple line plot. Suppose we have a list of numbers and
we want to visualize their distribution:

\begin{lstlisting}[language=Python]
x = [1, 2, 3, 4, 5]
y = [1, 4, 9, 16, 25]

plt.plot(x, y)
plt.show()
\end{lstlisting}

Here, `x` and `y` are the coordinates of the points. The `plt.plot`
function plots y versus x as lines and/or markers. The `plt.show`
function then displays the figure.

Creating a bar plot follows a similar approach:

\begin{lstlisting}[language=Python]
labels = ['A', 'B', 'C', 'D', 'E']
values = [5, 7, 9, 11, 13]

plt.bar(labels, values)
plt.show()
\end{lstlisting}

Here, `labels` are the categories we are plotting, and `values` are
the respective sizes of those categories. The `plt.bar` function
creates a bar plot.

Matplotlib provides a variety of other plot types and customization
options --- everything from scatter plots and histograms to custom line
styles and colors. Explore the official Matplotlib documentation to
learn more about what this powerful library can offer.
