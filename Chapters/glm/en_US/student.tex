\chapter{Generalized Linear Models}

In statistics, generalized linear models (GLMs) are a flexible
generalization of ordinary linear regression models for response
variables that are not normally distributed. If you're already
familiar with multiple linear regression, you're well on your way to
understanding GLMs.\index{GLMs} \index{generalized linear models}

\section{Components of a Generalized Linear Model}

A GLM consists of three components:

\begin{enumerate}
    \item A random component: This is a specification of the
      probability distribution of the response variable (e.g., normal,
      binomial, Poisson distributions, etc.). This differs from
      ordinary linear regression, which assumes that the response
      variable follows a normal distribution.
    \item A systematic component: This is the linear predictor, a
      linear combination of the explanatory variables, just as in
      ordinary linear regression.
    \item A link function: This is a function that connects the mean
      of the response variable to the linear predictor. The choice of
      link function depends on the nature of the response variable and
      the range of its possible values.
\end{enumerate}

\section{Formulation of a Generalized Linear Model}

The GLM can be formulated as follows:

\begin{equation}
g(E(Y)) = \eta = X\beta
\end{equation}

Here, $Y$ is the response variable, $X$ represents the matrix of
explanatory variables, $\beta$ is the vector of parameters to be
estimated, $\eta$ is the linear predictor, $E(Y)$ represents the
expected value of $Y$, and $g(.)$ is the link function.

\section{Fitting a Generalized Linear Model}

The parameters $\beta$ in a GLM are typically estimated using maximum
likelihood estimation (MLE). The specifics of this process depend on
the probability distribution of the response variable and the link
function.

\section{Examples of Generalized Linear Models}

Examples of GLMs include:

\begin{itemize}
    \item Logistic regression: This is a GLM with a binomial response variable and a logit link function.
    \item Poisson regression: This is a GLM with a Poisson response variable and a log link function.
\end{itemize}
