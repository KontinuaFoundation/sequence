\chapter{Data Tables and pandas}

Much of the data that you will encounter in your career will come to
you as a table.  Some of these tables are spreadsheets, some are in
relational databases, some will come to you as CSV files.

Typically each column will represent an attribute (like height or
acreage) and each row will represent an entity (like a person or a
farm). You might get a table like this:

\begin{tabular}{c | c | c | c}
  \texttt{property\_id} & \texttt{bedrooms} & \texttt{square\_meters} & \texttt{estimated\_value} \\
  \hline
  7927 &  3 & 921.4 & \$ 294,393 \\
  9329 &  2 & 829.1 & \$ 207,420 \\
\end{tabular}

Typically, one of the columns is guaranteed to be unique. We call this
the \newterm{primary key}.  In this table, \texttt{property\_id} is
the primary key: every property has one, and no two properties have
the same \texttt{property\_id}.

\section{Data types}

Each column in a table has a type, and these usually correspond pretty nicely with types in Python.

Here are some common datatypes:

\begin{tabular}{c | c | c }
  Type & Python type & Example \\
  \hline
  Integer & int & \texttt{910393} \\
  Float & float & \texttt{-23.19} \\
  String & string & \texttt{'Fred'} \\
  Boolean & bool & \texttt{False} \\
  Date & datetime.date & \texttt{2019-12-04} \\
  Timestamps & datetime.datetime & \texttt{2022-06-10T14:05:22Z} \\
\end{tabular}

Sometimes it is OK to have values missing.  For example, if you had a
table of data about employees, maybe one of the columns would be
\texttt{retirement}, a date that tells you when the person retired.  People who
had not yet retired would have no value in this column.  We would say
that they have \newterm{null} for \texttt{retirement}.\index{null}

Sometimes there are constraints on what values can appear in the
column.  For example, if the column were \texttt{height}, it would make no
sense to have a negative value.

Sometimes a column can only be one of a few values. For example, if
you ran a bike rental shop, each bicycle's status would be ``available'',
``rented'', or ``broken''.  Any other values in that column would not
be allowed.  We often call these columns \newterm{categorical}.

\section{pandas}

The Python community works with tables of data \emph{a lot}, so it
created the pandas library for reading, writing, and manipulating
tables of data.

When working with tables, you sometimes need to go through them
row-by-row. However, for large datasets, this is very slow. pandas
makes it easy (and very fast) to say things like ``Delete every row
that doesn't have a value for height'' instead of requiring you to
step through the whole table.

In pandas, there are two datatypes that you use a lot:
\begin{itemize}
\item a \texttt{Series} is a single column of data.
\item a \texttt{DataFrame} is a table of data: it has a \texttt{Series} for each column.
\end{itemize}

In the digital resources, you will fined \filename{bikes.csv}. If you
look at it in a text editor, it will start like this:
\begin{Verbatim}
bike_id,brand,size,purchase_price,purchase_date,status
5636248,GT,57,277.99,1986-09-07,available
4156134,Giant,56,201.52,2005-01-09,rented
7971254,Cannondale,54,292.25,1978-02-28,available
3600023,Canyon,57,197.62,2007-02-15,broken
\end{Verbatim}

The first line is a header and tells you the name of each column.
Then the values are separated by commas. (Thus the name: CSV stands
for ``Comma Separated Values''.)

\section{Reading a CSV with pandas}

Let's make a program that reads \filename{bikes.csv} into a pandas
dataframe.  Create a file called \filename{report.py} in the same
folder as \filename{bikes.csv}.

First, we will read in the csv file. pandas has one Series that acts
as the primary key; it calls this one the index. When reading in the
file, we will tell it to use the \texttt{bike\_id} as the index
series.

If you ask a dataframe for its shape, it returns a tuple containing
the number of rows and the number of columns. To confirm that we have
actually read the data in, let's print those numbers.  Add these lines
to \filename{report.py}:

\begin{Verbatim}
import pandas as pd

# Read the CSV and create a dataframe
df = pd.read_csv('bikes.csv', index_col="bike_id")

# Show the shape of the dataframe
(row_count, col_count) = df.shape
print(f"*** Basics ***")
print(f"Bikes: {row_count:,}")
print(f"Columns: {col_count}")
\end{Verbatim}

Build it and run it. You should see something like this:
\begin{Verbatim}
*** Basics ***
Bikes: 998
Columns: 5
\end{Verbatim}

Note that your table actually had 6 columns. The index series is
not included in the shape.

\section{Looking at a Series}

Let's get the lowest, the highest, and the mean purchase price of the
bikes.  The purchase price is a series, and you can ask the dataframe
for it. Add these lines to the end of your program:

\begin{Verbatim}
# Purchase price stats
print("\n*** Purchase Price ***")
series = df["purchase_price"]
print(f"Lowest:{series.min()}")
print(f"Highest:{series.max()}")
print(f"Mean:{series.mean():.2f}")
\end{Verbatim}

Now when you run it, you will see a few additional lines:
\begin{Verbatim}
*** Purchase Price ***
Lowest:107.37
Highest:377.7
Mean:249.01
\end{Verbatim}

What are all the brands of the bikes? Add a few more lines to your
program that shows how many of each brand:

\begin{Verbatim}
# Brand stats
print("\n*** Brands ***")
series = df["brand"]
series_counts = series.value_counts()
print(f"{series_counts}")
\end{Verbatim}

Now when you run it, your report will include the number of bikes for
each brand from most common to least:

\begin{Verbatim}
*** Brands ***
Canyon        192
BMC           173
Cannondale    170
Trek          166
GT            150
Giant         147
Name: brand, dtype: int64
\end{Verbatim}

\pyfunction{value\_counts} returns a Series.  To format this better we
need to learn about accessing individual rows in a series.

\section{Rows and the index}

In an array, you ask for data using an the location (as an int) of the
item you want. You can do this in pandas using \pyfunction{iloc}. Add
this to the end of your program:

\begin{Verbatim}
# First bike
print("\n*** First Bike ***")
row = df.iloc[0]
print(f"{row}")
\end{Verbatim}

When you run it, you will see the attributes of the first row of data:

\begin{Verbatim}
*** First Bike ***
brand                     GT
size                      57
purchase_price        277.99
purchase_date     1986-09-07
status             available
Name: 5636248, dtype: object
\end{Verbatim}

Notice that the data coming back is actually another series.

The last line says that the name (the value for the index column) for
this row is 5636248.  In pandas, we usually use this to locate
particular rows.  For example, there is a row with \texttt{bike\_id}
equal to 2969341. Let's ask for one entry from the 

\begin{Verbatim}
print("\n*** Some Bike ***")
brand = df.loc[2969341]['brand']
print(f"brand = {brand}")
\end{Verbatim}

Now you will see the information about that bike:

\begin{Verbatim}
*** Some Bike ***
brand = Cannondale
\end{Verbatim}

pandas has a few different ways of getting to that value.  All of these get you the same thing:
\begin{Verbatim}
brand = df.loc[2969341]['brand'] # Get row, then get value
brand = df['brand'][2969341]     # Get column, then get value
brand = df.loc[2969341, 'brand'] # One call with both row and value
\end{Verbatim}

\section{Changing data}

One of your attributes needs cleaning up. Every bike should have a
status and it should be one of the following strings:''available'',
``rented'', or ``broken''.  Get counts for each unique value in
status:

\begin{Verbatim}
print("\n*** Status ***")
series = df["status"]
missing = series.isnull()
print(f"{missing.sum()} bikes have no status.")
series_counts = series.value_counts()
for value in series_counts.index:
    print(f"{series_counts.loc[value]} bikes are \"{value}\"")
\end{Verbatim}

This will show you:

\begin{Verbatim}
*** Status ***
7 bikes have no status.
389 bikes are "rented"
304 bikes are "broken"
296 bikes are "available"
1 bikes are "Flat tire"
1 bikes are "Available"
\end{Verbatim}

Right away we can see two easily fixable problems: Someone typed
``Available'' instead of ``available''.  Right after you read the CSV
in, fix this in the data frame:

\begin{Verbatim}
mask = df['status'] == 'Available'
print(f"{mask}")
df.loc[mask, 'status'] = 'available'
\end{Verbatim}

When you run this, you will see that the mask is a series with
\texttt{bike\_id} as the index and \texttt{False} or \texttt{True} as the value,
depending on whether the row's status was equal to ``Available''.

When you use \pyfunction{loc} with this sort of mask, you are saying
``Give me all the rows for which the mask is True.''  So, the
assignment only happens in the one problematic row.

Let's get rid of the mask variable and do the same for turning \texttt{Flat tire} into \texttt{Broken}:

\begin{Verbatim}
df.loc[df['status'] == 'Available', 'status'] = 'available'
df.loc[df['status'] == 'Flat tire', 'status'] = 'broken'
\end{Verbatim}

Now those problems are gone:
\begin{Verbatim}
7 bikes have no status.
389 bikes are "rented"
305 bikes are "broken"
297 bikes are "available"
\end{Verbatim}

What about the rows with no values for status? We were pretty certain
that the bikes were available, we could just set them to 'available':

\begin{Verbatim}
missing_mask = df['status'].isnull()
df.loc[missing_mask, 'status'] = 'available'
\end{Verbatim}

Or maybe we would print out the IDs of the bikes so that we could go look for them:

\begin{Verbatim}
missing_mask = df['status'].isnull()
missing_ids = list(df[missing_mask].index)
print(f"These bikes have no status:{missing_ids}")
\end{Verbatim}

But lets just keep the rows where the status is not null:
\begin{Verbatim}
missing_mask = df['status'].isnull()
df = df[~missing_mask]
\end{Verbatim}

At the end of your program, write out the improved CSV:

\begin{Verbatim}
df.to_csv('bikes2.csv')
\end{Verbatim}

Run the program and open \filename{bikes2.csv} in a text editor.




