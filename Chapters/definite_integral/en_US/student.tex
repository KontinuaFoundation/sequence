\chapter{Definite Integrals}



Integrals are a fundamental concept in calculus, which are used to calculate areas, volumes, and many other things. A definite integral calculates the net area between the function and the x-axis over a given interval.

\section{Definition}

The definite integral of a function $f(x)$ over an interval $[a, b]$ is defined as the limit of a Riemann sum:

\begin{equation}
\int_{a}^{b} f(x) \, dx = \lim_{{n \to \infty}} \sum_{i=1}^{n} f(x_i^*) \Delta x
\end{equation}

where $x_i^*$ is a sample point in the $i^{th}$ subinterval of a partition of $[a, b]$, $\Delta x = \frac{b-a}{n}$ is the width of each subinterval, and the limit is taken as the number of subintervals $n$ approaches infinity.

\section{Fundamental Theorem of Calculus}

The Fundamental Theorem of Calculus provides an important link between differentiation and integration. It states that if $F(x)$ is an antiderivative of $f(x)$ on the interval $[a, b]$, then

\begin{equation}
\int_{a}^{b} f(x) \, dx = F(b) - F(a)
\end{equation}

This theorem provides a way to compute definite integrals without needing to take limits of Riemann sums.

\