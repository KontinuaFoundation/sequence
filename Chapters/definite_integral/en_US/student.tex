\chapter{Definite Integrals}

Integrals are a fundamental concept in calculus, which are used to
calculate areas, volumes, and many other things. A definite integral
calculates the net area between the function and the x-axis over a
given interval.

\section{Definition}

The definite integral of a function $f(x)$ over an interval $[a, b]$
is defined as the limit of a Riemann sum:

\begin{equation}
\int_{a}^{b} f(x) \, dx = \lim_{{n \to \infty}} \sum_{i=1}^{n} f(x_i^*) \Delta x
\end{equation}

where $x_i^*$ is a sample point in the $i^{th}$ subinterval of a
partition of $[a, b]$, $\Delta x = \frac{b-a}{n}$ is the width of each
subinterval, and the limit is taken as the number of subintervals $n$
approaches infinity.

\section{Applications in Physics}
You should recall that velocity is the derivative of displacement, and acceleration is the derivative of velocity. Therefore, velocity is the integral of acceleration, and displacement is the integral of velocity. Say we know the velocity function for some object and we want to know how far the object moved from time $a$ to time $b$. We take the definite integral from $t=a$ to $t=b$ of the velocity function. 

Suppose an object is moving in a straight line and its velocity can be described by the function $v(t) = 6t-t^2$. What is the particle's displacement from $t=0$ to $t=3$? First, we note that to find displacement we should take the integral of velocity. The definite integral we need is:
$$\int_{0}^{3} [6t-t^2]\,dt$$
We can separate this into two integrals:
$$\int_{0}^{3} 6t\,dt -\int_{0}^{3}t^2\,dt$$
The antidervative of $6t$ is $3t^2$, and the antiderivative of $t^2$ is $\frac{1}{3}t^3$. 
$$[3(3)^2-3(0)^2]-[\frac{1}{3}(3)^3-\frac{1}{3}(0)^3]=[27-0]-[9-0]=18$$
Therefore, the object has a displacement of 18 units (in the real world, you'll know if you're discussing meters, miles, etc.) from $t=0$ to $t=3$. 

