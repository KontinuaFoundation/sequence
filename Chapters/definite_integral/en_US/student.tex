\chapter{Definite Integrals}

Integrals are a fundamental concept in calculus, which are used to
calculate areas, volumes, and many other things. A definite integral
calculates the net area between the function and the x-axis over a
given interval.

Recall that you can use a Riemann sum to estimate the area under a function, and that as we increase the number of subintervals, the estimated area approaches the actual area. In sigma notation we can express a Riemann sum as $$\sum_{i=1}^{n} f(x_i)\Delta x$$

\section{Definition}

The definite integral of a function $f(x)$ over an interval $[a, b]$
is defined as the limit of a Riemann sum as $n$ approaches $\infty$:

\begin{equation}
\int_{a}^{b} f(x) \, dx = \lim_{{n \to \infty}} \sum_{i=1}^{n} f(x_i^*) \Delta x
\end{equation}

where $x_i^*$ is a sample point in the $i^{th}$ subinterval of a
partition of $[a, b]$, $\Delta x = \frac{b-a}{n}$ is the width of each
subinterval, and the limit is taken as the number of subintervals $n$
approaches infinity.

\begin{Exercise}[label=defint1]
Express $$\lim_{n \to \infty} \sum_{i=1}^{n} (x_i^3+x_i\sin{x_i})\Delta x$$ as an integral on the interval $[0, \pi]$. 
\end{Exercise}

\begin{Answer}[ref=defint1]
Following the structure shown in the formal definition of a definite integral, we can set $f(x) = x^3+x\sin{x}$ and re-write the limit of the sum as $\lim_{n \to \infty} \Sigma_{i=1}^{n} f(x)\Delta x=\int_{0}^{\pi} f(x) \, dx$. Therefore, the full definite integral would be written as $\int_{0}^{\pi} (x^3 + x\sin{x})\, dx$. 
\end{Answer}

\section{Positive and Negative Areas}
%explain when function is below graph, area is negative and so sum represents NET area

\section{Properties of Integrals}
The following properties of integrals apply when $f(x)$ is continuous or has a finite number of jump discontinuities on the interval $a \leq x \leq b$:
\begin{enumerate}
\item $$\int_{a}^{b}f(x)\, dx = -\int_{b}^{a}f(x)\, dx$$ %Property 1
\item If $a=b$, then $\Delta x = 0$ and therefore $$\int_{a}^{a}f(x)\, dx = 0$$%Property 2
\item $$\int_{a}^{b}c\,dx = c(b-a)\text{, where }c\text{ is any constant}$$%Property 3
\item $$\int_{a}^{b}[f(x) + g(x)]\,dx = \int_{a}^{b}f(x)\,dx + \int_{a}^{b}g(x)\,dx$$%Property 4
\item $$\int_{a}^{b}cf(x)\,dx = c\int_{a}^{b}f(x)\text{, where }c\text{ is any constant}$$%Property 5
\item $$\int_{a}^{b}[f(x) - g(x)]\,dx = \int_{a}^{b}f(x)\,dx - \int_{a}^{b}g(x)\,dx$$%Property 6
\item $$\int_{a}^{c} f(x)\,dx + \int_{c}^{b}f(x)\,dx = \int_{a}^{b}f(x)\,dx\text{, where }a<c<b$$%Property 7
\item $$\text{If }f(x) \geq 0\text{ for }a\leq x\leq b\text{, then }\int_{a}^{b}f(x)\,dx \geq0$$%Property 8
\item $$\text{If }f(x)\geq g(x)\text{ for }a\leq x \leq b\text{, then}\int_{a}^{b}f(x)\,dx \geq \int_{a}^{b}g(x)\,dx$$%Property 9
\item $$\text{If } m \leq f(x) \leq M\text{ for }a\leq x \leq b\text{, then }m(b-a)\leq \int_{a}^{b}f(x)\,dx \leq M(b-a)$$%Property 10
\end{enumerate}


\section{Applications in Physics}
You should recall that velocity is the derivative of displacement, and acceleration is the derivative of velocity. Therefore, velocity is the integral of acceleration, and displacement is the integral of velocity. Say we know the velocity function for some object and we want to know how far the object moved from time $a$ to time $b$. We take the definite integral from $t=a$ to $t=b$ of the velocity function. 

Suppose an object is moving in a straight line and its velocity can be described by the function $v(t) = 6t-t^2$. What is the particle's displacement from $t=0$ to $t=3$? First, we note that to find displacement we should take the integral of velocity. The definite integral we need is:
$$\int_{0}^{3} [6t-t^2]\,dt$$
We can separate this into two integrals:
$$\int_{0}^{3} 6t\,dt -\int_{0}^{3}t^2\,dt$$
The antidervative of $6t$ is $3t^2$, and the antiderivative of $t^2$ is $\frac{1}{3}t^3$. 
$$[3(3)^2-3(0)^2]-[\frac{1}{3}(3)^3-\frac{1}{3}(0)^3]=[27-0]-[9-0]=18$$
Therefore, the object has a displacement of 18 units (in the real world, you'll know if you're discussing meters, miles, etc.) from $t=0$ to $t=3$. 

\section{Practice Exercises}
\begin{Exercise}[label=defint2]
Given that $\int_{0}^{1}x^2\,dx=\frac{1}{3}$, use the properties of integrals to evaluate $\int_{0}^{1}(5-6x^2)\,dx$. 
\end{Exercise}

\begin{Answer}[ref=defint2]
By property 6, we know that $$\int_{0}^{1}(5-6x^2)\,dx=\int_{0}^{1}5\,dx-\int_{0}^{1}6x^2\,dx$$\\
By property 5, we know that $$\int_{0}^{1}5\,dx-\int_{0}^{1}6x^2\,dx=\int_{0}^{1}5\,dx-6\int_{0}^{1}x^2\,dx$$\\
By property 3, we know that $$\int_{0}^{1}5\,dx = 5(1-0) = 5$$\\
Putting it all together, we see that $$\int_{0}^{1}(5-6x^2)\,dx=5-6(\frac{1}{3})=5-2=3$$
\end{Answer}
