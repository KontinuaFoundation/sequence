\chapter{HTML}

HTML, an abbreviation for Hypertext Markup Language, is the standard
language for creating web pages and web applications. It is a
cornerstone technology of the World Wide Web and forms the structure
and layout of web content.\index{HTML}

\section{HTML Elements}
\index{HTML!elements of}
An HTML document is composed of a series of elements, which are
denoted by tags. Elements have an opening tag and a closing tag with
content in between. Some elements, however, are self-closing and do
not contain any content. For example, the paragraph tag `<p>` is used
to denote a paragraph:

\begin{verbatim}
<p>This is a paragraph.</p>
\end{verbatim}

\section{HTML Document Structure}
\index{HTML!structure of}
A typical HTML document has a specific structure, including the following elements:

\begin{itemize}
    \item \textbf{DOCTYPE declaration}: This informs the browser about
      the version of HTML. For HTML5, it is `<!DOCTYPE html>`.
    \item \textbf{html}: This tag encloses the entire HTML document.
    \item \textbf{head}: This contains meta-information about the
      document, such as its title, meta tags, and links to scripts and
      stylesheets.
    \item \textbf{body}: This contains the content of the web page
      that is rendered in the browser.
\end{itemize}

Here is a basic example of an HTML document:

\begin{verbatim}
<!DOCTYPE html>
<html>
<head>
    <title>My First HTML Page</title>
</head>
<body>
    <h1>Welcome to My First HTML Page!</h1>
    <p>This is a paragraph.</p>
</body>
</html>
\end{verbatim}

\section{Web Development}
\index{Web Development}
The Web Development trio is formed by the 3 languages \verb|HTML|, \verb|css|, and \verb|js|. 
FIXME expand on this
% helpful: https://youtu.be/OEV8gMkCHXQ
%          https://youtu.be/DHjqpvDnNGE 
