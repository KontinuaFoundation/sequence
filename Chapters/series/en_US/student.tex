\chapter{Series}
When writing a number with an infinite decimal, such as the Golden 
Ratio (also known as the Golden Number):
$$\phi = 1.618033988 \cdots$$

The decimal system means we can re-write the Golden Ratio (or any 
irrational number) as an infinite sum:
$$\phi = 1 + \frac{6}{10} + \frac{1}{10^2} + \frac{8}{10^3} + 
\frac{0}{10^4} + \frac{3}{10^5} + \cdots$$

You might recall from the chapter on Riemann Sums that we can 
represent the addition of many (or infinite) with big sigma notation:
$$\Sigma_{i = 1}^n a_i$$
Where i is the index as discussed in Sequences and n is the number of 
terms. For infinite sums, $n = \infty$.

\section{Partial Sums}
Let us quickly define a \textit{partial sum}. A partial sum is where 
we only look at the first $n$ terms of a series. For the general 
series, $\Sigma_{i=1}^{n} a_i$, the partial sums are:
$$s_1 = a_1$$
$$s_2 = a_1 + a_2$$
$$s_3 = a_1 + a_2 + a_3$$
$$\cdots$$
$$s_n = a_1 + a_2 + \cdots + a_n = \Sigma_{i=1}^{n} a_i$$

\textbf{Example}: A series is given by $\Sigma_{i=1}^\infty 
(\frac{-3}{4})^i$. What is the value of the partial sum $s_4$?

\textbf{Solution}: $s_4$ is the sum of the first 4 terms: 
$$(\frac{-3}{4})^1 + (\frac{-3}{4})^2 + (\frac{-3}{4})^3 + (\frac{-3}{4})^4$$
$$= \frac{-3}{4} + \frac{9}{16} + \frac{-27}{64} + \frac{81}{256} = \frac{-75}{256}$$


\section{Convergent and Divergent Series}
Just like sequences, series can also be convergent or divergent. 
Consider the series $\Sigma_{i=1}^\infty i$. Given what you already 
know about the meaning of "convergent" and "divergent", guess whether 
$\Sigma_{i=1}^\infty i$ is convergent or divergent. 

Let's determine the first few partial sums of the series (shown 
graphically in figure \ref{fig:divsum}):
\begin{center}
\begin{tabular}{|c|c|c|}\hline
n & Terms & Partial Sum\\
\hline
1 & 1 & 1\\
\hline
2 & 1+2 & 3\\
\hline
3 & 1+2+3 & 6\\
\hline
4 & 1+2+3+4 & 10\\
\hline
\end{tabular}
\end{center}

\begin{figure}[htbp]
\centering
    \begin{tikzpicture}
        \begin{axis}[axis lines = center, xmin = -0.5, xmax = 10, 
        ymin = 0, ymax = 55, xlabel=n, ylabel = $s_n$]
        \addplot[blue, mark=*](1,1);
        \addplot[blue, mark=*](2, 3);
        \addplot[blue, mark=*](3, 6);
        \addplot[blue, mark=*](4, 10);
        \addplot[blue, mark=*](5, 15);
        \addplot[blue, mark=*](6, 21);
        \addplot[blue, mark=*](7, 28);
        \addplot[blue, mark=*](8, 36);
        \addplot[blue, mark=*](9, 45);
        \addplot[blue, mark=*](10,55);
        \end{axis}
\end{tikzpicture}
    \caption{For the divergent series $\Sigma_{i=1}^n i$, the value of the 
    partial sum increases to infinity as $n$ increases}
    \label{fig:divsum}
\end{figure}

As you can see, as $n$ increases, the value of the partial sum 
increases without approaching a particular value. We can also see 
that the value of the first $n$ terms summed together is 
$\frac{n(n+1)}{2}$. This means that as $n$ approaches $\infty$, the 
sum also approaches $\infty$ and the series is divergent. 

Obviously, for a series to not become huge, the values of the terms 
should decrease as $i$ increases (that is, each subsequent term is 
smaller than the one before it). Take the series $\Sigma_{i=1}^\infty 
\frac{1}{2^i}$. As $i$ increases, $\frac{1}{2^i}$ decreases. Let's 
look at the first few partial sums of this series (shown graphically 
in figure \ref{fig:convsum}):
\begin{center}
\begin{tabular}{|c|c|c|}\hline
n & Terms & Partial Sum\\
\hline
1 & $\frac{1}{2}$ & $\frac{1}{2}$\\
\hline
2 & $\frac{1}{2} + \frac{1}{4}$ & $\frac{3}{4}$\\
\hline
3 & $\frac{1}{2} + \frac{1}{4} + \frac{1}{8}$ & $\frac{7}{8}$\\
\hline
4 & $\frac{1}{2} + \frac{1}{4} + \frac{1}{8} + \frac{1}{16}$ & 
$\frac{15}{16}$\\
\hline
\end{tabular}
\end{center}

\begin{figure}[htbp]
\centering
    \begin{tikzpicture}
        \begin{axis}[axis lines = center, xmin = -0.5, xmax = 8, 
        ymin = 0, ymax = 1.5, ytick = {1}, xlabel=n, ylabel = $s_n$]
        \addplot[blue, mark=*](1,0.5);
        \addplot[blue, mark=*](2, 3/4);
        \addplot[blue, mark=*](3, 7/8);
        \addplot[blue, mark=*](4, 15/16);
        \addplot[blue, mark=*](5, 31/32);
        \addplot[blue, mark=*](6, 63/64);
        \addplot[blue, mark=*](7, 127/128);
        \addplot[blue, mark=*](8, 255/256);
        \addplot[red, thin, dashed, domain = 0:8]{1};
        \end{axis}
\end{tikzpicture}
    \caption{For the convergent series $\Sigma_{i=1}^n \frac{1}{2^i}$, 
    the value of the partial sum approaches 1 as $n$ increases}
    \label{fig:convsum}
\end{figure}

Do you see the pattern? The $n^{th}$ partial sum is equal to 
$\frac{2^n - 1}{2^n} = 1 - \frac{1}{2^n}$. And as $n$ approaches 
$\infty$, the partial sum approaches 1. The series $\Sigma_{i=1}^\infty 
\frac{1}{2^i}$ is convergent. 

Let us define the sequence \{$s_n$\} where $s_n$ is the $n^{th}$ 
partial sum of a series:
$$s_n = \Sigma_{i=1}^n a_i$$. 

If the sequence \{$s_n$\} is convergent and $\lim_{n \to \infty} s_n$ 
exists, then the series $\Sigma_{i=1}^\infty a_i$ is also convergent. 
And if the sequence \{$s_n$\} is divergent, then the series 
$\Sigma_{i=1}^\infty a_i$ is also divergent.

\textbf{Example}: is the harmonic series, $\Sigma_{n = 1}^\infty \frac{1}{n}$ convergent or divergent?

\textbf{Solution}: You may think that the series is convergent, since $\lim_{n \to \infty} \frac{1}{n} = 0$. Let's see if we can confirm this. We begin by looking at the partial sums $s_2$, $s_4$, $s_8$, and $s_16$:
$$s_2 = 1 + \frac{1}{2}$$
$$s_4 = 1 + \frac{1}{2} + \left(\frac{1}{3} + \frac{1}{4} \right) > 1 + \frac{1}{2} + \left( \frac{1}{4} + \frac{1}{4} \right) = 1 + \frac{2}{2}$$
$$s_8 = 1 + \frac{1}{2} + \left(\frac{1}{3} + \frac{1}{4} \right) + \left( \frac{1}{5} + \frac{1}{6} + \frac{1}{7} + \frac{1}{8} \right) > 1 + \frac{1}{2} + \left(\frac{1}{4} + \frac{1}{4} \right) + \left( \frac{1}{8} + \frac{1}{8} + \frac{1}{8} + \frac{1}{8} \right) = 1 + \frac{3}{2}$$
$$s_{16} = 1 + \frac{1}{2} + \left(\frac{1}{3} + \frac{1}{4} \right) + \left( \frac{1}{5} + \frac{1}{6} + \frac{1}{7} + \frac{1}{8} \right) + \left(\frac{1}{9} + \cdots + \frac{1}{16} \right) > $$
$$1 + \frac{1}{2} + \left(\frac{1}{4} + \frac{1}{4} \right) + \left( \frac{1}{8} + \frac{1}{8} + \frac{1}{8} + \frac{1}{8} \right) + \left(\frac{1}{16} + \cdots + \frac{1}{16} \right) = 1 + \frac{4}{2}$$

Notice that, in general, $s_{2^n} > 1 + \frac{n}{2}$ for $n > 1$. Taking the limit as $n \to \infty$, we see that $\lim_{n \to \infty} s_{2^n} > \lim_{n \to \infty} 1 + \frac{n}{2} = \infty$. Therefore, $s_{2^n}$ also approaches $\infty$ as $n$ gets larger and the harmonic series $\Sigma_{n = 1}^\infty \frac{1}{n}$ is divergent. 

This example shows a very important point: the converse of a theorem is not always true! We do know that if a series is convergent, then the limit as n approaches infinity must be zero. The harmonic series shows that \textit{just because the limit as n approaches infinity is zero, the series is not necessarily convergent}. What we can say, though, is that the contrapositive statement is true: if the limit as n approaches infinity of a series does not exist or is not zero, then the series is divergent (i.e. not convergent). 

\subsection{Geometric Series}

Let's apply this definition of convergent and divergent to geometric 
series. You should recall from the chapter on sequences that a 
geometric sequence can be defined by $a_i = a(r)^{i-1}$, where $r$ 
is the common ratio and $a \neq 0$. We can express the 
\textit{geometric series} thusly:
$$a + ar + ar^2 + ar^3 = \Sigma_{i=1}^\infty ar^{i-1}$$

When are geometric series convergent? First, let's consider the case 
where $r=1$. If this is true, then $s_n = a + a + a + \cdots + a = 
na$. As $n$ approaches $\infty$, the sum will approach $\pm \infty$ 
(depending on whether $a$ is positive or negative), and the series is 
divergent. 

When $r \neq 1$, we can write $s_n$ and $rs_n$:
$$s_n = a + ar + ar^2 + \cdots + ar^{n-1}$$
$$rs_n = ar + ar^2 + ar^3 + \cdots + ar^n$$

Subtracting $rs_n$ from $s_n$, we get:
$$s_n - rs_n = (a + ar + ar^2 + \cdots + ar^{n-1}) - (ar + ar^2 + 
ar^3 + \cdots + ar^{n-1} + ar^n)$$
$$= a - ar^n$$

Solving for $s_n$, we find:
$$s_n = \frac{a(1-r^n)}{1-r}$$

We take the limit as $n \to \infty$ to determine for what values of 
$r$ the series converges:
$$\lim_{n \to \infty} s_n = \lim_{n \to \infty} \frac{a(1-r^n)}{1-r}$$
$$= \lim_{n\to \infty} \frac{a}{1-r} - \frac{ar^n}{1-r} = \frac{a}{1-r} 
- \frac{a}{1-r}\lim_{n \to \infty} r^n$$

This begs the question: when is $\lim_{n \to \infty} r^n$ convergent? 
From the sequences chapter, we know this limit converges if $|r| < 1$ 
(that is, $-1 < r < 1$). If this is true, then $\lim_{n \to \infty} 
r^n = 0$ and 
$$s_n = \frac{a}{1-r}$$

(see figures \ref{fig:geometric1} and \ref{fig:geometric2} for a visual)
\begin{figure}
    \centering
    \begin{tikzpicture}
        \begin{axis}[xmin = -0.5, xmax = 10, ymin = 0, axis lines = center, xlabel = $n$, clip = false, ytick = \empty]
        \addplot[red, mark=*] (1, 2); %r=1.5
        \addplot[red, mark=*] (2, 5);
        \addplot[red, mark=*] (3, 19/2);
        \addplot[red, mark=*] (4, 65/4);
        \addplot[red, mark=*] (5, 211/8);
        \addplot[red, mark=*] (6, 665/16);
        \addplot[red, mark=*] (7, 2059/32);
        \node[red] at (5.5, 60) {$r > 1$};
        
        \addplot[orange, mark=*] (1, 2); %r=1
        \addplot[orange, mark=*] (2, 4);
        \addplot[orange, mark=*] (3, 6);
        \addplot[orange, mark=*] (4, 8);
        \addplot[orange, mark=*] (5, 10);
        \addplot[orange, mark=*] (6, 12);
        \addplot[orange, mark=*] (7, 14);
        \addplot[orange, mark=*] (8, 16);
        \addplot[orange, mark=*] (9, 18);
        \addplot[orange, mark=*] (10, 20);
        \node[orange] at (9, 25) {$r = 1$};
        
        \addplot[blue, mark =*] (1, 2); %r = 0.5
        \addplot[blue, mark =*] (2, 3);
        \addplot[blue, mark =*] (3, 7/2);
        \addplot[blue, mark =*] (4, 15/4);
        \addplot[blue, mark =*] (5, 31/8);  
        \addplot[blue, mark =*] (6, 63/16);
        \addplot[blue, mark =*] (7, 127/32);
        \addplot[blue, mark =*] (8, 255/64);
        \addplot[blue, mark =*] (9, 511/128);
        \addplot[blue, mark =*] (10, 1023/256);
        \node[blue] at (10, 10) {$0 < r < 1$};
        \end{axis}
    \end{tikzpicture}
    \caption{Geometric sequences are divergent if $r \geq 1$}
    \label{fig:geometric1}
\end{figure}

\begin{figure}
    \centering
    \begin{tikzpicture}
        \begin{axis}[xmin = -0.5, xmax = 10, axis lines = center, xlabel = $n$, clip = false, ytick = \empty]
        \addplot[red, mark=*] (1, 2); %r=-1.5
        \addplot[red, mark=*] (2, -1);
        \addplot[red, mark=*] (3, 7/2);
        \addplot[red, mark=*] (4, -13/4);
        \addplot[red, mark=*] (5, 55/8);
        \addplot[red, mark=*] (6, -133/16);
        \addplot[red, mark=*] (7, 463/32);
        %\addplot[red, mark=*] (8, -1261/64);
        %\addplot[red, mark=*] (9, 4039/128);
        %\addplot[red, mark=*] (10, -11605/256);
        \node[red] at (5, 10) {$r > 1$};
        
        \addplot[orange, mark=*] (1, 2); %r=-1
        \addplot[orange, mark=*] (2, 0);
        \addplot[orange, mark=*] (3, 2);
        \addplot[orange, mark=*] (4, 0);
        \addplot[orange, mark=*] (5, 2);
        \addplot[orange, mark=*] (6, 0);
        \addplot[orange, mark=*] (7, 2);
        \addplot[orange, mark=*] (8, 0);
        \addplot[orange, mark=*] (9, 2);
        \addplot[orange, mark=*] (10, 0);
        \node[orange] at (9, -3) {$r = -1$};
        
        \addplot[blue, mark =*] (1, 2); %r = -0.5
        \addplot[blue, mark =*] (2, 1);
        \addplot[blue, mark =*] (3, 3/2);
        \addplot[blue, mark =*] (4, 5/4);
        \addplot[blue, mark =*] (5, 11/8);  
        \addplot[blue, mark =*] (6, 21/16);
        \addplot[blue, mark =*] (7, 43/32);
        \addplot[blue, mark =*] (8, 85/64);
        \addplot[blue, mark =*] (9, 171/128);
        \addplot[blue, mark =*] (10, 341/256);
        \node[blue] at (9, 5) {$-1 < r < 0$};
        \end{axis}
    \end{tikzpicture}
    \caption{Geometric sequences are divergent if $r \leq 1$. Notice that for $r = -1$, the partial sums alternate between the initial term and zero.}
    \label{fig:geometric2}
\end{figure}

\textbf{Example}: Find the sum of the geometric series given by $2 - 
\frac{2}{3} + \frac{2}{9} - \frac{2}{27} + \cdots$. 

\textbf{Solution}: The first term is $a = 2$ and each the common 
ratio is $r = \frac{-1}{3}$. Since $|r| < 1$, we know that the series 
converges. We can calculate the value of the sum using the geometric 
series formula: 
$$\Sigma_{i=1}^\infty a(r)^{i-1} = \frac{a}{1-r}$$
$$\Sigma_{i=1}^\infty 2(\frac{-1}{3})^{i-1} = \frac{2}{1-\frac{-1}{3}} = 
\frac{2}{\frac{4}{3}} = \frac{6}{4} = 1.5$$

We can confirm this graphically (see figure \ref{fig:geometric}). You 
can also write out the first several partial sequences: you should 
find the sums approach 1.5 as $n$ increases.

\begin{figure}[htbp]
\centering
    \begin{tikzpicture}
        \begin{axis}[axis lines = center, xmin = -0.5, xmax = 8, 
        ymin = -1.5, ymax = 2, xlabel=n, legend pos = south east]
        \addplot[blue, mark=*](1,2);
        \addlegendentry{$s_n$};
        \addplot[red, mark=o](1, 2);
        \addlegendentry{$a_n$};
        \addplot[blue, mark=*](2, 4/3);
        \addplot[blue, mark=*](3, 14/9);
        \addplot[blue, mark=*](4, 40/27);
        \addplot[blue, mark=*](5, 122/81);
        \addplot[blue, mark=*](6, 364/243);
        \addplot[blue, mark=*](7, 1094/729);
        \addplot[blue, mark=*](8, 3280/ 2187);
        \addplot[red, mark=o](2, -2/3);
        \addplot[red, mark=o](3, 2/9);
        \addplot[red, mark=o](4, -2/27);
        \addplot[red, mark=o](5, 2/81);
        \addplot[red, mark=o](6, -2/243);
        \addplot[red, mark=o](7, 2/729);
        \addplot[red, mark=o](8, -2/2187);
        \addplot[red, thin, dashed, domain = 0:8]{1.5};
        \end{axis}
\end{tikzpicture}
    \caption{the $n^{th}$ term and partial sums of $\Sigma_{i=1}^n 
    2(\frac{-1}{3})^{i-1}$}
    \label{fig:geometric}
\end{figure}

\section{Estimating the Value of Series}

\section{Ratio and Root Tests for Convergence}

\section{Power Series}
