\chapter{Fertilizer}

In 1950, there were 2.5 billion people on the planet, and about 65\%
were malnourished. In 2019, there were 7.7 billion people on the
planet, and only 15\% are malnourished. How did crop yields increase
so much? There were several factors: better crop varieties,
reliable irrigation, increased mechanization, and affordable fertilizers.\index{fertilizer}

When a plant grows, it takes molecules out of the soil and uses them
to build proteins. It primarily needs the elements nitrogen ($N$),
phosphorus ($P$), and potassium ($K$).\index{nitrogen} \index{phosphorus} \index{potassium}

When you buy a bag of fertilizer at the store, it has typically has
three number on the front.  For example, you might buy a bag of
``24-22-4''.  This means that 24\% of the mass of the bag is nitrogen,
22\% is phosphorus, and 4\% is potassium.

Potassium comes as potassium carbonate ($K_2CO_3$), potassium chloride
($KCl$), potassium sulfate ($K_2 SO_4$), and potassium nitrate
($KNO_3$). Any blend of these chemicals is known as ``potash''. Potash
is dug up out of mines. \index{potash}

Phosphorus is also mined, but is refined into phosphoric acid
($H_3PO_4$) before it is put into fertilizer.

Nitrogen is an especially interesting case for 2 reasons:
\begin{itemize}
\item Worldwide farmers apply more nitrogen to their soil than potassium or phosphorous combined.
\item 78\% of the air we breath is nitrogen in the form of $N_2$, but
  neither plants nor animals can utilize nitrogen in that form.
\end{itemize}

\section{The Nitrogen Cycle}

Converting the $N_2$ in the air into a form that a plant can use
is known as \newterm{nitrogen fixation}. For billions of years, there
was only two ways that nitrogen fixation occured on earth:
\begin{itemize}
\item The energy from lightning causes $N_2$ and $H_2O$ to reconfigure as ammonia ($NH_3$) and nitrate ($NO_3$). This accounts for about 10\% of all naturally occurring nitrogen fixation.
\item Cyanobacteria are responsible for the rest. They convert $N_2$ into ammonia.
\end{itemize}\index{nitrogen cycle} \index{nitrogen fixation}

Let's say that you are eating soybeans. There is a cyanobacteria
called \newterm{rhizobia} that has a symbiotic relationship with
soybean plants.  Rhizobia fixes nitrogen for the soybean plant. The
soybean plant performs photosynthesis and gives sugars to the
rhizobia.

The proteins in the soybeans contain nitrogen from the rhizobia. When
you eat them, you use some of the nitrogen to build new proteins. You
probably don't use all the nitrogen, so your cells release ammonia into your blood.

Ammonia likes to react with things, so your liver combines the ammonia
with carbon dioxide to make urea ($CO(NH_2)_2$).  Your kidneys take
the urea out of your blood and mix it with a bunch of water and salts
in your bladder.  When you urinate, the urea leaves your body.\index{urea}

If you urinate on the ground, the nearby plants can take the nitrogen out of
the urea.\index{urine}

When you die, the nitrogen in your proteins will return to the soil as
ammonia and nitrate.

For centuries, farms got their nitrogen from urine, feces, and rotting
organic material. There were two challenges with this:
\begin{itemize}
\item Human pathogens had to be kept away from human food.
\item There was simply not enough to support 7.7 billion people.
\end{itemize}

So we had to figure out how to do nitrogen fixation at an industrial
level.

\section{The Haber-Bosch Process}

During World War I, two German scientists, Fritz Haber and Carl Bosch,
figured out how to make ammonia from $N_2$ and $H_2$ using high
temperatures and pressures. This is how nearly all nitrogen fertilizer
is created today.\index{Haber-Bosch process}

Where do we get the $H_2$? From methane ($CH_4$) in natural gas. Today, 3-5\%
of the world's natural gas production is consumed in the Haber-Bosch
process.

The ammonia is converted into ammonium nitrate ($NH_4NO_3$) or urea
before it is shipped to farms.

\section{Other nutrients}

Healthy plants require several other elements that are sometimes
applied as fertilizer: calcium, magnesium, and sulfur.

Finally, tiny amount of copper, iron, manganese, molybdenum, zinc, and
boron are sometimes needed.
