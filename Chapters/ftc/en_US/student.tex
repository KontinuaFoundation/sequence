\chapter{The Fundamental Theorem of Calculus}

The Fundamental Theorem of Calculus is a theorem that connects the
concept of differentiating a function with the concept of integrating
a function. This theorem is divided into two parts:\index{fundamental theorem of calculus}

\section{First Part}

The first part of the Fundamental Theorem of Calculus states that if
$f$ is a continuous real-valued function defined on a closed interval
$[a, b]$ and $F$ is the function defined, for all $x$ in $[a, b]$, by:

\begin{equation}
F(x) = \int_a^x f(t) \, dt
\end{equation}

Then, $F$ is uniformly continuous and differentiable on the open
interval $(a, b)$, and $F'(x) = f(x)$ for all $x$ in $(a, b)$.

\section{Second Part}

The second part of the Fundamental Theorem of Calculus states that if
$f$ is a real-valued function defined on a closed interval $[a, b]$
that admits an antiderivative $F$ on $[a, b]$, and $f$ is integrable
on $[a, b]$ (it need not be continuous), then

\begin{equation}
\int_a^b f(t) \, dt = F(b) - F(a).
\end{equation}
