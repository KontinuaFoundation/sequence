\chapter{Concrete}

To make concrete, you mix cement with water and an aggregate (sand or
rock).  The cement is usually only about 10 to 15 percent of the
mixture. The cement reacts with the water, and the resulting solid
binds the aggregate together. In 2019, the world consumed 4.5 billion
tons of cement.\index{concrete} \index{cement}

Concrete is hard and durable. The mortar between the pyramids at Giza
is concrete --- it is now 5000 years old. Today, we use concrete to
build many structures including buildings, bridges, airport runways,
and dams.

There are many kinds of cement, but the most common is Portland
cement. It is made by heating limestone (calcium carbonate) with clay
(for silicon) in a kiln. Two things come out of the kiln: Carbon
dioxide and a hard substance called ``clinker''.  The clinker is
ground up with some gypsum before it is sent to market.

The carbon dioxide is released into the atmosphere. Cement manufacture
is responsible for about 8\% of the world's $CO_2$ emissions; it is a
major contributor to climate change.

Especially hard concrete, like that used in a nuclear power plant, can
support 3,000 kg per centimeter without being crushed. However, if
you pull on two ends of a piece of concrete, it comes apart relatively
easily. We say that concrete can handle a lot of \newterm{compressive
  stress}, but not much \newterm{tensile} stress.

\section{Steel reinforced concrete}

Many places where we use concrete (like in a bridge), we need both
compressive and tensile stress. Often, the top of a beam is undergoing
compression and the bottom of the beam is undergoing tension.

FIXME Picture here

Steel has tremendous tensile strength, but not as much compressive
strength as concrete. To get both tensile \emph{and} compressive
strength, we often bury steel bars or cables inside the concrete.
This is known as \newterm{steel-reinforced concrete}. The concrete
generally does a very good job protecting the steel, which keeps it
from rusting.\index{steel reinforced concrete}

You may have heard of \newterm{rebar} before. That is just short for
``reinforcing bar''.  Typically, rebar has bumps and ridges that keep
the bar and the concrete from moving independently.\index{rebar}

\section{Recycling concrete}

Many concrete structures only last about 100 years. When they are
demolished, the concrete can be reused as aggregate in other projects.
Often, the concrete bits are mixed with cement and made into concrete once more.

If the concrete to be reused is reinforced with steel, the steel has
to be removed and recycled separately.  The concrete is then crushed
into small pieces.
