\chapter{Complex Numbers}

Complex numbers are an extension of real numbers, which in turn
are an extension of rational numbers. In mathematics, the set of
complex numbers is a number system that extends the real number line
to a full two dimensions, using the imaginary unit, which is denoted by
$i$, with the property that $i^2 = -1$.\index{Complex Numbers}

An imaginary number is a number that is a multiple of the imaginary unit $i$.
For example, $5i$ and $-3i$ are both imaginary numbers. While imaginary numbers are a 
subset of complex numbers (with real part equal to zero), not all complex numbers are 
imaginary. Complex numbers can have both a real and an imaginary part, while imaginary 
numbers specifically refer to those with no real part.

\section{Definition}

A complex number is a number of the form $a + bi$, where $a$ and $b$
are real numbers, and $i$ is the imaginary unit, with the property
that $i^2 = -1$. The real part of the complex number is $a$, and the
imaginary part is $bi$.

\section{Why Are Complex Numbers Necessary?}

Complex numbers are essential to many fields of science and
engineering. Here are a few reasons why:

\subsection{Roots of Negative Numbers}

In the real number system, the square root of a negative number does
not exist, because there is no real number that you can square to get a
negative number. The introduction of the imaginary unit $i$, which has
the property that $i^2 = -1$, allows us to take square roots of
negative numbers and gives rise to complex numbers.

\subsection{Polynomial Equations}

The fundamental theorem of algebra states that every non-constant
polynomial equation with complex coefficients has a complex root. This
theorem guarantees that polynomial equations of degree $n$ always have
$n$ roots in the complex plane.

\subsection{Physics and Engineering}

In physics and engineering, complex numbers are used to represent
waveforms in control systems, in quantum mechanics, and many other
areas. Their properties make many mathematical manipulations more
convenient.

\section{Adding Complex Numbers}

The addition of complex numbers is straightforward. If we have two
complex numbers $z_1 = a + bi$ and $z_2 = c + di$, their sum is
defined as:

\begin{equation}
z_1 + z_2 = (a + c) + (b + d)i
\end{equation}

In other words, you add the real parts to get the real part of the
sum, and add the imaginary parts to get the imaginary part of the sum.

\section{Multiplying Complex Numbers}

The multiplication of complex numbers is a bit more involved. If we
have two complex numbers $z_1 = a + bi$ and $z_2 = c + di$, their
product is defined as:

\begin{equation}
z_1 \cdot z_2 = (a + bi) \cdot (c + di) = ac + adi + bci - bd = (ac - bd) + (ad + bc)i
\end{equation}

Note the last term comes from $i^2 = -1$. You multiply the real parts
and the imaginary parts just as you would in a binomial
multiplication, and remember to replace $i^2$ with $-1$. See the Khan academy video
in the digital resources for a more in-depth explanation.

