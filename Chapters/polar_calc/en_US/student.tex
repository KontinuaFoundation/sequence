\chapter{Calculus with Polar Coordinates}

We've been working in Cartesian coordinates, which are rectangular, with $x$ 
representing the horizontal position and $y$ representing the vertical 
position. Another way to represent a position in $2D$ space is with \textbf{
polar coordinates}\index{polar coordinates}. In this coordinate system, the 
first number and independent variable is $\theta$ and represents the degrees 
of rotation from the the $x$ axis. The second number is $r$ and represents how 
far the point is from the origin (see figure \ref{polarex}). 

\begin{figure}[htbp]
\centering
    \begin{tikzpicture}
	\begin{polaraxis}[xtick = {0,0,deg((pi)/6),deg((2*pi)/6),deg((3*pi)/6),
	deg(4*pi)/6,deg((5*pi)/6), deg((6*pi)/6), deg((7*pi)/6), deg((8*pi)/6), 
	deg((9*pi)/6), deg((10*pi)/6), deg((11*pi)/6)}, xticklabels={,0,
	$\frac{\pi}{6}$,$\frac{\pi}{3}$,$\frac{\pi}{2}$,$\frac{2\pi}{3}$, 
	$\frac{5\pi}{6}$, $\pi$,$\frac{7\pi}{6}$, $\frac{4\pi}{3}$, 
	$\frac{3\pi}{2}$, $\frac{5\pi}{3}$, $\frac{11\pi}{6}$}, ymax = 2.25, clip = false]
		\addplot[white, thick, domain = 0:360, samples = 200]{2.25};
        \addplot[blue, only marks]coordinates {(0,1) (60, 1.5) (240, 2)};
        \addplot[red, domain = 0:60]{1.5};
        \draw[red, -latex](58, 1.5)--(60,1.5);
        \node[] at (65, 1.6) {$(\frac{\pi}{3}, 1.5)$};
        \node[] at (-15, 1) {$(0, 1)$};
        \addplot[red, domain = -120:0]{2};
        \draw[red, -latex](-118, 2) -- (-120,2);
        \node[] at (-120, 1.8) {$(-\frac{5\pi}{6}, 2)$};
        \end{polaraxis}
    \end{tikzpicture}
    \label{polarex}
    \caption{Polar coordinates give a degree of rotation, $\theta$, and a 
    distance from the origin, $r$}
    \end{figure}

\section{Derivatives of Polar Functions}
Consider the cardioid $r = 2 + \sin{\theta}$ (see figure \ref{cardioid}). What 
is the slope of the line tangent to the curve at $\theta = \frac{\pi}{2}$? 

\begin{figure}[htbp]
\centering
    \begin{tikzpicture}
	\begin{polaraxis}[xtick = {0,0,deg((pi)/6),deg((2*pi)/6),deg((3*pi)/6),
	deg(4*pi)/6,deg((5*pi)/6), deg((6*pi)/6), deg((7*pi)/6), deg((8*pi)/6), 
	deg((9*pi)/6), deg((10*pi)/6), deg((11*pi)/6)}, xticklabels={,0,
	$\frac{\pi}{6}$,$\frac{\pi}{3}$,$\frac{\pi}{2}$,$\frac{2\pi}{3}$, 
	$\frac{5\pi}{6}$, $\pi$,$\frac{7\pi}{6}$, $\frac{4\pi}{3}$, $\frac{3\pi}{2}$, 
	$\frac{5\pi}{3}$, $\frac{11\pi}{6}$}, ymax = 2.25, clip = false]
		\addplot[white, thick, domain = 0:360, samples = 200]{2.25};
        \addplot[blue, thick, domain = 0:360, samples = 100]{1 + sin(x)};
        \end{polaraxis}
    \end{tikzpicture}
    \label{cardioid}
    \caption{$r = 2 + \sin{\theta}$}
    \end{figure}

From a visual inspection, we can guess that the slope of the tangent line is 
zero. Let's prove this mathematically:

First, recall that to convert polar coordinates to Cartesian coordinates, we 
can use the trigonometric identities:
$$x = r\cos{\theta}$$
$$y = r\sin{\theta}$$

So we can write the parametric equation:
$$x = \left[2 + \sin{\theta} \right]\cos{\theta}$$
$$y = \left[ 2 + \sin{\theta} \right]\sin{\theta}$$

Recall from parametric equations that we can use implicit differentiation to 
find $\frac{dy}{dx}$:
$$\frac{dy}{dx} = \frac{\frac{dy}{d\theta}}{\frac{dx}{d\theta}}$$

Finding $\frac{dy}{d\theta}$ and $\frac{dx}{d\theta}$:
$$\frac{dy}{d\theta} = \frac{d}{d\theta} \left( 2\sin{\theta} + \sin^2{\theta} 
\right) = 2\cos{\theta} + 2\sin{\theta}\cos{\theta}$$
$$\frac{dx}{d\theta} = \frac{d}{d\theta} \left( 2\cos{\theta} + \sin{\theta}
\cos{\theta} \right) = \cos^2{\theta} - \sin^2{\theta} - 2\sin{\theta}$$

Substituting $\theta = \frac{pi}{2}$, we find that:
$$\frac{dy}{d\theta} = 2(0) + 2(1)(0) = 0$$
$$\frac{dx}{d\theta} = (0)^2 - (1)^2 - 2(1) = -3$$

And therefore,
$$\frac{dy}{dx} = \frac{0}{-3} = 0$$

Which is the result we expected from examining the graph of $r = 2 + \sin{
\theta}$. 


\begin{Exercise}[label = polar1]
[This problem was originally presented as a no-calculator, multiple-choice 
question on the 2012 AP Calculus BC exam.] What is the slope of the line 
tangent to the polar curve $r = 1 + 2\sin{\theta}$ at $\theta = 0$?
\end{Exercise}

\begin{Answer}[ref = polar1]
Recall that for a polar function, $\frac{dy}{dx} = \frac{\frac{dy}{d\theta}}{
\frac{dx}{d\theta}}$. We also know that $x = r \cos{\theta}$, which equals 
$\left[ 1 + 2\sin{\theta} \right] \cdot \cos{\theta} = \cos{\theta} + 2\sin{
\theta}\cos{\theta}$ in this case. And we know that $y = r \cdot \sin{\theta}$, 
which equals $\left[ 1 + 2 \sin{\theta} \right] \cdot \sin{\theta} = \sin{
\theta} + 2 \sin^2{\theta}$ in this case. Taking the derivative with respect 
to $\theta$:
$$\frac{dy}{d\theta} = \frac{d}{d\theta} \left[ \sin{\theta} + 2 \sin^2{\theta} 
\right]$$
$$\frac{dy}{d\theta} = \cos{\theta} + 4\sin{\theta}\cos{\theta}$$

And

$$\frac{dx}{d\theta} = \frac{d}{d\theta} \left[ \cos{\theta} + 2\sin{\theta}\cos{
\theta} \right]$$
$$\frac{dx}{d\theta} = -\sin{\theta} - 2\sin^2{\theta} + 2\cos^2{\theta}$$

Evaluating each at $\theta = 0$:
$$\frac{dy}{d\theta} = \cos{0} + 4\sin{0}\cos{0} = 1 + 0 = 1$$
$$\frac{dx}{d\theta} = -\sin{0} - 2\sin^2{0} + 2\cos^2{0} = 0 - 0 + 2 = 2$$

And therefore $\frac{dr}{d\theta} = \frac{dy/d\theta}{dx/d\theta} = \frac{1}{2}$
\end{Answer}

\section{Integrals of Polar Functions}
Similar to Cartesian functions, an integral of a polar function tells us the 
area within the function. We say "within" as opposed to "under" because a 
polar function describes how far from the origin the graph is based on the 
angle. Consider the graph of $r = 2\sin{\theta}$ (figure \ref{fig:2sine}). 
Geometrically, we expect the area inside the curve to be $\pi r^2 = \pi$. 
However, this is not the result we get from directly integrating the function 
(we only integrate from $\theta = 0$ to $\theta = \pi$ because the circle is 
complete when $\theta$ reaches $\pi$): 
$$\int_0^{\pi} 2\sin{\theta}\,d\theta = -2\cos{\theta}|_{\theta = 0}^{\theta 
= \pi}$$
$$= -2 \left[ \cos{\pi} - \cos{0} \right] = -2 \left[ -1 - 1 \right] = 4 \neq 
\pi$$

\begin{figure}[htbp]
\centering
    \begin{tikzpicture}
        \begin{polaraxis}[xtick = {0,0,deg((pi)/6), deg((2*pi)/6), deg((3*pi)/6), 
        deg(4*pi)/6, deg((5*pi)/6), deg((6*pi)/6), deg((7*pi)/6), deg((8*pi)/6), 
        deg((9*pi)/6), deg((10*pi)/6), deg((11*pi)/6)}, xticklabels={,0, $\frac{
        \pi}{6}$, $\frac{\pi}{3}$, $\frac{\pi}{2}$, $\frac{2\pi}{3}$, $\frac{5
        \pi}{6}$, $\pi$, $\frac{7\pi}{6}$, $\frac{4\pi}{3}$, $\frac{3\pi}{2}$,
         $\frac{5\pi}{3}$, $\frac{11\pi}{6}$}, ymax = 2.25, clip = false, xmin 
         = 0, xmax = 180]
            \addplot[blue, thick, domain = 0:360, samples = 100]{2*sin(x)};
            \addplot[white, thick, domain = 0:360, samples = 100]{2.25};
            \addplot[blue, mark=*] coordinates {(90, 1)};
            \draw[black, dashed] (90, 1) -- (135, 1.414);
            \draw[black, |-|] (90, 1.1) -- (130, 1.464);
            \node[] at (110, 1.4) {$r = 1$};
        \end{polaraxis}
    \end{tikzpicture}
    \caption{The graph of $r = 2\sin{\theta}$ is a circle of radius $2$ 
    centered at $(1, \frac{\pi}{2})$}
    \label{fig:2sine}
\end{figure}


Clearly something else is happening here. We can just take the integral of a 
Cartesian function because the area of a rectangle is the base times the 
height. When integrating Cartesian functions, the base is given by the $dx$ 
and the height by the function, $f(x)$. In polar coordinates, the integral 
sweeps across a $\theta$ interval, making a wedge, not a rectangle (see 
figure \ref{fig:sector}). 

\begin{figure}[htbp]
\centering
    \begin{tikzpicture}
        \begin{polaraxis}[xtick = {0,0,deg((pi)/6), deg((2*pi)/6), deg((3*pi)/6), 
        deg(4*pi)/6, deg((5*pi)/6), deg((6*pi)/6), deg((7*pi)/6), deg((8*pi)/6), 
        deg((9*pi)/6), deg((10*pi)/6), deg((11*pi)/6)}, xticklabels={,0, $\frac{
        \pi}{6}$, $\frac{\pi}{3}$, $\frac{\pi}{2}$, $\frac{2\pi}{3}$, $\frac{5
        \pi}{6}$, $\pi$, $\frac{7\pi}{6}$, $\frac{4\pi}{3}$, $\frac{3\pi}{2}$, 
        $\frac{5\pi}{3}$, $\frac{11\pi}{6}$}, ymax = 2.25, clip = true, xmin = 0, 
        xmax = 90]
            \addplot[blue, thick, domain = 0:180, samples = 100]{2*sin(x)};
            \addplot[white, thick, domain = 0:360, samples = 100]{2.25};
            \addplot[blue, thin] coordinates {(30,0) (30, 1)};
            \addplot[blue, thin] coordinates {(75, 0) (75, 1.932)};
            \fill[blue!30, opacity = 0.4] (30, 0) -- (30, 1) -- (37.5, 1.218) -- 
            (45, 1.414) -- (52.5, 1.587) -- (60, 1.732) -- (67.5, 1.848) -- 
            (75, 1.932) -- (75, 0) -- cycle;
            \node[red] at (30, 1.25) {$\theta_1$};
            \node[red] at (75, 2.1) {$\theta_2$};
        \end{polaraxis}
    \end{tikzpicture}
    \caption{Sweeping from $\theta_1$ to $\theta_2$ creates a sector}
    \label{fig:sector}
\end{figure}

Suppose we divided the polar function into $i$ small sectors (similar to how 
we divided Cartesian functions into many thin rectangles) (see figure fixme 
make figure). Then each small sector has a central angle $\Delta \theta$ and 
a radius $r(\theta_i^*)$, where $\theta_{i-1} < \theta_i^* < \theta_i$. What 
is the area of the $i^{th}$ sector? Recall from the chapter on circles that 
the area of a sector with angle $\theta$ and length $r$ is $A = \frac{1}{2} 
r^2 \theta$. Substituting, we see the area of the $i^{th}$ sector is:
$$A_i = \frac{1}{2} \left[ r(\theta_i^*) \right]^2 d\theta$$

And therefore the total area of the whole sector from $\theta_1$ to $\theta_2$ 
is:
$$A = \lim_{n \to \infty} \sum_{i = 1}^n \frac{1}{2} \left[ r( \theta_i^*) 
\right] ^2 \Delta \theta$$

Does this look familiar? It's the definition of an integral!
$$\lim_{n \to \infty} \sum_{i = 1}^n \frac{1}{2} \left[ r( \theta_i^*) \right] 
^2 \Delta \theta = \int_a^b \frac{1}{2} \left[ r(\theta) \right]^2\,d\theta$$

\begin{mdframed}[style=important, frametitle={Area of a Polar Function}]
The area of a polar function is given by the integral 
$$\int_a^b \frac{1}{2} r^2\,d\theta$$

Where $r$ is a function of $\theta$. 
\end{mdframed}

\textbf{Example}: The graph of $r = 3\sin{2\theta}$ is shown below. What is 
the total area enclosed by the graph?

\begin{figure}[htbp]
\centering
    \begin{tikzpicture}
        \begin{polaraxis}[xtick = {0,0,deg((pi)/6), deg((2*pi)/6), deg((3*pi)/6), 
        deg(4*pi)/6, deg((5*pi)/6), deg((6*pi)/6), deg((7*pi)/6), deg((8*pi)/6), 
        deg((9*pi)/6), deg((10*pi)/6), deg((11*pi)/6)}, xticklabels={,0, $\frac{
        \pi}{6}$, $\frac{\pi}{3}$, $\frac{\pi}{2}$, $\frac{2\pi}{3}$, $\frac{5
        \pi}{6}$, $\pi$, $\frac{7\pi}{6}$, $\frac{4\pi}{3}$, $\frac{3\pi}{2}$, 
        $\frac{5\pi}{3}$, $\frac{11\pi}{6}$}, ymax = 3.25, clip = true, xmin = 0, 
        xmax = 360]
            \addplot[blue, thick, domain = 0:360, samples = 150]{3*sin(2*x)};
            \addplot[white, thick, domain = 0:360, samples = 100]{3.25};
        \end{polaraxis}
    \end{tikzpicture}
    \caption{$r = 3\sin{2\theta}$}
    \label{fig:fourlobe}
\end{figure}

\textbf{Solution}: Since each lobe is symmetric to the others, we can find the 
area of one lobe and multiply it by four. To find the area of one lobe, we 
need to determine an interval for $\theta$ that defines one lobe. You can 
imagine each lobe being draw out from the center and then back in. So we will 
find where $r = 0$:
$$0 = 3\sin{2\theta}$$
$$\sin{2\theta} = 0$$
$$2\theta = n\pi$$
$$\theta = \frac{n\pi}{2}$$

Taking the first 2 solutions, $\theta = 0$ and $\theta = \frac{\pi}{2}$, as our 
limits of integration, we see that the area of one lobe is:
$$A_{lobe} = \frac{1}{2} \int_0^{\pi/2} \left[3 \sin{2 \theta} \right]^2\,d
\theta$$
$$A_{lobe} = \frac{9}{2} \int_0^{\pi/2} \sin^2{2\theta}\,d\theta$$

Applying the half-angle formula $\sin^2{\theta} = \frac{1 - 2\cos{2\theta}}{2}$, 
we see that:
$$A_{lobe} = \frac{9}{2} \int_0^{\pi/2} \frac{1 - 2\cos{4\theta}}{2}\,d\theta$$
$$A_{lobe} = \frac{9}{4} \int_0^{\pi/2} 1 - 2\cos{4\theta}\,d\theta$$
$$A_{lobe} = \frac{9}{4} \left[ \theta - \frac{\sin{4\theta}}{2} \right]_{0}^{
\pi/2}$$
$$A_{lobe} = \frac{9}{4} \left[ \left( \frac{\pi}{2} - \frac{\sin{4\pi/2}}{2} 
\right) - \left(0 - \frac{\sin{0}}{2} \right) \right]$$
$$A_{lobe} = \frac{9}{4} \left[ \frac{\pi}{2} - 0 - 0 + 0 \right] = \frac{9
\pi}{8}$$

Since the area of one lobe is $\frac{9\pi}{8}$, the area of all four lobes is 
$\frac{9\pi}{2}$. 

\begin{Exercise}[label = polar2]
[This question was originally presented as a multiple-choice, calculator-
allowed problem on the 2012 AP Calculus BC exam.] The figure below shows the 
graphs of polar curves $r = 2\cos{3\theta}$ and $r = 2$. What is the sum of 
the areas of the shaded regions to three decimal places?
\begin{center}
\begin{tikzpicture}
	\begin{polaraxis}[ymax = 2.25, clip = false, xtick = {0, 30, 60, 90, 120, 150, 
	180, 210, 240, 270, 300, 330}, xticklabels = {$0$, $\frac{\pi}{6}$, $\frac{\pi
	}{3}$, $\frac{\pi}{2}$, $\frac{2\pi}{3}$, $\frac{5\pi}{6}$, $\pi$, $\frac{7\pi
	}{6}$, $\frac{4\pi}{3}$, $\frac{3\pi}{2}$, $\frac{5\pi}{3}$, $\frac{11\pi}{6}$
	}]
            \addplot[white, thick, domain = 0:360, samples = 200]{2.25};
		\addplot[name path = A, blue, thick, domain = 0:180, samples = 200]{
		2*cos(3*x)};
        \filldraw[gray!30, opacity = 0.4](0,2) -- (5, 1.932) -- (10, 1.732) -- 
        (15, 1.414) -- (20, 1) -- (25, 0.518) -- (30, 0) -- (90, 0) -- 
        (95, 0.518) -- (100, 1) -- (105, 1.414) -- (110, 1.732) -- 
        (115, 1.932) -- (120, 2) -- (110, 2) -- (100, 2) -- (90, 2) -- (80, 2) 
        -- (70, 2) -- (60, 2) -- (50, 2) -- (40, 2) -- (30, 2) -- (20, 2) -- 
        (10, 2) -- cycle;
        \filldraw[gray!30, opacity = 0.4](120, 2) -- (125, 1.932) -- 
        (130, 1.732) -- (135, 1.414) -- (140, 1) -- (145, 0.518) -- (150, 0) 
        -- (30, 0) -- (35, -0.518) -- (40, -1) -- (45, -1.414) -- (50, -1.732) 
        -- (55, -1.932) -- (60, -2) -- (240, 2) -- (230, 2) -- (220, 2) -- 
        (210, 2) -- (200, 2) -- (190, 2) -- (180, 2) -- (170, 2) -- (160, 2) 
        -- (150, 2) -- (140, 2) -- (130, 2) -- cycle;
        \filldraw[gray!30, opacity = 0.4](240, 2) -- (245, 1.932) -- 
        (250, 1.732) -- (255, 1.414) -- (260, 1) -- (265, 0.518) -- (270, 0) 
        -- (330, 0) -- (335, 0.518) -- (340, 1) -- (345, 1.414) -- 
        (350, 1.732) -- (355, 1.932) -- (360, 2) -- (350, 2) -- (340, 2) -- 
        (330, 2) -- (320, 2) -- (310, 2) -- (300, 2) -- (290, 2) -- (280, 2) 
        -- (270, 2) -- (260, 2) -- (250, 2) -- cycle;
	\end{polaraxis}
    \end{tikzpicture}
\end{center}
\end{Exercise}

\begin{Answer}[ref = polar2]
We know the area of the circle is $\pi r^2 = \pi(2)^2 = 4\pi$. To find the 
area of the shaded regions, we need to subtract the area of the trefoil from 
the area of the circle. The trefoil has three equal areas. We can find the 
area of the leaf that is formed on the interval $\frac{\pi}{6} \leq \theta 
\leq \frac{\pi}{2}$ (see figure below)

\begin{center}
  \begin{tikzpicture}
	\begin{polaraxis}[ymax = 2.25, clip = false, xmin = 180, xmax = 270, 
	xtick = {180, 210, 240, 270}, xticklabels = {$\pi$, $\frac{7\pi}{6}$, 
	$\frac{4\pi}{3}$, $\frac{3\pi}{2}$}]
            \addplot[white, thick, domain = 0:360, samples = 200]{2.25};
		\addplot[blue, thick, domain = 30:90, samples = 50]{2*cos(3*x)};
	\end{polaraxis}
\end{tikzpicture}  
\end{center}

The area of one leaf of the trefoil is given by $\frac{1}{2} \int_{\pi/6}^{
\pi/2} \left[2\cos{3\theta} \right]^2\,d\theta$. Using a calculator, the area 
of one leaf is $\approx 1.0472$. The area of the circle is given by $\pi r^2 = 
\pi (2)^2 \approx 12.5664$. Then the area of the shaded region is the area of 
the circle minus three times the area of a single leaf: $12.5664 - 3 \cdot 
1.0472 = 9.4248 \approx 9.425$. 
\end{Answer}