\chapter{Sailboats}

Imagine that you have a canoe, and you are about to paddle from one island to another that is directly east of where you are standing.  Imagine, also, that there 
is a steady wind coming from the west, and you have a big piece of plywood.  You might be inspired to use it as a sail.

This situation is the most simple form of sailing: Wind comes from behind the boat and hits the sail which generates a force that pushes the 
boat in the direction of the wind.

The sail has two sides:  The \newterm{windward} side is the one that is getting hit with the wind.  The \newterm{leeward} side is the side away from the wind.

\section{Magnitude of the Wind Force}

The first natural question is: How much force will I have pushing my canoe through the water?

\begin{mdframed}[style=important, frametitle={Wind Force}]

When the sail is perpendicular to the wind,  the force of the wind on the sail in newtons ($F_w$) will be given by:

$$F_w = A \frac{d v^2}{2}$$

where $A$ is the area of the sail in square meters  $d$ is the density of the gas in kg per cubic meter, and $v$ is the wind speed in meters per second.

For air at STP,  $d$ is about 1.225 kg per cubic meter.

We call $\frac{d v^2}{2}$  the \newterm{wind pressure}.   It is the amount of pressure that the windward side of plywood is experiencing that is above the 
 pressure that the leeward side of the plywood is experiencing.  (The leeward side might experience some turbulence,  but the pressure it is experiencing is 
 approximately 1 atmosphere.)

\end{mdframed}

Let's your canoe is standing still and the wind is 0.5 m/s.  Then the wind pressure is

$$P =  \frac{1.225 (0.5^2)}{2} =  0.153125 \text{ newtons per square meter}$$

Let's say your plywood sail is 2 meters tall and 1.5 meters wide.  What will be the force of the wind?

$$F_w  = A P = (3)(0.153125) \approx 0.46 \text{ newtons}$$

This is a very intuitive idea:  There is a difference between the pressure on the windward side and the pressure on the leeward side,  and the plywood experiences a force that pushes
the boat through the water.

\section{Direction and Location of the Wind Force}

If there is low pressure on one side of the sail and high pressure on the other,  the force vector will be perpendicular to the sail. 
 
Where is this force vector applied?  We can think of the force as being applied at the geometric center of the sail.  This is called \newterm{the center of effort}.   In this case,  the center of effort is the exact center of the rectangular plywood.

The mast on a windsurfing board can be tilted from side to side.  When the center of effort is over the center of lateral resistance,  the board goes straight.
To steer,   the sailor moves the mast to one side of the center of lateral resistance which rotates the board.

\section{Beam Reach}

When you are sailing in the same direction as the wind,  sailors say you are \newterm{running}.   What if you want to go east and now the wind (still 0.5 m per second) is coming from the south?  Sailing perpendicular to the wind is known as a \newterm{beam reach}.

To do a beam reach,  instead of mounting the plywood perpendicular to the boats direction of travel,  you would mount it at a 45 degree angle.   The wind pressure will build on the windward
side of the plywood,  and the plywood will experience a force pushing it at a $45^\circ$ angle to the boat.

We can think of this force as having two components: 
\begin{itemize}
\item One component pushes the boat forward (Yay!)
\item One component pushes the boat sideways (Ugh.)
\end{itemize}

To minimize the effect of the sideways force,  sailboats typically have  a keel -- a long fin on its underside that slows its sideways sliding.

Notice also that the "wind shadow" of the plywood is smaller when it is at a $45^\circ$ angle to the wind.  How much smaller?  The effective area of your plywood has 
gone from 3 square meters to $\frac{3}{\sqrt{2}} \approx 2.12$ square meters.   So the force generated will be smaller, and some of it will be wasted pushing the boat sideways.

If we assume that the wind pressure is still 0.153125 newtons per square meter.  The force on the plywood will be about

$$F_w = A P = \frac{3}{\sqrt{2}}(0.154125) \approx 0.325 \text{ newtons}$$

However,   the direction of that force is not all in the direction you want to go,  so the effective force is:

$$F = F_w \frac{1}{\sqrt{2}} =  \frac{3}{2}(0.154125) = 0.2311875 \text{ newtons}$$

Notice that we got twice as much effective force when we were running with the wind as when we are on a beam reach.   However,  any sailor will tell you that you can go much faster
on a beam reach than you can running.  Why?

\section{Apparent Wind}

When you are running,  you can never go faster than the wind.   As you go faster and faster,  the wind that the boat experience decreases.   For example,  if you are going 0.2 m/s in a wind 
of 0.5 m/s,  you (and your sail) will only experience wind at 0.3 m/s.We call the wind as experienced by the boat 
the \newterm{apparent wind}.  The wind as observed by a stationary observer is called the \newterm{actual wind}.

If you are running with the wind,   as you approach the speed of the wind,  the force of that wind will decrease towards zero.

On a beam reach, as you go faster,  the direction of the wind seems to change.  If you are going 0.2 m/s east and the actual wind is 0.5 m/s from the south, the direction of the 
apparent wind will seem to come from about 22 degrees east of true south.  The speed of the apparent wind will be about 0.54 m/s.

\section{Close Reach}

What if you want to go east,  and the wind is coming from 40 degrees east of south?  This would mean that you were sailing just 50 degrees away from 
straight into the wind.   Is this possible?

If you put your sail at a 25 degree angle,   you will still catch some wind and create some pressure on one side of the sail.  Most of the resulting force would be trying to push
your boat sideways,  but some of it would be in the direction you were trying to travel. 

Picking an angle for your sail that creates high pressure that makes a desirable force is known as the "angle of attack".

This is a non-intuitive result: A boat can sail into the wind!?  The boat can't sail directly into the wind -- with each degree that the boat gets closer
to straight into the wind,  the force pushing it forward decreases and the force pushing it back increases.   However,  most boats can get within 45\% if they have a well-shaped sail.

\section{Shaping the Sail}

Most of the power of the wind can be captured with a piece of flat plywood.  The wind hits it and creates a high pressure on the windward side.   What about the other side of the plywood?

It turns out that if we can get the wind to travel smoothly over the back side of the plywood,   the pressure on that side will be a little lower than if we had turbulence there.   (I'm not going
to go into this too deeply into why.  If you want to know more,  look up the Coanda effect.)

For example,   if we were on a close reach,   the very best sail we could have would gently pull the wind along its backside.  It would look like this:

Of course,   for the sail to work on either side of the boat,  this asymmetrical design would not work.  (Although,  we should note that this design works great
for airplane wings.)

When we make a sail out of cloth,  we give it some curve known as \newterm{camber}.  Slow winds require just a little camber,  fast winds require more.

Some newer sailboats have wing sails that have two pieces that can be arranged to redirect the most air possible.

Note: when running with the wind,  the turbulence on the leeward side of the sail is unavoidable.   But when traveling perpendicular to the wind or on a close reach,  the air should move smoothly over the leeward side of the sail.   Many sailors have a piece of yarn taped on each side of the sail so they can see if the air is moving smoothly.

Many sailboats also have multiple sails.  Besides the increase in the sail area,  each sail also redirects the wind to pass smoothly over the leeward surface of the sail
behind it.

\section{Tacking into the Wind}

What if the wind is coming from the east and you really need to go directly into the wind?  

The boat will not travel straight into the wind,  instead you will travel on a close reach with the wind coming from one side of the boat.   Then you will turn into the wind and continue turning
until you are on a close reach with the wind coming from the other side.  This is known as \newterm{tacking}.

\section{Heeling}

The center of effort is near the center of your sail,  which is usually pretty high in the air.   Yes, the force generated will push your boat,  but it will also rotate your boat.  Sailors call
the resulting lean \newterm{heeling}.    Heeling too far is problematic -- the rudder gets pulled out of the water, which makes it hard to steer the boat.

If a boat is heeling too far,  sailors will move their weight to the windward side of the boat -- some even wear harnesses that them push their weight out beyond the edge of the hull.  
If that doesn't work, they will reduce the sideways force on the sail.

Can the boat flip-over?  (The word sailors use is \newterm{capsize}.)  Most larger boats should not be able to capsize -- as the sail gets pushed down toward the water,  it loses power.  So putting some weight in the keel will ensure that the boat doesn't turn upside-down.  Small boats (think 3 or 4 meters long) can capsize,  but they are small enough that the sailor can usually
get them upright again without assistance.

There are boats with multiple hulls.   A catamaran has two identical hulls that are side-by-side.  As a result,  the catamaran will not heel much at all.  However,  if a big gust of wind comes up, one hull can be pulled completely out of the water.  Catamarans can be capsized.

If a boat is running when a big gust comes up,  that rotating force will try to push the bow underwater.   If the boat is going very fast,  this can result in a somersault as 
the front of the boat slows and dips suddenly and the back of the boat it pitched up over it.


