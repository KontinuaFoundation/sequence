\chapter{The Multivariate Normal Distribution}

The multivariate normal distribution is a generalization of the one-dimensional (univariate) normal distribution to higher dimensions. It is used in statistics to describe any set of correlated real-valued random variables.

\section{Multivariate Normal Distribution}
A random vector $X = [X_1, X_2, ..., X_n]^T$ follows a multivariate normal distribution if every linear combination of its components has a univariate normal distribution. The distribution is parameterized by a mean vector and a covariance matrix.

The probability density function (pdf) of an $n$-dimensional multivariate normal distribution is given by:

\begin{equation*}
f(\mathbf{x}|\boldsymbol\mu, \mathbf{\Sigma}) = \frac{1}{\sqrt{(2\pi)^n|\mathbf{\Sigma}|}}\exp\left(-\frac{1}{2}(\mathbf{x}-\boldsymbol\mu)^T\mathbf{\Sigma}^{-1}(\mathbf{x}-\boldsymbol\mu)\right)
\end{equation*}

where:
\begin{itemize}
\item $\mathbf{x} = [x_1, x_2, ..., x_n]^T$ is the point up to which the function is integrated,
\item $\boldsymbol\mu = [\mu_1, \mu_2, ..., \mu_n]^T$ is the mean vector,
\item $\mathbf{\Sigma}$ is the covariance matrix,
\item $|\mathbf{\Sigma}|$ denotes the determinant of the covariance matrix,
\item $T$ denotes the matrix transpose.
\end{itemize}

\section{Covariance Matrix}
The covariance matrix, $\mathbf{\Sigma}$, is a symmetric matrix that contains information about the variance of each variable and the covariance between every pair of variables in the distribution.

The element $\Sigma_{ij}$ is the covariance between the $i$-th and the $j$-th random variable, and $\Sigma_{ii}$ is the variance of the $i$-th random variable.

The covariance matrix provides a measure of how much each of the dimensions varies from the mean with respect to each other. A positive covariance between two variables indicates that the variables increase or decrease together, whereas a negative covariance indicates that one variable increases when the other decreases.