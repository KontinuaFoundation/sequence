\chapter{Bayesian Networks}

A Bayesian network, also known as a Bayes network, belief network, or
decision network, is a probabilistic graphical model that represents a
set of variables and their conditional dependencies via a directed
acyclic graph (DAG).\index{Bayesian Network}

\section{Components}

A Bayesian Network consists of two main components:

\begin{enumerate}
    \item A directed acyclic graph (DAG) where each node represents a
      variable, and the absence or presence of a directed edge between
      nodes denotes the conditional dependence or independence
      respectively between the variables.
    \item A conditional probability table (CPT) associated with each
      node which contains the conditional probability distribution of
      that node given its parents in the DAG.
\end{enumerate}

\section{Inferences}

Bayesian Networks are typically used for reasoning and making
inferences under uncertainty. Given observations of a set of
variables, we can compute the posterior probabilities of the other
variables using Bayes' rule.

There are three main types of inferences that we can make:

\begin{itemize}
    \item \textbf{Causal reasoning (prediction)}: Given the causes,
      what are the effects?
    \item \textbf{Evidential reasoning (diagnosis)}: Given the
      effects, what are the causes?
    \item \textbf{Intercausal reasoning (explaining away)}: Given an
      effect and some of its causes, what can we say about the other
      causes?
\end{itemize}

\section{Learning}

Learning a Bayesian Network from data involves two main tasks:

\begin{itemize}
    \item \textbf{Structure learning}: Determining the DAG structure
      that best fits the data.
    \item \textbf{Parameter learning}: Estimating the parameters
      (conditional probabilities) of the CPTs given the DAG and data.
\end{itemize}
