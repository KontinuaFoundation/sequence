\chapter{Diodes, Rectifiers, and Photovoltaics}

Early in this course, we discussed some basic electronics components: resistors, capacitors, and inductors. In this chapter, we are going to introduce a semiconductor: \newterm{Diodes} ensure that current flows in only one direction.\index{diode}

You will learn a lot about semiconductors from studying diodes.  We can then apply what you have learned to solar cells, transistors, and integrated circuits like CPUs.

We are going to focus on semiconductors made of silicon. You can make semiconductors out of other materials, but silicon has proved to be very versatile and cost-effective.

\section{Silicon}

A silicon atom has 14 protons and 14 electrons.  The inner 10 electrons are arranged in two complete and stable shells.  The outer four electrons are in an incomplete shell, which allows them to form bonds with other atoms. We call these \newterm{valence electrons} because they aren't tightly bound to their atoms. A complete outer shell would have eight electrons.\index{silicon}

Atoms can share electrons to get complete shells.  This is called \newterm{covalent bonding}.  When atoms share electrons, they form a \newterm{covalent bond}.  The electrons are shared between the atoms, and the atoms are held together by the shared electrons.

In a silicon crystal, each silicon atom shares its outer four electrons with four neighboring atoms. With its four electrons and an electron borrowed from each of it four neighbors, each silicon atom has eight electrons in its outer shell. This is a complete shell, and the atoms are stable. This is why glass is hard and stable.  It also why electricity doesn't flow through glass -- all the electrons are tightly bound to their atoms.

Phosphorus has five valence electrons. If you put a phosphorus atom into silicon crystal, it shares four of its electrons with its neighbors, but it has one electron left over after the complete shell of eight is built. As you add more phosphorus to a silicon crystal, it conducts more electricity.  We call this \newterm{doping}.\index{phosphorus}\index{doping}

\begin{figure}[H]
    \centering
    \includegraphics[width=0.5\textwidth]{doping-27.png}
    \caption{Phosphorus atom in a silicon crystal}
    \label{fig:n-doping}
\end{figure}

Because this kind of doping frees up some electrons, which are negatively charged, we call this \newterm{n-doping}.\index{n-doped}

Boron has only three valence electrons. If you put a boron atom into silicon crystal, it shares three of its electrons with its neighbors, which leaves a "hole" that would like to be filled by an electron. We call this \newterm{p-doping}.\index{p-doped}\index{boron}

\begin{figure}[H]
    \centering
    \includegraphics[width=0.5\textwidth]{doping-26.png}
    \caption{Boron atom in a silicon crystal}
    \label{fig:p-doping}
\end{figure}

Both n-doped and p-doped silicon are conductors of electricity.  In n-doped silicon, we say the loose electrons are the \newterm{carriers} of the current. In p-doped silicon, we say the holes are the carriers.

Note that doped silicon is still mostly silicon. In normal doping, maybe 1 in 10,000 atoms are replaced with boron or phosphorus.

\section{Diodes}

A diode is a semiconductor device that allows current to flow in one direction only.

\begin{figure}[H]
    \centering
    \includegraphics[width=0.4\textwidth]{diode_Diode-14.png}
    \includegraphics[width=0.4\textwidth]{diode_Diode-15.png}
    \caption{Current flows in one direction through the diode, not in the other direction.}
    \label{fig:go-diode}
\end{figure}

Diodes can be made with silicon that has been n-doped and p-doped. A diode looks like this:

\begin{figure}[H]
    \centering
    \includegraphics[width=0.4\textwidth]{diodeReal.png}
    \caption{What Diodes Look Like}
    \label{fig:diode-real}
\end{figure}

We say that current flows from the positive terminal to the negative terminal of a battery. However, remember that the electrons are actually moving from the negative terminal to the positive terminal. This is a common source of confusion in any discussion of diodes.

One end of the diode is called the \newterm{anode}.  The anode is p-doped.  The other end is called the \newterm{cathode}.  The cathode is n-doped.\index{anode}\index{cathode}

\begin{figure}[H]
    \centering
    \includegraphics[width=0.4\textwidth]{diodeDiagram2.png}
    \caption{A p-n Diode}
    \label{fig:diode-real}
\end{figure}

The region close to where the anode and cathode meet is called the \newterm{depletion region}. In this space, the spare electrons in the cathode (the n-doped material) migrate over to fill the holes in anode (the p-doped material). This creates an electric field: there are more electrons than protons on the anode side, and more protons than electrons on the cathode side. (How strong is this field? It is typically about 0.7 volts.)\index{depletion region}

\begin{figure}[H]
    \centering
    \includegraphics[width=0.4\textwidth]{diodeProcess-31.png}
    \caption{The depletion region in a diode}
    \label{fig:diode-depletion}
\end{figure}

What happens if we overpower that 0.7 volts? What if we put 2 volts in the opposite direction? This shrinks the depletion region and the diode becomes a conductor. For example, if the negative end of a battery is connected to the cathode of a diode and the positive end is connected to the anode, current will flow through the diode.

\begin{figure}[H]
    \centering
    \includegraphics[width=0.4\textwidth]{diodeProcess-32.png}
    \caption{Electrons flow through the diode}
    \label{fig:diode-flow}
\end{figure}

If we reverse the battery, the depletion region expands, and the diode becomes an insulator.

\begin{figure}[H]
    \centering
    \includegraphics[width=0.4\textwidth]{diodeProcess-33.png}
    \caption{Reverse the electric field? No flow.}
    \label{fig:diode-no-flow}
\end{figure}

Here is the symbol for a diode:

\begin{figure}[H]
    \centering
    \includegraphics[width=0.4\textwidth]{Diode_symbol.png}
    \caption{Diode Symbol}
    \label{fig:diode-symbol}
\end{figure}

Current will flow in the direction of the arrow. It will not flow in the opposite direction.

\subsection{Rectifiers}

One of the really useful things we can do with a diode is to convert alternating current (AC) into direct current (DC). This is called rectification. Remember that AC pushes and pulls the electrons back and forth in the wire. Rectification ensures that the electrons only move in one direction.\index{rectifier}

A \newterm{half-wave rectifier} uses a single diode to ensure that the current flows in one direction through the load.

\begin{figure}[H]
    \centering
    \includegraphics[width=0.4\textwidth]{halfwave.png}
    \caption{Half-wave Rectifier}
    \label{fig:halfwave-rectifier}
\end{figure}

Current flows in the direction of the arrow on the diode. If the voltage is reversed, the diode will block the current from moving.\index{half-wave rectifier}

A \newterm{full-wave rectifier} uses four diodes to swap the direction of the current through the load when necessary.\index{full-wave rectifier}

\begin{figure}[H]
    \centering
    \includegraphics[width=0.4\textwidth]{fullwave.png}
    \caption{Full-wave Rectifier}
    \label{fig:fullwave-rectifier}
\end{figure}

\subsection{Photovoltaics}

Photovoltaics (or "Solar cells") are similar to diodes. They have a p-n junction with a 0.7 volt electrical field. The difference is that sunlight is allowed to shine on the cathode side. When an atom is struck by a photon of light energy, the electron becomes excited.  Sometimes is it excited enough to leap across that 0.7 volt barrier.\index{solar cells}\index{photovoltaics}

When it leaps across, there is an extra electron on the anode side of the barrier and an extra proton on the cathode side. The electron is pushed back to where it came from, but it can't go back over the barrier (unless more than 0.7 volts of charge have built up across the barrier).  So, it goes the long way: through the load.

\begin{figure}[H]
    \centering
    \includegraphics[width=0.4\textwidth]{solarcell.png}
    \caption{Solar Cell}
    \label{fig:solarcell}
\end{figure}
