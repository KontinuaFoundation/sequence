\chapter{Continuous Probability Distributions and Cumulative Density Functions}

In statistics, a probability distribution describes how the values of a random variable are distributed. For continuous random variables, the probability distribution can be described by a probability density function (pdf), while the cumulative distribution function (cdf) gives the probability that the random variable is less than or equal to a certain value.

\section{Continuous Probability Distributions}

A continuous probability distribution is a probability distribution that has a pdf, which is a function that provides the probabilities of occurrence of different possible outcomes in an experiment. For a continuous distribution, the pdf $f(x)$ is such that for any two numbers $a$ and $b$ with $a \leq b$:

\begin{equation}
P(a \leq X \leq b) = \int_{a}^{b} f(x) \, dx
\end{equation}

Note that for a pdf $f(x)$, we have:

\begin{equation}
f(x) \geq 0 \quad \text{for all } x
\end{equation}

and

\begin{equation}
\int_{-\infty}^{\infty} f(x) \, dx = 1
\end{equation}

\section{Cumulative Density Functions}

The cumulative distribution function of a random variable $X$, denoted by $F(x)$, is defined as the probability that $X$ will take a value less than or equal to $x$:

\begin{equation}
F(x) = P(X \leq x) = \int_{-\infty}^{x} f(t) \, dt
\end{equation}

Note that the cdf is always a non-decreasing function, and:

\begin{equation}
\lim_{x\to-\infty} F(x) = 0 \quad \text{and} \quad \lim_{x\to\infty} F(x) = 1
\end{equation}

\section{Example: Normal Distribution}

One common example of a continuous probability distribution is the normal (or Gaussian) distribution, characterized by its bell-shaped curve. The pdf of a normal distribution with mean $\mu$ and standard deviation $\sigma$ is given by:

\begin{equation}
f(x) = \frac{1}{\sigma\sqrt{2\pi}} e^{ -\frac{1}{2} \left(\frac{x-\mu}{\sigma}\right)^2 }
\end{equation}

The cdf of the normal distribution cannot be expressed in terms of elementary functions and is often computed using numerical methods. However, it can be represented as:

\begin{equation}
F(x) = \frac{1}{2}[1 + \text{erf}(\frac{x - \mu}{\sigma\sqrt{2}})]
\end{equation}

where $\text{erf}(x)$ is the error function.

\