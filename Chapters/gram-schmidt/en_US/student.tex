\chapter{The Gram-Schmidt Process}

The Gram-Schmidt process is a method for orthonormalizing a set of
vectors in an inner product space, most commonly the Euclidean space
$\mathbb{R}^n$. The process takes a finite, linearly independent set
$S = \{v_1, v_2, \ldots, v_k\}$ for $k \leq n$, and generates an
orthogonal set $S' = \{u_1, u_2, \ldots, u_k\}$ that spans the same
k-dimensional subspace of $\mathbb{R}^n$ as $S$.\index{Gram-Schmidt process}

\section{Process}

Given a set of vectors $S = \{v_1, v_2, \ldots, v_k\}$, the
Gram-Schmidt process is as follows:

\begin{enumerate}
    \item Let $u_1 = v_1$.
    \item For $j = 2, 3, \ldots, k$:
    \begin{enumerate}
        \item Let $w_j = v_j - \sum_{i=1}^{j-1} \frac{\langle v_j, u_i \rangle}{\langle u_i, u_i \rangle} u_i$
        \item Let $u_j = w_j$
    \end{enumerate}
\end{enumerate}

Here, $\langle . , . \rangle$ denotes the inner product. 

\section{Orthonormalization}

The set of vectors $S' = \{u_1, u_2, \ldots, u_k\}$ obtained from the
process above is orthogonal, but not necessarily orthonormal. To form
an orthonormal set, we simply need to normalize each vector $u_i$ to
unit length. That is, $u_i' = \frac{u_i}{\|u_i\|}$, where $\|.\|$
denotes the norm (or length) of a vector.


