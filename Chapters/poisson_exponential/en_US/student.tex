\chapter{Poisson and Exponential Probability Distributions}

In this chapter, we will explore two essential probability
distributions: the Poisson distribution and the exponential
distribution. These distributions play a crucial role in modeling
random events and phenomena, providing insights into the occurrence of
events over time or in a discrete set of outcomes.

The Poisson distribution is widely used to describe the number of
events that occur within a fixed interval of time or space. It is
particularly useful when dealing with rare events or events that occur
independently at a constant average rate. For example, the Poisson
distribution can model the number of customer arrivals at a store in a
given hour, the number of phone calls received by a call center in a
day, or the number of defects in a production process.

The Poisson distribution is characterized by a single parameter, often
denoted as $\lambda$, which represents the average rate of event
occurrences in the specified interval. The probability mass function
of the Poisson distribution gives the probability of observing a
specific number of events within that interval.

On the other hand, the exponential distribution is used to model the
time between events occurring at a constant average rate. It is
commonly employed in reliability analysis, queuing theory, and
survival analysis. For example, the exponential distribution can
represent the time between customer arrivals at a service desk, the
lifespan of electronic components, or the duration between consecutive
earthquakes.

The exponential distribution is characterized by a parameter often
denoted as $\lambda$, which represents the average rate of event
occurrence. The probability density function of the exponential
distribution describes the likelihood of observing a specific time
interval between events.

In this chapter, we will explore the following key aspects of the
Poisson and exponential probability distributions:

\begin{itemize}
\item Probability mass function and probability density function: We
  will dive into the mathematical representation of these
  distributions and learn how to calculate probabilities and densities
  for specific events or time intervals.

\item Mean and variance: We will discuss how to calculate the mean and
  variance of the Poisson and exponential distributions, providing
  measures of central tendency and variability.

\item Applications and examples: We will examine real-world scenarios
  where these distributions find practical applications. From
  analyzing customer arrival patterns to modeling equipment failure
  rates, we will explore a range of contexts where the Poisson and
  exponential distributions prove valuable.

\item Relationship between the Poisson and exponential distributions:
  We will explore the connection between these distributions, as the
  exponential distribution can emerge as the waiting time between
  events following a Poisson process.

\item Limitations and assumptions: We will also discuss the
  assumptions and limitations associated with the Poisson and
  exponential distributions, helping you understand when these models
  are suitable and when alternative approaches may be necessary.
\end{itemize}

By developing a solid understanding of the Poisson and exponential
probability distributions, you will gain powerful tools for modeling
and analyzing random events in various fields. These distributions
provide valuable insights into event occurrences, time intervals, and
rates, supporting decision-making processes and improving our
understanding of stochastic phenomena.
