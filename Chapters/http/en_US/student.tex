\chapter{Making Web Requests with HTTP}

The Hypertext Transfer Protocol (HTTP) is the protocol used for
transmitting hypertext over the World Wide Web. It is the foundation
of any data exchange on the web, and it is a protocol used for
transmitting hypertext requests from clients (like a user's browser)
to servers, which respond with the requested resources.\index{HTTP}

\section{HTTP Requests}

An HTTP request is made up of several components:

\begin{itemize}
    \item \textbf{Method}: The HTTP method, like GET (retrieve data), POST (send data), PUT (update data), DELETE (remove data), and so on.
    \item \textbf{URL}: The URL of the resource to retrieve, send data to, update or delete.
    \item \textbf{Headers}: Additional information about the request or response, like the content type of the body.
    \item \textbf{Body}: The body of the request, used when sending data in POST or PUT requests.
\end{itemize}

\section{Using HTTP with Web-Based APIs}

Software developers often use HTTP to interact with web-based APIs. An
Application Programming Interface (API) is a set of rules that allows
programs to talk to each other. The developer creates the API on the
server and allows the client to talk to it.

When a developer makes a request to an API endpoint, they're asking
the server to either send them some data or receive some data from
them. The response from the server will often be in a format like JSON
or XML, which the developer can then use in their own application.

For example, a developer might make a GET request to
`\texttt{https://api.example.com/users}` to retrieve a list of all
users. The server would respond with a list of users in a format like
JSON.

