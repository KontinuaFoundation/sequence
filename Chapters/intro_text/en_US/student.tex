\chapter{Introduction to Text}


In computer systems, text is represented in files as a sequence of
characters, each of which corresponds to a specific number known as a
character code. These character codes are then stored in the file as
binary data.\index{text}

\section{Newlines and Carriage Returns}

Two of the character codes that have special meanings are the newline
(often represented as '\textbackslash n') and the carriage return
(often represented as '\textbackslash r').

The newline character signifies the end of a line of text and the
beginning of a new one. The carriage return character moves the cursor
to the beginning of the line. The use of these characters can vary
between operating systems. Unix-based systems (like Linux and MacOS)
use the newline character to indicate the end of a line, while Windows
systems use a combination of a carriage return and a newline
('\textbackslash r\textbackslash n').

\section{ASCII}

The American Standard Code for Information Interchange (ASCII) is one
of the earliest character encodings. It uses 7 bits to represent each
character, allowing it to define up to $2^7 = 128$ different
characters. These include the English alphabet (in both lower and
upper cases), digits, punctuation symbols, control characters (like
newline and carriage return), and some other symbols.

\section{UTF-8}

UTF-8 (8-bit Unicode Transformation Format) is a variable-width
character encoding that can represent every character in the Unicode
standard, yet remains backward-compatible with ASCII. For the ASCII
range (0-127), UTF-8 is identical to ASCII. But it can use additional
bytes (up to 4 bytes in total) to represent characters that are not
included in ASCII, such as characters from other languages, emojis,
and many other symbols. This has made UTF-8 a widely used encoding in
many modern systems.
