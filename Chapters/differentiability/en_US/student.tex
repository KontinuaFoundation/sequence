\chapter{Differentiation}

We have done some differentiation, but you haven't been given the real
definition because it is based on limits.

The idea is that we can find the slope between two points on the graph
$a$ and $b$ like this:

$$m = \frac{f(b) - f(a)}{b - a}$$

\begin{tikzpicture} [scale=2]
\draw[->,thick,sdkblue] (0,0) -- (4,0) node[right] {$x$};
\draw[->,thick,sdkblue] (0,0) -- (0, 3.45) node[above] {$y$};
% curve
\draw[<->,thick,draw=black, domain=0:3.7,samples=300,variable=\x]  
plot (\x,{\x * \x * 0.25});
\draw [sdkblue] (1, 0.25) -- (3, 2.251);
\draw [dashed] (1,0.25) -- (1, 0) node[below]{$a$};
\draw [dashed] (3, 2.251) -- (3, 0) node[below]{$b$};
\draw [dashed] (1,0.25) -- (3, 0.25) node[right] {$f(a)$};
\draw (3,2.251) node[right] {$f(b)$};
\draw (1,0.25) circle (0.05);
\draw (3, 2.251) circle (0.05);
\draw (2, 0.25) node[above]{$b - a$};
\draw (3, 1.25) node[right] {$f(b) - f(a)$};
\end{tikzpicture}

If we want to find the slope at $a$ we take the limit of this as the 
$b$ goes to $a$:

$$f'(a) = \lim_{b \rightarrow b}\frac{f(b) - f(a)}{b - a}$$

This idea is usually expressed using $\Delta x$ as the difference 
between $b$ and $a$:

\begin{tikzpicture} [scale=2]
\draw[->,thick,sdkblue] (0,0) -- (4,0) node[right] {$x$};
\draw[->,thick,sdkblue] (0,0) -- (0, 3.45) node[above] {$y$};
% curve
\draw[<->,thick,draw=black, domain=0:3.7,samples=300,variable=\x]  
plot (\x,{\x * \x * 0.25});
\draw [sdkblue] (1, 0.25) -- (3, 2.251);
\draw [dashed] (1,0.25) -- (1, 0) node[below]{$a$};
\draw [dashed] (3, 2.251) -- (3, 0) node[below]{$a + \Delta x$};
\draw [dashed] (1,0.25) -- (3, 0.25) node[right] {$f(a)$};
\draw (3,2.251) node[right] {$f(a + \Delta x)$};
\draw (1,0.25) circle (0.05);
\draw (3, 2.251) circle (0.05);
\draw (2, 0.25) node[above]{$\Delta x$};
\draw (3, 1.25) node[right] {$f(a + \Delta x) - f(a)$};
\end{tikzpicture}

Then the formula becomes:

$$f'(a) = \lim_{\Delta x \rightarrow 0}\frac{f(a + \Delta x) - f(a)}{\Delta x}$$

Now, at any point $a$ we can compute the slope of the line tangent to 
the function at $a$:

\begin{tikzpicture} [scale=2]
\draw[->,thick,sdkblue] (0,0) -- (4,0) node[right] {$x$};
\draw[->,thick,sdkblue] (0,0) -- (0, 3.45) node[above] {$y$};
% curve
\draw[<->,thick,draw=black, domain=0:3.7,samples=300,variable=\x]  
plot (\x,{\x * \x * 0.25});
\draw [sdkblue] (0, -0.25) -- (3, 1.25) node[midway, right]{slope = $f'(a)$};;
\draw [dashed] (1,0.25) -- (1, 0) node[below]{$a$};
\draw (1,0.25) circle (0.05);
\end{tikzpicture}

\section{Differentiability}

Warning: Not every function is differentiable everywhere.  For
example, if $f(x) = |x|$, you get a corner at zero.

\begin{tikzpicture} [scale=1]
\draw[->,thick,sdkblue] (-4,0) -- (4,0) node[right] {$x$};
\draw[->,thick,sdkblue] (0,0) -- (0, 3.45) node[above] {$y$};
% curve
\draw[->,thick,draw=black, domain=0:3.7,samples=300,variable=\x]  
plot (\x,\x);
\draw[<-,thick,draw=black, domain=-3.7:0,samples=300,variable=\x]  
plot (\x,{-1 * \x});
\end{tikzpicture}

To the left of zero, the slope is -1. To the right of zero, the slope
is 1.  At zero?  The derivative is not defined.

If a function has a derivative everywhere, it is said to be
\newterm{differentiable}. Generally, you can think of differentiable
functions as smooth -- their graphs have no corners.

\begin{Exercise}[label=diff1]
	[This problem was originally presented as a no-calculator, 
	multiple-choice question on the 2012 AP Calculus BC exam.] Let $f$ 
	be the function defined by $f(x) = \sqrt{|x - 2|}$ for all $x$. 
	Classify each of the following statements as true or false. 
	\begin{enumerate}
		\item $f$ is continuous at $x = 2$. 
		\item $f$ is differentiable at $x = 2$.
		\item $\lim_{x \to 2} f(x) = 0$.
		\item $x = 2$ is a vertical asymptote of the graph of $f(x)$. 
	\end{enumerate}
\end{Exercise}

\begin{Answer}[ref=diff1]
	\begin{enumerate}
		\item True. $f(2)$ exists and $lim_{x \to 2^+}f(x) = 
		\lim_{x \to 2^-}f(x) = f(2) = 0$. 
		\item False. Because of the absolute value, there is a corner in 
		the graph of $f$ at $x=2$. $\lim_{x \to 2^+}f'(x) < 0$ and 
		$\lim_{x \to 2^-}f'(x) < 0$. Therefore there is a discontinuity in 
		$f'(x)$ at $x = 2$ and $f(x)$ is not differentiable at $x = 2$. 
		\item True. $\sqrt{|2-2|} = \sqrt{0} = 0$.
		\item False. $f(2)$ is defined at $x = 2$. 
	\end{enumerate}
\end{Answer}

\section{Using the definition of derivative}

Let's say that you want to know the slope of $f(x) = -3x^2$ at $x = 2$.
Using the definition of the derivative, that would be:

$$f'(2) = \lim_{\Delta x \to 0}\frac{f(2 + \Delta x) - f(2)}{\Delta x} 
= \lim_{\Delta x \to 0}\frac{-3(2 + \Delta x)^2- \left(-3(2)^2\right)}
{\Delta x} = \lim_{\Delta x \to 0}\frac{-12 - 12\Delta x + 
-3(\Delta x)^2 + 12}{\Delta x} = -12$$ 


