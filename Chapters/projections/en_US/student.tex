\chapter{Projections}

A projection is shows the relationship between two vectors. Given two vectors, $\mathbf{a}$ and $\mathbf{b}$, the projection of $\mathbf{a}$ onto $\mathbf{b}$ separates $\mathbf{a}$ into two components. The first component signifies how much $\mathbf{a}$ lies in the direction of $\mathbf{b}$. The second signifies the component of $\mathbf{a}$ that is perpendicular (orthogonal) to $\mathbf{b}$. \index{projection}

Projections are useful in many diverse fields. There are just a few examples, but there are numerous other applications in science, math, engineering, and finance.

\begin{itemize}
\item Investors evaluate risk and return of a portfolio by projecting an asset’s return onto a reference portfolio.
\item Astronomers analyze the motion of stellar objects by projecting the object’s true motion onto the plane of the sky.
\item Robotics engineers use projections to prevent robots from running into obstacles by projecting the robot’s position onto the optimal path.
\end{itemize}

As you work your way through this course, you'll have a chance to apply the calculations you learn in this chapter to a variety of problems.

To calculate the projection of two vectors
$\mathbf{a}$ onto $\mathbf{b}$, denoted as
$\mathbf{proj}_\mathbf{b}(\mathbf{a})$, use this formula:

$$\mathbf{proj}_\mathbf{b}(\mathbf{a}) = \left(\frac{{\mathbf{a} \cdot \mathbf{b}}}{{\|\mathbf{b}\|^2}}\right) \mathbf{b}$$

where $\mathbf{a} \cdot \mathbf{b}$ denotes the dot product and $||\mathbf{b}||$ represents the magnitude (or length) of vector $\mathbf{b}$.

The numerator measures the extent to
which $\mathbf{a}$ and $\mathbf{b}$ are aligned with each
other. Dividing this by $||\mathbf{b}||^2$ scales the projection to
ensure it represents the correct length along $\mathbf{b}$.

Finally, multiplying the scaled value with $\mathbf{b}$ gives the
projection vector itself.

In summary, you use the dot product to determine the alignment between
two vectors, and by appropriately scaling one vector and multiplying
it with the other vector, you obtain the projection of one vector
onto the other.

Let's look at a specific example:

$$a = (1,4,6)$$ 
$$b = (-2,6,2)$$ 

First, compute the dot product: 
$$1*-2 + 4*6 + 6*2 = -2 +24 +12 = 34$$

Then find the magnitude of vector $\mathbf{b}$:
$$(b_1^{2}+b_2^{2}+b_3^{2}) = 44 $$

Finally compute the projection vector.
$$34/44 * b = (-1.545 , 4.64, 1.545)$$

\begin{Exercise}[title={Projections}, label=projections]
	Find the projection of $\mathbf{a}$ on $\mathbf{b}$ where:
	$$a = (1,3)$$
	$$b = (-4,6)$$
\end{Exercise}
\begin{Answer}[ref=project_vector]
	Compute dot product
	$$1*-4 + 3*6 = -4 +18 = 14$$
	Find the magnitude of $\mathbf{b}$  
	$$magnitude = (b_1^{2}+b_2^{2}) = 52 $$
	$$14/52 * b = (-1.076 , 1.61)$$
\end{Answer}
 
\section{Projections in Python}

Create a file called \filename{vectors\_projections.py} and enter this code:
\begin{Verbatim}

# import numpy to perform operations on vector
import numpy as np
  
a = np.array([1, 4, 6])   # vector a
b = np.array([-2, 6, 2])   # vector b:
  
# Find the magnitude (length) of vector b
b_mag = sum(b**2)    
  
# Find the projection
# Use np.dot() to calculate the dot product
projection_a_on_b = (np.dot(a, b)/b_mag)*b
  
print("The projection of vector a on vector b is:", projection_a_on_b)
\end{Verbatim}
 
\section{Where to Learn More}

Watch this Introduction to Projections from Khan Academy 
https://www.khanacademy.org/math/linear-algebra/matrix-transformations/lin-trans-examples/v/introduction-to-projections

