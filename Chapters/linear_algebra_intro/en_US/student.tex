\chapter{Introduction to Linear Algebra}

Welcome to the world of linear algebra, a branch of mathematics that relies on 
vectors, matrices, and linear transformations. You are familiar with most of 
these concepts, so in this workbook you will see how you can use them together 
to solve problems.  \index{linear algebra}

Let's review what you know.

\begin{itemize}
\item \textbf{Vectors}. In workbook 7, you saw how vectors can represent 
forces, as well as how to add and multiply them to figure out such things as 
rocket engine force and direction. 
\item \textbf{Matrices}. In workbook 3, you learned to use spreadsheets to 
solve problems numerically, such as how to figure out the number of barrels a 
cooper has to produce to achieve a certain take-home pay. Spreadsheets are 
essentially matrices --- a row by column structure that contains values. 
\item \textbf{Linear transformations}. When you studied congruence in workbook 
5, you were introduced to a few linear transformations, such as translation 
and reflection. 
\end{itemize}

\section{What's With the Linear?}

You might be thinking, “Hey, haven't I been doing algebra already?” 

You have! You have come a long way in your problem solving journey. You have 
used algebra to solve simple equations like $7x + 10 = 24$ and quadratic 
equations like $4x^{2} + 9x + 31 = 0$. What distinguishes linear algebra is 
the focus on linear combinations. Any equation with a power greater than 1, 
such as a quadratic, is nonlinear.
\index{nonlinear operations}
We will first take a look at linear combinations

\section{Linear Combinations}

You won't see any \textbf{sin}, \textbf{cos}, or \textbf{tan} operations in 
this section. Linear operations do not use trigonometric functions; those 
operations are all nonlinear. A linear combination consists solely of addition and 
scalar multiplication. You will see that linear combinations allow you to 
solve many types of problems in science and engineering. Before we get deep 
into the numbers, let's take a look at a few linear operations you can perform 
on images. This will give you an intuition for the underlying math. After 
that, we'll take a look at some numbers.

\section{Image Operations}
\index{linear image operations}
The simplest image, a bitmap, can be represented by a two-dimensional matrix 
of values --- either 0 for black, or 1 for white. Grayscale images are also 
represented by a two-dimensional matrix of values, but the values typically 
range from 0 to 255. 0 is black, 255 is white, and the values in between 
represent shades of gray. 

Color images are more complex. The simplest color image is a three-dimensional 
matrix of values. You can think of it as three 2D matrices, one to represent 
red values (R), another for green values (G), and the third for blue values 
(B). The combination of R, G, and B determines the color you see.

Working with images means working with millions of pixels. Fortunately, 
modern techniques make this a snap. Let's look at some common operations on an 
image of a rocket.
Flipping is a linear operation.
\begin{figure}[htbp]
    \centering
    \includegraphics[width=0.25\textwidth]{flying-rocket.png}
    \caption{Rocket image.}
    \label{fig:flying-rocket}
\end{figure}

\begin{figure}[htbp]
    \centering
    \includegraphics[width=0.25\textwidth]{rocket-flipped.png}
    \caption{The rocket being flipped involves a binary linear equation.}
    \label{fig:rocket_flipped}
\end{figure}

Next, the image is rotated 90 degrees. This rotation is linear, but if you 
want to rotate it at an angle that isn't a multiple of 90, you would need 
trigonometry. This would be treading into nonlinear territory, but that happens 
in the field of linear algebra. You will learn about nonlinear extensions 
later, which use trigonometric functions and imaginary numbers.

\begin{figure}[htbp]
    \centering
    \includegraphics[width=0.25\textwidth]{rocket-rotated-90.png}
    \caption{Rotating the rocket by $90^\circ$}
    \label{fig:rocket-rotated}
\end{figure}

Inversion is an interesting linear operation that involves redefining the red, 
green, and blue values, such that the new value is 1.0 minus the old value. 
The resulting black background gives the impression the rocket is in deep 
space, don't you think?

\begin{figure}[htbp]
    \centering
    \includegraphics[width=0.25\textwidth]{rocket-inverted.png}
    \caption{Inverting the rocket replaces each pixel with a "opposite" color.}
    \label{fig:rocket-inverted}
\end{figure}

It is possible to redefine the red, green, and blue values in many ways. Visit 
the NASA website and search for false color images. NASA and other scientists 
redefine colors to communicate such things as the amount of vegetation or 
water in an area, the temperatures of the sun's surface, and so on. 
Photographers often do this for artistic effect. For example, the image on the 
left was taken with an infrared camera. (This infrared is not the same as 
thermal kind you have likely seen before. This is the infrared that is emitted 
by living plants.) The image is further processed to swap channels. For 
example, the matrix representing red might be swapped with the matrix 
representing blue. The image on the right shows the image after swapping color 
values. All these swapping operations are linear.

\begin{figure}[htbp]
    \centering
    \includegraphics[width=1.0\textwidth]{infrared.png}
    \caption{Left: image with infrared camera. Right: image shown post swapping color values}
    \label{fig:infrared}
\end{figure}

\section{The Numbers Behind Some Image Operations}
\index{matrices}
You will see a few \newterm{matrices} in this section. Let's first look at how a 
spreadsheet can be represented as a matrix. Recall the barrel-making shop 
example. This is part of that spreadsheet. 
\begin{figure}[htbp]
    \centering
    \includegraphics[width=1.0\textwidth]{spreadsheet.png}
    \caption{Spreadsheet of the barrel example.}
    \label{fig:barrel-img}
\end{figure}

Represented as a matrix, it looks like the following. Note the differences. A 
matrix contains only values, no labels. This matrix uses floating point values, 
hence the inclusion of decimal points. 

$$\begin{bmatrix}
115.0 & 120.0 & 125.0\\
45.0 & 45.0 & 45.0\\
100.0 & 100.0 & 100.0\\
2000.0 & 2000.0 & 2000.0\\
4325.0 & 4600.0 & 4875.0\\
865.0 & 920.0 & 975.0\\
3460.0 & 3680.0 & 3900.0
\end{bmatrix}$$
 
A matrix that represents an image contains only pixel values, whereas the 
barrel-making shop matrix represents seven kinds of variables: barrels 
produced, materials cost, sales price, rent, pretax earnings, taxes, and take 
home pay. 

The simplest image to create is a bitmap, because that requires a matrix of 
zeros and ones. This is a matrix for a 10 pixel by 10 pixel image. Why use 
decimal points when this is obviously a matrix of integers? It turns out that 
when you use tools like Python to process matrices, you must be conscious of 
data types. Most of the Python methods we use for image operations expect 
floating points. A few expect integer types, but you'll see how to handle type 
conversion later, in the section on Python. 

$$\begin{bmatrix}
0. & 0. & 0. & 0. & 1. & 0. & 0. & 0. & 1. & 0\\
0. & 0. & 0. & 0. & 1. & 1. & 1. & 1. & 1. & 1\\
0. & 0. & 0. & 0. & 1. & 0. & 0. & 0. & 1. & 0\\
0. & 0. & 0. & 0. & 1. & 0. & 0. & 0. & 1. & 0\\
0. & 0. & 0. & 0. & 1. & 0. & 0. & 0. & 1. & 0\\
0. & 0. & 0. & 0. & 1. & 0. & 0. & 0. & 1. & 0\\
0. & 0. & 0. & 0. & 1. & 0. & 0. & 0. & 1. & 0\\
0. & 0. & 0. & 0. & 1. & 1. & 1. & 1. & 1. & 1\\
0. & 0. & 0. & 0. & 1. & 0. & 0. & 0. & 1. & 0\\
0. & 0. & 0. & 0. & 1. & 0. & 0. & 0. & 1. & 0
\end{bmatrix}$$

When converted to an image, it is very tiny. This is an enlarged version, so 
you can see the pattern.

\begin{figure}[htbp]
    \centering
    \includegraphics[width=0.25\textwidth]{normal.png}
    \caption{Resulting bitmap vectors as a black and white image.}
    \label{fig:normal}
\end{figure}

We can create an inverse of this image by changing all the values in the matrix 
so that 0 becomes 1 and 1 becomes 0. (Technically this is not the way you would 
invert a matrix, as you will see in the Python section. For this example, you 
will get the same visual result, but when you start inverting matrices 
programmatically, you will learn the formal definition.)

$$\begin{bmatrix}
1. & 1. & 1. & 1. & 0. & 1. & 1. & 1. & 0. & 1\\
1. & 1. & 1. & 1. & 0. & 0. & 0. & 0. & 0. & 0\\
1. & 1. & 1. & 1. & 0. & 1. & 1. & 1. & 0. & 1\\
1. & 1. & 1. & 1. & 0. & 1. & 1. & 1. & 0. & 1\\
1. & 1. & 1. & 1. & 0. & 1. & 1. & 1. & 0. & 1\\
1. & 1. & 1. & 1. & 0. & 1. & 1. & 1. & 0. & 1\\
1. & 1. & 1. & 1. & 0. & 1. & 1. & 1. & 0. & 1\\
1. & 1. & 1. & 1. & 0. & 0. & 0. & 0. & 0. & 0\\
1. & 1. & 1. & 1. & 0. & 1. & 1. & 1. & 0. & 1\\
1. & 1. & 1. & 1. & 0. & 1. & 1. & 1. & 0. & 1
\end{bmatrix}$$

When converted to an image and enlarged, it looks like this:

\begin{figure}[htbp]
    \centering
    \includegraphics[width=0.25\textwidth]{inverse.png}
    \caption{Inversion of the linearr matrix.}
    \label{fig:example}
\end{figure}

Rotating the original matrix by 90 degrees gives this:

$$\begin{bmatrix}
0. & 1. & 0. & 0. & 0. & 0. & 0. & 1. & 0. & 0.\\
1. & 1. & 1. & 1. & 1. & 1. & 1. & 1. & 1. & 1.\\
0. & 1. & 0. & 0. & 0. & 0. & 0. & 1. & 0. & 0.\\
0. & 1. & 0. & 0. & 0. & 0. & 0. & 1. & 0. & 0.\\
0. & 1. & 0. & 0. & 0. & 0. & 0. & 1. & 0. & 0.\\
1. & 1. & 1. & 1. & 1. & 1. & 1. & 1. & 1. & 1.\\
0. & 0. & 0. & 0. & 0. & 0. & 0. & 0. & 0. & 0.\\
0. & 0. & 0. & 0. & 0. & 0. & 0. & 0. & 0. & 0.\\
0. & 0. & 0. & 0. & 0. & 0. & 0. & 0. & 0. & 0.\\
0. & 0. & 0. & 0. & 0. & 0. & 0. & 0. & 0. & 0.
\end{bmatrix}$$

This is the resulting enlarged image:

\begin{figure}[htbp]
    \centering
    \includegraphics[width=0.25\textwidth]{rotate90.png}
    \caption{Rotating the image by 90deg.}
    \label{fig:rotate90matrix}
\end{figure}

You'll transpose many matrices in the upcoming pages. It requires swapping 
rows for columns. 

$$\begin{bmatrix}
0. & 0. & 0. & 0. & 0. & 0. & 0. & 0. & 0. & 0.\\
0. & 0. & 0. & 0. & 0. & 0. & 0. & 0. & 0. & 0.\\
0. & 0. & 0. & 0. & 0. & 0. & 0. & 0. & 0. & 0.\\
0. & 0. & 0. & 0. & 0. & 0. & 0. & 0. & 0. & 0.\\
1. & 1. & 1. & 1. & 1. & 1. & 1. & 1. & 1. & 1.\\
0. & 1. & 0. & 0. & 0. & 0. & 0. & 1. & 0. & 0.\\
0. & 1. & 0. & 0. & 0. & 0. & 0. & 1. & 0. & 0.\\
0. & 1. & 0. & 0. & 0. & 0. & 0. & 1. & 0. & 0.\\
1. & 1. & 1. & 1. & 1. & 1. & 1. & 1. & 1. & 1.\\
0. & 1. & 0. & 0. & 0. & 0. & 0. & 1. & 0. & 0.
\end{bmatrix}$$

The resulting image looks like this:
\begin{figure}[htbp]
    \centering
    \includegraphics[width=0.25\textwidth]{transpose.png}
    \caption{Transposed image reversing rows and columns.}
    \label{fig:transposedMatrix}
\end{figure}

What about adding images? That fits the definition of a linear combination. 
Recall that grayscale images have values from 0 to 255. To make things simple, 
let's define two matrices with values ranging from 0.0 to 1.0. When we want a 
grayscale image, it is easy to multiply the matrix by 255. 

Let's call this matrix f.

$$\begin{bmatrix}
1.0 & 0.5 & 0.0\\
1.0 & 1.0 & 1.0\\
0.5 & 0.0 & 0.0 
\end{bmatrix}$$

When multiplied by 255 and converted to a grayscale image:
\begin{figure}[htbp]
    \centering
    \includegraphics[width=0.25\textwidth]{fBitmap.png}
    \caption{Bitmap of matrix f.}
    \label{fig:fBitmap}
\end{figure}

Let's call this matrix g:

$$\begin{bmatrix}
0.5 & 0.0 & 0.0\\
1.0 & 0.5 & 1.0\\
1.0 & 1.0 & 1.0 
\end{bmatrix}$$

When multiplied by 255 and converted to a grayscale image:
\begin{figure}[htbp]
    \centering
    \includegraphics[width=0.25\textwidth]{gBitmap.png}
    \caption{Bitmap of matrix g.}
    \label{fig:gBitmap}
\end{figure}

When we add f and g we get k:

$$\begin{bmatrix}
1.5 & 0.5 & 0.0\\
2.0 & 1.5 & 2.0\\
1.5 & 1.0 & 1.0 
\end{bmatrix}$$

However, the values in k exceed the range of 0.0 to 1.0, so we normalize by 
dividing the matrix by 2.0:
\index{matrices!normalization}
$$\begin{bmatrix}
0.75 & 0.25 & 0.00\\
1.00 & 0.75 & 1.00\\
0.75 & 0.50 & 0.50  
\end{bmatrix}$$

When multiplied by 255 and converted to grayscale, we get:
\begin{figure}[htbp]
    \centering
    \includegraphics[width=0.25\textwidth]{fgBitmapAdded.png}
    \caption{Adding f+g represented as a bitmap.}
    \label{fig:fgBitmapAdded}
\end{figure}

Let's go back to the first small grayscale image (from Figure~\ref{fig:fBitmap}):

\includegraphics[width=0.25\textwidth]{fBitmap.png}

If you want to keep the pattern in the first column, you could multiply the 
matrix by a vector, [1.0 0.0 0.0]. The 1.0 will keep the values in the first 
column, but the 0.0 will knock out the other values because 0.0 times anything 
equals 0.0.
\begin{figure}[htbp]
    \centering
    \includegraphics[width=0.25\textwidth]{onechannel.png}
    \caption{Multiplying channels by 0 results in them being represented as black bitmaps, while 1 keeps them the same.}
    \label{fig:onechannel}
\end{figure}

What do you think will happen if you use the vector [0.0 1.0 0.0] or 
[0.0 0.0 1.0]? You'll get a chance later to use Python to perform image 
operations.

All the operations we performed on these images satisfy the requirement for 
linear combinations: preserving addition and scalar multiplication. 


\section{Applications of Linear Algebra}

So far you've seen how linear operations on matrices can process images by:
\begin{itemize}
\item multipling a matrix using a scalar (e.g., normalize, change the range)
\item adding one matrix to another to get a composite image 
\item multipling two matrices to perform a transform (e.g., flipping)
\item mulitpling a matrix with a vector (isolating a channel)
\end{itemize}

Many areas in engineering and science rely on the matrix operations defines by 
linear algebra. Besides image processing, linear algebra is used for: 
\begin{itemize}
\item \textbf{Computer Graphics}. When you play a video game or watch the latest CG 
animation, matrix operations transform objects in the scene to make them appear 
as if moving, getting closer, and so on. You can represent the vertices of 
objects as vectors, and then apply a transformation matrix.
\item \textbf{Data Analysis}. We live in an era in which it's easy to collect so much 
data that it's difficult to make sense of the data by just looking at it. You 
can represent the data in matrix form and then find a solution vector. For 
example, scientists use this technique to figure out the effectiveness of drug 
treatments on disease.
\item \textbf{Economics}. Take a look at financial section of any news organization and 
you'll see headlines such as "Economic Data Points to Faster Growth" or "Is 
the Inflation Battle Won?" Economists can use systems of linear equations to 
represent economic indicators, such as consumer consumption, government 
spending, investment rate, and gross national product. By using various methods 
that you'll learn about later, they can get a good idea of the state of the 
economy.
\item \textbf{Engineering}. Engineers couldn't do without linear algebra. For example, 
the orbital dynamics of space travel relies on it. Engineers must predict and 
calculate the the motion of planetary bodies, satellites, and spacecraft. By 
solving systems of linear equations engineers can make sure that a spacecraft 
travels to its destination without running into a satellite or space rock.
\end{itemize}

\section{Let's Observe the Sun!}
India recently sent the Aditya spacecraft on a mission to study the Sun. 
Without a thorough understanding of linear algebra (among other things), the 
engineers would not have accomplished the amazing feat of getting Aditya in a 
stable orbit around a Lagrange point. In previous chapters you learned about 
gravity and its effects. 

A Lagrange point is a point in space between two bodies (e.g. Earth and Sun) 
where there is gravitational equilibrium. With the right trajectory, a 
spacecraft will orbit around a Lagrange point in a stable position that 
doesn’t require much energy to maintain. That’s called a Halo orbit. Because 
that there are no fueling stations in space, a Halo orbit will allow Aditya to 
maintain position for about 5 years. Pretty good mileage!

\begin{figure}[htbp]
    \centering
    \includegraphics[width=0.5\textwidth]{adityaOrbit.png}
    \caption{The Aditya Orbit.}
    \label{fig:adityaOrbit}
\end{figure}

Aditya’s engineers had to calculate a looping maneuver that would precisely 
inject the Aditya spacecraft into the Halo orbit. They determined the angles 
and burn times for the thrust engine. If they were wrong in one direction, the 
spacecraft would fly off to the sun. The other direction would send the 
spacecraft back in the direction of Earth. Their success is due to a solid 
understanding of vectors and linear algebra.

\begin{figure}[htbp]
    \centering
    \includegraphics[width=0.5\textwidth]{gimbalEngine.png}
    \caption{The gimbal of the rocket engine.}
    \label{fig:gimbalEngine}
\end{figure}

\section{Images in Python}
One of the wonderful things about python is the availability of libraries for 
specialized computation. The Python Imaging Library, PIL, is what you'll use 
to create images, read existing images from disk, and perform operation on 
images. To create and manipulate arrays, you will use NumPy. 

Create a file called \filename{image\_creation.py} and enter this code:

\begin{Verbatim}
# Import necessary modules
import numpy as np
import PIL
from PIL import Image
from PIL import ImageOps

# Create a 10 by 10 pixel bitmap Image. 
# Using a decimal point ensure python see the values as floating point numbers
# Some image operations assume floats

bitmapArray = np.array([
[0., 0., 0., 0., 1., 0., 0., 0., 1., 0.],
[0., 0., 0., 0., 1., 1., 1., 1., 1., 1.],
[0., 0., 0., 0., 1., 0., 0., 0., 1., 0.],
[0., 0., 0., 0., 1., 0., 0., 0., 1., 0.],
[0., 0., 0., 0., 1., 0., 0., 0., 1., 0.],
[0., 0., 0., 0., 1., 0., 0., 0., 1., 0.],
[0., 0., 0., 0., 1., 0., 0., 0., 1., 0.],
[0., 0., 0., 0., 1., 1., 1., 1., 1., 1.],
[0., 0., 0., 0., 1., 0., 0., 0., 1., 0.],
[0., 0., 0., 0., 1., 0., 0., 0., 1., 0.]])

# Image.fromarray assumes a range of 0 to 255, so scale by 255

myImage = Image.fromarray(bitmapArray*255)
myImage.show()

# A window opens with an image so tiny you might think nothing is there
# Zoom in to see the pattern  

# Transpose the array, create an image, and then show it. 
# Note that you operate on the original array (not the image)
# Remember to zoom in to see the pattern 

myImageTransposed = bitmapArray.transpose()
myImageTransposed.show()

# Invert the array. You'll use the NumPy invert method.
# The invert method assumes integer values. You need to convert the data type
# Numpy has a method for that

intBitmapArray = np.asarray(bitmapArray,dtype="int")
invertedArray = np.invert(intBitmapArray)

# Take a look at the array

invertedArray

# The values range from -2 to -1. Image values are positive.
# You need to change the range so the values are from 0 to 255
# Further you need to change back to floating point values because
# the PIL method requires them

invertedArray = (invertedArray + 2)*1.0
invertedImage = Image.fromarray(255*invertedArray)
invertedImage.show()

# Zoom in on the image and compare the pattern with the original

\end{Verbatim}

\section{Exercise}

Create a python program that creates matrix $f$ and matrix $g$ from the 
previous section, and then performs all the operations shown in that section. 
If you are not sure how to accomplish something, consult the online 
documentation for the PIL and NumPy python libraries. 
