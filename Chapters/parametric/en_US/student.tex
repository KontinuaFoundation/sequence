\chapter{Parametric Functions}

Throughout your study of functions, you have seen functions that look like this:
\begin{equation*}
y = f(x)
\end{equation*}

Sometimes, to describe certain shapes or curves on a graph, we must use more than one function to do so. This means a shape, like a circle on the graph, can be represented using
\begin{equation*}
y = f(t) \quad \text{and} \quad x = g(t).
\end{equation*}

In this case, there are two functions, $x(t)$ and $y(t)$, that depend on a third variable $t$, and together they help us visualize and define a curve.

The central word you should think about as you go through this chapter is \emph{parameter}. A parameter is a variable that both $x$ and $y$ depend on. The most common example of a parametric function is the circle. To represent a circle on a graph, we can use the equations
\begin{equation*}
x = \cos(\theta) \quad \text{and} \quad y = \sin(\theta).
\end{equation*}

FIXME: Insert graph of a circle showing parametric function

In this case, the graph of the circle is determined by the value of the angle $\theta$.

So, why parametric functions? When you look up the function that describes a circle, you might see something like
\begin{equation*}
x^2 + y^2 = r^2.
\end{equation*}
The equation $x^2 + y^2 = r^2$ does not describe $y$ as a single function of $x$. You would need two functions (one for the top half and one for the bottom half), and many curves cannot be written as a single function at all. Parametric equations solve this problem by allowing us to describe the entire curve with one pair of equations, using a parameter $t$. This parameter can be denoted by other letters, but we will use $t$ in this chapter.

Imagine a ball rolling across graph paper. At each moment, the ball is at a different location. If we label time with a variable $t$, then for every value of $t$ the ball has an $x$-coordinate and a $y$-coordinate. What if we had two equations, one that told us its $x$-position at time $t$ and one that told us its $y$-position? As $t$ changes, these two equations would trace the entire path of the ball. This is the basic idea behind parametric equations.

\section{Conversion to Rectangular Form}

In many cases, we can eliminate the parameter to convert a pair of parametric equations,
\begin{equation*}
x = f(t), \quad y = g(t),
\end{equation*}
into a single equation relating $x$ and $y$. 

In many cases, a parametric curve can be rewritten as a single equation
involving only $x$ and $y$. This process is called eliminating the
parameter, and the resulting equation is known as the rectangular
form of the curve.

To eliminate the parameter, we solve one of the parametric equations for
$t$ and substitute into the other equation. 

\textbf{Example:}

Consider the parametric equations
\begin{equation*}
x = 2t + 1, \quad y = 3t - 4.
\end{equation*}

\textbf{Solution:}
Solve the equation for $x$ to isolate $t$:
\begin{equation*}
t = \frac{x - 1}{2}.
\end{equation*}

Substitute into the equation for $y$:
\begin{equation*}
y = 3\left(\frac{x - 1}{2}\right) - 4.
\end{equation*}

Simplifying,
\begin{equation*}
y = \frac{3}{2}x - \frac{11}{2}.
\end{equation*}

\textbf{Example}

Consider the parametric equations
\begin{equation*}
x = \cos t, \quad y = \sin t.
\end{equation*}

\textbf{Solution:}
In this case, solving for $t$ directly is not convenient. Instead, we use
the trigonometric identity
\begin{equation*}
\cos^2 t + \sin^2 t = 1.
\end{equation*}

Substituting $x = \cos t$ and $y = \sin t$ gives
\begin{equation*}
x^2 + y^2 = 1.
\end{equation*}

This rectangular equation describes a circle of radius 1 centered at the
origin. 

\textbf{Note}

Converting to rectangular form describes the shape of a curve, but it
does not show the direction of motion or how many
times the curve is traced. This information is contained in the parametric
equations and the domain of the parameter!


\section{Sketching Parametric Equations}

Parametric equations describe curves by expressing both $x$ and $y$ in terms
of a third variable, usually $t$, called the parameter. When sketching a
parametric curve, it is important to remember that the curve is traced as the
parameter changes, rather than being drawn all at once as a function of $x$.

To sketch a parametric curve,
\begin{itemize}
    \item choose several values of the parameter $t$,
    \item compute the corresponding $x$- and $y$-values,
    \item plot the resulting points in the $xy$-plane,
    \item and indicate the direction in which the curve is traced as $t$
    increases.
\end{itemize}

Unlike rectangular graphs, parametric sketches also depend on the direction
and speed of motion, which are determined by how $x(t)$ and $y(t)$ change as
$t$ varies.

FIXME Sketched example
%\subsection*{Example: Sketching a Parametric Curve}
%
%Consider the parametric equations
%\begin{equation*}
%x = t + 1, \quad y = t^2.
%\end{equation*}
%
%As $t$ varies, each value of $t$ produces a point $(x,y)$ in the plane. By
%evaluating the equations at several values of $t$, we can sketch the curve and
%determine how it is traced.

\begin{Exercise}[label = parametric2]

Sketch the parametric equations
\begin{equation*}
x = t + 1, \quad y = t^2.
\end{equation*}

In your sketch:
\begin{itemize}
    \item create a table of values for $t$, $x$, and $y$,
    \item plot the corresponding points,
    \item and indicate the direction of motion as $t$ increases.
\end{itemize}

\vspace{80mm}

\end{Exercise}

\begin{Answer}[ref = parametric2]

To sketch the curve, we begin by choosing several values of $t$ and computing
the corresponding $x$- and $y$-values.

\begin{center}
\begin{tabular}{c|c|c}
$t$ & $x $ & $y  \\
\hline
$-2$ & $-1$ & $4$ \\
$-1$ & $0$ & $1$ \\
$0$ & $1$ & $0$ \\
$1$ & $2$ & $1$ \\
$2$ & $3$ & $4$
\end{tabular}
\end{center}

Plotting these points produces a parabola opening upward. As $t$ increases,
the $x$-values increase and the curve is traced from left to right.

Although the rectangular equation of the curve is
\[
y = (x - 1)^2,
\]
the parametric equations determine the direction in which the curve is drawn,
which cannot be seen from the rectangular form alone.

\end{Answer}



%\begin{Answer}[ref = parametric2]
% Figure cannot be formatted into the Answers to Excercises section so replace with a list of values.
%\begin{figure}[htbp]
%\centering
%\begin{tikzpicture}[scale=0.07]
%
%% ---------------- Frame ----------------
%\draw[thick] (-20,-95) rectangle (6,5);
%
%% ---------------- Axes ----------------
%\draw[thick,-stealth] (-20,0) -- (6,0) node[right] {$x$};
%\draw[thick,-stealth] (0,-95) -- (0,5) node[above] {$y$};
%
%% ---------------- Grid ----------------
%\draw[step=10, lightgray] (-20,-95) grid (6,5);
%
%% ---------------- Parametric curve ----------------
%\draw[thick, sdkblue, domain=0:10, samples=200, smooth]
%  plot ({4 - 2*\x}, {1 + \x - \x*\x});
%
%\end{tikzpicture}
%\caption{Parametric curve $x=4-2t$, $y=1+t-t^2$, $0\le t \le 10$.}
%\label{fig:param_curve}
%\end{figure}
%\end{Answer}

\section{Using Python to Explore Parametric Curves}

Python can be used to visualize parametric curves and explore how the point $(x(t), y(t))$ moves as $t$ changes. For example, we could write code that:
\begin{itemize}
    \item plots points along a circle defined by $x = \cos(t)$ and $y = \sin(t)$,
    \item shows the distance from the origin,
    \item traces more complicated parametric curves like spirals or ellipses.
\end{itemize}

Below is a Python script that creates an animation for sketching parametric curves. There is a provision for changing the equations based on exercises and questions you come across.

\begin{verbatim}
import numpy as np
import matplotlib.pyplot as plt
from matplotlib.animation import FuncAnimation

t = np.linspace(0, 3, 50)

x = 4 - 2*t
y = 3 + 6*t - 4*t**2

fig, ax = plt.subplots()
ax.set_aspect("equal", "box")

ax.axhline(0, color="black", linewidth=0.5)
ax.axvline(0, color="black", linewidth=0.5)

ax.set_xlabel("x(t)")
ax.set_ylabel("y(t)")
ax.set_title("Parametric Equation of Curve")

ax.set_xlim([x.min() - 1, x.max() + 1])
ax.set_ylim([y.min() - 1, y.max() + 1])

(line,) = ax.plot([], [], "b--")
(point,) = ax.plot([], [], "ro")

def update(frame):
    line.set_data(x[:frame + 1], y[:frame + 1])
    point.set_data([x[frame]], [y[frame]])
    return line, point

ani = FuncAnimation(fig, update, frames=len(t), interval=50, blit=True)
ani.save("parametric_animation.mp4")

plt.show()
\end{verbatim}

\begin{Exercise}[label = parametric1]

Consider the parametric equations
\begin{equation*}
x = 4 - 2t, \quad y = 3 + 6t - 4t^2,
\end{equation*}
for $0 \le t \le 3$.

\begin{itemize}
    \item[(a)] Sketch the curve by creating a table of values for $t$, $x$, and $y$.
    \item[(b)] Use the provided Python animation code to visualize the curve.
    \item[(c)] Describe the direction in which the curve is traced as $t$ increases.
    \item[(d)] Explain how the Python animation supports or clarifies your hand-drawn sketch.
\end{itemize}
\vspace{80mm}
\end{Exercise}

\begin{Answer}[ref = parametric1]

To sketch the curve, we begin by evaluating the parametric equations at several
values of $t$.

\begin{center}
\begin{tabular}{c|c|c}
$t$ & $x$ & $y$ \\
\hline
0 & 4 & 3 \\
1 & 2 & 5 \\
2 & 0 & -1 \\
3 & -2 & -15
\end{tabular}
\end{center}

Plotting these points produces a curve that opens downward. As $t$ increases,
the $x$-values decrease, indicating that the curve is traced from right to left.

The Python animation confirms this behavior by showing the curve being drawn
progressively as $t$ increases. The moving point illustrates both the direction
of motion and how the curve is traced over time, reinforcing the hand-drawn
sketch.

\end{Answer}



\section{Derivatives of Parametric Functions}
Just like any other type of curve, you can be asked to determine the slope of the tangent to the point on a parametric curve. To do this, you should know how to find the derivative of the curve! You could do this by eliminating the variable t and solving for the derivative of the rectangular equation. However, there is an easier way!

The derivative of a parametric curve can be found by dividing $\frac{dy}{dt}$ by $\frac{dx}{dt}$:
\begin{equation*}
\frac{dy}{dx} = \frac{dy/dt}{dx/dt}.
\end{equation*}

Derivatives of parametric functions are used to describe the slope of a curve at
a given value of the parameter. This slope indicates how the curve is changing
at that point. In some problems, the slope is used to find the equation of a tangent line to
the curve. In other problems, the slope is used to analyze the behavior of the
curve, such as determining where the curve has horizontal or vertical tangents.


\subsection*{Example}

Find $\dfrac{dy}{dx}$ for the parametric equations
\[
x = t^2 + 1, \quad y = 3t - 2.
\]

\textbf{Solution}

First compute the derivatives with respect to $t$:
\[
\frac{dx}{dt} = 2t, \quad \frac{dy}{dt} = 3.
\]

Now divide:
\[
\frac{dy}{dx} = \frac{3}{2t}.
\]

As a check, eliminate the parameter. From $x = t^2 + 1$, we get
\[
t = \sqrt{x - 1}.
\]
Substitute into $y = 3t - 2$:
\[
y = 3\sqrt{x - 1} - 2.
\]
Differentiating gives
\[
\frac{dy}{dx} = \frac{3}{2\sqrt{x - 1}},
\]
which matches the parametric result.


\begin{Exercise}[label = parametric3]

Consider the parametric equations
\begin{equation*}
x = t^3 - 3t, \quad y = t^2.
\end{equation*}

\begin{itemize}
    \item[(a)] Compute $\dfrac{dx}{dt}$ and $\dfrac{dy}{dt}$.
    \item[(b)] Find the values of $t$ for which the curve has horizontal tangents.
    \item[(c)] Find the values of $t$ for which the curve has vertical tangents.
    \item[(d)] Identify the points on the curve where these tangents occur.
\end{itemize}
\vspace{100mm}
\end{Exercise}

\begin{Answer}[ref=parametric3]

First compute the derivatives:
\begin{equation*}
\frac{dx}{dt} = 3t^2 - 3, \quad \frac{dy}{dt} = 2t.
\end{equation*}

Horizontal tangents occur when $\dfrac{dy}{dt} = 0$ and $\dfrac{dx}{dt} \neq 0$.
\begin{equation*}
2t = 0 \Rightarrow t = 0.
\end{equation*}

At $t = 0$, the point on the curve is
\begin{equation*}
x = 0, \quad y = 0.
\end{equation*}

Vertical tangents occur when $\dfrac{dx}{dt} = 0$ and $\dfrac{dy}{dt} \neq 0$.
\begin{equation*}
3t^2 - 3 = 0 \Rightarrow t = \pm 1.
\end{equation*}

At $t = 1$, the point is
\begin{equation*}
( -2, 1 ),
\end{equation*}
and at $t = -1$, the point is
\begin{equation*}
( 2, 1 ).
\end{equation*}

\end{Answer}


\begin{Exercise}[label = parametric4]

Consider the parametric equations
\begin{equation*}
x = 2t + 1, \quad y = t^2.
\end{equation*}

\begin{itemize}
    \item[(a)] Find $\dfrac{dy}{dx}$ in terms of $t$.
    \item[(b)] Evaluate $\dfrac{dy}{dx}$ at $t = 1$.
    \item[(c)] Interpret your result as the slope of the tangent line at the corresponding point.
\end{itemize}
\vspace{100mm}
\end{Exercise}

\begin{Answer}[ref=parametric4]

First compute the derivatives with respect to $t$:
\begin{equation*}
\frac{dx}{dt} = 2, \quad \frac{dy}{dt} = 2t.
\end{equation*}

The derivative of the curve is
\begin{equation*}
\frac{dy}{dx} = \frac{dy/dt}{dx/dt} = \frac{2t}{2} = t.
\end{equation*}

At $t = 1$,
\begin{equation*}
\frac{dy}{dx} = 1.
\end{equation*}

Thus, the slope of the tangent line at the corresponding point on the curve is $1$.

\end{Answer}

FIXME Expand on section
\section{Length of Curve Traced by Parametric Equations}

Speed is the magnitude of the velocity vector:
\begin{equation*}
\text{Speed} = \sqrt{(x'(t))^2 + (y'(t))^2}.
\end{equation*}

Given
\begin{equation*}
x'(t) = 8t - t^2, \quad y'(t) = -t + \sqrt{t^{1.2} + 20},
\end{equation*}
the speed at $t = 2$ seconds is $12.3$ cm/s.

\begin{Exercise}[label = parametric5]
%Could be a word excercise, you  can edit later. 
How can two sets of parametric equations represent the same graph but different curve? 
\end{Exercise}

\begin{Answer}[ref=parametric5]

Two sets of parametric equations can represent the same graph but
describe different curves because the parameter controls how the graph
is traced, not just the shape of the graph itself.

Different parametric equations may generate the same set of points in the
plane while differing in one or more of the following ways:
\begin{itemize}
    \item the direction of motion along the curve,
    \item the speed at which the curve is traced,
    \item the starting point of the curve,
    \item or the number of times the curve is traced.
\end{itemize}

\end{Answer}

