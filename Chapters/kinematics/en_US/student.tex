\chapter{Kinematics}

How can we describe the motion of objects other than falling bodies? Kinematics 
is the description of motion. 

%intro explaining usefulness of kinematics
%derive kinematics eqs from concrete examples

\begin{mdframed}[style = important]
$$x_f = x_0 + v_0 t + \frac{1}{2}at^2$$
$$v_f = v_0 + at$$
$$x_f = x_0 + \frac{1}{2}\left(v_f + v_0 \right)t$$
$$v_f^2 = v_0^2 + 2a\left(x_f - x_0 \right)$$
\end{mdframed}

\textbf{Example}: Terri and Jerry are running a race. Terri has a maximum 
acceleration of 3.4 $m/s^2$ and a top speed of 9.2 $m/s$. Jerry has a maximum 
acceleration of 3.7 $m/s^2$ and a top speed of 8.7 $m/s$. If the race is 200 $m$ 
long, who will win? Assume that each runner has maximum acceleration until they 
reach their top speed and that they will maintain that top speed once they reach 
it. 

\textbf{Solution}: We want to know how long it takes each runner to complete the 
200 $m$ race. For each runner, we will divide their run into two sections:
\begin{enumerate}
\item The time they are accelerating to their top speed
\item The time they are maintaining their top speed
\end{enumerate}

We can use a table to track the results of our calculations:
\begin{center}
\begin{tabular}{|c|c|c|}
\hline
Runner & Terri & Jerry\\\hline
Leg 1 (s) & & \\\hline
Leg 2 (s) & & \\\hline
Total (s) & & \\\hline
\end{tabular}
\end{center}

We'll begin with Terri's first leg. Taking the starting line as $x = 0$, we know 
that:
$$x_0 = 0\text{ }m$$
$$v_0 = 0\text { } \frac{m}{s}$$
$$a = 3.4\text{ }\frac{m}{s^2}$$

And since we want to know how long it takes Terri to reach her top speed, we 
also know that $v_f = 9.2$ $m/s$. Since we don't know how far Terri will run 
before she reaches her top speed (and we're not looking for that quantity), we 
need to select an equation that does not include $x$:
$$v_f = v_0 + at$$
$$9.2 \frac{m}{s} = 0 \frac{m}{s} + \left(3.4 \frac{m}{s^2} \right)t$$
$$t = \frac{9.2 \frac{m}{s}}{3.4 \frac{m}{s^2}} \approx 2.7\text{ }s$$

With a similar method, we can find how long it takes Jerry to reach his top 
speed:
$$t = \frac{v_f}{a} = \frac{8.7 \frac{m}{s}}{3.7 \frac{m}{s^2}} \approx 2.4
\text{ }s$$

Let's go ahead and record this in our table:
\begin{center}
\begin{tabular}{|c|c|c|}
\hline
Runner & Terri & Jerry\\\hline
Leg 1 (s) & 2.7 & \\\hline
Leg 2 (s) & 2.4 & \\\hline
Total (s) & & \\\hline
\end{tabular}
\end{center}

Now that we know how much time it takes each runner to reach their top speed, we 
need to figure out how much time it takes them to complete the race from the 
point at which each reaches their top speed. To do this, we will first have to 
find \textit{where} each runner hits their top speed. (This is because we can't 
use $v_f = v_0 + at$ to find a time anymore, since from now on the runners' 
accelerations are zero, and all the other equations involve $x$.) For each 
runner, we know $a$, $v_0$, $v_f$, $x_0$, and $t$. You could choose any equation, 
but we will use this one:
$$x_f = x_0 + \frac{1}{2} \left( v_f + v_0 \right)t$$

Since each runner begins on the starting line, $x_0 = 0$ $m$ and $v_0 = 0$ $m/s$:
$$x_f = \frac{v_f \cdot t}{2}$$

For Terri:
$$x_f = \frac{\left(9.2 \frac{m}{s} \right) \left(2.7\text{ }s \right)}{2} 
\approx 12.4 \text{ }m$$

For Jerry:
$$x_f = \frac{\left( 8.7 \frac{m}{s} \right) \left( 2.4 \text{ }s \right)}{2} 
\approx 10.2 \text{ }m$$

Now that we know where they reach their top speed, we can take that position as 
$x_0$ and find how long it takes each runner to reach the finish line at $x_f = 
200\text{ }m$. Since $a = 0$ $m/s^2$, we can use:
$$x_f = x_0 + v_0 t$$

Rearranging to solve for $t$:
$$t = \frac{x_f - x_0}{v_0}$$

For Terri:
$$t = \frac{200 \text{ } m - 12.4 \text{ } m}{9.2 \text{ } \frac{m}{s}} \approx 
20.4\text{ }s$$

And for Jerry:
$$t = \frac{200 \text{ } m - 10.2 \text{ } m}{8.7 \text{ } \frac{m}{s}} \approx 
21.8 \text{ } s$$

Completing our table, we see that Terri will win the race by finishing in the 
least amount of time:
\begin{center}
\begin{tabular}{|c|c|c|}
\hline
Runner & Terri & Jerry\\\hline
Leg 1 (s) & 2.7 & 2.4 \\\hline
Leg 2 (s) & 20.4 & 21.8 \\\hline
Total (s) & 23.1 & 24.2 \\\hline
\end{tabular}
\end{center}

\section{Graphing Motion}



%interpretation of x-t and v-t graphs without explicit calculus


\section{Separation of Components}
In the current chapter, we have learned how to use kinematics to describe one-dimensional motion. In the next chapter, you will learn to describe two-dimensional motion. It turns out that you can treat the different dimensions (horizontal and vertical motion) separately! %image and video link horizontal launch vs drop from rest hang time equivalence