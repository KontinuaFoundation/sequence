\chapter{Differentiating Polynomials}

If you had a function that gave you the height of an object, it would
be handy to be able to figure out a function that gave you the
velocity at which it was rising or falling. The process of converting
the position function into a velocity function is known as
\emph{differentiation} or \emph{finding the derivative}.

There are a bunch of rules for finding a derivative, but
differentiating polynomials only requires three:
\begin{itemize}
\item The derivative of a sum is equal to the sum of the derivatives.
\item The derivative of a constant is zero.
\item The derivative of a nonconstant monomial $at^b$ ($a$ and $b$ are constant numbers, $t$ is time) is $abt^{b-1}$ 
\end{itemize}\index{differentiation!polynomials}

So, for example, if I tell you that the height in meters of quadcopter
at second $t$ is given by $2t^3 - 5t^2 + 9t + 200$. You could tell me
that its vertical velocity is $6t^{2} - 10t + 9$.

We indicate the derivative of a function with an apostrophe (read as "prime") between the name of the function and the variable. For example, the derivative of $h(t)$ is $h'(t)$ (which is read out loud as "h prime of t"). 

\begin{Exercise}[title={Differentiation of polynomials}, label=diffpoly]
  Differentiate the following polynomials.
  \begin{enumerate}
  \item $f(t)=2t^3-3t^2-4t$
  \item $g(t)=2t^{-3/4}$
  \item $F(r) = \frac{5}{r^3}$
  \item $H(u) = (3u-1)(u+2)$
  \end{enumerate}
  \end{Exercise}
\begin{Answer}[ref=diffpoly]
\begin{enumerate}
\item $f'(t) = 3t^2-6t-4$
\item $g'(t) = (\frac{-3}{4})2t^{-3/4-1}=\frac{-3}{2}t^{-7/4}$
\item $F'(r) = \frac{-15}{r^4}$
\item First, we expand the function by multiplying out the two binomials: $(3u-1)(u+2)=3u^2+6u-u-2$. Therefore, $H(u) = 3u^2+5u-2$, and we can differentiate using what we've learned about differentiating polynomials. $H'(u) = 6u+5$. In a later chapter, you will learn the Product rule, which will allow you to differentiate this function without multiplying out the binomials. 
\end{enumerate}
\end{Answer}
Notice that the degree of the derivative is one less than the degree
of the original polynomial. (Unless, of course, the degree of the
original is already zero.)

Now, if you know that a position is given by a polynomial, you can
differentiate it to find the object's velocity at any time.

The same trick works for acceleration: Let's say you know a function
that gives an object's velocity. To find its acceleration at any time,
you take the derivative of the velocity function (the second derivative). 

\begin{Exercise}[title={Differentiation of polynomials in Python}, label=pydiffpoly]
  Write a function that returns the derivative of a polynomial in \filename{poly.py}. It should look like this:
\begin{Verbatim}
def derivative_of_polynomial(pn):
  ...Your code here...
\end{Verbatim}
When you test it in \filename{test.py}, it should look like this:
\begin{Verbatim}
# 3x**3 + 2x + 5
p1 = [5.0, 2.0, 0.0, 3.0]
d1 = poly.derivative_of_polynomial(p1)
# d1 should be 9x**2 + 2
print("Derivative of", poly.polynomial_to_string(p1),"is", poly.polynomial_to_string(d1))

# Check constant polynomials
p2 = [-9.0]
d2 = poly.derivative_of_polynomial(p2)
# d2 should be 0.0
print("Derivative of", poly.polynomial_to_string(p2),"is", poly.polynomial_to_string(d2))
\end{Verbatim}
\end{Exercise}
\begin{Answer}[ref=pydiffpoly]
\begin{Verbatim}
def derivative_of_polynomial(pn):

    # What is the degree of the resulting polynomial?
    original_degree = len(pn) - 1
    if original_degree > 0:
        degree_of_derivative = original_degree - 1
    else:
        degree_of_derivative = 0

    # We can ignore the constant term (skip the first coefficient)
    current_degree = 1
    result = []

    # Differentiate each monomial
    while current_degree < len(pn):
        coefficient = pn[current_degree]
        result.append(coefficient * current_degree)
        current_degree = current_degree + 1

    # No terms? Make it the zero polynomial
    if len(result) == 0:
        result.append(0.0)

    return result
\end{Verbatim}
\end{Answer}

\section{Second order and higher derivatives}
As seen from the example with height, velocity, and acceleration, you can take the derivative of a derivative, which is called the second derivative and indicated with 2 marks, like so: 
$$\frac{d}{dx}f'(x) = f''(x)$$
When you have the height function (or position function, in the case of horizontal motion) of an object, the first derivative describes the velocity of the object, and the second derivative describes the acceleration. Suppose the motion of a particle is given by $s(t) = t^3-5t$, where $s$ is in meters and $t$ is in seconds.  What is the acceleration when the velocity is $0$? First, we find the velocity function, $s'(t)$, and the acceleration function, $s''(t)$:
$$s'(t) = 3t^2-5$$
$$s''(t) = 6t$$
To find where the velocity is $0$, set $s'(t) = 0$:
$$3t^2-5=0$$
$$3t^2=5$$
$$t^2=\frac{5}{3}$$
$$t = \sqrt{\frac{5}{3}} \approx 1.29s$$ (we ignore the other solution, $t=-\sqrt{\frac{5}{3}}$ because it is usual for time to start at zero.)

Next, we use $t\approx 1.29s$ in the acceleration function, $s''(t)$:
$$s''(\sqrt{\frac{5}{3}}) = 6\sqrt{\frac{5}{3}} \approx 7.75 \frac{m}{s^2}$$

For higher order derivatives, you just keep taking the derivative! So a third derivative is found by taking the derivative of the second derivative, and so on. 
\begin{Exercise}[title=Using Derivatives to Describe Motion, label=diffpoly2]
The position of a particle is described by the equation $s(t) = t^4-2t^3+t^2-t$, where $s$ is in meters and $t$ is in seconds. 

(a) Find the velocity and acceleration as functions of $t$.

(b) Find the velocity after 1.5 s. 

(c) Find the acceleration after 1.5 s.

(d) Is the object speeding up or slowing down at $t=1.5$? How do you know? 

\end{Exercise}
\begin{Answer}[ref=diffpoly2]
(a) Velocity is the first derivative of the position function, $s'(t) = 4t^3-6t^2+2t-1$. And acceleration is the derivative of the velocity function, $s''(t) = 12t^2-12t+2$. 

(b) $s'(1.5) = 4(1.5)^3-6(1.5)^2+2(1.5)-1 = 2$ We should note that this is a measurement and needs units to make sense. Since $s$ is in meters and $t$ is in seconds, our velocity should have units of $\frac{m}{s}$, so our final answer is $s'(1.5s) = 2\frac{m}{s}$. 

(c) $s''(1.5) = 12(1.5)^2-12(1.5)+2 = 11$. Similarly to part (b), our answer needs units. The units for acceleration are the units for velocity divided by the unit for time (because acceleration is a rate of change of velocity), and our final answer should be $s''(1.5s) = 11\frac{m}{s^2}$.

(d) When velocity and acceleration are occurring in the same direction (i.e. have the same sign), the speed (the absolute value of velocity) is increasing. Since $s'(1.5s)$ and $s''(1.5s)$ are both $>0$, the speed of the object is increasing. 
\end{Answer}