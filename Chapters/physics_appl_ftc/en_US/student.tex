\chapter{Physics Applications of the Fundamental Theorem of Calculus}

\section{Position, Velocity, and Acceleration}

When we discussed derivatives, you saw that:
$$v(t) = \frac{d}{dt} x(t)$$
$$a(t) = \frac{d}{dt} v(t)$$

If we integrate both sides of these equations and apply the Fundamental Theorem of Calculus, we see that:
$$\int v(t)\,dt + C_x = x(t)$$
$$\int a(t)\,dt + C_v = v(t)$$

Remember: we lose any constants when taking a derivative, so $C_x$ and $C_v$ represent the unknown constants. These constants will be the initial position and initial velocity, respectively. Since these are the position and velocity when $t = 0$, we call them $x_0$ and $v_0$. 
$$x(t) = x_0 + \int v(t)\,dt$$
$$v(t) = v_0 + \int a(t)\,dt$$

%fixme graphic arrows showing relationship among position, velocity, and acceleration
%fixme example integration

\begin{Exercise}[label = velocity]
[This question was adapted from a multiple-choice question originally presented
in the 2012 AP Physics C public practice exam.] A particle of mass $m$ accelerates
from rest starting at time $t = 0$. The acceleration is caused by a single force, 
$F = bt$, where $b$ is a constant. Derive a formula for the particle's velocity, 
$v_x(t)$, in terms of $b$, $t$, and $m$. 
\vspace{50mm}
\end{Exercise}

\begin{Answer}[ref = velocity]
If $F = bt$, then $a(t) = \frac{bt}{m}$. Since the particle starts from rest, we 
know that $v_0 = 0 \frac{m}{s}$. And we know the relationship between acceleration
and velocity:
$$v(t) = v_0 + \int a(t) \, dt$$

Substituting for $v(0)$ and $a(t)$:
$$v(t) = 0 + \int \frac{bt}{m} \, dt = \frac{bt^2}{2m}$$
\end{Answer}

\section{Work and Kinetic Energy}
%intuitive explanation
%equation
%example
%practice

\section{Force, Momentum, and Impulse}
%intuitive explanation
%equation
%example
%practice