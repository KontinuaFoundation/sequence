\chapter{Optimization}

Optimization is a branch of mathematics that involves finding the best
solution from all feasible solutions. In the field of operations
research, optimization plays a crucial role. Whether it is minimizing
costs, maximizing profits, or reducing the time taken to perform a
task, optimization techniques are employed to make decisions
effectively and efficiently.\index{optimization}

In this chapter, the primary goal is to find the maximum or minimum
value of a function, often referred to as the objective function, by finding the point on the graph where the derivative is equal to zero. 
This is usually referred to as the critical point! \index{critical point}

At a critical point where the derivative is zero, the graph of the
function has a horizontal tangent line. At such a point, the function
may be reaching a peak or a valley.

Not every critical point corresponds to a maximum or minimum. Some
critical points may instead be points of inflection. This is when the second derivative of the point is also equal to zero! You will see this in a few examples!

%A standard form of an optimization problem is:
%
%\begin{equation*}
%\begin{aligned}
%& \underset{x}{\text{minimize}}
%& & f(x) \
%& \text{subject to}
%& & g_i(x) \leq 0, ; i = 1, \ldots, m \
%&
%& & h_j(x) = 0, ; j = 1, \ldots, p
%\end{aligned}
%\end{equation*}
%
%where
%\begin{itemize}
%\item $f(x)$ is the objective function,
%\item $g_i(x) \leq 0$ are the inequality constraints,
%\item $h_j(x) = 0$ are the equality constraints.
%\end{itemize}
%
%\section{Types of Optimization Problems}
%There are several different types of optimization problems, such as:
%
%\begin{itemize}
%\item \textbf{Linear Programming:} The objective function and the constraints are all linear.
%\item \textbf{Integer Programming:} The solution space is restricted to integer values.
%\item \textbf{Nonlinear Programming:} The objective function and/or the constraints are nonlinear.
%\item \textbf{Stochastic Programming:} The objective function and/or constraints involve random variables.
%\end{itemize}
%
%These problems are solved using different techniques and algorithms,
%many of which are a subject of active research.
%
%\section{Applications}
%
%Optimization techniques have a wide variety of applications in many
%fields, such as economics, engineering, transportation, and scheduling
%problems.
%


\section{Extrema}

There are two types of extrema: relative extrema and absolute extrema.

A relative maximum or minimum occurs when the function reaches a peak
or valley compared to nearby values. This does not guarantee that the
value is the highest or lowest value of the function overall.\index{relative extrema} \index{relative maximum} \index{relative minimum}

An absolute maximum or minimum is the greatest or least value that the
function takes on a given interval. \index{absolute extrema} \index{absolute maximum} \index{absolute minimum}

An absolute maximum or minimum can occur either at a critical point or
at an endpoint of the interval.

For this reason, when solving optimization problems on a restricted
domain, it is important to test both critical points and endpoints.

\section{The Second Derivative Test}

Once a critical point has been found, we often want to determine whether
it corresponds to a maximum or a minimum. One method for doing this is
the Second Derivative Test. \index{second derivative test}

If a function has a critical point at $x = c$, then:
\begin{itemize}
\item if the second derivative is positive at $x = c$, the function is
concave up and the point is a relative minimum;
\item if the second derivative is negative at $x = c$, the function is
concave down and the point is a relative maximum;
\item if the second derivative is zero at $x = c$, the test is
inconclusive.
\end{itemize}

If the second derivative is always positive or always negative on an
interval, then any critical point in that interval must be an absolute
minimum or absolute maximum, respectively.

\section{Solving Optimization Problems}

Optimization problems apply the ideas of derivatives, critical points,
and extrema to real situations. Although the context of each problem may
be different, the method used to solve them follows the same general
pattern.

\section{General Strategy for Solving Optimization Problems}

Most calculus optimization problems follow these steps:

\begin{enumerate}
\item Identify the quantity to be optimized.
\item Write an equation for the objective function. Sometimes, you might be given an equation already. In that case, you will skip this step.  For word problems, you will typically be required to write the equations yourself. 
\item Use the given constraints to rewrite the function using one
variable.
\item Determine the appropriate domain.
\item Find the derivative of the function.
\item Find critical points by setting the derivative equal to zero.
\item Interpret the result in the context of the problem.
\end{enumerate}

When solving optimization problems, some common mistakes include
forgetting to define variables, not using the constraints of the
problem appropriately, misinterpreting or failing to interpret the
final answer, and giving answers without appropriate units. For word problems, be very vigilant baout how your answer relates back to the question given! 
It serves as a small check for accuracy.

\subsection{Example}

Consider the function
\[
f(x) = x^2 - 4x + 1.
\]

We want to find the maximum or minimum value of this function.

First, take the derivative of the function:
\[
f'(x) = 2x - 4.
\]

Next, set the derivative equal to zero to find the critical point:
\[
2x - 4 = 0.
\]

Solving for $x$ gives
\[
x = 2.
\]

Now determine whether this critical point corresponds to a maximum or a
minimum by using the second derivative.

The second derivative is
\[
f''(x) = 2.
\]

Because the second derivative is positive, the function is concave up,
and the critical point corresponds to a minimum.

To find the minimum value of the function, substitute $x = 2$ back into
the original function:
\[
f(2) = 2^2 - 4(2) + 1 = -3.
\]

Therefore, the function has a minimum value of $-3$ at $x = 2$.

\begin{Exercise}[label = optimization1]
Consider the function
\[
f(x) = x^2 - 6x + 5.
\]

Find the maximum or minimum value of the function using calculus.
\end{Exercise}

\begin{Answer}[ref=optimization1]

First, take the derivative of the function:
\[
f'(x) = 2x - 6.
\]

Set the derivative equal to zero to find the critical point:
\[
2x - 6 = 0.
\]

Solving this equation gives
\[
x = 3.
\]

Next, take the second derivative:
\[
f''(x) = 2.
\]

Since the second derivative is positive, the function is concave up, and
the critical point corresponds to a minimum.

Finally, substitute $x = 3$ back into the original function:
\[
f(3) = 3^2 - 6(3) + 5 = -4.
\]

Therefore, the function has a minimum value of $-4$ at $x = 3$.
\end{Answer}

Now, let's look at a word example:
\subsection*{Example}

The cost, in dollars, of producing $x$ units of a product is given by the
function
\[
C(x) = 2x^2 - 24x + 100.
\]

Find the number of units that should be produced in order to minimize
the cost, and determine the minimum cost.

\medskip

To minimize the cost, take the derivative of the cost function:
\[
C'(x) = 4x - 24.
\]

Set the derivative equal to zero to find the critical point:
\[
4x - 24 = 0.
\]

Solving for $x$ gives
\[
x = 6.
\]

Next, take the second derivative:
\[
C''(x) = 4.
\]

Since the second derivative is positive, the cost function is concave
up, and the critical point corresponds to a minimum.

Substitute $x = 6$ back into the original cost function:
\[
C(6) = 2(6)^2 - 24(6) + 100 = 28.
\]

Therefore, the cost is minimized when 6 units are produced, and the
minimum cost is \$28.

\begin{Exercise}[label = optimization2]
A rectangular enclosure is to be built using 60 units of fencing. Three
sides of the enclosure require fencing, while the fourth side is along
a wall and does not require fencing.

Let $x$ represent the length of the side parallel to the wall and $y$
represent the width of the enclosure.

Find the dimensions of the enclosure that minimize the amount of fencing
used.
\end{Exercise}

\begin{Answer}[ref=optimization2]

Because only three sides require fencing, the total amount of fencing is
given by
\[
x + 2y = 60.
\]

Solving this equation for $x$ gives
\[
x = 60 - 2y.
\]

The area of the enclosure is
\[
A = xy.
\]

Substituting for $x$ yields
\[
A(y) = y(60 - 2y) = 60y - 2y^2.
\]

Take the derivative:
\[
A'(y) = 60 - 4y.
\]

Set the derivative equal to zero:
\[
60 - 4y = 0.
\]

Solving for $y$ gives
\[
y = 15.
\]

The second derivative is
\[
A''(y) = -4.
\]

Since the second derivative is negative, this critical point corresponds
to a maximum area.

Substituting $y = 15$ into the constraint gives
\[
x = 60 - 2(15) = 30.
\]

Therefore, the enclosure has dimensions 30 units by 15 units.
\end{Answer}

\section{Helpful Table for Optimization Problems}
\begin{center}
\begin{tabular}{|p{4cm}|p{6cm}|p{4cm}|}
\hline
\textbf{Condition at a point $x = c$} & \textbf{What it tells you} & \textbf{What kind of point it could be} \\
\hline
$f'(c) \neq 0$ &
The graph has a non-horizontal tangent &
Not a maximum or minimum \\
\hline
$f'(c) = 0$ &
The graph has a horizontal tangent &
Possible maximum, minimum, or neither \\
\hline
$f'(c)$ does not exist &
The graph may have a corner, cusp, or vertical tangent &
Possible maximum or minimum so run second derivative test to check! \\
\hline
$f'(c) = 0$ and $f''(c) > 0$ &
Graph is concave up at $c$ &
Local minimum \\
\hline
$f'(c) = 0$ and $f''(c) < 0$ &
Graph is concave down at $c$ &
Local maximum \\
\hline
$f'(c) = 0$ and $f''(c) = 0$ &
Second derivative test fails &
Could be max, min, or neither \\
\hline
\end{tabular}
\end{center}


% FIXME Using python to find absolute extrema, derivative test. Also update these sections. 

\section{Using Python to Visualize Your Optimization Problem}
You may be familiar with the use of Python to find the derivative of functions. We are going 
to use Python to supplement your knowledge of Optimization problems! 
Below is a script that can be used to visualize what your world problems may be asking you to do! 
This script will help you with finding the first derivative, performing the second derivative test, 
and graphs of your equations showing these points. 

\begin{verbatim}
import sympy as sp
import numpy as np
import matplotlib.pyplot as plt

# 1) Define the variable
x = sp.symbols('x')

# 2) Define the objective function (STUDENTS EDIT THIS)
f = 20*x - x**2

# 3) Derivatives
f_prime = sp.diff(f, x)
f_double_prime = sp.diff(f_prime, x)

# 4) Solve f'(x) = 0
critical_point = sp.solve(f_prime, x)[0]

# 5) Classify using second derivative
second_derivative_value = f_double_prime.subs(x, critical_point)

if second_derivative_value > 0:
    classification = "minimum"
elif second_derivative_value < 0:
    classification = "maximum"
else:
    classification = "inconclusive"

print("f(x) =", f)
print("f'(x) =", f_prime)
print("f''(x) =", f_double_prime)
print("Critical point:", critical_point)
print("Classification:", classification)

# 6) Plot
f_num = sp.lambdify(x, f, "numpy")

X = np.linspace(0, 20, 400)
Y = f_num(X)

xc = float(critical_point)
yc = float(f_num(xc))

plt.plot(X, Y)
plt.scatter([xc], [yc])
plt.title("Optimization")
plt.xlabel("x")
plt.ylabel("f(x)")
plt.grid(True)

plt.annotate(
    classification,
    (xc, yc),
    xytext=(xc + 1, yc),
    arrowprops=dict(arrowstyle="->")
)

plt.show()
\end{verbatim}