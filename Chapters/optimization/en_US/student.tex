\chapter{Optimization}

Optimization is a branch of mathematics that involves finding the best
solution from all feasible solutions. In the field of operations
research, optimization plays a crucial role. Whether it is minimizing
costs, maximizing profits, or reducing the time taken to perform a
task, optimization techniques are employed to make decisions
effectively and efficiently.\index{optimization}

\section{Optimization Problems}

An optimization problem consists of maximizing or minimizing a real
function by systematically choosing the values of real or integer
variables from within an allowed set. This function is known as the
objective function.

A standard form of an optimization problem is:

\begin{equation*}
\begin{aligned}
& \underset{x}{\text{minimize}}
& & f(x) \
& \text{subject to}
& & g_i(x) \leq 0, ; i = 1, \ldots, m \
&
& & h_j(x) = 0, ; j = 1, \ldots, p
\end{aligned}
\end{equation*}

where
\begin{itemize}
\item $f(x)$ is the objective function,
\item $g_i(x) \leq 0$ are the inequality constraints,
\item $h_j(x) = 0$ are the equality constraints.
\end{itemize}

\section{Types of Optimization Problems}
There are several different types of optimization problems, such as:

\begin{itemize}
\item \textbf{Linear Programming:} The objective function and the constraints are all linear.
\item \textbf{Integer Programming:} The solution space is restricted to integer values.
\item \textbf{Nonlinear Programming:} The objective function and/or the constraints are nonlinear.
\item \textbf{Stochastic Programming:} The objective function and/or constraints involve random variables.
\end{itemize}

These problems are solved using different techniques and algorithms,
many of which are a subject of active research.

\section{Applications}

Optimization techniques have a wide variety of applications in many
fields, such as economics, engineering, transportation, and scheduling
problems.

FIXME expand this chapter with examples of calculus optimization using min max from previous chapter

\section{Calculus and Optimization}

Calculus plays a key role in optimization because maximum and minimum
values of a function often occur at critical points. A critical point
occurs when the derivative of a function is equal to zero or does not
exist. For a function defined on an interval, the maximum or minimum
value may occur at a critical point or at an endpoint of the interval.
This idea provides a systematic method for solving optimization
problems.

In order to solve optimization problems, two key components are
important:

\begin{itemize}
\item \textbf{The objective function}, which is the quantity that we
want to maximize or minimize. Common objective functions include
volume, area, and cost.
\item \textbf{The constraint}, which is the condition that limits the
possible values of the objective function. Fixed values of volume,
perimeter, time, or weight are common constraints. When solving
examples, a constraint may be worded like "The weight must be
50 kg'' or "We have 30 years to maximize total profit.'' Whatever
value is fixed in the problem statement is the constraint.
\end{itemize}

\section{General Strategy for Solving Optimization Problems}

Most calculus optimization problems follow these steps:

\begin{enumerate}
\item Identify the quantity to be optimized.
\item Write an equation for the objective function.
\item Use the given constraints to rewrite the function using one
variable.
\item Determine the appropriate domain.
\item Find the derivative of the function.
\item Find critical points by setting the derivative equal to zero.
\item Interpret the result in the context of the problem.
\end{enumerate}

When solving optimization problems, some common mistakes include
forgetting to define variables, not using the constraints of the
problem appropriately, misinterpreting or failing to interpret the
final answer, and giving answers without appropriate units.

% FIXME Using python to find absolute extrema, derivative test. Also update these sections. 

\section{Using Python to Visualize Your Optimization Problem}
You may be familiar with the use of Python to find the derivative of functions. We are going 
to use Python to supplement your knowledge of Optimization problems! 
Below is a script that can be used to visualize what you 
are solving for including maximums, minimums, and 