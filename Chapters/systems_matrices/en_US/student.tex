\chapter{Systems of Equations as Matrices}\label{chap:syseqma}
You have already gained experience with matrices earlier in this module, as well as when you have used spreadsheets. In this chapter, you will learn the types of matrices and get an introduction to some of the special matrices used for various calculations. 

As you know, a matrix is a rectangular array of numbers arranged in rows and columns. The individual numbers in the matrix are called elements or entries. Matrices can be described by their dimensions. For example, a matrix with 2 rows and 3 columns is a 2 by 3 matrix.

$$\begin{bmatrix}
1 & 2 & 3\\
4 & 5 & 6 
\end{bmatrix}
$$

More generally, a matrix with $m$ rows and $n$ columns is referred to as an $m \times n$ matrix, or simply an $m$-by-$n$ matrix; $m$ and $n$ are its dimensions.

The general form of a $2 \times 3$ matrix $A$ is:
$$
A = \begin{bmatrix}
a_{1,1} & a_{1,2} & a_{1,3} \\
a_{2,1} & a_{2,2} & a_{2,3}
\end{bmatrix}
$$

\section{Types of Matrices}
\index{matrices!shapes of}

Matrices can be described by their shape:
\begin{description}
    \item[\textbf{Row Matrix}] A matrix of size $1\times n$ with $1$ row and $n$ columns.
    \item[\textbf{Column Matrix}] A matrix of size $n\times 1$ with $n$ rows and $1$ column, typically used to represent vectors. 
    \item[\textbf{Square Matrix}] A matrix of size $n\times n$, containing the same number of rows and columns
    \item[\textbf{Rectangular Matrix}] A matrix of size $m \times n$, that has an unequal number of rows and columns.
    
\end{description}
\index{matrices!properties of}
They can also be described by their unique numerical properties. Special matrices that come in handy for certains types of computations. These are a few of the most common special matrices:
\begin{description}
	\item[\textbf{Zero Matrix}] A matrix that only contains entries that are zero. $$\vecb{0 & 0 \\ 0 & 0}$$ The matrix above could also be $$0_{2\times 2}$$
       \item[\textbf{Diagonal Matrix}] A square matrix with nonzero entries along the diagonal, and zeroes everywhere else. $$\vecb{2 & 0 & 0 \\ 0 & 4 & 0 \\ 0 & 0 & 6}$$
	\item[\textbf{Identity Matrix}] A diagonal matrix with entries of 1 along the diagonal. $$\vecb{1 & 0 & 0 \\ 0 & 1 & 0 \\ 0 & 0 & 1}$$
	\item[\textbf{Symmetric Matrix}] A Symmetric matrix is a matrix, when transposed, becomes itself.
	$$
       A = \begin{bmatrix}
       20 & 40 & 60 \\
       40 & 50 & 80 \\
       60 & 80 & 100
       \end{bmatrix}
       $$
	\item[\textbf{Triangular Matrix}] This is a special square matrix that can be upper triangular or lower triangular. If upper, the main diagonal and all entries above it are nonzero while the lower entries are all zero. If lower, the main diagonal and all the entries below it are nonzero, while the upper entries are all zero. A matrix that is both upper and lower triangular becomes a diagonal matrix.
	
       Upper Triangular Matrix
       $$
       \begin{bmatrix}
       2 & 3 & 1 \\
       0 & 5 & 4 \\
       0 & 0 & 6
       \end{bmatrix}
       $$
       Lower Triangular matrix
       $$
       \begin{bmatrix}
       7 & 0 & 0 \\
       2 & 5 & 0 \\
       4 & 3 & 9
       \end{bmatrix}
       $$
\end{description}

\subsection{Symmetric Matrices}
\index{symmetric matrices}
If you want to find out if a square matrix is symmetric, you need to transpose it. If the transpose is equal to the original matrix, then the matrix is symmetric.

To transpose a matrix, flip it over its diagonal so that the rows and columns are switched, like this:
$$
A = \begin{bmatrix}
a_{1,1} & a_{1,2} & a_{1,3} \\
a_{2,1} & a_{2,2} & a_{2,3} \\
a_{3,1} & a_{3,2} & a_{3,3}
\end{bmatrix}
$$
After transposing:
$$
A^T = \begin{bmatrix}
a_{1,1} & a_{2,1} & a_{3,1} \\
a_{1,2} & a_{2,2} & a_{3,2} \\
a_{1,3} & a_{2,3} & a_{3,3}
\end{bmatrix}
$$
Note that $A^T$ means the transpose of A. 

Let's see how this works for the following square matrix, A.
$$
A = \begin{bmatrix}
1 & 2 & 3 \\
2 & 4 & 5 \\
3 & 5 & 6
\end{bmatrix}
$$
Switch the rows and columns to get the transpose:
$$
A^T = \begin{bmatrix}
1 & 2 & 3 \\
2 & 4 & 5 \\
3 & 5 & 6
\end{bmatrix}
$$
Notice that $A = A^T$, so the matrix is symmetric.

What about matrix B? 
$$
B = \begin{bmatrix}
1 & 2 & 3 \\
2 & 4 & 5 \\
7 & 8 & 9
\end{bmatrix}
$$
Switch the rows and columns to get the transpose:
$$
B = \begin{bmatrix}
1 & 2 & 7 \\
2 & 4 & 8 \\
3 & 5 & 9
\end{bmatrix}
$$

Note that $B \neq B^T$, so B is not symmetrical.

\begin{Exercise}[title={Matrix Transposition}, label=matrix-transpose01]
Find the transpose of this matrix. Is it symmetric? 
$$A = \begin{bmatrix}
 		3 & -2 &4  \\
 		-2 & 6 &2 \\
 		4 & 2 & 3 
	  \end{bmatrix}$$
\end{Exercise}
\begin{Answer}[ref=matrix-transpose01]
$$A =  A^t = 
	  \begin{bmatrix}
 		3 & -2 & 4  \\
 		-2 & 6 & 2 \\
 		4 & 2 & 3 
	\end{bmatrix}$$
\end{Answer}
\section{Systems of Equations, REF, and RREF}

We have talked a lot about matrices and vectors. Now we can use matrices and vectors to solve systems of equations. A system of linear equations can be written compactly using matrices and vectors. This matrix representation allows us to use algebraic tools such as row operations and matrix multiplication to analyze and solve the system more efficiently. Let's look at a very simple example from a previous section.

\[\begin{cases}
1x_1 + 2x_2 = -1 \\
2x_1 + 0x_2 = 4
\end{cases}\]

We can write this in the form of a matrix! This may come as a surprise to some of you. How can we do this? Let's rewrite the the coefficients as a matrix and the $x_n$'s as column vector, and the constant vector, or right-hand-side vector, called $\vec{b}$. 

Here is the coefficient matrix which we will call $A$:
\[A = \vecb{1 & 2 \\ 2 & 0}\]

Then we can write the column vector of $x$, which we will terminalize as $\vec{x}$
\[\vec{x} = \vecb{x_1 \\ x_2}\] 

And we can write the $\vec{b}$ as the following column vector:

\[
\vec{b} = \vecb{-1 \\ -4}
\]

Now, if we multiply $A$ and $\vec{x}$,
$$A\vec{x} = \vecb{1 & 2 \\ 2 & 0}\vecb{x_1 \\ x_2} = \vecb{1x_1 + 2x_2 \\ 2x_1 + 0x_2}$$

We get exactly the left-hand sides of our original system of equations!
Therefore, when we set this equal to the right-hand-side vector $\vec{b}$:
$$\vec{b}=\vecb{-1 \\ 4}$$
we get the fundamental system of equation formula represented as a matrix Equation
\begin{equation}
       A\vec{x} = \vec{b}
       \label{eq:matrixform}
\end{equation}

This is a very important equation for linear algebra as a whole. Notice that the size of matrix has to respect the matrix multiplication rules:
\[\underset{m \times n}{A} \cdot \underset{n \times 1}{\vec{b}} = \underset{m\times 1}{\vec{p}}\]

\begin{Exercise}[title=Matrix Equation, label=matrixeq1]
       Rewrite the systems of equations in matrix form and identify $A$, $\vec{x}$, and $\vec{b}$. 
       \[\begin{cases}
       -4x_1 + 9x_2 - 8x_3= 5\\
       -1x_1 + 0x_2 + 6x_3= 7
       \end{cases}\]
\end{Exercise}

\begin{Answer}[ref=matrixeq1]
     \[A = \vecb{-4 & 9 & -8 \\ -1 & 0 & 6}, \quad \vec{x} = \vecb{x_1 \\ x_2 \\ x_3}, \quad \vec{b} = \vecb{5 \\ 7}\]
     \[A\vec{x}=\vec{b} \implies \vecb{-4 & 9 & -8 \\ -1 & 0 & 6}\vecb{x_1 \\ x_2 \\ x_3} = \vecb{5 \\ 7}\]
\end{Answer}

\section{Row-Echelon Form}
This brings up the question, how can we solve for the $\vec{x}$? There are multiple ways, but most commonly we form an augmented matrix and solve for Row-Echelon Form. 
\subsection{The Augmented Matrix}
What is the augmented matrix? It is a way of represented the $A$ matrix and $\vec{b}$ vector. We write the $A$ matrix, seperated by a vertical line, and then writing the $\vec{b}$ vector:
$$\begin{array}{c} [A \mid \vec{b}] =
\begin{bmatrix}
a_{11} & a_{12} & \cdots & a_{1n} & \vert & b_1 \\
a_{21} & a_{22} & \cdots & a_{2n} & \vert & b_2 \\
\vdots & \vdots &        & \vdots & \vert & \vdots \\
a_{m1} & a_{m2} & \cdots & a_{mn} & \vert & b_m
\end{bmatrix} \end{array}$$

The left block contains all the coefficients of the system of equations, and the right block contains only the solutions. Then, to solve for $\vec{x}$, we perform \textit{Elementary Row Operations} to find the \textit{Row-Echelon Form}, mathematically stated as $\text{rref}(A\vert \vec{b})$. Let's continue with the matrix vector set we were experimenting with:
$$A\vec{x} = \vecb{1 & 2 \\ 2 & 0}\vecb{x_1 \\ x_2} = \vecb{1x_1 + 2x_2 \\ 2x_1 + 0x_2}$$
This can be written as an augmented matrix:

$$\vecb{A\vert \vec{b}} = \vecb{1 & 2 &\vert & -1 \\ 2 & 0 & \vert & 4}$$


\subsection{Elementary Row Operations}
The Row-Echelon Form of a matrix is an \emph{equivalent} form of a matrix obtained through Gaussian and Gauss-Jordan elimination. There are lots of videos on this process online, so multiple videos included in your digital resources to walk you through the process. The rules of Gaussian Elimination are simple, only 3 different elementary row operations, or ERO's can be performed:

\begin{itemize}
       \item Swapping two rows
       \item multiply a row by a non-zero scalar
       \item dd a multiple of one row to another row
\end{itemize}
Why do these operations preserve the given linear system? Swapping two rows only changes the order in which the equations are written and therefore has no effect on the solution set. Multiplying a row by a nonzero scalar corresponds to multiplying both sides of an equation by the same scalar, which does not alter its solutions. Finally, if two equations are satisfied by the same solution, then any linear combination of those equations is also satisfied by that solution; in particular, adding a multiple of one equation to another preserves the solution set. Before and after each operation, the augmented matrix contains exactly the same information as the system of equations and therefore represents the same solution set.

With these ERO's, we are trying to achieve a matrix with the following properties:
\begin{itemize}
\item Leading entries (not necessarily 1) move to the right as you go down
\item Rows of zeros, if exist, are at the bottom
\item Each pivot has zeros in all positions below it in the same column.
\end{itemize}

Lets walk through the steps to simplify $\text{ref}(A\vert \vec{b})$:

\begin{align*}
\vecb{1 & 2 &\vert & -1 \\ 2 & 0 & \vert & 4} \\
R_2 \rightarrow R_2 - 2R_1 &\implies \vecb{1 & 2 &\vert & -1 \\ 0 & -4 & \vert & 6} \\
\end{align*}

As you can see, we set Row 2 to be Row 2 minus 2 times Row 1. This gives us a zero in the first column of Row 2. Technically, this matrix is now in Row-Echelon Form (REF). However, we can continue applying elementary row operations to obtain an even more structured form called the \emph{Reduced Row-Echelon Form}.

\subsection{Reduced Row Echelon Form}
A matrix is in \emph{Reduced Row-Echelon Form} (RREF) if it satisfies all of the conditions for Row-Echelon Form, and in addition:

\begin{itemize}
\item Each leading entry is equal to $1$
\item Each leading entry is the only nonzero entry in its column
\end{itemize}

In other words, not only do pivots move to the right as we move down the rows, but each pivot column contains zeros both above and below the pivot, and each pivot is normalized to $1$. These additional conditions make RREF especially useful, since the solutions to the system can often be read directly from the matrix.

Continuing our example, we start with the matrix in REF:
\begin{align*}
\vecb{1 & 2 &\vert & -1 \\ 0 & -4 & \vert & 6}
\end{align*}

First, make the leading entry in Row 2 equal to $1$ by dividing the row by $-4$:
\begin{align*}
R_2 \rightarrow -\frac{1}{4}R_2 
&\implies 
\vecb{1 & 2 &\vert & -1 \\ 0 & 1 & \vert & -\frac{3}{2}}
\end{align*}

Next, eliminate the entry above the pivot in column 2:
\begin{align*}
R_1 \rightarrow R_1 - 2R_2 
&\implies 
\vecb{1 & 0 &\vert & 2 \\ 0 & 1 & \vert & -\frac{3}{2}}
\end{align*}

The matrix is now in Reduced Row-Echelon Form.

\subsection{Writing the Solution Vector}

Recall that the augmented matrix represents a system of linear equations. After reducing the matrix to Reduced Row-Echelon Form, we obtain
\begin{align*}
\vecb{1 & 0 &\vert & 2 \\ 0 & 1 & \vert & -\frac{3}{2}}
\end{align*}

Each row now corresponds directly to an equation in the variables $x_1$ and $x_2$:
\begin{align*}
x_1 &= 2 \\
x_2 &= -\frac{3}{2}
\end{align*}

Thus, the solution to the system can be written as the vector
\[
\vec{x}
=
\vecb{2 \\-\frac{3}{2}}
\]

Since every variable corresponds to a pivot column, the system has a \emph{unique solution}.

We have sucessfully solved the system of equations using matrix methods:
$$\vecb{1 & 2 \\ 2 & 0}\vecb{2 \\ -\frac{3}{2}} = \vecb{-1 \\ 4}$$

This is a fundamental technique in linear algebra, and you will use it often in this module and beyond!

\section{Larger RREF Example}

Let's solve a larger system by finding $\vec{x}$ in the matrix equation
\[
A\vec{x}=\vec{b}.
\]

\subsection{Write the System and Identify $A$, $\vec{x}$, and $\vec{b}$}

Consider the following system of equations:
\[
\begin{cases}
x_1 + x_2 + x_3 + x_4 = 4 \\
2x_1 + x_2 + 3x_3 + x_4 = 9 \\
x_1 + 3x_2 + 2x_3 + 2x_4 = 10 \\
3x_1 + x_2 + 2x_3 + 4x_4 = 13
\end{cases}
\]

The coefficient matrix, variable vector, and constant vector are:
\[
A=\vecb{
1 & 1 & 1 & 1\\
2 & 1 & 3 & 1\\
1 & 3 & 2 & 2\\
3 & 1 & 2 & 4
},
\qquad
\vec{x}=\vecb{x_1\\x_2\\x_3\\x_4},
\qquad
\vec{b}=\vecb{4\\9\\10\\13}.
\]

So the matrix equation is:
\[
A\vec{x}=\vec{b}.
\]

\subsection{Form the Augmented Matrix}

\[
\vecb{A\vert \vec{b}}
=
\vecb{
1 & 1 & 1 & 1 & \vert & 4\\
2 & 1 & 3 & 1 & \vert & 9\\
1 & 3 & 2 & 2 & \vert & 10\\
3 & 1 & 2 & 4 & \vert & 13
}.
\]

\subsection{Row-Reduce to REF}

Use the pivot in Row 1 to clear entries below it in column 1:
\begin{align*}
R_2 &\rightarrow R_2 - 2R_1 \\
R_3 &\rightarrow R_3 - R_1 \\
R_4 &\rightarrow R_4 - 3R_1
\end{align*}

\begin{align*}
\implies
\vecb{
1 & 1 & 1 & 1 & \vert & 4\\
0 & -1 & 1 & -1 & \vert & 1\\
0 & 2 & 1 & 1 & \vert & 6\\
0 & -2 & -1 & 1 & \vert & 1
}.
\end{align*}

Now use the pivot in Row 2 (column 2) to clear entries below it in column 2:
\begin{align*}
R_3 &\rightarrow R_3 + 2R_2 \\
R_4 &\rightarrow R_4 - 2R_2
\end{align*}

\begin{align*}
\implies
\vecb{
1 & 1 & 1 & 1 & \vert & 4\\
0 & -1 & 1 & -1 & \vert & 1\\
0 & 0 & 3 & -1 & \vert & 8\\
0 & 0 & -3 & 3 & \vert & -1
}.
\end{align*}

Next use the pivot in Row 3 (column 3) to clear the entry below it in column 3:
\begin{align*}
R_4 \rightarrow R_4 + R_3
\end{align*}

\begin{align*}
\implies
\vecb{
1 & 1 & 1 & 1 & \vert & 4\\
0 & -1 & 1 & -1 & \vert & 1\\
0 & 0 & 3 & -1 & \vert & 8\\
0 & 0 & 0 & 2 & \vert & 7
}.
\end{align*}

At this point, the matrix is in Row-Echelon Form (REF).

\subsection{Continue to RREF}

Make the pivot in Row 4 equal to $1$:
\begin{align*}
R_4 \rightarrow \frac{1}{2}R_4
\implies
\vecb{
1 & 1 & 1 & 1 & \vert & 4\\
0 & -1 & 1 & -1 & \vert & 1\\
0 & 0 & 3 & -1 & \vert & 8\\
0 & 0 & 0 & 1 & \vert & \frac{7}{2}
}.
\end{align*}

Clear the entries above the pivot in column 4:
\begin{align*}
R_1 &\rightarrow R_1 - R_4\\
R_2 &\rightarrow R_2 + R_4\\
R_3 &\rightarrow R_3 + R_4
\end{align*}

\begin{align*}
\implies
\vecb{
1 & 1 & 1 & 0 & \vert & \frac{1}{2}\\
0 & -1 & 1 & 0 & \vert & \frac{9}{2}\\
0 & 0 & 3 & 0 & \vert & \frac{23}{2}\\
0 & 0 & 0 & 1 & \vert & \frac{7}{2}
}.
\end{align*}

Make the pivot in Row 3 equal to $1$:
\begin{align*}
R_3 \rightarrow \frac{1}{3}R_3
\implies
\vecb{
1 & 1 & 1 & 0 & \vert & \frac{1}{2}\\
0 & -1 & 1 & 0 & \vert & \frac{9}{2}\\
0 & 0 & 1 & 0 & \vert & \frac{23}{6}\\
0 & 0 & 0 & 1 & \vert & \frac{7}{2}
}.
\end{align*}

Clear the entries above the pivot in column 3:
\begin{align*}
R_1 &\rightarrow R_1 - R_3\\
R_2 &\rightarrow R_2 - R_3
\end{align*}

\begin{align*}
\implies
\vecb{
1 & 1 & 0 & 0 & \vert & -\frac{10}{3}\\
0 & -1 & 0 & 0 & \vert & \frac{2}{3}\\
0 & 0 & 1 & 0 & \vert & \frac{23}{6}\\
0 & 0 & 0 & 1 & \vert & \frac{7}{2}
}.
\end{align*}

Make the pivot in Row 2 equal to $1$:
\begin{align*}
R_2 \rightarrow -R_2
\implies
\vecb{
1 & 1 & 0 & 0 & \vert & -\frac{10}{3}\\
0 & 1 & 0 & 0 & \vert & -\frac{2}{3}\\
0 & 0 & 1 & 0 & \vert & \frac{23}{6}\\
0 & 0 & 0 & 1 & \vert & \frac{7}{2}
}.
\end{align*}

Clear the entry above the pivot in column 2:
\begin{align*}
R_1 \rightarrow R_1 - R_2
\end{align*}

\begin{align*}
\implies
\vecb{
1 & 0 & 0 & 0 & \vert & -\frac{8}{3}\\
0 & 1 & 0 & 0 & \vert & -\frac{2}{3}\\
0 & 0 & 1 & 0 & \vert & \frac{23}{6}\\
0 & 0 & 0 & 1 & \vert & \frac{7}{2}
}.
\end{align*}

This matrix is now in Reduced Row-Echelon Form (RREF).

\subsection{Write the Solution Vector}

Reading each row as an equation gives:
\begin{align*}
x_1 &= -\frac{8}{3}\\
x_2 &= -\frac{2}{3}\\
x_3 &= \frac{23}{6}\\
x_4 &= \frac{7}{2}
\end{align*}

Thus the solution vector is:
\[
\vec{x}=\vecb{
-\frac{8}{3}\\[4pt]
-\frac{2}{3}\\[4pt]
\frac{23}{6}\\[4pt]
\frac{7}{2}
}.
\]

You can verify the solution by confirming that $A\vec{x}=\vec{b}$ using matrix multiplication.

FIXME exercises
FIXME problem with free variable

\section{Creating Matrices in Python}
Recall that matrices can be represented by 2D arrays in Python. Let's explore this idea further with the module \mintinline{python}|numpy|. Create a file called \filename{matrices\_creation.py} and enter this code:
\begin{minted}{python}
# import the python module that supports matrices
import numpy as np
# Use the function np.array to define a matrix that 
# contains specific values that you supply.
A = np.array([[ 5, 1, 3], 
              [ 1, -1, 8], 
              [ 6, 2, 1]])
# The transpose function returns 
A.transpose()
\end{minted}
When you run it, you should see:
\begin{minted}{python}
array([[ 5, 1, 6], 
       [ 1, -1, 2], 
       [ 3, 8, 1]])
\end{minted}
As you can see, $A\neq A^T$, so A is not symmetric.
Try another: 
\begin{minted}{python}
# create a matrix, B
B = np.array([[ 5, 1, 6], 
              [ 1, -1, 2], 
              [ 6, 2, 1]])
B.transpose()
\end{minted}
When you run it, you should see:
\begin{minted}{python}
array([[ 5, 1, 6], 
       [ 1, -1, 2], 
       [ 6, 2, 1]])
\end{minted}
B is symmetric. You can actually transpose any matrix using this function, but a matrix cannot be symmetric unless it is square. 

Try this code to see what happens when you transpose a rectangular matrix. 
\begin{minted}{python}
# create a matrix, J
J = np.array([[ 5, 1, 3, 0], 
              [ 1, -1, 8, 11], 
              [ 6, 2, 1,-7]])
J.transpose()
\end{minted}
Note that transposing a rectangular matrix changes its dimension from 3 by 4 to 4 by 3. You should see a transposed matrix, but it's not symmetric.
\begin{minted}{python}
array([[ 5,  1,  6],
       [ 1, -1,  2],
       [ 3,  8,  1],
       [ 0, 11, -7]])
\end{minted}

\subsection{Creating Special Matrices in Python}
 Use the same file to add this code for creating a zero matrix.
\begin{minted}{python}
# create an 8 by 10 Zero matrix.
 C = np.zeros((8, 10))
 C
\end{minted}
When you run it, you should see:
\begin{minted}{python}
array([[0., 0., 0., 0., 0., 0., 0., 0., 0., 0.],
       [0., 0., 0., 0., 0., 0., 0., 0., 0., 0.],
       [0., 0., 0., 0., 0., 0., 0., 0., 0., 0.],
       [0., 0., 0., 0., 0., 0., 0., 0., 0., 0.],
       [0., 0., 0., 0., 0., 0., 0., 0., 0., 0.],
       [0., 0., 0., 0., 0., 0., 0., 0., 0., 0.],
       [0., 0., 0., 0., 0., 0., 0., 0., 0., 0.],
       [0., 0., 0., 0., 0., 0., 0., 0., 0., 0.]])
\end{minted}
Add the following code to create an 8 by 8 Identity matrix.
\begin{minted}{python}
# create an 8 by 8 Identity matrix 
 D = np.eye(8)
 D
\end{minted}
When you run it, you should see:
\begin{minted}{python}
array([[1., 0., 0., 0., 0., 0., 0., 0.],
       [0., 1., 0., 0., 0., 0., 0., 0.],
       [0., 0., 1., 0., 0., 0., 0., 0.],
       [0., 0., 0., 1., 0., 0., 0., 0.],
       [0., 0., 0., 0., 1., 0., 0., 0.],
       [0., 0., 0., 0., 0., 1., 0., 0.],
       [0., 0., 0., 0., 0., 0., 1., 0.],
       [0., 0., 0., 0., 0., 0., 0., 1.]])
\end{minted}
As you progress in your studies, you will learn the importance of diagonal matrices and of extracting the diagonal of a matrix. Let's see how to extract a diagonal, then create a diagonal matrix.
\begin{minted}{python}
# create a  matrix 
W = np.array([[1, 2, 3, 4], 
              [5, 6, 7, 8], 
              [-8, -7, -6, -5],
              [-4, -3, -2, -1]])
\end{minted}
Extract the main diagonal using np.diag(<array>,<diagonal to extract>). 
Passing 0 as the second parameter specifies the main diagonal. A positive value extracts a diagonal from the upper part. A negative value extracts a diagonal from the lower part. Run this code then experiment passing other values to see what you get.
\begin{minted}{python}
print(np.diag(W,0))
\end{minted}
When you run it, you should see:
\begin{minted}{python}
array([ 1,  6, -6, -1])
\end{minted}
You can also use np.diag() to create a diagonal matrix from a 1D array. In this case, do not pass a second paramenter.
\begin{minted}{python}
Q = np.array([1, 2, 3])
DiagArray = np.diag(Q))
print(DiagArray)
\end{minted}
When you run it you should see;
\begin{minted}{python}
[[1 0 0]
 [0 2 0]
 [0 0 3]]
\end{minted}
Python has functions for extracting upper and lower triangular matrices. Try these:
\begin{minted}{python}
print(np.triu(W))
print(np.tril(W))
\end{minted}
You should see:
\begin{minted}{python}
[[ 1  2  3  4]
 [ 0  6  7  8]
 [ 0  0 -6 -5]
 [ 0  0  0 -1]]
 
[[ 1  0  0  0]
 [ 5  6  0  0]
 [-8 -7 -6  0]
 [-4 -3 -2 -1]]
\end{minted}


\section{Conclusion}
In this module, we have explored the fundamental concepts of matrices, including their types, properties, and how they can be used to represent and solve systems of linear equations. We have also learned how to perform row operations to convert matrices into Row-Echelon Form (REF) and Reduced Row-Echelon Form (RREF), which are essential techniques in linear algebra. In the next chapter~\ref{chap:subspaces}, we will talk deeper about the concepts of vector spaces and subspaces and how to find RREF, building on our understanding of matrices and their applications.

FIXME digital resources