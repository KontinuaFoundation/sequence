\chapter{Link Functions}


In generalized linear models, the link function provides the
relationship between the linear predictor and the mean of the
distribution function. Different choices of link function can be used
to model different types of relationships. Here are a few commonly
used link functions:\index{link functions}

\begin{enumerate}

\item \textbf{Identity link:} The identity link function is the
  simplest form of link function, where the response variable is
  expected to be the linear combination of the predictors. This is the
  default link function for Gaussian family distributions.\index{identity link function}

\[
g(\mu) = \mu
\]

\item \textbf{Log link:} The log link function is used when modeling
  positive data and count data. This link function is the default for
  Poisson and exponential family distributions.\index{log link function}

\[
g(\mu) = \log(\mu)
\]

\item \textbf{Logit link:} The logit link function is often used when
  modeling binary response data, and is the default link function for
  binomial family distributions. It gives the log-odds, or the
  logarithm of the odds $p/(1-p)$. \index{logit link function}

\[
g(\mu) = \log\left(\frac{\mu}{1-\mu}\right)
\]

\item \textbf{Probit link:} The probit link function is another common
  choice for binary response data. It is based on the cumulative
  distribution function of the standard normal distribution.\index{probit link function}

\[
g(\mu) = \Phi^{-1}(\mu)
\]

where $\Phi^{-1}(\cdot)$ is the inverse cumulative distribution
function of the standard normal distribution.

\item \textbf{Inverse link:} The inverse link function is often used
  in modeling rates or times. It is the canonical (or default) link
  function for the Gamma family distributions.

\[
g(\mu) = \mu^{-1}
\]

\end{enumerate}

Different link functions can substantially impact the model's
interpretation, so it's crucial to choose a link function that aligns
with the nature of the data and the scientific question at hand.
