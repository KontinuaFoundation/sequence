\chapter{Vector Spans and Independence}


In linear algebra, the span of a set of vectors is the set of all possible linear combinations of those vectors. If the set $S = \{v_1, v_2, ..., v_n\}$ contains vectors from a vector space $V$, then the span of $S$ is given by:

\begin{equation}
\text{Span}(S) = \{a_1v_1 + a_2v_2 + ... + a_nv_n : a_1, a_2, ..., a_n \in \mathbb{R}\}
\end{equation}

This means that any vector in the Span$(S)$ can be written as a linear combination of the vectors in $S$.

\section{Vector Independence}

A set of vectors $S = \{v_1, v_2, ..., v_n\}$ is said to be linearly independent if the only solution to the equation 

\begin{equation}
a_1v_1 + a_2v_2 + ... + a_nv_n = 0
\end{equation}

is $a_1 = a_2 = ... = a_n = 0$. This means that no vector in the set can be written as a linear combination of the other vectors.

If there exists a nontrivial solution (i.e., a solution where some $a_i \neq 0$), then the vectors are said to be linearly dependent. This means that at least one vector in the set can be written as a linear combination of the other vectors.

The concept of vector independence is crucial in many areas of linear algebra, including the study of vector spaces, bases, and rank.

