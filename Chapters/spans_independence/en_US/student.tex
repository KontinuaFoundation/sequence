\chapter{Vector Independence}
Think back to a time when you played with blocks. If you had two blocks, you 
couldn't make many shapes out of them. With three, you had a few more options. 
With a dozen, you were able to make many more shapes. In the world of blocks, 
a \newterm{span} would be all the things you could make with a given set of blocks. 
\index{span}
A vector span is similar, but in a mathematical sense. If I give you the 
coordinates for a vector and ask you to make everything you can from that 
vector using only the original vector, the result is the span. You can scale 
the vector, add it to itself --- anything that is a linear combination of only 
that vector. As with the blocks, you'll find what you can make from one vector 
is limited. The span will be a line. However, when you are given two or more 
vectors to "play" with, you will be able to create much more. The span will be 
larger than in the case of having only one vector. The size of the span 
(sometimes referred to as a subspace) will depend on whether the vectors are 
linearly independent or dependent. In this chapter, we will examine what 
independence and dependence mean for vectors. In the next chapter, we will 
apply what we've learned about independence and dependence to determine the 
span of a set of vectors. 

\section{Overview: Independence and Dependence}
You saw some linearly dependent vectors in the previous chapter. Now, we will 
expand this concept. A set of linearly independent vectors means that no 
vector is a combination of any other vector. Let's look at these three:
$$[1, 0, 0]$$
$$[0, 1, 0]$$
$$[0, 0, 1]$$

If you scale each vector as much as possible, the span encompasses the entire 
3D real space. Any linear combination of these three vectors from the origin to any point in $\mathbb{R}^3$ is reachable using scalars. 

A set of linearly dependent vectors means one or more of the vectors can be 
written as a combination of one of the vectors.

For example:
$$v_1 = [7 -2 2]$$
$$v_2 = [14 -4 4]$$

You can see that $v_2$ is $2\cdot v_1$. They are linearly dependent. This is a 
simple example, but when you encounter larger matrices, it won't be as obvious. 
You will learn computational techniques for figuring out independence.

Vector spans have practical applications in a number of fields. Computer 
graphics and physics are two of them. For example, in space travel, knowing 
the vector span is essential to calculating a slingshot maneuver that will 
give spacecraft a gravity boost from a planet. For this, you'd need to know 
the gravity vector of the planet relative to the sun and the velocity vectors 
that characterize the spacecraft. Engineers would use this information to 
figure out the trajectory angle that would allow the spacecraft to achieve a 
particular velocity in the desired direction. The span constrains the set of 
successful solutions.

\section{Vector Independence}
A set of vectors $S = \{\mathbf{v_1}, \mathbf{v_2}, \dots, \mathbf{v_n}\}$ is 
linearly independent if the only solution to the equation:
$$a_1 \mathbf{v_1} + a_2 \mathbf{v_2} + \dots + a_n \mathbf{v_n} = 0$$

is 
$$a_1 = a_2 = ... = a_n = 0$$ 

This means that no vector in the set can be written as a linear combination of 
the other vectors.

If there exists a nontrivial solution (i.e., a solution where some $a_i \neq 
0$), then the vectors are said to be linearly dependent. This means that at 
least one vector in the set can be written as a linear combination of the 
other vectors.

The concept of vector independence is fundamental to the study of vector 
spaces, bases, and rank. You will learn more about these concepts in future 
modules. 

\subsection{Dependent Vectors}
Let's start by looking at two vectors. 

$$\mathbf{v_1} = \begin{bmatrix}
			2 \\
			4
		\end{bmatrix}$$
$$\mathbf{v_2} = \begin{bmatrix}
			-14 \\
			-28
\end{bmatrix}$$

These two vectors are dependent, because $\mathbf{v_2} = -7 \cdot \mathbf{v_2}$. This is an obvious example, but let's show it mathematically. If linearly independent, the two vectors must satisfy:

	$$a_1 \mathbf{v_1} + a_2 \mathbf{v_2} = 0$$

which is:
	$$2a_1 -14a_2 = 0$$
	$$4a_1 -28a_2 = 0$$

To solve, multiply the top equation by -2 and add it to the bottom: 
$$-4a_1 + 28a_2 = 0 $$
$$ 0  + 0     = 0 $$

The bottom equation drops out. Now, solve for $a_1$ in the remaining equation:
$$a_1 = -7a_2$$
As you can see, one vector is a multiple of another. $$a_1 \neq a_2 \neq 0$$

\subsection{Independent Vectors}
\index{vector independence}
Let's see if these two vectors are independent.
$$\mathbf{v_1} = \begin{bmatrix}
1 \\
0
\end{bmatrix}$$
$$\mathbf{v_2} = \begin{bmatrix}
0 \\
-1
\end{bmatrix}$$

To be independent, the two vectors must satisfy:
	$$a_1 \mathbf{v_1} + a_2 \mathbf{v_2} = \vec{0}$$
	
which is:
$$\begin{bmatrix}
	a_1 + 0\cdot a_2 = 0 \\
	0\cdot a_1 + -a_2 = 0
\end{bmatrix}$$

So:
$$a_1 = a_2 = 0$$
These vectors are not only independent, but they are orthogonal (perpendicular)
to one another. You'll learn more about orthogonality later.

Here is an example whose solution isn't as obvious. You can solve using 
Gaussian elimination.
$$\mathbf{v_1} = [2, 1]$$
$$\mathbf{v_2} = [1, -6]$$

Rewrite as a system of equations:
$$a_1\cdot 2 + a_2 \cdot 1 = 0 $$
$$a_1 \cdot 1 + a_2 \cdot (-6) = 0$$

First, swap the equations so that the the top equation has a coefficient of 1 
for $a_1$:
$$a_1 - 6a_2 = 0$$ 
$$2a_1 + a_2 = 0$$ 
Next, multiply row 1 by -2 and add it to row 2:
$$a_1 - 6a_2 = 0$$ 
$$0  - 11a_2 = 0$$ 

Multiply row 2 by 1 divided by 11.
$$a_1 - 6a_2 = 0$$ 
$$0 + a_2 = 0$$ 

Back substitute $a_2$ solution into the first equation:
$$a_1  = 0$$
$$a_2 = 0$$ 
Therefore, $a_1 = a_2 = 0$ and the two vectors are linearly independent.

\begin{Exercise}[title={Vector Independence}, label=vector_independence]
    Are these vectors independent? 
$$\begin{matrix}[2 & 1  & 4]\end{matrix}$$
$$\begin{matrix}[2  & -1  & 2]\end{matrix}$$ 
$$\begin{matrix}[0  & 1  & -2]\end{matrix}$$
Show your work.
\end{Exercise}

\begin{Answer}[ref=vector_independence]
    Rewrite as a system of equations:
        $$\begin{matrix}
			2\cdot a_1 +2\cdot a_2 + 0\cdot a_3 = 0 \\
			1\cdot a_1 - 1\cdot a_2 + 1\cdot a_3 = 0 \\
			4\cdot a_1 + 2\cdot a_2 - 2\cdot a_3 = 0
		  \end{matrix} $$
	Simplify
		$$\begin{matrix}
			2a_1 +2\cdot a_2 = 0 \\
			a_1 - a_2 + a_3 = 0 \\
			4a_1 + 2a_2 - 2a_3 = 0
		  \end{matrix} $$
	Swap row 2 and 1:
		$$\begin{matrix}
			a_1 - a_2 + a_3 = 0 \\
			2a_1 + 2\cdot a_2 = 0 \\
			4a_1 + 2a_2 - 2a_3 = 0
		  \end{matrix} $$
	Multiply row 1 by -2 and add to row 2:
	   $$\begin{matrix}
			a_1 - a_2 + a_3 = 0 \\
			0 +  3\cdot a_2 - 2a_3  = 0 \\
			4a_1 + 2a_2 - 2a_3 = 0
		  \end{matrix} $$
	Multiply row 1 by -4 and add to row 3:	
	    $$\begin{matrix}
			a_1 - a_2 + a_3 = 0 \\
			0 + 3\cdot a_2 -2a_3 = 0 \\
			0 + 6a_2 - 6a_3 = 0
		  \end{matrix} $$
	Multiply row 2 by -4 and add to row 3:
	   $$\begin{matrix}
			a_1 - a_2 + a_3 = 0 \\
			0 + 3\cdot a_2 -2a_3 = 0 \\
			0 + 0 - 2a_3 = 0
		  \end{matrix} $$
	Multiply row 3 by -1 and add to row 2:
		$$\begin{matrix}
			a_1 - a_2 + a_3 = 0 \\
			0 + 3\cdot a_2 + 0 = 0 \\
			0 + 0 - 2a_3 = 0
		\end{matrix} $$
    Divide row 3 by -2 and row 2 by $\frac{1}{3}$:
    	$$\begin{matrix}
			a_1 - a_2 + a_3 = 0 \\
			0 +  a_2 +0  = 0 \\
			0  + 0  + a_3 = 0
		\end{matrix} $$
	Backsubstitute $a_2$ and $a_3$ into row 1:
	 	$$\begin{matrix}
			a_1 + 0 + 0 = 0 \\
			0 +  a_2 + 0   = 0 \\
			0   + 0  + a_3 = 0
		\end{matrix} $$
	 Therefore $$a_1 = a_2 = a_3 = 0$$.
\end{Answer}
    
\section{Checking for Linear Independence Using Python}  
One way to use Python to check for linear independence is to use the 
linalg.solve() function to solve the system of equations. You need to create 
an array that contains the coefficients of the variable and a vector that 
contains the values on the right-side of each equation. So far, you have either 
been given equations that equal 0 or you have manipulated each equation to be 
equal to 0. 

Let's first see how to use Python to solve the equations in the previous 
exercise. If the equations are linearly independent, then $a_1 = a_2 = a_3 = 0$.

Create a file called span\_independence.py and enter this code:
\begin{Verbatim}
import numpy as np

A = np.array([[2, 2, 0], 
              [1, -1, 1],
              [4, 2, -2]])
b = np.array([0, 0, 0])
c = np.linalg.solve(A,b)
print(c)
\end{Verbatim}
You should get this result, which shows the equations are linearly independent.
\begin{Verbatim}
[0., -0.,  0.]
\end{Verbatim}
However, what happens if the equations are not independent? Let's make the 
first two equations dependent by making equation 1 two times equation 2. Enter 
this code into your file:
\begin{Verbatim}
import numpy as np

D = np.array([[2, -2, 2], 
              [1, -1, 1],
              [4, 2, -2]])
e = np.array([0, 0, 0])
f = np.linalg.solve(D,f)
print(f)

You should get many lines indicating an error. Among the spew, you should see:

raise LinAlgError("Singular matrix")
\end{Verbatim}
So, while the linalg.solve() function is quite useful for solving a system of 
independent linear equations, raising an error is not the most elegant way to 
figure out if the equations are dependent. That is where the concept of a 
determinant comes in. You will learn about that in the next section, but for 
now, let's use the  linalg.solve() function to find a solution for a set of 
equations known to be linearly independent.
$$4x_1 + 3x_2 - 5x_3 = 2$$
$$-2x_1- 4x_2 - 5x_3 = 5$$
$$       8x_2 + 8x_3  = -3$$
You will create a matrix that contains all the coefficients and a vector that 
contains the values on the right-side of the equations. 

Enter this code into your file. 
\begin{Verbatim}
G = np.array([[4, 3, -5], 
              [-2, -4, 5], 
              [8, 8, 0]])
h = np.array([2, 5, -3])

j = np.linalg.solve(G, h)
print(j)
\end{Verbatim}
You should get this answer:
\begin{Verbatim}
[2.20833333, -2.58333333, -0.18333333]
\end{Verbatim}

