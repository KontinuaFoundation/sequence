\chapter{Representing Natural Numbers}

Natural numbers are positive whole numbers, such as 1, 2, 3, and so on.  -5 is not a natural
number.  $\pi$ is not a natural number. $\frac{1}{2}$ is not a natural
number.

You are used to seeing the natural numbers represented in a base-10
\newterm{Hindu-Arabic} numeral system.  That is, when you see 2531 you
think ``2 thousands, 5 hundreds, 3 tens, and 1 one.''  Rewritten, this is:

$$2 \times 10^3 + 5 \times 10^2 + 3 \times 10^1 + 1 \times 10^0$$

In any Hindu-Arabic system, the location of the digits is meaningful:
101 is different from 110.  Here are those numbers in Roman numerals:
CI and CX.  Roman numerals didn't have a symbol for zero at all.

The Hindu-Arabic system gave us really straightforward algorithms for
addition and multiplication.  For addition, you memorized the following table:

\begin{tabular}{ c || c | c | c | c | c | c| c| c| c| c }
  & 0 & 1 & 2 & 3 & 4 & 5 & 6 & 7 & 8 & 9 \\
  \hline
0 & 0 & 1 & 2 & 3 & 4 & 5 & 6 & 7 & 8 & 9 \\
1 & 1 & 2 & 3 & 4 & 5 & 6 & 7 & 8 & 9 & 10 \\
2 & 2 & 3 & 4 & 5 & 6 & 7 & 8 & 9 & 10 & 11\\
3 & 3 & 4 & 5 & 6 & 7 & 8 & 9 & 10 & 11 & 12\\
4 & 4 & 5 & 6 & 7 & 8 & 9 & 10 & 11 & 12 & 13\\
5 & 5 & 6 & 7 & 8 & 9 & 10 & 11 & 12 & 13 & 14\\
6 & 6 & 7 & 8 & 9 & 10 & 11 & 12 & 13 & 14 & 15\\
7 & 7 & 8 & 9 & 10 & 11 & 12 & 13 & 14 & 15 & 16\\
8 & 8 & 9 & 10 & 11 & 12 & 13 & 14 & 15 & 16 & 17\\
9 & 9 & 10 & 11 & 12 & 13 & 14 & 15 16 & 17 & 18\\
\end{tabular}

When you multiplied two number together, you simply multiplied each
pair of digits. $254 \times 26$  might look like this:

\begin{tabular} {r c c c c | c}
 & & 2 & 5 & 4 & \\
 & & $\times$  & 2 & 6 & \\
  \hline
&  &   & 2 & 4 & $6 \times 4$\\
&  & 3 & 0 & & $6 \times 5$  \\
&  1 & 2 & & & $6 \times 2$  \\
&  &  &  8 & & $2 \times 4$ \\
& 1 & 0 & & & $2 \times 5$\\
+ & 4 &  & & & $2 \times 2$\\
 \hline
 6 & 6 & 0 & 4
 \end{tabular}
  

For multiplication, you memorized this table:

\begin{tabular}{ c || c | c | c | c | c | c| c| c| c| c}
  & 0 & 1 & 2 & 3 & 4 & 5 & 6 & 7 & 8 & 9 \\
  \hline
  0  & 0  & 0  & 0  & 0  & 0  & 0  & 0  & 0  & 0  &  0 \\
  1  & 0  & 1  & 2  & 3  & 4  & 5  & 6  & 7  & 8  &  9 \\
  2  & 0  & 2  & 4  & 6  & 8  & 10 & 12 & 14 & 16 & 18 \\
  3  & 0  & 3  & 6  & 9  & 12 & 15 & 18 & 21 & 24 & 27 \\
  4  & 0  & 4  & 8  & 12 & 16 & 20 & 24 & 28 & 32 & 36 \\
  5  & 0  & 5  & 10 & 15 & 20 & 25 & 30 & 35 & 40 & 45 \\
  6  & 0  & 6  & 12 & 18 & 24 & 30 & 36 & 42 & 48 & 54 \\
  7  & 0  & 7  & 14 & 21 & 28 & 35 & 42 & 49 & 56 & 63 \\
  9  & 0  & 9  & 18 & 27 & 36 & 45 & 54 & 63 & 72 & 81 
  \end{tabular}

