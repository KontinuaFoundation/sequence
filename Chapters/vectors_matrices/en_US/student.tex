\chapter{Vectors and Matrices}
  
The last chapter provided an overview of linear algebra, using several image 
examples. In this chapter, we will focus primarily on vector-matrix 
multiplications. First, we will show how matrices can be used to represent a 
set of linear equations. Then, we will provide you with a general definition 
of vector-matrix multiplication, followed by a few examples. You will have an 
opportunity to solve a problem manually, then by using Python. In this 
chapter, we will use two-dimensional matrices for simplicity, but a matrix can 
have any number of dimensions.  

\section{Matrices}
We've been looking at vectors, which are usually represented as a column of 
numbers. For example, while we may write $\mathbf{v} = \left[ 1, 2, 3 \right]$ 
in line, the vector is really:
$$\mathbf{v} = \begin{bmatrix}
1\\
2\\
3
\end{bmatrix}$$

A matrix can be made of many columns, like the $3 \times 2$ matrix shown below:
$$\begin{bmatrix}
1 & 1 & 1\\
2 & 4 & 8\\
3 & 9 & 27
\end{bmatrix}$$

We describe the size and shape of matrices by saying \textit{an }$m \times 
n$\textit{ matrix}, where $m$ is the number of rows and $n$ is the number of 
columns. A vector is simply a one-column matrix. For example, the vectors 
\textbf{v} above is $3 \times 1$. Matrices aren't restricted to 2 dimensions: 
a matrix can be 3, 4, or any number of dimensions. For example, a $3 \times 2 
\times 4$ matrix would be made of 4 stacked $3 \times 2$ matrices.

\begin{Exercise}[title = {Matrix Dimensions 1}, label = mat_dim1]
Write the dimensions of the following matrices:
\begin{enumerate}
\item $\begin{bmatrix}
-3 & 0 & 4 & -2 & -4\\
-1 & 5 & 3 & 4 & -2\\
-3 & 2 & 3 & -5 & 1
\end{bmatrix}$
\item $\begin{bmatrix}
-3 & 1
\end{bmatrix}$
\item $\begin{bmatrix}
-3 & 2 & -3\\
4 & 0 & -3\\
-5 & -4 & 1\\
0 & -2 & 2\\
\end{bmatrix}$
\end{enumerate}
\end{Exercise}

\begin{Answer}[ref = mat_dim1]
\begin{enumerate}
\item $3 \times 5$
\item $1 \times 2$
\item $4 \times 3$
\end{enumerate}
\end{Answer}

\begin{Exercise}[title = {Matrix Dimensions 2}, label = mat_dim2]
Create a matrix with the indicated dimensions.
\begin{enumerate}
\item $1 \times 3$
\item $2 \times 4$
\item $4 \times 3$
\end{enumerate}
\end{Exercise}

\begin{Answer}[ref = mat_dim2]
\begin{enumerate}
\item The matrix should have 1 row and 3 columns. For example, 
$$\begin{bmatrix}
1 & 2 & 3
\end{bmatrix}$$
\item The matrix should have 2 rows and 4 columns. For example,
$$\begin{bmatrix}
1 & 2 & 3 & 4\\
5 & 6 & 7 & 8
\end{bmatrix}$$
\item The matrix should have 4 rows and 3 columns. For example,
$$\begin{bmatrix}
1 & 2 & 3\\
4 & 5 & 6\\
7& 8 & 9\\
10 & 11 & 12
\end{bmatrix}$$
\end{enumerate}
\end{Answer}

\subsection{Zero Matrices}
Recall that we can represent a generic zero vector as \textbf{0}, which 
indicates a vector of any number of dimensions filled with zeros. Just like 
vectors, there are \textit{zero matrices}\index{zero matrix}, which can by of 
any number of dimensions, all filled with zeros. In two dimensions, zero 
matrices are denoted as $\mathbf{\mathit{O}}_{m \times n}$, where the 
subscript is the dimension of the matrix. The subscript can be expanded to 
denote any number of dimensions. 

\section{Matrix Arithmetic}
\subsection{Adding and Subtracting Matrices}
Matrices that are the same dimension can be added and subtracted. Just like 
vectors, to add matrices you add the elements in the same position:
$$\begin{bmatrix}
-2 & -1\\
2 & 4
\end{bmatrix}
+ \begin{bmatrix}
5 & 2\\
-1 & -4
\end{bmatrix}
= \begin{bmatrix}
-2 + 5 & -1 + 2\\
2 + -1 & 4 + -4
\end{bmatrix} = 
\begin{bmatrix}
3 & 1\\
1 & 0
\end{bmatrix}$$

And to subtract matrices, you subtract the elements in the same position:
$$\begin{bmatrix}
-2 & -1\\
2 & 4
\end{bmatrix}
- \begin{bmatrix}
5 & 2\\
-1 & -4
\end{bmatrix}
= \begin{bmatrix}
-2 - 5 & -1 - 2\\
2 - (-1) & 4 - (-4)
\end{bmatrix} = 
\begin{bmatrix}
-7 & -3\\
3 & 8
\end{bmatrix}$$

Formally, for 2-dimensional matrices, we can say that:
\begin{mdframed}[style = important, frametitle={Adding and Subtracting Matrics}]
For two $m \times n$ matrices, the sum of the matrices is the matrix of the 
sums of the elements in analogous positions:
$$\begin{bmatrix}
x_{11} & x_{12} & x_{13} & \cdots & x_{1n}\\
x_{21} & x_{22} & x_{23} & \cdots & \vdots\\
\vdots & \vdots & \vdots & \cdots & \vdots\\
x_{m1} & x_{m2} & x_{m3} & \cdots & x_{mn}
\end{bmatrix}
+ \begin{bmatrix}
y_{11} & y_{12} & y_{13} & \cdots & y_{1n}\\
y_{21} & y_{22} & y_{23} & \cdots & \vdots\\
\vdots & \vdots & \vdots & \cdots & \vdots\\
y_{m1} & y_{m2} & y_{m3} & \cdots & y_{mn}
\end{bmatrix} = $$
$$\begin{bmatrix}
x_{11} + y_{11} & x_{12} + y_{12} & x_{13} + y_{13} & \cdots & x_{1n} + y_{1n}\\
x_{21} + y_{21} & x_{22} + y_{22} & x_{23} + y_{23} & \cdots & \vdots\\
\vdots & \vdots & \vdots & \cdots & \vdots\\
x_{m1} + y_{m1} & x_{m2} + y_{m2} & x_{m3} + y_{m3} & \cdots & x_{mn} + y_{mn}
\end{bmatrix}$$

To subtract matrices, simply add the negative of the second matrix (that is, 
\textbf{\textit{A}} - \textbf{\textit{B}} = \textbf{\textit{A}} + -\textbf{
\textit{B}}). Additionally, matrix addition is commutative (\textbf{\textit{A}}
+ \textbf{\textit{B}} = \textbf{\textit{B}} + \textbf{\textit{B}}).
\end{mdframed}

\begin{Exercise}[title = {Adding and Subtracting Matrices}, label = add_mat]
Find \textbf{\textit{A}} + \textbf{\textit{B}}, \textbf{\textit{A}} - 
\textbf{\textit{B}}, and \textbf{\textit{B}} - \textbf{\textit{A}}.
\begin{enumerate}
\item $\textbf{\textit{A}} = \begin{bmatrix}
0 & 4 & 0 & 5
\end{bmatrix}$ and $\textbf{\textit{B}} = \begin{bmatrix}
-2 & 3 & -2 & 5
\end{bmatrix}$
\item $\textbf{\textit{A}} = \begin{bmatrix}
4 & -4 & -2\\
1 & -3 & 5\\
-5 & 3 & 0
\end{bmatrix}$ and $\textbf{\textit{B}} = \begin{bmatrix}
5 & 0 & -1\\
-5 & -3 & -2\\
-5 & 3 & -4
\end{bmatrix}$.
\item $\textbf{\textit{A}} = \begin{bmatrix}
-2 & -1 & -5 & -1\\
5 & -4 & 4 & 3
\end{bmatrix}$ and $\textbf{\textit{B}} = \begin{bmatrix}
-5 & -2 & 3 & -5\\
0 & 5 & -4 & -3
\end{bmatrix}$.
\end{enumerate}
\end{Exercise}

\begin{Answer}[ref = add_mat]
\begin{enumerate}
\item \textbf{\textit{A}} + \textbf{\textit{B}} $= \begin{bmatrix}
-2 & 7 & -2 & 10
\end{bmatrix}$. $\textbf{\textit{A}} - \textbf{\textit{B}} = \begin{bmatrix}
2 & 1 & 2 & 0
\end{bmatrix}$. $\textbf{\textit{B}} - \textbf{\textit{A}} = \begin{bmatrix}
-2 & -1 & -2 & 0
\end{bmatrix}$
\item $\textbf{\textit{A}} + \textbf{\textit{B}} = \begin{bmatrix}
9 & -4 & -3\\
-4 & -6 & 3\\
-10 & 6 & -4
\end{bmatrix}$. $\textbf{\textit{A}} - \textbf{\textit{B}} = \begin{bmatrix}
-1 & -4 & -1\\
6 & 0 & 7\\
0 & 0 & 4
\end{bmatrix}$. $\textbf{\textit{B}} - \textbf{\textit{A}} = \begin{bmatrix}
1 & 4 & 1\\
-6 & 0 & -7\\
0 & 0 & -4
\end{bmatrix}$.
\item $\textbf{\textit{A}} + \textbf{\textit{B}} = \begin{bmatrix}
-7 & -3 & -2 & -6\\
5 & 1 & 0 & 0
\end{bmatrix}$. $\textbf{\textit{A}} - \textbf{\textit{B}} = \begin{bmatrix}
3 & 1 & -8 & 4\\
5 & -9 & 8 & 6
\end{bmatrix}$. $\textbf{\textit{B}} - \textbf{\textit{A}} = \begin{bmatrix}
-3 & -1 & 8 & -4\\
-5 & 9 & -8 & -6
\end{bmatrix}$.
\end{enumerate}
\end{Answer}

\subsection{Multiplying Matrices}
Surprisingly (it may be to you), matrix multiplication has dimension limits. 
We cannot multiply any two matrices: the first matrix must have the same 
number of columns as the second has number of rows. Let's examine the origin 
of the dimension limits on matrix multiplication. We begin with a review of 
the vector dot product. 

Recall that in order to find the dot product of two vectors, they must be the 
same length (that is, the same number of dimensions). The result is always a 
scalar: one number. You can review finding the dot product of vectors and 
practice the dimension limits on the vector dot product in the next exercise.

\begin{Exercise}[title = {Vector Dot Product Review}, label = vect_dot]
Find all possible pairs of vectors that can be used to find a dot product, 
then find the dot products. 
\begin{enumerate}
\item $\mathbf{a} = \begin{bmatrix}
1\\
2
\end{bmatrix}$
\item $\mathbf{b} = \begin{bmatrix}
-3\\
3\\
5\\
-5
\end{bmatrix}$
\item $\mathbf{c} = \begin{bmatrix}
1\\
2\\
-1
\end{bmatrix}$
\item $\mathbf{d} = \begin{bmatrix}
-5\\
-1
\end{bmatrix}$
\item $\mathbf{e} = \begin{bmatrix}
1\\
-5\\
3\\
1
\end{bmatrix}$
\item $\mathbf{f} = \begin{bmatrix}
4\\
1\\
-3
\end{bmatrix}$
\end{enumerate}
\end{Exercise}

\begin{Answer}[ref = vect_dot]
It is possible to compute $\mathbf{a} \cdot \mathbf{d}$, $\mathbf{b} \cdot 
\mathbf{e}$, and $\mathbf{c} \cdot \mathbf{f}$:
\begin{enumerate}
    \item $\mathbf{a} \cdot \mathbf{d} = 1(-5) + 2(-1) = -5 + (-2) = -7$
    \item $\mathbf{b} \cdot \mathbf{e} = -3(1) + 3(-5) + 5(3) + -5(1) = -3 + 
    (-15) + 15 - 5 = -8$
    \item $\mathbf{c} \cdot \mathbf{f} = 1(4) + 2(1) + -1(-3) = 4 + 2 + 3 = 9$
\end{enumerate}
\end{Answer}

To multiply two matrices, it is helpful to think of the rows of the first 
matrix and the columns of the second matrix as vectors. Let's see how this 
shakes out for two $2 \times 2$ matrices:

\begin{figure}
    \centering
    \begin{tikzpicture}
        \node[] at (0,0) {$\begin{bmatrix}
            5 & 4\\
            -5 & 1
        \end{bmatrix} \cdot \begin{bmatrix}
            -1 & -2\\
            -5 & -4
        \end{bmatrix} = \begin{bmatrix}
             & & & & & & & \\
             & & & & & & & 
        \end{bmatrix}$};
        \node[] at (-2.75, -1) {\textbf{\textit{A}}};
        \node[] at (-2, -1) {$\cdot$};
        \node[] at (-1, -1) {\textbf{\textit{B}}};
        \node[text = red] at (-3.8, 0.25) {$\mathbf{a}_1 \to$};
        \node[text = red] at (-3.8, -0.25) {$\mathbf{a}_2 \to$};
        \node[text = blue] at (-1.3, 0.7) {$\downarrow$};
        \node[text = blue] at (-1.3, 1) {$\mathbf{b}_1$};
        \node[text = blue] at (-0.55, 0.7) {$\downarrow$};
        \node[text = blue] at (-0.55, 1) {$\mathbf{b}_2$};
        \node[text = red] at (0.95, 0.15) {$\mathbf{a}_1$};
        \node[] at (1.2, 0.15) {$\cdot$};
        \node[text = blue] at (1.5, 0.15) {$\mathbf{b}_1$};
        \node[text = red] at (0.95, -0.25) {$\mathbf{a}_2$};
        \node[] at (1.2, -0.25) {$\cdot$};
        \node[text = blue] at (1.5, -0.25) {$\mathbf{b}_1$};
        \node[text = red] at (2.15, 0.15) {$\mathbf{a}_1$};
        \node[] at (2.4, 0.15) {$\cdot$};
        \node[text = blue] at (2.7, 0.15) {$\mathbf{b}_2$};
        \node[text = red] at (2.15, -0.25) {$\mathbf{a}_2$};
        \node[] at (2.4, -0.25) {$\cdot$};
        \node[text = blue] at (2.7, -0.25) {$\mathbf{b}_2$};
    \end{tikzpicture}
    \caption{Each entry in \textbf{\textit{C}}, $c_{ij}$, is the dot product 
    of the $i^{th}$ row of \textbf{\textit{A}}, $a_i$, and the $j^{th}$ column 
    of \textbf{\textit{B}}, $b_j$.}
    \label{fig:mat_mult}
\end{figure}

 Let's look at this more concretely. For two-dimensional matrices, it can be 
 helpful to move your left index finger across the row and right index finger 
 down the column, as shown in figure \ref{fig:fingers}.
 
 \begin{figure}[htbp]
    \centering
    \begin{tikzpicture}
        \node[] at (0,3) {$\begin{bmatrix}
            5 & 4\\
            -5 & 1
        \end{bmatrix} \cdot \begin{bmatrix}
            -1 & -2\\
            -5 & -4
        \end{bmatrix} = \begin{bmatrix}
             & & & & & & & \\
             & & & & & & & 
        \end{bmatrix}$};
        \draw[thick, -latex, red] (-3, 3.25) -- (-2, 3.25);
        \draw[thick, -latex, blue] (-1.25, 3.4) -- (-1.25, 2.6);
        \node[font = \tiny] at (1.25, 3.25) {
        \textcolor{red}{5}$\times$\textcolor{blue}{-1}$+$\textcolor{red}{4}$\times$\textcolor{blue}{-4}};

        \node[] at (0,2) {$\begin{bmatrix}
            5 & 4\\
            -5 & 1
        \end{bmatrix} \cdot \begin{bmatrix}
            -1 & -2\\
            -5 & -4
        \end{bmatrix} = \begin{bmatrix}
             & & & & & & & \\
             & & & & & & & 
        \end{bmatrix}$};
        \draw[thick, -latex, red] (-3, 2.25) -- (-2, 2.25);
        \draw[thick, -latex, blue] (-.35, 2.4) -- (-.35, 1.6);
        \node[font = \normalsize] at (1.25, 2.25) {-21};
        \node[font = \tiny] at (2.25, 2.25) {
        \textcolor{red}{5}$\times$\textcolor{blue}{-2}$+$\textcolor{red}{4}$\times$\textcolor{blue}{-4}};
        
        \node[] at (0, 1) {$\begin{bmatrix}
            5 & 4\\
            -5 & 1
        \end{bmatrix} \cdot \begin{bmatrix}
            -1 & -2\\
            -5 & -4
        \end{bmatrix} = \begin{bmatrix}
             & & & & & & & \\
             & & & & & & & 
        \end{bmatrix}$};
        \draw[thick, -latex, red] (-3, 0.75) -- (-2, 0.75);
        \draw[thick, -latex, blue] (-1.25, 1.4) -- (-1.25, 0.6);
        \node[font = \normalsize] at (1.25, 1.25) {-21};
        \node[font = \normalsize] at (2.25, 1.25) {-26};
        \node[font = \tiny] at (1.25, 0.8) {
        \textcolor{red}{-5}$\times$\textcolor{blue}{-1}$+$\textcolor{red}{1}$\times$\textcolor{blue}{-5}};

        \node[] at (0,0) {$\begin{bmatrix}
            5 & 4\\
            -5 & 1
        \end{bmatrix} \cdot \begin{bmatrix}
            -1 & -2\\
            -5 & -4
        \end{bmatrix} = \begin{bmatrix}
             & & & & & & & \\
             & & & & & & & 
        \end{bmatrix}$};
        \draw[thick, -latex, red] (-3, -0.25) -- (-2, -0.25);
        \draw[thick, -latex, blue] (-.35, 0.4) -- (-.35, -0.4);
        \node[font = \normalsize] at (1.25, 0.25) {-21};
        \node[font = \normalsize] at (2.25, 0.25) {-26};
        \node[font = \normalsize] at (1.25, -0.25) {0};
        \node[font = \tiny] at (2.25, -0.25) {
        \textcolor{red}{-5}$\times$\textcolor{blue}{-2}$+$\textcolor{red}{1}$\times$\textcolor{blue}{-4}};
        \node[] at (0,-1) {$\begin{bmatrix}
            5 & 4\\
            -5 & 1
        \end{bmatrix} \cdot \begin{bmatrix}
            -1 & -2\\
            -5 & -4
        \end{bmatrix} = \begin{bmatrix}
             & & & & & & & \\
             & & & & & & & 
        \end{bmatrix}$};
        \node[font = \normalsize] at (1.25, -0.75) {-21};
        \node[font = \normalsize] at (2.25, -0.75) {-26};
        \node[font = \normalsize] at (1.25, -1.25) {0};
        \node[font = \normalsize] at (2.25, -1.25) {6};
    \end{tikzpicture}
    \caption{You can use your fingers to trace across matrix \textbf{\textit{A}} 
    and down matrix \textbf{\textit{B}} to find $\textbf{\textit{A}} \cdot 
    \textbf{\textit{B}}$.}
    \label{fig:fingers}
\end{figure}

Since each entry in the product matrix is the dot product between a row of the 
first matrix and a column of the second matrix, the first matrix must have the 
same number of elements in each row as the second has in each column. Another 
way to say this is that the number of columns of the first matrix must match 
the number of rows in the second matrix. 

\begin{mdframed}[style = important, frametitle = {Matrix Multiplication}]
For two-dimensional matrices, the inner dimensions must match in order to 
carry out matrix multiplication. That is, if we want to find \textbf{\textit{
A}}$\cdot$\textbf{\textit{B}}, and \textbf{\textit{A}} has dimensions $m 
\times n$, then \textbf{\textit{B}} must have dimensions $n \times p$, where 
$m$, $n$, and $p$ are integers. The resulting matrix will have dimensions $m 
\times p$ ($m$ and $p$ may be equal or unequal).
\end{mdframed}

\begin{Exercise}[title = {Multiplying Matrices 1}, label = mult_mat1]
Multiply the matrices.
\begin{enumerate}
\item $\begin{bmatrix}
-5 & -2 & 2 & 1
\end{bmatrix} \cdot \begin{bmatrix}
-3 & -1 & -5\\
3 & 0 & 3\\
4 & -1 & -4\\
-1 & -4 & 2
\end{bmatrix}$
\item $\begin{bmatrix}
1\\
5\\
-5\\
4\\
1
\end{bmatrix} \cdot \begin{bmatrix}
0 & 5 & 1
\end{bmatrix}$
\item $\begin{bmatrix}
-1 & 4 & -4\\
5 & -3 & 5\\
-1 & -4 & 4\\
-4 & 1 & 4
\end{bmatrix} \cdot \begin{bmatrix}
-3 & 5 & 1\\
-3 & 0 & -3\\
0 & 3 & 0
\end{bmatrix}$
\end{enumerate}
\end{Exercise}

\begin{Answer}[ref = mult_mat1]
\begin{enumerate}
    \item $\begin{bmatrix}
        -2 & -1 & 5
    \end{bmatrix}$
    \item $\begin{bmatrix}
        0 & 5 & 1\\
        0 & 25 & 5\\
        0 & -25 & -5\\
        0 & 20 & 4\\
        0 & 5 & 1
    \end{bmatrix}$
    \item $\begin{bmatrix}
        -9 & -17 & 11\\
        -6 & 40 & -4\\
        15 & 7 & -13\\
        9 & -8 & -1
    \end{bmatrix}$
\end{enumerate}
\end{Answer}

\begin{Exercise}[title = {Multiplying Matrices 2}, label = mult_mat2]
Find $\mathbf{\mathit{A}} \cdot \mathbf{\mathit{B}}$ and $\mathbf{\mathit{B}} \cdot \mathbf{\mathit{A}}$. 
\begin{enumerate}
\item $\mathbf{\mathit{A}} = \begin{bmatrix}
-2\\
2\\
1\\
-2
\end{bmatrix}$ and $\mathbf{\mathit{B}} = \begin{bmatrix}
-4 & 3 & -5 & -2
\end{bmatrix}$
\item $\mathbf{\mathit{A}} = \begin{bmatrix}
-4 & -2\\
2 & 5\\
-3 & -4\\
\end{bmatrix}$ and $\mathbf{\mathit{B}} = \begin{bmatrix}
0 & -2 & -4\\
1 & -4 & 0
\end{bmatrix}$
\item $\mathbf{\mathit{A}} = \begin{bmatrix}
2 & 0 & 1 & 4\\
-4 & 0 & -5 & -1
\end{bmatrix}$ and $\mathbf{\mathit{B}} = \begin{bmatrix}
0 & -3\\
-4 & -1\\
-2 & 3\\
-5 & 1
\end{bmatrix}$
\end{enumerate}
\end{Exercise}

\begin{Answer}[ref = mult_mat2]
\begin{enumerate}
\item $\mathbf{\mathit{A}} \cdot \mathbf{\mathit{B}} = \begin{bmatrix}
8 & -6 & 10 & 4\\
-8 & 6 & -10 & -4\\
-4 & 3 & -5 & -2\\
8 & -6 + 1- & 4
\end{bmatrix}$ and $\mathbf{\mathit{B}} \cdot \mathbf{\mathit{A}} = 
\begin{bmatrix}
13
\end{bmatrix}$
\item $\mathbf{\mathit{A}} \cdot \mathbf{\mathit{B}} = \begin{bmatrix}
-2 & 16 & 16\\
5 & -24 & -8\\
-4 & 22 & 12
\end{bmatrix}$ and $\mathbf{\mathit{B}} \cdot \mathbf{\mathit{A}} = 
\begin{bmatrix}
8 & 6\\
-12 & -22
\end{bmatrix}$
\item $\mathbf{\mathit{A}} \cdot \mathbf{\mathit{B}} = \begin{bmatrix}
-22 & 1\\
15 & -4
\end{bmatrix}$ and $\mathbf{\mathit{B}} \cdot \mathbf{\mathit{A}} = 
\begin{bmatrix}
12 & 0 & 15 & 3\\
-4 & 0 & 1 & -15\\
-16 & - & -17 & -11\\
-14 & 0 & -10 & -21
\end{bmatrix}$
\end{enumerate}
\end{Answer}

What have you noticed about the results of $\mathbf{\mathit{A}} \cdot 
\mathbf{\mathit{B}}$ as compared to $\mathbf{\mathit{B}} \cdot \mathbf{
\mathit{A}}$? You should have noticed that the product matrices are 
\textit{different dimensions}. This leads us to the next unusual property of 
matrix multiplication: it is \textit{non-commutative}. That is, the \textit{
order} in which you multiply matrices affects the result. This is very 
different from scalar values! 

As you saw in the second matrix multiplication exercise, \textbf{\textit{A}} 
is a $2 \times 4$ matrix and \textbf{\textit{B}} is a $4 \times 2$ matrix, 
then \textbf{\textit{A}}\textbf{\textit{B}} is a $2 \times 2$ matrix, while 
\textbf{\textit{B}}\textbf{\textit{A}} is a $4 \times 4$ matrix. It is obvious, 
then, that $\mathbf{\mathit{A}} \cdot \mathbf{\mathit{B}} \neq \mathbf{\mathit{
B}} \cdot \mathbf{\mathit{A}}$. What if \textbf{\textit{A}} and \textbf{\textit{
B}} are square matrices?

\textbf{Example}: Find $\mathbf{\mathit{A}} \cdot \mathbf{\mathit{B}}$ and 
$\mathbf{\mathit{B}} \cdot \mathbf{\mathit{A}}$ if $\mathbf{\mathit{A}} = 
\begin{bmatrix}
-3 & 5\\
-1 & 0
\end{bmatrix}$ and $\mathbf{\mathit{B}} = \begin{bmatrix}
-1 & 1\\
4 & -3
\end{bmatrix}$.

\textbf{Solution}: $$\mathbf{\mathit{A}} \cdot \mathbf{\mathit{B}} = 
\begin{bmatrix}
-3 & 5\\
-1 & 0
\end{bmatrix} \cdot \begin{bmatrix}
-1 & 1\\
4 & -3
\end{bmatrix} = \begin{bmatrix}
-3(-1) + 5(4) & -3(1) + 5(-3)\\
-1(-1) + 0(4) & -1(1) + 0(-3)
\end{bmatrix} = \begin{bmatrix}
23 & -18\\
1 & -1
\end{bmatrix}$$

$$\mathbf{\mathit{B}} \cdot \mathbf{\mathit{A}} = \begin{bmatrix}
-1 & 1\\
4 & -3
\end{bmatrix} \cdot \begin{bmatrix}
-3 & 5\\
-1 & 0
\end{bmatrix} = \begin{bmatrix}
-1(-3) + 1(-1) & -1(5) + 1(0)\\
4(-3) + -3(-1) & 4(5) + -3(0)
\end{bmatrix} = \begin{bmatrix}
2 & -5\\
-9 & 20
\end{bmatrix}$$

As you can see, even if \textbf{\textit{A}} and \textbf{\textit{B}} are square, 
matrix multiplication is still not commutative. 

\begin{mdframed}[style = important, frametitle = {Non-Commutation of Matrix Multiplication}]
For two matrices \textbf{\textit{A}} and \textbf{\textit{B}}, where neither is 
an identity matrix or a zero matrix:
$$\mathbf{\mathit{A}} \cdot \mathbf{\mathit{B}} \neq \mathbf{\mathit{B}} 
\cdot \mathbf{\mathit{A}}$$
\end{mdframed}
	
\subsubsection{Properties of the Zero Matrix}
Just like the number 0, the zero matrix, \textbf{\textit{O}} has unique 
mathematical properties:
\begin{mdframed}[style = important, frametitle = {Properties of the Zero Matrix}]
For a matrix, \textbf{\textit{A}}, and a zero matrix, \textbf{\textit{O}}
\begin{enumerate}
\item $\mathbf{\mathit{A}} + \mathbf{\mathit{O}} = \mathbf{\mathit{A}}$
\item $\mathbf{\mathit{A}} + \mathbf{\mathit{-A}} = \mathbf{\mathit{O}}$
\item $0 \cdot \mathbf{\mathit{A}} = \mathbf{\mathit{O}}$
\end{enumerate}
\end{mdframed}

\subsubsection{The Identity Matrix}
There is another special matrix, called the \textit{identity matrix}\index{
identity matrix}, usually denoted with \textbf{\textit{I}}. An identity matrix 
is all zeroes except for a diagonal line of ones. A $3 \times 3$ identity 
matrix is shown below:

$$\begin{bmatrix}
1 & 0 & 0\\
0 & 1 & 0\\
0 & 0 & 1
\end{bmatrix}$$

All identity matrices are square (that is, they have the same number of rows 
as they do columns). The identity matrix has the special property that 
whenever a vector or matrix is multiplied by \textbf{\textit{I}}, it doesn't 
change. Let's look at some examples:

\textbf{Example}: If $\mathbf{x} = \begin{bmatrix}
2\\
-3
\end{bmatrix}$, what is $\mathbf{\mathit{I}}\mathbf{x}$? (Take \textbf{\textit{
I}} to be a $2 \times 2$ identity matrix.)

\textbf{Solution}: $$\mathbf{\mathit{I}} \mathbf{x} = \begin{bmatrix}
1 & 0 \\
0 & 1
\end{bmatrix} \cdot \begin{bmatrix}
2 \\
-3
\end{bmatrix} = \begin{bmatrix}
1 \cdot (2) + 0 \cdot (-3)\\
0 \cdot (2) + 1 \cdot (-3)
\end{bmatrix} = \begin{bmatrix}
2\\
-3
\end{bmatrix}$$

\textbf{Example}: If $\mathbf{\mathit{B}} = \begin{bmatrix}
-2 & 5\\
3 & -4
\end{bmatrix}$, what is $\mathbf{\mathit{I}} \cdot \mathbf{\mathit{B}}$?

\textbf{Solution}: $$\mathbf{\mathit{I}} \cdot \mathbf{\mathit{B}} = 
\begin{bmatrix}
1 & 0 \\
0 & 1
\end{bmatrix} \cdot \begin{bmatrix}
-2 & 5\\
3 & -4
\end{bmatrix} = \begin{bmatrix}
1 \cdot (-2) + 0 \cdot (5) & 1 \cdot (5) + 0 \cdot (-4)\\
0 \cdot (-2) + 1 \cdot (5) & 0 \cdot (5) + 1 \cdot (-4)
\end{bmatrix} = \begin{bmatrix}
-2 & 5\\
3 & -4
\end{bmatrix}$$

\begin{mdframed}[style = important, frametitle = {Properties of the Identity Matrix}]
An $n \times n$ identity matrix, \textbf{\textit{I}}, does not change any 
vectors or matrices it multiplies. That is: 
\begin{enumerate}
\item $\mathbf{\mathit{I}} \cdot \mathbf{x} = \mathbf{x}$
\item $\mathbf{\mathit{I}} \cdot \mathbf{\mathit{B}} = \mathbf{\mathit{B}}$
\end{enumerate}
where \textbf{x} is an $n \times 1$ vector and \textbf{\textit{B}} is an $n 
\times p$ matrix ($p$ may be, but is not necessarily, equal to $n$). 
\end{mdframed}

\subsection{Can We Divide Matrices?}
Matrices cannot be divided. Suppose we have a matrix, \textbf{\textit{A}}, a vector \textbf{x}, and another vector \textbf{b} such that:

$$\textbf{\textit{A}} \cdot \textbf{x} = \textbf{b}$$

Now, if we know \textbf{\textit{A}} and \textbf{x}, it is easy to find \textbf{b}. What if, on the other hand, we know \textbf{\textit{A}} and \textbf{b} and want to find \textbf{x}? We might be tempted to do something like this:

$$\textbf{x} = \frac{\textbf{b}}{\textbf{\textit{A}}}$$

While this would be correct if \textbf{x}, \textbf{b}, and \textbf{\textit{A}} were scalars, but it is not for matrices. However, there is an analogy we can make. Instead of trying to divide by \textbf{\textit{A}}, we can multiply by its \textit{inverse}:

\begin{mdframed}[style = important, frametitle = {Inverse Matrices}]
Given a matrix \textbf{\textit{A}}, and vectors \textbf{b} and \textbf{x}, if

$$\textbf{\textit{A}} \cdot \textbf{x} = \textbf{b}$$

Then,

$$\textbf{x} = \textbf{\textit{A}}^{-1} \cdot \textbf{b}$$
\end{mdframed}


\section{Systems of Linear Equations}
matrix-vector multiplication, representing linear combination
system of linear equations connection
linear combo example
system of equations example (trail mix, be more explicit about eqs and representation)
practice
	use matrix multiplication to find...
	grading question moved here

\section{Trail Mix for Mars}
Let's look at an applied problem. Three astronauts, Pat, Kai, and River, are getting ready for a trip to Mars. NASA food service is preparing trail mix for the voyage, tailored to each astronaut's taste. The chef needs to submit a budget based on the cost of the trail mix for each astronaut. The mix is made up of raisins, almonds, and chocolate.

Pat prefers a raisins:almonds:chocolate ratio of 6:10:4, Kai likes 2:3:15, and River wants 14:1:5. The chef can buy a kg of raisins for \$7.50, a kg of almonds for \$14.75, and a kg of chocolate for \$22.25. Assuming each astronaut will get 20 kg of trail mix, which astronaut will cost more to feed?

First, set up a matrix to represent the raisins:almonds:chocolate ratios. (Conveniently, these ratios already add to 20.)

$$MixRatios = \begin{bmatrix}
6 & 10 & 4  \\
2 & 3 & 15 \\
14 & 1 & 5
\end{bmatrix}$$

Use a vector to represent the cost of each item:
$$IngredientCost = 
\begin{bmatrix}
7.50 \\
14.75 \\
22.25
\end{bmatrix}$$

To find the cost of trail mix for each astronaut, we simply find the dot product between the mix ratios and the ingredient costs to get:

Pat =  \$281.50 \newline
Kai =  \$615.50 \newline
River = \$231.00 \newline


\begin{Exercise}[title={Vector Matrix Multiplication}, label=vector-matrix-multiply01]
Multiply the array $A$ with the vector $v$. Compute this by hand, and make sure to show your computations. 
$$A = \begin{bmatrix}
1 & -2 & 3 & 5  \\
-4 & 2 & 7 & 1 \\
3 & 3  & -9 & 1
\end{bmatrix}$$

$$v = 
\begin{bmatrix}
 2 \\
 2 \\
 6 \\
 -1
 \end{bmatrix}$$
\end{Exercise}

\begin{Answer}[ref=vector-matrix-multiply01]
$$Av = (11 \text{ }37 \text{ } -43)$$
\end{Answer}

\begin{Exercise}[title = {Using Vector Matrix Multiplication}, label = vmm02]
A college professor offers three different methods of determining a student's final grade. In method A, the student's grade is 20\% based on attendance, 50\% homework, 15\% midterm, and 15\% final. This professor knows many students can learn the material without attending every class, so with method B the student's grade is 50\% homework, 20\% midterm, and 30\% final. Last, the professor knows some students don't do the homework but still show they understand the material by doing well on the tests. With method C, a student's grade is 40\% midterm and 60\% final. The professor uses whatever method results in the highest grade to determine each student's final grade. 

Suppose Suzy has attended 65\% of classes, has an average homework grade of 30\%, earned a 80\% on the midterm, and earned a 75\% on the final. What final grade will her professor post?
\end{Exercise}

\begin{Answer}[ref = vmm02]
The different methods can be represented in a matrix:
$$\begin{bmatrix}
0.20 & 0.50 & 0.15 & 0.15\\
0 & 0.50 & 0.20 & 0.30\\
0 & 0 & 0.4 & 0.6
\end{bmatrix}$$

And Suzy's individual grades can be represented by a vector:
$$\begin{bmatrix}
65\\
30\\
80\\
75
\end{bmatrix}
$$

To see the results of the three different methods, we can multiply the matrix and the vector:
$$\begin{bmatrix}
0.20 & 0.50 & 0.15 & 0.15\\
0 & 0.50 & 0.20 & 0.30\\
0 & 0 & 0.4 & 0.6
\end{bmatrix}
\cdot 
\begin{bmatrix}
65\\
30\\
80\\
75
\end{bmatrix} = 
\begin{bmatrix}
0.2(65)+0.5(30)+0.15(80) + 0.15(75)\\
0(65) + 0.5(30) + 0.2(80) + 0.3(75)\\
0(65) + 0(30) + 0.4(80) + 0.6(75)
\end{bmatrix}$$

Which yields:
$$\begin{bmatrix}
51.25\\
53.5\\
77
\end{bmatrix}$$

Since method C yields the highest grade, the professor will post a final grade of 77. 
\end{Answer}

\subsection{Vector-Matrix Multiplication in Python}
Most real-world problems use very large matrices, where it becomes impractical to do calculations by hand. As long as you understand how matrix-vector multiplication is performed, you will be equipped to use a computing language, like Python, to do the calculations for you. 

Create a file called \filename{vectors\_matrices.py} and enter this code:

\begin{Verbatim}
# import the python module that supports matrices
import numpy as np

# create an array
a = np.array([[5, 1, 3, -2], 
              [1, -1, 8, 4], 
              [6, 2, 1, 3]])

# create a vector 
b = np.array([1, 2, 3, -8])

# calculate the dot product
print(a.dot(b))
\end{Verbatim}

When you run it, you should see:
\begin{Verbatim}
[16, 6, 8]
\end{Verbatim}

\section{Where to Learn More}
Watch this video from Khan Academy about matrix-vector products: \url{https://rb.gy/frga5}

