\chapter{Vectors and Matrices}
  
The last chapter provided an overview of linear algebra, using many image examples. In this chapter, we'll focus primarily on vector-matrix multiplications. First you'll see the general definition,followed by a few examples. You'll have an opportunity to solve a problem manually and then you'll get to use Python again. In this chapter, we will use two-dimensional matrices for simplicity. But a matrix can have any number of dimensions.  

\section{Vector-Matrix Multiplication}
Let's take a look at the general form of vector-matrix multiplication. Given a matrix $A$ of size $m \times n$ and a vector $x$ of size $n \times 1$, the product $Ax$ is a new vector of size $m \times 1$. 

You compute the $i$-th component of the product vector $Av$ by taking the dot product of the $i$-th row of $A$ and the vector $v$:

$$(Av)_i = \sum_{j=1}^n a_{i,j}x_j$$

where $a_{i,j}$ is the element in the $i$-th row and $j$-th column of $A$, and $v_j$ is the $j$-th element of $v$.

This is the general form of a matrix and a vector, written to show the specific components of each:

 $$A = \begin{bmatrix}
 a_{1,1} & a_{1,2} & a_{1,3} & ... & a_{1,n}  \\
 a_{2,1} & a_{2,2} & a_{2,3} & ... & a_{2,n}  \\
 ... \\
 a_{m,1} & a_{m,2} & a_{m,3} & ... & a_{m,n}  
\end{bmatrix}$$

$$v = \begin{bmatrix}
 v_{1} \\
 v_{2} \\
 v_{3} \\
 ... \\
 v_{m} 
\end{bmatrix}$$

 $$Av =\begin{bmatrix}
 v_{1}*a_{1,1} +v_{2}*a_{1,2}  +v_{3}*a_{1,3} +... +v_{m}*a_{1,n}  \\
 v_{1}*a_{2,1} +v_{2}*a_{2,2}  +v_{3}*a_{2,3} +... +v_{m}*a_{2,n}  \\
 ... \\
 v_{1}*a_{m,1} +v_{2}*a_{m,2}  +v_{3}*a_{m,3} +... +v_{m}*a_{m,n}  
\end{bmatrix}$$

Let's look at a specific example.

$$A = \begin{bmatrix}
 2  & 4 & 6  \\
 3  & 5 & 7  \\
 1  & 2 & 3  \\
 8  & 6 & 2 
\end{bmatrix}$$

$$v = \begin{bmatrix}
 -2 \\
 1 \\
 3 
\end{bmatrix}$$

Solution:
$$= \begin{bmatrix}
-2*2+1*4+3*6\\
-2*3+1*5+3*7\\
-2*1+1*2+3*3\\
-2*8+1*6+3*2
\end{bmatrix}$$

$$= \begin{bmatrix}
18 \\
20\\
9\\
-4 
\end{bmatrix}$$

\section{Trail Mix for Mars}
Let's look at an applied problem. Three astronauts (Pat, Kai, and River) are getting ready for a trip to Mars. NASA food service is preparing trail mix for the voyage, tailored to each astronaut's taste. The chef needs to submit a budget based on the cost of the trail mix for each astronaut. The mix is made up of raisins, almonds, and chocolate.

Pat prefers a raisins:almonds:chocolate ratio of 6:10:4. Kai likes 2:3:15. River wants 14:1:5. The chef can buy a kg of raisins for \$7.50, a kg of almonds for \$14.75, and a kg of chocolate for \$22.25. Assuming each astronaut will get 20 kg of trail mix, which astronaut will cost more to feed?

First, set up a matrix to represent the raisins:almonds:chocolate ratios. (Conveniently, these ratios already add to 20.)

$$MixRatios = \begin{bmatrix}
6 & 10 & 4  \\
2 & 3 & 15 \\
14 & 1 & 5
\end{bmatrix}$$

Use a vector to represent the cost of each item:
$$IngredientCost = 
\begin{bmatrix}
7.50 \\
14.75 \\
22.25
\end{bmatrix}$$

To find the cost of trail mix for each astronaut, we simply find the dot product between the mix ratios and the ingredient costs to get:

Pat =  \$281.50 \newline
Kai =  \$615.50 \newline
River = \$231.00 \newline


\begin{Exercise}[title={Vector Matrix Multiplication}, label=vector-matrix-multiply01]
Multiply the array $A$ with the vector $v$. Compute this by hand, and make sure to show your computations. 
$$A = \begin{bmatrix}
1 & -2 & 3 & 5  \\
-4 & 2 & 7 & 1 \\
3 & 3  & -9 & 1
\end{bmatrix}$$

$$v = 
\begin{bmatrix}
 2 \\
 2 \\
 6 \\
 -1
 \end{bmatrix}$$
\end{Exercise}

\begin{Answer}[ref=vector-matrix-multiply01]
$$Av = (11 37 -43)$$
\end{Answer}

\subsection{Vector-Matrix Multiplication in Python}
Most real-world problems use very large matrices where it becomes impractical to do calculations by hand. As long as you understand how matrix-vector multiplication is performed, you'll be equipped to use a computing language, like Python, to do the calculations for you. 

Create a file called \filename{vectors\_matrices.py} and enter this code:

\begin{Verbatim}
# import the python module that supports matrices
import numpy as np

# create an array
a = np.array([[5, 1, 3, -2], 
              [1, -1, 8, 4], 
              [6, 2, 1, 3]])

# create a vector 
b = np.array([1, 2, 3, -8])

# calculate the dot product
print(a.dot(b))
\end{Verbatim}

When you run it, you should see:
\begin{Verbatim}
[16, 6, 8]
\end{Verbatim}

\section{Where to Learn More}
Watch this video from Khan Academy about matrix-vector products: \url{https://rb.gy/frga5}

