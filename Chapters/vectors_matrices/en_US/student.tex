\chapter{Vectors and Matrices}
  
The last chapter provided an overview of linear algebra, using several image 
examples. In this chapter, we will focus primarily on vector-matrix 
multiplications. First, we will show how matrices can be used to represent a 
set of linear equations. Then, we will provide you with a general definition 
of vector-matrix multiplication, followed by a few examples. You will have an 
opportunity to solve a problem manually, then by using Python. In this 
chapter, we will use two-dimensional matrices for simplicity, but a matrix can 
have any number of dimensions.  

\section{Matrices}
We've been looking at vectors, which are usually represented as a column of 
numbers. For example, while we may write $\mathbf{v} = \left[ 1, 2, 3 \right]$ 
in line, the vector is really:
$$\mathbf{v} = \begin{bmatrix}
1\\
2\\
3
\end{bmatrix}$$

A matrix can be made of many columns, like the $3 \times 2$ matrix shown below:
$$\begin{bmatrix}
1 & 1 & 1\\
2 & 4 & 8\\
3 & 9 & 27
\end{bmatrix}$$

We describe the size and shape of matrices by saying \textit{an }$m \times 
n$\textit{ matrix}, where $m$ is the number of rows and $n$ is the number of 
columns. A vector is simply a one-column matrix. For example, the vectors 
\textbf{v} above is $3 \times 1$. Matrices aren't restricted to 2 dimensions: 
a matrix can be 3, 4, or any number of dimensions. For example, a $3 \times 2 
\times 4$ matrix would be made of 4 stacked $3 \times 2$ matrices (see figure)

%fixme image of 3D matrix as many stacked 2D matrices

\begin{Exercise}[title = {Matrix Dimensions 1}, label = mat_dim1]
%different size 1, 2, 3d matrices, student writes dimensions in correct order
\end{Exercise}

\begin{Answer}[ref = mat_dim1]

\end{Answer}

\begin{Exercise}[title = {Matrix Dimensions 2}, label = mat_dim2]

\end{Exercise}

\begin{Answer}[ref = mat_dim2]

\end{Answer}

\subsection{Zero Matrices}
Recall that we can represent a generic zero vector as \textbf{0}, which 
indicates a vector of any number of dimensions filled with zeros. Just like 
vectors, there are \textit{zero matrices}\index{zero matrix}, which can by of 
any number of dimensions, all filled with zeros. In two dimensions, zero 
matrices are denoted as $\mathbf{\mathit{O}}_{m \times n}$, where the 
subscript is the dimension of the matrix. The subscript can be expanded to 
denote any number of dimensions. 

\section{Matrix Arithmetic}
\subsection{Adding and Subtracting Matrices}
Matrices that are the same dimension can be added and subtracted. Just like 
vectors, to add matrices you add the elements in the same position:
$$\begin{bmatrix}
-2 & -1\\
2 & 4
\end{bmatrix}
+ \begin{bmatrix}
5 & 2\\
-1 & -4
\end{bmatrix}
= \begin{bmatrix}
-2 + 5 & -1 + 2\\
2 + -1 & 4 + -4
\end{bmatrix} = 
\begin{bmatrix}
3 & 1\\
1 & 0
\end{bmatrix}$$

And to subtract matrices, you subtract the elements in the same position:
$$\begin{bmatrix}
-2 & -1\\
2 & 4
\end{bmatrix}
- \begin{bmatrix}
5 & 2\\
-1 & -4
\end{bmatrix}
= \begin{bmatrix}
-2 - 5 & -1 - 2\\
2 - (-1) & 4 - (-4)
\end{bmatrix} = 
\begin{bmatrix}
-7 & -3\\
3 & 8
\end{bmatrix}$$

Formally, for 2-dimensional matrices, we can say that:
\begin{mdframed}[style = important, frametitle={Adding and Subtracting Matrics}]
For two $m \times n$ matrices, the sum of the matrices is the matrix of the 
sums of the elements in analogous positions:
$$\begin{bmatrix}
x_{11} & x_{12} & x_{13} & \cdots & x_{1n}\\
x_{21} & x_{22} & x_{23} & \cdots & \vdots\\
\vdots & \vdots & \vdots & \cdots & \vdots\\
x_{m1} & x_{m2} & x_{m3} & \cdots & x_{mn}
\end{bmatrix}
+ \begin{bmatrix}
y_{11} & y_{12} & y_{13} & \cdots & y_{1n}\\
y_{21} & y_{22} & y_{23} & \cdots & \vdots\\
\vdots & \vdots & \vdots & \cdots & \vdots\\
y_{m1} & y_{m2} & y_{m3} & \cdots & y_{mn}
\end{bmatrix} = $$
$$\begin{bmatrix}
x_{11} + y_{11} & x_{12} + y_{12} & x_{13} + y_{13} & \cdots & x_{1n} + y_{1n}\\
x_{21} + y_{21} & x_{22} + y_{22} & x_{23} + y_{23} & \cdots & \vdots\\
\vdots & \vdots & \vdots & \cdots & \vdots\\
x_{m1} + y_{m1} & x_{m2} + y_{m2} & x_{m3} + y_{m3} & \cdots & x_{mn} + y_{mn}
\end{bmatrix}$$

To subtract matrices, simply add the negative of the second matrix (that is, 
\textbf{\textit{A}} - \textbf{\textit{B}} = \textbf{\textit{A}} + -\textbf{
\textit{B}}). Additionally, matrix addition is commutative (\textbf{\textit{A}}
+ \textbf{\textit{B}} = \textbf{\textit{B}} + \textbf{\textit{B}}).
\end{mdframed}

\begin{Exercise}[title = {Adding and Subtracting Matrices}, label = add_mat]
Find \textbf{\textit{A}} + \textbf{\textit{B}}, \textbf{\textit{A}} - 
\textbf{\textit{B}}, and \textbf{\textit{B}} - \textbf{\textit{A}}.
\begin{enumerate}
\item $\textbf{\textit{A}} = \begin{bmatrix}
0 & 4 & 0 & 5
\end{bmatrix}$ and $\textbf{\textit{B}} = \begin{bmatrix}
-2 & 3 & -2 & 5
\end{bmatrix}$
\item $\textbf{\textit{A}} = \begin{bmatrix}
4 & -4 & -2\\
1 & -3 & 5\\
-5 & 3 & 0
\end{bmatrix}$ and $\textbf{\textit{B}} = \begin{bmatrix}
5 & 0 & -1\\
-5 & -3 & -2\\
-5 & 3 & -4
\end{bmatrix}$.
\item $\textbf{\textit{A}} = \begin{bmatrix}
-2 & -1 & -5 & -1\\
5 & -4 & 4 & 3
\end{bmatrix}$ and $\textbf{\textit{B}} = \begin{bmatrix}
-5 & -2 & 3 & -5\\
0 & 5 & -4 & -3
\end{bmatrix}$.
\end{enumerate}
\end{Exercise}

\begin{Answer}[ref = add_mat]
\begin{enumerate}
\item \textbf{\textit{A}} + \textbf{\textit{B}} $= \begin{bmatrix}
-2 & 7 & -2 & 10
\end{bmatrix}$. $\textbf{\textit{A}} - \textbf{\textit{B}} = \begin{bmatrix}
2 & 1 & 2 & 0
\end{bmatrix}$. $\textbf{\textit{B}} - \textbf{\textit{A}} = \begin{bmatrix}
-2 & -1 & -2 & 0
\end{bmatrix}$
\item $\textbf{\textit{A}} + \textbf{\textit{B}} = \begin{bmatrix}
9 & -4 & -3\\
-4 & -6 & 3\\
-10 & 6 & -4
\end{bmatrix}$. $\textbf{\textit{A}} - \textbf{\textit{B}} = \begin{bmatrix}
-1 & -4 & -1\\
6 & 0 & 7\\
0 & 0 & 4
\end{bmatrix}$. $\textbf{\textit{B}} - \textbf{\textit{A}} = \begin{bmatrix}
1 & 4 & 1\\
-6 & 0 & -7\\
0 & 0 & -4
\end{bmatrix}$.
\item $\textbf{\textit{A}} + \textbf{\textit{B}} = \begin{bmatrix}
-7 & -3 & -2 & -6\\
5 & 1 & 0 & 0
\end{bmatrix}$. $\textbf{\textit{A}} - \textbf{\textit{B}} = \begin{bmatrix}
3 & 1 & -8 & 4\\
5 & -9 & 8 & 6
\end{bmatrix}$. $\textbf{\textit{B}} - \textbf{\textit{A}} = \begin{bmatrix}
-3 & -1 & 8 & -4\\
-5 & 9 & -8 & -6
\end{bmatrix}$.
\end{enumerate}
\end{Answer}

\subsection{Multiplying Matrices}
Surprisingly (it may be to you), matrix multiplication has dimension limits. 
We cannot multiply any two matrices: the first matrix must have the same 
number of columns as the second has number of rows. Let's examine the origin 
of the dimension limits on matrix multiplication. We begin with a review of 
the vector dot product. 

Recall that in order to find the dot product of two vectors, they must be the 
same length (that is, the same number of dimensions). The result is always a 
scalar: one number. You can review finding the dot product of vectors and 
practice the dimension limits on the vector dot product in the next exercise.

\begin{Exercise}[title = {Vector Dot Product Review}, label = vect_dot]
Find all possible pairs of vectors that can be used to find a dot product, 
then find the dot products. 
\begin{enumerate}
\item $\mathbf{a} = \begin{bmatrix}
1\\
2
\end{bmatrix}$
\item $\mathbf{b} = \begin{bmatrix}
-3\\
3\\
5\\
-5
\end{bmatrix}$
\item $\mathbf{c} = \begin{bmatrix}
1\\
2\\
-1
\end{bmatrix}$
\item $\mathbf{d} = \begin{bmatrix}
-5\\
-1
\end{bmatrix}$
\item $\mathbf{e} = \begin{bmatrix}
1\\
-5\\
3\\
1
\end{bmatrix}$
\item $\mathbf{f} = \begin{bmatrix}
4\\
1\\
-3
\end{bmatrix}$
\end{enumerate}
\end{Exercise}

\begin{Answer}[ref = vect_dot]
It is possible to compute $\mathbf{a} \cdot \mathbf{d}$, $\mathbf{b} \cdot 
\mathbf{e}$, and $\mathbf{c} \cdot \mathbf{f}$:
\begin{enumerate}
    \item $\mathbf{a} \cdot \mathbf{d} = 1(-5) + 2(-1) = -5 + (-2) = -7$
    \item $\mathbf{b} \cdot \mathbf{e} = -3(1) + 3(-5) + 5(3) + -5(1) = -3 + 
    (-15) + 15 - 5 = -8$
    \item $\mathbf{c} \cdot \mathbf{f} = 1(4) + 2(1) + -1(-3) = 4 + 2 + 3 = 9$
\end{enumerate}
\end{Answer}

To multiply two matrices, it is helpful to think of the rows of the first matrix and the columns of the second matrix as vectors. Let's see how this shakes out for two $2 \times 2$ matrices:

\begin{figure}[h!]
    \centering
    \begin{tikzpicture}
        \node[] at (0,0) {$\begin{bmatrix}
            5 & 4\\
            -5 & 1
        \end{bmatrix} \cdot \begin{bmatrix}
            -1 & -2\\
            -5 & -4
        \end{bmatrix} = \begin{bmatrix}
             & & & & & & & & \\
             & & & & & & & & 
        \end{bmatrix}$};
        \node[] at (-2.75, -1) {\textbf{\textit{A}}};
        \node[] at (-2, -1) {$\cdot$};
        \node[] at (-1, -1) {\textbf{\textit{B}}};
    \end{tikzpicture}
    \caption{Each entry in \textbf{\textit{C}}, $c_{ij}$, is the dot product of the $i^{th}$ row of \textbf{\textit{A}}, $a_i$, and the $j^{th}$ column of \textbf{\textit{B}}, $b_j$.}
    \label{fig:mat_mult}
\end{figure}

 Let's look at this more concretely. For two-dimensional matrices, it can be helpful to move your left index finger across the row and right index finger down the column, as shown below:

explanation
	non-commutative
	in 2 dimensions, inner dimensions must match
	visual of arrow method for tracking
	dimensions of resulting matrix
	expand to higher dimensions
example
	can these matrices be multiplied? if so in what order(s)?
	multiply these matrices
	
\subsubsection{Properties of the Zero Matrix}
Just like the number 0, the zero matrix, \textbf{\textit{O}} has unique mathematical properties:
\begin{mdframed}[style = important, frametitle = {Properties of the Zero Matrix}]
For a matrix, \textbf{\textit{A}}, and a zero matrix, \textbf{\textit{O}}
\begin{enumerate}
\item $\mathbf{\mathit{A}} + \mathbf{\mathit{O}} = \mathbf{\mathit{A}}$
\item $\mathbf{\mathit{A}} + \mathbf{\mathit{-A}} = \mathbf{\mathit{O}}$
\item $0 \cdot \mathbf{\mathit{A}} = \mathbf{\mathit{O}}$
\end{enumerate}
\end{mdframed}

\subsection{Can We Divide Matrices?}
no - but we can multiply by the inverse
will be addressed in next chapter(special matrices?)
\section{Systems of Linear Equations}
matrix-vector multiplication, representing linear combination
system of linear equations connection
linear combo example
system of equations example (trail mix, be more explicit about eqs and representation)
practice
	use matrix multiplication to find...
	grading question moved here

\section{Trail Mix for Mars}
Let's look at an applied problem. Three astronauts, Pat, Kai, and River, are getting ready for a trip to Mars. NASA food service is preparing trail mix for the voyage, tailored to each astronaut's taste. The chef needs to submit a budget based on the cost of the trail mix for each astronaut. The mix is made up of raisins, almonds, and chocolate.

Pat prefers a raisins:almonds:chocolate ratio of 6:10:4, Kai likes 2:3:15, and River wants 14:1:5. The chef can buy a kg of raisins for \$7.50, a kg of almonds for \$14.75, and a kg of chocolate for \$22.25. Assuming each astronaut will get 20 kg of trail mix, which astronaut will cost more to feed?

First, set up a matrix to represent the raisins:almonds:chocolate ratios. (Conveniently, these ratios already add to 20.)

$$MixRatios = \begin{bmatrix}
6 & 10 & 4  \\
2 & 3 & 15 \\
14 & 1 & 5
\end{bmatrix}$$

Use a vector to represent the cost of each item:
$$IngredientCost = 
\begin{bmatrix}
7.50 \\
14.75 \\
22.25
\end{bmatrix}$$

To find the cost of trail mix for each astronaut, we simply find the dot product between the mix ratios and the ingredient costs to get:

Pat =  \$281.50 \newline
Kai =  \$615.50 \newline
River = \$231.00 \newline


\begin{Exercise}[title={Vector Matrix Multiplication}, label=vector-matrix-multiply01]
Multiply the array $A$ with the vector $v$. Compute this by hand, and make sure to show your computations. 
$$A = \begin{bmatrix}
1 & -2 & 3 & 5  \\
-4 & 2 & 7 & 1 \\
3 & 3  & -9 & 1
\end{bmatrix}$$

$$v = 
\begin{bmatrix}
 2 \\
 2 \\
 6 \\
 -1
 \end{bmatrix}$$
\end{Exercise}

\begin{Answer}[ref=vector-matrix-multiply01]
$$Av = (11 \text{ }37 \text{ } -43)$$
\end{Answer}

\begin{Exercise}[title = {Using Vector Matrix Multiplication}, label = vmm02]
A college professor offers three different methods of determining a student's final grade. In method A, the student's grade is 20\% based on attendance, 50\% homework, 15\% midterm, and 15\% final. This professor knows many students can learn the material without attending every class, so with method B the student's grade is 50\% homework, 20\% midterm, and 30\% final. Last, the professor knows some students don't do the homework but still show they understand the material by doing well on the tests. With method C, a student's grade is 40\% midterm and 60\% final. The professor uses whatever method results in the highest grade to determine each student's final grade. 

Suppose Suzy has attended 65\% of classes, has an average homework grade of 30\%, earned a 80\% on the midterm, and earned a 75\% on the final. What final grade will her professor post?
\end{Exercise}

\begin{Answer}[ref = vmm02]
The different methods can be represented in a matrix:
$$\begin{bmatrix}
0.20 & 0.50 & 0.15 & 0.15\\
0 & 0.50 & 0.20 & 0.30\\
0 & 0 & 0.4 & 0.6
\end{bmatrix}$$

And Suzy's individual grades can be represented by a vector:
$$\begin{bmatrix}
65\\
30\\
80\\
75
\end{bmatrix}
$$

To see the results of the three different methods, we can multiply the matrix and the vector:
$$\begin{bmatrix}
0.20 & 0.50 & 0.15 & 0.15\\
0 & 0.50 & 0.20 & 0.30\\
0 & 0 & 0.4 & 0.6
\end{bmatrix}
\cdot 
\begin{bmatrix}
65\\
30\\
80\\
75
\end{bmatrix} = 
\begin{bmatrix}
0.2(65)+0.5(30)+0.15(80) + 0.15(75)\\
0(65) + 0.5(30) + 0.2(80) + 0.3(75)\\
0(65) + 0(30) + 0.4(80) + 0.6(75)
\end{bmatrix}$$

Which yields:
$$\begin{bmatrix}
51.25\\
53.5\\
77
\end{bmatrix}$$

Since method C yields the highest grade, the professor will post a final grade of 77. 
\end{Answer}

\subsection{Vector-Matrix Multiplication in Python}
Most real-world problems use very large matrices, where it becomes impractical to do calculations by hand. As long as you understand how matrix-vector multiplication is performed, you will be equipped to use a computing language, like Python, to do the calculations for you. 

Create a file called \filename{vectors\_matrices.py} and enter this code:

\begin{Verbatim}
# import the python module that supports matrices
import numpy as np

# create an array
a = np.array([[5, 1, 3, -2], 
              [1, -1, 8, 4], 
              [6, 2, 1, 3]])

# create a vector 
b = np.array([1, 2, 3, -8])

# calculate the dot product
print(a.dot(b))
\end{Verbatim}

When you run it, you should see:
\begin{Verbatim}
[16, 6, 8]
\end{Verbatim}

\section{Where to Learn More}
Watch this video from Khan Academy about matrix-vector products: \url{https://rb.gy/frga5}

