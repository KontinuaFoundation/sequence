\chapter{Vectors and Matrices}
Linear algebra is a specialized form of algebra that can represent and manipulate sets of variables that are linearly related to one another. One of the basic operations is the multiplication of a matrix by a vector. As you work through this module, you’ll see that you already know the fundamentals (vectors, scalars, dot products) and how to apply these concepts to practical problems. You’ve also had some experience with matrices in the form of spreadsheets. 

Matrices can represent:
\begin{itemize}
\item A linear transformation, such as rotation, scaling, and skewing. You apply a transformation to a vector by multiplying the vector by a matrix.  
\item A system of linear equations. Linear algebra provides various methods that you can use to find the solution vector. 
\end{itemize}

\section{Applications of Matrix-Vector Multiplication}
Many areas in engineering and science rely on matrix-vector multiplication. These are just a few examples. As you encounter more topics in science and engineering, you will find that matrix-vector multiplication is crucial to many other fields.

\subsection{Computer Graphics}
When you play a video game or watch the latest CG animation, matrix operations transform objects in the scene to make them appear as if moving, getting closer, and so on. You can represent the vertices of objects as vectors, and then apply a transformation matrix.

\subsection{Data Analysis}
We live in an era in which it's easy to collect so much data that it's difficult to make sense of the data by just looking at it. You can represent the data in matrix form and then find a solution vector. For example, scientists use this technique to figure out the effectiveness of drug treatments on disease.

\subsection{Economics}
Take a look at financial section of any news organization and you'll see headlines such as "Economic Data Points to Faster Growth" or "Is the Inflation Battle Won?" Economists can use systems of linear equations to represent economic indicators, such as consumer consumption, government spending, investment rate, and gross national product. By using various methods that you'll learn about later, they can get a good idea of the state of the economy.

\subsection{Engineering}
Engineering couldn't do without vector-matrix multiplication. For example, the orbital dynamics of space travel relies on it. Engineers must predict and calculate the the motion of planetary bodies, satellites, and spacecraft. By solving systems of linear equations engineers can make sure that a spacecraft travels to its destination without running into a satellite or space rock.

\subsection{Image Processing}
An image is a matrix of pixel values. When you take a selfie and apply a filter, the image app applies a transformation to the image matrix. A simple operation would be to change the tint of the image. A more complex operation would be to skew the image to make it distorted, like a funhouse mirror.

\section{Vector-Matrix Multiplication}
Let's take a look at the general form of vector-matrix multiplication. Given a matrix $A$ of size $m \times n$ and a vector $x$ of size $n \times 1$, the product $Ax$ is a new vector of size $m \times 1$. 

You compute the $i$-th component of the product vector $Av$ by taking the dot product of the $i$-th row of $A$ and the vector $v$:

\begin{equation*}
(Av)_i = \sum_{j=1}^n a_{ij}x_j
\end{equation*}

where $a_{ij}$ is the element in the $i$-th row and $j$-th column of $A$, and $v_j$ is the $j$-th element of $v$.

This is the general form of a matrix and a vector, written to show the specific components of each:


 $$A = \begin{bmatrix}
 a_{11} &a_{12}  &a_{13} &... &a_{1n}  \\
 a_{21} &a_{22}  &a_{23} &... &a_{2n}  \\
 ... \\
 a_{m1} &a_{m2}  &a_{m3} &... &a_{mn}  
\end{bmatrix}$$

$$v = \begin{bmatrix}
 v_{1}  \\
 v_{2} \\
 v_{3} \\
 ... \\
 v_{m} 
\end{bmatrix}$$

 $$Av =\begin{bmatrix}
 v_{1}*a_{11} +v_{2}*a_{12}  +v_{3}*a_{13} +... +v_{m}*a_{1n}  \\
 v_{1}*a_{21} +v_{2}*a_{22}  +v_{3}*a_{23} +... +v_{m}*a_{2n}  \\
 ... \\
 v_{1}*a_{m1} +v_{2}*a_{m2}  +v_{3}*a_{m3} +... +v_{m}*a_{mn}  
\end{bmatrix}$$

Let's look at a specific example.

$$A = \begin{bmatrix}
 2  &4 &6  \\
 3  &5 &7  \\
 1  &2 &3  \\
 8  &6 &2 
\end{bmatrix}$$

$$v = \begin{bmatrix}
 -2  \\
 1 \\
 3 
\end{bmatrix}$$

Solution:
$$= \begin{bmatrix}
-2*2+1*4+3*6\\
-2*3+1*5+3*7\\
 -2*1+1*2+3*3\\
-2*8+1*6+3*2
\end{bmatrix}$$

$$= \begin{bmatrix}
18 \\
20\\
9\\
-4 
\end{bmatrix}$$
$$= (18,20,9,-4)$$

\begin{Exercise}[title={Vector Matrix Multiplication}, label=vector-matrix-multiply01]
Multiply the array $A$ with the vector $v$. Compute this by hand, and make sure to show your computations. 
$$A = \begin{bmatrix}
1 &-2  &3 &5  \\
-4  &2  &7 &1 \\
3  &3  &-9 &1
\end{bmatrix}$$
	$$v = 
	\begin{bmatrix}
		2 \\
 		2 \\
 		6 \\
 		-1
	\end{bmatrix}$$
\end{Exercise}
\begin{Answer}[ref=vector-matrix-multiply01]
$$Av = (11, 37, -43)$$
\end{Answer}

\subsection{Vector-Matrix Multiplication in Python}
Most real-world problems use very large matrices where it becomes impractical to perform calculations by hand. As long as you understand how matrix-vector multiplication is done, you'll be equipped to use a computing language, like Python, to do the calculations for you. 

Create a file called \filename{vectors\_matrices.py} and enter this code:

\begin{Verbatim}
// import the python module that supports matrices
import numpy as np

// create an array
a = np.array([[ 5, 1 ,3, -2], 
              [ 1, -1 ,8, 4], 
              [ 6, 2 ,1, 3]])

// create a vector 
b = np.array([1, 2, 3,-8])

//calculate the dot product
print(a.dot(b))
\end{Verbatim}

When you run it, you should see:
\begin{Verbatim}
[16  6  8]
\end{Verbatim}

\section{Where to Learn More}
Watch this video from Khan Academy about matrix-vector products: \url{https://rb.gy/frga5}

