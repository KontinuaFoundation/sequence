\chapter{Matrices and Systems of Linear Equations}

In the chapter on linear combinations, we saw that we can linearly combine vectors to create other vectors. Consider 3 vectors:

$$\textbf{x} = \begin{bmatrix}
-1\\
2\\
0
\end{bmatrix}\text{ 	} \textbf{y} = \begin{bmatrix}
-1\\
2\\
1
\end{bmatrix}\text{ 	} \textbf{z} = \begin{bmatrix}
-2\\
1\\
0
\end{bmatrix}$$

We can write a linear combination of these vectors:
$$c\textbf{x} + d\textbf{y} + e\textbf{z}$$

Which we can expand to show the vectors:
$$c \begin{bmatrix}
-1\\
2\\
0
\end{bmatrix} + d \begin{bmatrix}
-1\\
2\\
1
\end{bmatrix} + e \begin{bmatrix}
-2\\
1\\
0
\end{bmatrix} = \begin{bmatrix}
-c - d - 2e\\
2c + 2d + e\\
d
\end{bmatrix}$$

We can also represent this combination with a matrix where each column is one of the vectors:
$$\begin{bmatrix}
-1 & -1 & -2\\
2 & 2 & 1\\
0 & 1 & 0
\end{bmatrix} \cdot \begin{bmatrix}
c\\
d\\
e
\end{bmatrix} = \begin{bmatrix}
-c - d - 2e\\
2c + 2d + e\\
d
\end{bmatrix}$$

%\subsection{Thinking about Rows}
	
%\subsection{Thinking about Columns}


%\section{Systems of Linear Equations}
%matrix-vector multiplication, representing linear combination
%system of linear equations connection
%linear combo example
%system of equations example (trail mix, be more explicit about eqs and representation)
%practice
%	use matrix multiplication to find...
%	grading question moved here

\section{Trail Mix for Mars}
Let's look at an applied problem. Three astronauts, Pat, Kai, and River, are getting ready for a trip to Mars. NASA food service is preparing trail mix for the voyage, tailored to each astronaut's taste. The chef needs to submit a budget based on the cost of the trail mix for each astronaut. The mix is made up of raisins, almonds, and chocolate.

Pat prefers a raisins:almonds:chocolate ratio of 6:10:4, Kai likes 2:3:15, and River wants 14:1:5. The chef can buy a kg of raisins for \$7.50, a kg of almonds for \$14.75, and a kg of chocolate for \$22.25. Assuming each astronaut will get 20 kg of trail mix, which astronaut will cost more to feed?

First, set up a matrix to represent the raisins:almonds:chocolate ratios. (Conveniently, these ratios already add to 20.)

$$MixRatios = \begin{bmatrix}
6 & 10 & 4  \\
2 & 3 & 15 \\
14 & 1 & 5
\end{bmatrix}$$

Use a vector to represent the cost of each item:
$$IngredientCost = 
\begin{bmatrix}
7.50 \\
14.75 \\
22.25
\end{bmatrix}$$

To find the cost of trail mix for each astronaut, we simply find the dot product between the mix ratios and the ingredient costs to get:

Pat =  \$281.50 \newline
Kai =  \$615.50 \newline
River = \$231.00 \newline


\begin{Exercise}[title={Vector Matrix Multiplication}, label=vector-matrix-multiply01]
Multiply the array $A$ with the vector $v$. Compute this by hand, and make sure to show your computations. 
$$A = \begin{bmatrix}
1 & -2 & 3 & 5  \\
-4 & 2 & 7 & 1 \\
3 & 3  & -9 & 1
\end{bmatrix}$$

$$v = 
\begin{bmatrix}
 2 \\
 2 \\
 6 \\
 -1
 \end{bmatrix}$$
\end{Exercise}

\begin{Answer}[ref=vector-matrix-multiply01]
$$Av = (11 \text{ }37 \text{ } -43)$$
\end{Answer}

\begin{Exercise}[title = {Using Vector Matrix Multiplication}, label = vmm02]
A college professor offers three different methods of determining a student's final grade. In method A, the student's grade is 20\% based on attendance, 50\% homework, 15\% midterm, and 15\% final. This professor knows many students can learn the material without attending every class, so with method B the student's grade is 50\% homework, 20\% midterm, and 30\% final. Last, the professor knows some students don't do the homework but still show they understand the material by doing well on the tests. With method C, a student's grade is 40\% midterm and 60\% final. The professor uses whatever method results in the highest grade to determine each student's final grade. 

Suppose Suzy has attended 65\% of classes, has an average homework grade of 30\%, earned a 80\% on the midterm, and earned a 75\% on the final. What final grade will her professor post?
\end{Exercise}

\begin{Answer}[ref = vmm02]
The different methods can be represented in a matrix:
$$\begin{bmatrix}
0.20 & 0.50 & 0.15 & 0.15\\
0 & 0.50 & 0.20 & 0.30\\
0 & 0 & 0.4 & 0.6
\end{bmatrix}$$

And Suzy's individual grades can be represented by a vector:
$$\begin{bmatrix}
65\\
30\\
80\\
75
\end{bmatrix}
$$

To see the results of the three different methods, we can multiply the matrix and the vector:
$$\begin{bmatrix}
0.20 & 0.50 & 0.15 & 0.15\\
0 & 0.50 & 0.20 & 0.30\\
0 & 0 & 0.4 & 0.6
\end{bmatrix}
\cdot 
\begin{bmatrix}
65\\
30\\
80\\
75
\end{bmatrix} = 
\begin{bmatrix}
0.2(65)+0.5(30)+0.15(80) + 0.15(75)\\
0(65) + 0.5(30) + 0.2(80) + 0.3(75)\\
0(65) + 0(30) + 0.4(80) + 0.6(75)
\end{bmatrix}$$

Which yields:
$$\begin{bmatrix}
51.25\\
53.5\\
77
\end{bmatrix}$$

Since method C yields the highest grade, the professor will post a final grade of 77. 
\end{Answer}

\subsection{Vector-Matrix Multiplication in Python}
Most real-world problems use very large matrices, where it becomes impractical to do calculations by hand. As long as you understand how matrix-vector multiplication is performed, you will be equipped to use a computing language, like Python, to do the calculations for you. 

Create a file called \filename{vectors\_matrices.py} and enter this code:

\begin{Verbatim}
# import the python module that supports matrices
import numpy as np

# create an array
a = np.array([[5, 1, 3, -2], 
              [1, -1, 8, 4], 
              [6, 2, 1, 3]])

# create a vector 
b = np.array([1, 2, 3, -8])

# calculate the dot product
print(a.dot(b))
\end{Verbatim}

When you run it, you should see:
\begin{Verbatim}
[16, 6, 8]
\end{Verbatim}

\section{Where to Learn More}
Watch this video from Khan Academy about matrix-vector products: \url{https://rb.gy/frga5}

