\chapter{Data tables in SQL}

Most organizations keep their data as tables inside a relational
database management system. Developers talk to those systems using a
language called SQL (``Structured Query Language'').

Some relational database managers are pricey products you may have
heard of before: Oracle, Microsoft SQL Server. Some are free:
PostgreSQL or MySQL.  These are server software that client programs
talk to over the companies network.

There is a library, called \texttt{sqlite}, that lets us create files that hold
tables. We can use SQL to create, edit, and browse those tables.
sqlite is free, fast, and very easy to install.  So we will use sqlite
instead of a networked database management system.

If you look in your digital resources, you will find a file called
\filename{bikes.db}. I created this file using sqlite, and now you
will use sqlite to access it.

In the terminal, get to the directory where \filename{bikes.db} lives. To open the sqlite tool on that file:

\begin{Verbatim}[commandchars=\\\{\}]
> textbf{sqlite3 bikes.db}
\end{Verbatim}

(If your system complains that there is no sqlite3 tool, you need to install sqlite. See this website: \url{https://sqlite.org/})

Please follow along: type each command shown here into the terminal
and see what happens.

We mostly run SQL commands in this tool, but there are a few non-SQL
 commands that all start with a period.  To see the tables and their
 columns, you can run \texttt{.schema}:

\begin{Verbatim}[commandchars=\\\{\}]
sqlite> \textbf{.schema}
CREATE TABLE bike (bike_id int PRIMARY KEY, brand text, size int,
                   purchase_price real, purchase_date date, status text);
\end{Verbatim}

That is the SQL command that I used to create the \texttt{bike}
table. You can see all the columns and their types.

You want to see all the rows of data in that table?

\begin{Verbatim}[commandchars=\\\{\}]
sqlite> \textbf{select * from bike;}
4997391|GT|57|269.61|2009-05-03|rented
5429447|Cannondale|50|215.91|2002-02-17|broken
5019171|Trek|58|251.17|1985-07-11|rented
3000288|Cannondale|57|211.08|1993-01-05|broken
880965|GT|52|281.75|1995-08-02|available
...
\end{Verbatim}

You will see 1000 rows of data!

The SQL language is not case-sensitive, so you can also write it like this:
\begin{Verbatim}[commandchars=\\\{\}]
sqlite> \textbf{SELECT * FROM BIKE;}    
\end{Verbatim}

Often you will see SQL with just the SQL keywords in all caps:
\begin{Verbatim}[commandchars=\\\{\}]
sqlite> \textbf{SELECT * FROM bike;}    
\end{Verbatim}
The semicolon is not part of SQL, but it tells sqlite that you are done writing a command and that it should be executed.

SQL lets you choose which columns you would like to see:
\begin{Verbatim}[commandchars=\\\{\}]
sqlite> \textbf{SELECT bike_id, brand FROM bike;}
4997391|GT
5429447|Cannondale
5019171|Trek
3000288|Cannondale
...
\end{Verbatim}

Using WHERE, SQL lets you choose which rows you would like to see:
\begin{Verbatim}[commandchars=\\\{\}]
sqlite> \textbf{SELECT * FROM bike WHERE purchase_date > '2009-01-01' AND brand = 'GT';}
4997391|GT|57|269.61|2009-05-03|rented
326774|GT|56|165.0|2009-06-27|available
264933|GT|52|302.43|2009-07-09|available
5931243|GT|55|173.56|2009-11-26|rented
4819848|GT|51|221.71|2009-12-11|rented
9347713|GT|52|232.32|2009-06-13|available
3019205|GT|58|262.94|2009-08-22|available    
\end{Verbatim}

Using DISTINCT, SQL lets you get just one copy of each value:
\begin{Verbatim}[commandchars=\\\{\}]
sqlite> \textbf{SELECT DISTINCT status FROM bike;}
rented
broken
available

Busted
Flat tire
good
out
Rented
\end{Verbatim}

You can also edit these rows.  For example, if you wanted every status
that is \texttt{Busted} to be changed to \texttt{broken}. You can use an UPDATE statement:

\begin{Verbatim}[commandchars=\\\{\}]
sqlite> \textbf{UPDATE bike SET status='broken' WHERE status='Busted';}
sqlite> \textbf{SELECT DISTINCT status FROM bike;}
rented
broken
available

Flat tire
good
out
Rented
\end{Verbatim}

You can insert new rows:
\begin{Verbatim}[commandchars=\\\{\}]
sqlite> \textbf{INSERT INTO bike (bike_id, brand, size, purchase_price, purchase_date, status)}
   ...> \textbf{VALUES (1, 'GT', 53, 123.45, '2020-11-13', 'available');}
sqlite> \textbf{SELECT * FROM bike WHERE bike_id = 1;}
1|GT|53|123.45|2020-11-13|available
\end{Verbatim}

You can delete rows:
\begin{Verbatim}[commandchars=\\\{\}]
sqlite> \textbf{DELETE FROM bike WHERE bike_id = 1;}
sqlite> \textbf{SELECT * FROM bike WHERE bike_id = 1;}
\end{Verbatim}

To get out of sqlite, type \texttt{.exit}.

\begin{Exercise}[title={SQL Query}, label=sql_where]
  Execute an SQL query that returns the \texttt{bike\_id} (no other
  columns) of every Trek bike that cost more than \$300.
\end{Exercise}
\begin{Answer}[ref=sql_where]
\begin{Verbatim}
  SELECT bike_id FROM bike WHERE purchase_price > 330 AND brand='Trek'
\end{Verbatim}
\end{Answer}

\section{Using SQL from Python}

The people behind sqlite created a library for Python that lets you
execute SQL and fetch the results from inside a python program.

Let's create a simple program that fetches and displays the bike ID
and purchase date of every Trek bike that cost more than \$300.

Create a file called \filename{report.py}:
\begin{Verbatim}
import sqlite3 as db

con = db.connect('bikes.db')
cur = con.cursor()

cur.execute("SELECT bike_id, purchase_date FROM bike WHERE purchase_price > 330 AND brand='Trek'")
rows = cur.fetchall()

today = datetime.date.today()
for row in rows:
    print(f"Bike {row[0]}, purchased {row[1]}")

con.close()
\end{Verbatim}

When you execute it, you should see:
\begin{Verbatim}[commandchars=\\\{\}]
> \textbf{python3 report.py}
Bike 4128046, purchased 2007-08-06
Bike 7117808, purchased 1995-03-12
Bike 7176903, purchased 1986-07-03
Bike 827899, purchased 2009-03-14
Bike 363983, purchased 1970-08-16
\end{Verbatim}
