\chapter{Force, Mass, and Acceleration}

\section{Mass and Acceleration}

Each atom has a mass, which means everything made up of those atoms has mass as 
well (and that's pretty much everything!). We measure mass in grams. A paper clip 
is about 1 gram of steel. An adult human can have a mass of 70,000 grams, so for 
larger things, we often talk about kilograms, which is 1000 grams.

The first interesting thing about mass is that objects with more mass
require more force to accelerate. For example, pushing a bicycle so
that it accelerates from a standstill to jogging speed in 2 seconds
requires much less force than pushing a train so that it accelerates
at the same rate.


\begin{mdframed}[style=important, frametitle={Newton's Second Law of Motion}]

The force necessary to accelerate an object of mass $m$ at an acceleration of
$a$ is given by:
$$F = m a$$

This means the force is equal to the mass times the acceleration.

\end{mdframed}

What are the units here? We already know that mass is measured in
kilograms. We can measure velocity in meters per second, but that is
different from acceleration. Acceleration is the rate of change in
velocity. So if we want to go from 0 to 5 meters per second (that's
jogging speed) in two seconds, that is a change in velocity of 2.5
meters per second every second. We would say this acceleration is $2.5
m/s^2$.

\subsection{Calculating Acceleration}
As suggested above, acceleration is a change in velocity. It is calculated by
dividing the change in velocity by the time it takes to make that change.

\begin{mdframed}[style = important, frametitle = {Calculating Acceleration}]
The acceleration of an object from an initial velocity, $v_i$, to a final
velocity, $v_f$, over a period of time, $t$, is given by:

$$a = \frac{v_f - v_i}{t}$$
\end{mdframed}

\textbf{Example}: Your car can go from zero to 60 mph in 3 seconds. What is the
acceleration in $m / s^2$?

\textbf{Solution}: First, let's convert from the imperial units of miles per
hour to the SI units of meters per second. You can do this using a search engine, 
but we will show how to do it by hand below. (You will learn more about this 
method in the Units chapter).

$$\frac{60 \text{ miles}}{1 \text{ hour}} \cdot \frac{1.61 \text{ km}}{1 
\text{ mile}} \cdot \frac{1000\text{ m}}{1\text{ km}} \cdot \frac{1\text{ hour}}{
3600\text{ seconds}} \approx \frac{26.82\text{ m}}{s}$$

Now we have the starting velocity (0 m/s), the ending velocity (26.82 m/s), and 
the time (3 s), and we can find the acceleration:
$$a = \frac{v_f - v_i}{t} = \frac{26.82\frac{m}{s} - 0\frac{m}{s}}{3s} \approx 
8.94 \frac{m}{s^2}$$

\subsection{Determining Force}
What about measuring force? Newton decided to name the unit after himself: The
force necessary to accelerate one kilogram at $1 m/s^2$ is known as \textit{a
newton}. It is often denoted by the symbol $N$.

$$1 N = 1 \frac{kg \cdot m}{s^2}$$

\textbf{Example}: If the car in the above example has a mass of 1500 kg, how much 
force does the engine use to accelerate the car?

\textbf{Solution}: We have already found the car's acceleration: 8.94 $m/s^2$. 
With the mass and acceleration, we can use Newton's Second Law to find the force 
needed to accelerate the car:
$$F = m \cdot a = 1500\text{ kg} \cdot 8.94 \frac{m}{s^2} = 13410\text{ N}$$

\begin{Exercise}[title={Acceleration}, label=acceleration_train]

While driving a bulldozer, you come across a train car (with no brakes
and no locomotive) sitting on a track in the middle of a city. The train car
has a label telling you that it has a mass of 2,400 kg. There is a time-bomb
welded to the interior of the train car, and the timer tells you that
you can safely push the train car for 120 seconds. To get the train
car to where it can explode safely, you need to accelerate it to 20 meters per
second. Fortunately, the track is level and the train car's wheels have
almost no rolling resistance.

With what force, in newtons, do you need to push the train for those 120 seconds?

\end{Exercise}
\begin{Answer}[ref=acceleration_train]
If you accelerate to 20 m/s in 120 s, the acceleration is:
$$a = \frac{v_f - v_i}{t} = \frac{20\text{ m/s} - 0\text{ m/s}}{120\text{ s}} = 
\frac{1}{6} \frac{m}{s^2}$$

To achieve this acceleration, you will need to apply a force of:
$$F = m \cdot a = 2400\text{ kg} \cdot \frac{1}{6} \frac{m}{s^2} = 400\text{ }N$$
\end{Answer}
