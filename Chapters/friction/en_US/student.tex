\chapter{Friction}

Imagine there is a large and heavy steel box resting in the middle of a large floor   Imagine you push it hard enough to get it moving.   If you stop pushing,  will it continue to glide gracefully across the floor? 

Probably not.  Unless the floor is very slippery for some reason,  the box will come to a halt immediately after you stop pushing.  We would say that it is stopped by the force of \newterm{friction}. 

What's really happening?  The kinetic energy of the box is being converted into heat 
between the bottom of the box and floor.   As the bottom of the box and the floor get warmer,  the speed of the box decreases.

The amount of friction is proportional to the force with which the box is pressing against the floor -- so you should expect a box that is twice as heavy to experience twice as much frictional force.

That is,  the frictional force is proportional to the normal force.  (FIXME: picture here)

The amount of friction is also determined by the materials that are sliding against each other.  For example,  if the floor is ice,  the frictional force will be less than if the floor is made of wood. 

If you are pushing the box with a force of $F$ and it is moving but neither accelerating nor decelerating,  then the force you are applying is exactly balanced by the frictional force.  If the box is pressing against the floor with a force of $N$, then we say the \newterm{coefficient of friction} between the steel box and the floor is given by

$$\mu = \frac{F}{N}$$

\begin{Exercise}[title={Bicycle Stopping},  label=bike_stop]
  
You are riding your bicycle at 11 meters per second when you slam on the brakes and lock up the wheels.  
You weigh 55 kg.   
When any piece of rubber is skidding across a dry road,  the coefficient of friction will be about 0.7.

Answer the following questions: 

\begin{itemize}
\item How much kinetic energy do you have when you engage the brakes?
\item As you skid,  how much frictional force is decelerating you?
\item For how many meters will you slide?
\end{itemize}

\end{Exercise}
\begin{Answer}[ref=bike_stop]

Kinetic energy? $E = mv^2 = (55)(11^2) = 6,655 \frac{kg m^2}{s^2} = 6,655$ joules.

Frictional force? $F = \mu N = (0.7)(55)(9.8) = 377.3$ newtons.

Distance?  $D = \frac{6,655}{377.3} = 17.6$ seconds.

\end{Answer}

Notice that the force of friction is not determined by how much of the tire is touching the ground.  The coefficient of friction of the two materials and the normal force all all you need to compute the friction.

Why, then, do drag racing cars have such big tires?  Remember that the energy of friction becomes heat and that rubber melts.  If a drag racer had skinny tires,  all the heat from the friction would be in a very small patch of tire, which would melt.

\section{Static vs Dynamic Friction Coefficients}

Once again, imagine the box resting on the floor.  As you start to push it,  it will sit still until your force is greater than the force of friction.   However, once it starts moving,  the force of friction seems to be less.

Between two materials,  there is actually 2 different friction coefficients:

\begin{itemize}
\item Dynamic friction coefficient:  The coefficient you use once the box is sliding against the floor.
\item Static friction coefficient: The coefficient you use to figure out how much force you need to get the box to start to move.
\end{itemize}

The dynamic friction coefficient is always less than the static friction coefficient.  For a car on a dry road,  the dynamic friction coefficient is about 


