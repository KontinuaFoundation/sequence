\chapter{Tides and Eclipses}

You live with a lot of orbital paths:
\begin{itemize}

\item The earth is spinning.  If you are standing at the equator,  you are traveling at 1,674 km per hour around the center of the planet.  We are all spinning east,  thus the sun comes up in the east and sets in the west.

\item  The earth is orbiting the sun.    It takes 365.242 days for the earth to go once around the sun.  This is why different constellations appear at different times during the year -- we only see the stars at night and the direction of night shifts as the earth moves around the sun.  

\item The moon is orbiting the earth.   The moon travels once around the earth once every 27.3 days.  

\end{itemize}

You can see the effects of these orbits on our planet.  Let's go over a few.

\section{Leap Years}

Note that it takes 365.242 days for the earth to go around the sun.  If we declared "The calendar will \emph{always} be 365 days per year!"  then gradually the seasons would shift by 0.242 days every year.  After a century,  they would have migrated 24 days.

So, we made a rule: "Every fourth year,  we will add an extra day to the calendar!"  The years 2021, 2022,and 2023 get no February 29th.  2024 gets a February 29th.

That got us a calendar with 365.25 days, so the seasons would not have migrated as quickly,  but they would
have migrated about three days every four hundred years.

So, we made another rule: "There will be no February 29th in the three century years (multiples of 100) that are not multiples of 400."  So the year 1900 had no Feb 29,  but the year 2000 had one.   Now the average number of days per year is 365.2425.

\section{Phases of the Moon}

The earth, the moon, and the sun form a triangle.   If you were standing on the moon,  you could measure the angle between the light coming from the sun and the the light going to the earth.   That angle would fluctuate between 0 degrees and 180 degrees.  
\begin{itemize}
\item When the angle was close to 0,  the people on earth would see a full moon.  
\item When the angle was close to 90 degrees,  the people on earth would see a half moon. 
\item When the angle was nearing 180 degrees, the people on earth would see a slim crescent.
\item When the angle was very close to 180 degrees,  the moon would be dark.  This is called a "new moon." 
\end{itemize}


Even though it takes 27.3 for the moon to travel around the earth once,   it takes 29.5 days to get from one full moon to the next. Why?  In the 27.3 days that it took the moon to travel around the earth,  the earth has moved about 17 degrees around the sun.  To get back into the same triangle takes another 2.2 days.

\section{Eclipses}

While the earth orbits the sun and the moon orbits the earth,  the two orbits are \emph{not in the same plane}.
We call the plane that the earth orbits the sun in the \newterm{ecliptic plane}.   The plane of the moon's orbit is about  5 degrees tilted from the ecliptic plane.

Note that the moon passes through the ecliptic plane only twice every 27.3 days.   Imagine that at the instant it passed through the ecliptic plane was also the precise instant of a full moon.    The sun, the earth, and the moon would be in a straight line!  The earth would cast a shadow upon the moon -- it would go from a bright full moon to a dark moon until the moon moved back out of the shadow of the earth.   This is known as a \newterm{lunar eclipse}.

The diameter of the moon is a little more than a quarter the diameter of the earth,  so they don't have to be in perfect alignment for the moon to be darkened.   Lunar eclipses actually happen once or twice per year.

Now imagine that at the instant the moon passed through the ecliptic plane was also the precise instant of a new moon.    The sun,  the moon, and the earth would be in a straight line!  The moon would cast a shadow upon some part of the earth.  To a person in that shadow,  the sun disappear behind the moon.   This is known as a \newterm{solar eclipse}.

The sun is pretty big,  so if the moon blots out just part of it,  we call it a \newterm{partial solar eclipse}.  There are a few partial solar eclipses every year.   Note that because the moon's shadow is too small to shade the whole earth,   only certain parts of the world will experience any solar eclipses. 

Every 18 months or so,  there is a total eclipse of the sun.  Once again,  only certain parts of the world experience it.   You can expect to experience a total eclipse of the sun at your home about once every 375 years.
 

