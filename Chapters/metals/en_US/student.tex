\chapter{Metals}

Elements that transmit electricity well, even at low temperatures, are
called \newterm{metals}. Here are some metals that you are probably familiar
with: aluminum, iron, copper, tin, gold, silver, and platinum. Aluminum and
iron are particularly common; together they make up about 14\% of the
earth's crust.

An \newterm{alloy} is a mixture of elements that includes at least one
metal. Brass, for example, is an alloy of copper and zinc.  Bronze is
an alloy of copper and tin.

\section{Steel}

One of the most common alloys is steel, an alloy of iron and carbon.
In pure iron, the molecules slip easily past each other, so pure iron
is relatively soft and easily deformed. The carbon in steel prevents
that slipping, thus steel is much, much harder than iron.

How much carbon? If you put less than 0.002\% by weight, you end up
with something very much like pure iron.  As you increase the carbon,
it gets harder and harder.  Once it gets above about 2\%, the result
is very brittle.

If you add about 11\% chromium to steel, you get \newterm{stainless
  steel} which resists rusting.

\begin{Exercise}[title={Tensile Strength}, label=tensile-mpa]

The tensile strength of steel is usually between 400 MPa and 1200
MPa. A Mega Pascal (MPa) is the strength necessary to hold 1,000,000 newtons of
force with a cable that has a 1 square meter cross section. Or,
equivalently, to hold 1 newton of force with a cable that has a 1
square millimeter cross section. 

If you have are buying a round cable that has a tensile strength of
700 Mpa and must hold a 100 kg man aloft, what the diameter of the
smallest cable you can use?
  
\end{Exercise}
\begin{Answer}[ref=tensile-mpa]
On earth, holding a 100 kg man aloft requires 980 Newtons of force.

$980/700 = 1.4$, so you need a cable with a cross-section area of 1.4
square millimeters.

$$\pi r^2 = 1.4$$

So $r = \sqrt{1.4/\pi} \approx .67$ millimeters.  So the cable would
have to have a diameter of at least 1.34 millimeters.

\end{Answer}

Here are some approximate tensile strengths of ather materials:

\begin{tabular}{c|c}
  Material & Tensile strength (MPa) \\
  \hline
  Iron & 3 \\
  Concrete & 4 \\
  Rubber & 16 \\
  Glass & 33 \\
  Wood & 40 \\
  Nylon & 100 \\
  Human hair & 200 \\
  Aluminum  & 300 \\
  Steel & 700 \\
  Spider webs & 1000 \\
  Carbon fiber & 4000
\end{tabular}

\section{What metal for what task?}

You will see copper used a lot for electrical wires in your house and
appliances because it is very efficient at moving electricity (very
little power is lost as heat). It is also very good a transmitting
heat, so you will often see copper pots and pans.

Aluminum is less dense than copper, and is still a pretty good
conductor of electricity. Thus, the overhead wires in a power system
are often made of aluminum.

Aluminum is not as strong as steel, but considerably lighter. It is
often used structurally where weight is a concern: skyscrapers, cars,
airplanes, and ships.

Titanium is about as strong as steel, but it weights about half as
much. Titanium is very difficult to work with, so it is used in places
where weight and strength are very important and cost is not:
airplanes and bicycles.

(Carbon fiber, which is light, strong, and very easy to work with, is
replacing aluminum and titanium in many applications. 20 years ago,
many expensive bicycles were made of titanium. These days the vast
majority are made with carbon fiber.)

Zinc and tin are very resistant to corrosion, so they are often used
as a coating to prevent steel from rusting. They are also used in many
alloys for the same reason.  In the United States, the penny is 97.5\%
zinc and only 2.5\% copper.

