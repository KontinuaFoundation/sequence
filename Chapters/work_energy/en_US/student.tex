\chapter{Work and Energy}

In this chapter, we are going to talk about how engineers define work
and energy. It frequently takes force to get work done. Let's start with thinking about the relationship between force and energy. As we learned earlier, Force is measured in
newtons, and one newton is equal to the force necessary to accelerate one
kilogram at a rate of $1 m/s^2$.

When you lean on a wall, you are exerting a force on the wall, but you
aren't doing any work. On the other hand, if you push a car for a mile,
you are clearly doing work. Work, to an engineer, is the force you
apply to something, as well as the distance that it moves, in the direction
of the applied force. We measure work in \textit{joules}. A joule is one
newton of force over one meter.\index{Joule}

\includegraphics[width=0.6\textwidth]{workvsforce.png}

For example, if you push a car uphill with a force of 10 newtons for 12
meters, you have done 120 joules of work.\index{work}

\begin{mdframed}[style=important, frametitle={Formula for Work}]
\[
W = F \cdot d
\]

where $W$ is the work in joules, $F$ is the \textit{force} in newtons, and $d$ is the distance in meters.

If the force is not in the same direction as the distance, we can use the cosine of the angle between the force and the distance:
\[
W = F \cdot d \cdot \cos(\theta)
\]

where $\theta$ is the angle between the force and the distance.
\end{mdframed}

The work-energy theorem (or work-energy principle) states that the work done on an object is equal to the \textbf{change in its energy}. In other words, if you do work on an object or give it movement, you change its energy. Most commonly, this is used to relate the work done on an object to its kinetic energy or potential. This is derived from Newton's second law of motion, covered in the previous chapter. % link here to the previous chapter

\[
W = \Delta E = \Delta KE = \Delta PE \text{(with units of Joules (J) or Newton-meters (Nm))} 
\]


Work is how energy is transferred from one thing to another. When you
push the car, you also burn sugars (energy of the body) in your blood. That energy is then
transferred to the car after it has been pushed uphill.

Thus, we measure the energy something consumes or generates in
units of work: joules, kilowatt-hours, horsepower-hours, foot-pounds,
BTUs (British Thermal Unit), and calories.

Let's go over a few different forms that energy can take.

\section{Forms of Energy}\index{energy!Forms of}

In this section we are going to learn about several different types of energy:
\begin{itemize}
\item Heat
\item Electricity
\item Chemical Energy
\item Kinetic Energy
\item Gravitational Potential Energy
\end{itemize}

\subsection{Heat}\index{heat}

When you heat something, you are transferring energy to it. The BTU
 is a common unit for heat. One BTU is the
amount of heat required to raise the temperature of one pound of water
by one degree. One BTU is about 1,055 joules. In fact, when you buy and sell
natural gas as fuel, it is priced by the BTU.\index{heat} \index{BTU}

\subsection{Electricity}\index{electricity}

Electricity is the movement of electrons. When you push electrons
through a space that resists their passage (like a light bulb),
energy is transferred from the power source (like a battery)
 into the source of the resistance.

Let's say your lightbulb consumes 60 watts of electricity, and you leave it on for 24 hours.
We would say that you have consumed 1.44 kilowatt hours, or 3,600,000 joules.


\subsection{Chemical Energy}\index{chemical energy}

As mentioned early, some chemical reactions consume energy and some
produce energy. This means energy can be stored in the structure of a
molecule. When a plant uses photosynthesis to rearrange water and
carbon dioxide into a sugar molecule, it converts the energy in
the sunlight (solar energy) into chemical energy. Remember that photosythesis is a process that releases energy.
Therefore, the sugar molecule has more chemical energy than the carbon dioxide and water molecules that were
used in its creation.
% ADD: photosythesis equation 
% This was included in one of the previous quizzes, but it is worth repeating here.
% C6H12O6 + 6 O2 -> 6 CO2 + 6 H2O + energy
% KA: https://www.khanacademy.org/science/ap-biology/cellular-energetics/photosynthesis/a/intro-to-photosynthesis

In our diet, we measure this energy in \textit{kilocalories}. A
calorie is the energy necessary to raise one gram of water one degree
Celsius, and is about 4.19 joules. This is a very small unit. An apple
has about 100,000 calories (100 kilocalories), so people working with food started
measuring everything in kilocalories.\index{calories}
% ADD: Conversion chapter should come before this chapter

Here is where things get tricky: People who work with food got tired of
saying ``kilocalories'', so they just started using ``Calorie'' to
mean 1,000 calories. This has created a great deal of confusion over the
years. So if the C is capitalized, ``Calorie'' probably means kilocalorie.

\subsection{Kinetic Energy}\index{kinetic energy}

A mass in motion has energy. For example, if you are in a moving car
and you slam on the breaks, the energy from the motion of the
car will be converted into heat in the breaks and under the tires.

How much energy does the car have?
\[
E = \frac{1}{2} m v^2
\]

\begin{mdframed}[style=important, frametitle={Formula for Kinetic Energy}]

\[
E = \frac{1}{2} m v^2
\]

where $E$ is the energy in joules, $m$ is the mass in kilograms, and
$v$ is the speed in meters per second.

\end{mdframed}

\subsection{Gravitational Potential Energy}\index{potential energy!gravitational}


When you lift something heavy onto a shelf, you are giving it
\textit{potential energy}. The amount of energy that you transferred
to it is proportional to its weight and the height that you lifted it.

\[
E = mgh
\]
a rate of $9.8 m/s^2$.

\begin{mdframed}[style=important, frametitle={Formula for Gravitational Potential Energy}]
The formula for gravitational potentional energy is
\[
E = (9.8)mh
\]
\[
E = mgh
\]

where $E$ is the energy in joules, $m$ is the mass of the object you
lifted, $g$ is acceleration due to gravity, and $h$ is the height that you lifted it.

On earth, then, gravitational potential energy is given by

\[
E = (9.8)mh
\]

since objects accelerate at $9.8 m/s^2$.

\end{mdframed}


There are other kinds of potential energy. For example, when you draw
a bow in order to fire an arrow, you have given that bow potential energy. When you release it,
the potential energy is transferred to the arrow, which expresses it
as kinetic energy.
% ADD: section about KE and U

\section{Conservation of Energy}

The first law of thermodynamics says ``Energy is neither created nor
destroyed.''\index{energy!conservation of}

Energy can change forms. Your cells consume chemical energy to give
gravitational potential energy to a car you push up a hill. However, the total amount of
energy in a closed system stays constant.
% ADD: Create Systems chapter before introducing concept here

\begin{Exercise}[title={The Energy of Falling}, label=energy_falling]

A 5 kg cannonball falls off the top of a 3 meter ladder. As it falls,
its gravitational potential energy is converted into kinetic energy.
How fast is the cannonball traveling just before
it hits the floor?

\end{Exercise}
\begin{Answer}[ref=energy_falling]

  At the top of the ladder, the cannonball has $(9.8)(5)(3) = 147$ joules of potential energy.

  At the bottom, the kinetic energy $\frac{1}{2}(5)v^2$ must be equal
  to 147 joules. So $v^2 = \frac{294}{5}$.  This means it is going about
  $7.7$ meters per second.

  (You may be wondering about air resistance. Yes, a tiny amount of energy is lost to air resistance, but for a dense
  object moving at these relatively slow speeds, this energy is
  neglible.)

\end{Answer}


\section{Efficiency}


% Depending on how important this is, we may want to expand this section or give it its own chapter. - Arjan
Although energy is always conserved as it moves through different
forms, scientists aren't always that good at controlling it.\index{efficiency}

In terms of an equation, efficiency is the ratio of the useful energy
output to the total energy input. It is usually expressed as a
percentage.
\begin{mdframed}[style=important, frametitle={Formula for Efficiency}]
\[
\text{Efficiency} = \frac{\text{Useful Energy Output}}{\text{Total Energy Input}} \times 100\%
\]
\end{mdframed}
where the useful energy output is the energy that is actually used to do work or complete a task, and the total energy input is the total energy consumed by the system.

A machine is considered 100\% efficient only if all the input work is converted into useful output work, with no energy lost to heat, friction, or sound.

For example, when a car engine consumes the chemical energy in gasoline, only
about 20\% of the energy consumed is used to turn the wheels. Most of
the energy is actually lost as heat. If you run a car for a while, the engine
gets very hot, as does the exhaust coming from the tailpipe.

A human is about 25\% efficient. Most of the loss is in the heat produced
during the chemical reactions that turns food into motion.
% ADD: Cellular Respiration

In general, if you are trying to increase efficiency in any system,
the solution is usually easy to identify by the heat that is produced. Reduce the heat, increase the efficiency.

Light bulbs are an interesting case. To get the light of a 60 watt
incandescent bulb, you can use an 8 watt LED or a 16 watt fluorescent
light. This is why we say that the LED light is much more efficient. If you
run both, the incandescent bulb will consume 1.44 kilowatt-hours; the
LED will consume only 0.192 kilowatt-hours.

In addition to light, the incandescent bulb is producing a lot of heat. If it
is inside your house, what happens to the heat? It warms your house.

In the winter, when you want light and heat, the incandescent bulb is
100\% efficient!

Of course, this also means the reverse is true. In the summer, if you are running the air conditioner to cool down your house, the
incandescent bulb is worse than just ``inefficient at making light'' ---
it is actually counteracting the air conditioner!
