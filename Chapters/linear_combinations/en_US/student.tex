\chapter{Linear Combinations}

A linear combination of vectors is the addition of two or more scaled vectors. For example, given two vectors, ${v}_1, {v}_2$ and two scalars $a_1,a_2$, you'd write their linear combination as:

\[x
\mathbf{w} = a_1\mathbf{v}_1 + a_2\mathbf{v}_2
\]

The scalars can be any real number. The vectors can be of any dimension. 

Let's take a more generalized approach. Given vectors $\mathbf{v}_1,
\mathbf{v}_2, ..., \mathbf{v}_n \in \mathbb{R}^m$ and scalars $a_1,
a_2, ..., a_n \in \mathbb{R}$, a linear combination of these vectors
is any vector of the form\index{linear combinations}

\[
\mathbf{w} = a_1\mathbf{v}_1 + a_2\mathbf{v}_2 + ... + a_n\mathbf{v}_n
\]

Each scalar $a_i$ scales the corresponding vector $\mathbf{v}_i$, and
added together, the results are produce a new vector $\mathbf{w}$.

Let's look at an example with 4 vectors their scalars. 
\begin{itemize}
\item $a_1 = 1, v_1 = [9, 1, 2]$
\item $a_2 = -1, v_2 = [8, -3, 4]$
\item $a_3 = 3, v_3 = [6, 0, 1]$
\item $a_4 = -4, v_4 = [3, 7, 2]$
\end{itemize}
As a linear combination:
\[
\mathbf{w} = 1*[9,1,2] + (-1)*[8, -3, 4]+ 3*[6,0,1] + (-4)*[3,7,2]
\]
After multiplying each vector by its associated scalar.
 \[
\mathbf{w} = [9, 1, 2] + [-8, 3, -4] + [18, 0, 3] + [-12, -28, -8]
\]
When combined:
\[
\mathbf{w} = [7, -24, -7]
\]

\begin{Exercise}[title={Linear Combination}, label=linearCombo]
Calculate the linear combination 
for vectors $v_1, v_2, v_3$ and
scalars $a_1, a_2,a_3$ where:   
	\begin{itemize}
      \item a1 = 2, v1 =[2, 4, 8]
      \item a2 = -2,v2 =[8, -6, 3]
      \item a3 = 4,v3 =[7, 9, 2]
	\end{itemize}
Make sure to show all your work. 
\end{Exercise}

\begin{Answer}[ref=linearCombo]
     \[
	\mathbf{w} = 2*[2,4,8] + (-2)*[8,-6,3] + 4*[7,9,2] 
	\]
  	\[
	\mathbf{w} = [4,8,16] + [-16,12,-6] + [28,36,8] 
	\]  
 	\[
	\mathbf{w} = [16, 56, 18]  
	\]   
\end{Answer}

\section{Weighted Averages of Vectors}
A weighted average of vectors is a specific type of linear combination
where the coefficients (or weights) $a_i$ are non-negative and sum to
1:\index{weighted averages}
\[
\sum_{i=1}^{n} a_i = 1, \quad a_i \geq 0
\]

A weighted average of vectors $\mathbf{v}_1, \mathbf{v}_2, ...,
\mathbf{v}_n$ is then defined as

\[
\mathbf{w} = a_1\mathbf{v}_1 + a_2\mathbf{v}_2 + ... + a_n\mathbf{v}_n
\]

In this case, each $a_i$ not only scales the corresponding vector
$\mathbf{v}_i$, but also represents the proportion of that vector in
the final average vector $\mathbf{w}$.

Weighted averages are useful when you want to attribute the contribution of one feature or item over another. For example, a teacher might figure a student's final grade using exam scores, class participation, and a final project. The exam scores might make up 65\% of the final grade, class participation 10\%, and a final project 25\%. Thus giving the formula for a grade as:

\[
\mathbf{Grade} = .65*ExamScores + .10*Participation + .25*FinalProject
\]

The teacher defines the weights, making sure they sum to 1.0. 

Let's look at an example where the weights don't sum to 1.0. A store that sells umbrellas might have to get the umbrella stock from three different manufacturers. The store owner buys 100 umbrellas at a cost of \$2.10 each, 50 umbrellas cost \$1.85 each, and 200 umbrellas cost \$2.00. 

\[
\mathbf{TotalCost} = 2.10*100 + 1.85*50 + 2.00*200 = 702.5
\]

To calculate the weighted average, divide the total cost by the number of items.

\[
\mathbf{WeightedAverage} = 702.5/350 = 2.01 
\]

\begin{Exercise}[title={Weighted Average}, label=weightedAverage]
  A concert sells 300 tickets in the balcony at \$50 each, 100 tickets on the main floor at \$75 each, and 50 tickets in the section closest to the stage at \$150 each. What's the weighted average?
\end{Exercise}
\begin{Answer}[ref=weightedAverage]
 \begin{itemize}
    \item\[\mathbf{TotalSales} = 50*300 + 75*100 + 150*50 = 30,000\]
	\item\[\mathbf{NumberTickets} = 300 + 100 + 50 = 450\]
	\item\[\mathbf{WeightedAverage} = 30,000/450 = 66.67\]
 \end{itemize} 
\end{Answer}

\section{Weighted Averages of Vectors in Python}

Create a file called \filename{linearCombos.py} and enter this code:

\begin{Verbatim}
// import the python module that supports matrices
import numpy as np

// an array for number of umbrellas by manufacturer
items = np.array([100, 50, 200])

// weights are the cost of item by manufacturer
weights = np.array([2.10, 1.85, 2.00])

// create an array for total cost for each manufacturer
costPerManufacturer=items * weights

// sum the individuals costs to get the total
totalCost = np.sum(costPerManufacturer)

// get number of items
numItems = np.sum(items) 

// you are ready to calculated the weighted average
weightedAverage = totalCost/numItems
print(weightedAverage)

\end{Verbatim}

When you run this code, you should get a weighted average of \$2.01 when rounded to the nearest cent.
