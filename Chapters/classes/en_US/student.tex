\chapter{Python Classes}
FIXME integrate this better with polynomials, but we need to cover the following
\section{Object-Oriented Programming Introduction}
Imagine you want to implement multiple dogs in python. It gets a bit complicated to do the following:

\begin{verbatim}
    dog1name = "Teddy"
    dog1age = 6
    dog1sound = "woof"

    dog2name = "Fluffers"
    dog2age = 2
    dog2sound = "bark"

    dog3name = "Bella"
    dog3age = 3
    dog3sound = "grr"

    print(f"{dog1name} is {dog1age} and says {dog1sound}!")
    print(f"{dog2name} is {dog2age} and says {dog2sound}!")
    print(f"{dog3name} is {dog3age} and says {dog3sound}!")
    

\end{verbatim}

Instead, we can use classes, which are a way to create your own datatype.\index{classes}
Classes can contain custom methods that either return, print, or calculate different values for you, 
and they con contain custom variables referred to as attributes.\index{attributes}

\begin{verbatim}
class Dog:
    """A simple model of a dog."""
    # constructor - which creates a new model of a dog. 
    def __init__(self, name: str, age: int, sound: str):
        self.name = name
        self.age = age
        self.sound = sound
    # method speak 
    def speak(self) -> None:
        """Print a sentence describing the dog."""
        print(f"{self.name} is {self.age} and says {self.sound}!")

# create dog objects called “instances” with the given variables
dog1 = Dog("Teddy", 6, "woof")
dog2 = Dog("Fluffers", 2, "bark")
dog3 = Dog("Bella", 3, "grr")

# call the behavior on each object
dog1.speak()
dog2.speak()
dog3.speak()

\end{verbatim}

Now lets talk about classes in context of polynomials!

The built-in types, such as strings, have functions associated with
them. So, for example, if you needed a string converted to uppercase,
you would call its \pyfunction{upper()} function:
-
\begin{Verbatim}
my_string = "houston, we have a problem!"
louder_string = my_string.upper()
\end{Verbatim}
This would set \pyvar{louder\_string} to "HOUSTON, WE HAVE A PROBLEM!"
When a function is associated with a datatype like this, it called a
\emph{method}. A datatype with methods is known as a \emph{class}. Creating a new version of a class in a variable is called an \emph{instance}.\index{instance} For
example, in the example, we would say ``\pyvar{my\_string} is an instance of
the class \pytype{str}. \pytype{str} has a method called \pyfunction{upper}''

The function \pyfunction{type} will tell you the type of any data:
\begin{Verbatim}
  print(type(my_string))
\end{Verbatim}
This will output
\begin{Verbatim}
<class 'str'>
\end{Verbatim}

A class can also define operators.  \pyfunction{+}, for example, is
redefined by \pytype{str} to concatenate strings together:
\begin{Verbatim}
long_string = "I saw " + "15 people"
\end{Verbatim}

\section{Parent classes}
\index{classes!parent}
\index{classes!child}
\index{subclass}
Classes can have classes they inherit from (called ``parent classes'') or classes that 
inherit from them (called ``child classes''). 
Parent classes (ie `Animal') give a subclass (or child) different attributes or methods. Lets check out an example:

\begin{verbatim}
class Animal:
    """Generic animal base-class."""

    def __init__(self, name: str, age: int) -> None:
        self.name = name
        self.age = age

    def describe(self) -> None:
        """Print a basic description common to all animals."""
        print(f"{self.name} is {self.age} years old.")

class Dog(Animal):
    """Dog inherits name and age from Animal, adds its own sound."""

    def __init__(self, name: str, age: int, sound: str) -> None:
        super().__init__(name, age)     # initialize the Animal part
        self.sound = sound

    def speak(self) -> None:
        """Dog-specific implementation of speak()."""
        print(f"{self.name} is {self.age} and says {self.sound}!")


# demonstration
dogs = [
    Dog("Teddy", 6, "woof"),
    Dog("Fluffers", 2, "bark"),
    Dog("Bella", 3, "grr")
]

for dog in dogs:
    dog.describe()   # common behavior from Animal
    dog.speak()      # overridden behavior in Dog

\end{verbatim}

Here, we have made a parent class for `dog' called `animal'. 
\section{Making a Polynomial class}

You have created a bunch of useful python functions for dealing with
polynomials. Notice how each one has the word ``polynomial'' in the
function name like \pyfunction{derivative\_of\_polynomial}.  Wouldn't it
be more elegant if you had a Polynomial class with a
\pyfunction{derivative} method? Then you could use your polynomial
like this:\index{class in python}
\begin{Verbatim}
a = Polynomial([9.0, 0.0, 2.3])
b = Polynomial([-2.0, 4.5, 0.0, 2.1])

print(a, "plus", b , "is", a+b)
print(a, "times", b , "is", a*b)
print(a, "times", 3 , "is", a*3)
print(a, "minus", b , "is", a-b)

c = b.derivative()

print("Derivative of", b ,"is", c)
\end{Verbatim}

And it would output:
\begin{Verbatim}
2.30x^2 + 9.00 plus 2.10x^3 + 4.50x + -2.00 is 2.10x^3 + 2.30x^2 + 4.50x + 7.00
2.30x^2 + 9.00 times 2.10x^3 + 4.50x + -2.00 is 4.83x^5 + 29.25x^3 + -4.60x^2 + 40.50x + -18.00
2.30x^2 + 9.00 times 3 is 6.90x^2 + 27.00
2.30x^2 + 9.00 minus 2.10x^3 + 4.50x + -2.00 is -2.10x^3 + 2.30x^2 + -4.50x + 11.00
Derivative of 2.10x^3 + 4.50x + -2.00 is 6.30x^2 + 4.50  
\end{Verbatim}

Create a file for your class definition called \filename{Polynomial.py}. Enter the following:
\begin{Verbatim}
class Polynomial:
    def __init__(self, coeffs):
        self.coefficients = coeffs.copy()

    def __repr__(self):
        # Make a list of the monomial strings
        monomial_strings = []

        # For standard form we start at the largest degree
        degree = len(self.coefficients) - 1

        # Go through the list backwards
        while degree >= 0:
            coefficient = self.coefficients[degree]

            if coefficient != 0.0:
                # Describe the monomial
                if degree == 0:
                    monomial_string = "{:.2f}".format(coefficient)
                elif degree == 1:
                    monomial_string = "{:.2f}x".format(coefficient)
                else:
                    monomial_string = "{:.2f}x^{}".format(coefficient, degree)
                
                # Add it to the list
                monomial_strings.append(monomial_string)
        
            # Move to the previous term
            degree = degree - 1

        # Deal with the zero polynomial
        if len(monomial_strings) == 0:
            monomial_strings.append("0.0")
    
        # Separate the terms with a plus sign
        return " + ".join(monomial_strings)

    def __call__(self, x):
        sum = 0.0  
        for degree, coefficient in enumerate(self.coefficients):
            sum = sum + coefficient * x ** degree
        return sum

    def __add__(self, b):
        result_length = max(len(self.coefficients), len(b.coefficients))
        result = []
        for i in range(result_length):
            if i < len(self.coefficients):
                coefficient_a = self.coefficients[i]
            else:
                coefficient_a = 0.0

            if i < len(b.coefficients):
                coefficient_b = b.coefficients[i]
            else:
                coefficient_b = 0.0
            result.append(coefficient_a + coefficient_b)
            
        return Polynomial(result)

    def __mul__(self, other):

        # Not a polynomial?
        if not isinstance(other, Polynomial):
            # Try to make it a constant polynomial
            other = Polynomial([other])
        
        # What is the degree of the resulting polynomial?
        result_degree = (len(self.coefficients) - 1) + (len(other.coefficients) - 1)

        # Make a list of zeros to hold the coefficents
        result = [0.0] * (result_degree + 1)

        # Iterate over the indices and values of a
        for a_degree, a_coefficient in enumerate(self.coefficients):

            # Iterate over the indices and values of b
            for b_degree, b_coefficient in enumerate(other.coefficients):

                # Calculate the resulting monomial
                coefficient = a_coefficient * b_coefficient
                degree = a_degree + b_degree
            
                # Add it to the right bucket
                result[degree] = result[degree] + coefficient
            
        return Polynomial(result)

    __rmul__ = __mul__

    def __sub__(self, other):
        return self + other * -1.0
    
    def derivative(self):

        # What is the degree of the resulting polynomial?
        original_degree = len(self.coefficients) - 1
        if original_degree > 0:
            degree_of_derivative = original_degree - 1
        else:
            degree_of_derivative = 0

        # We can ignore the constant term (skip the first coefficient)
        current_degree = 1
        result = []

        # Differentiate each monomial
        while current_degree < len(self.coefficients):
            coefficient = self.coefficients[current_degree]
            result.append(coefficient * current_degree)
            current_degree = current_degree + 1

        # No terms? Make it the zero polynomial
        if len(result) == 0:
            result.append(0.0)

        return Polynomial(result)
\end{Verbatim}

Create a second file called \filename{test\_polynomial.py} to test it:
\begin{Verbatim}[commandchars=\\\{\}]
from Polynomial import Polynomial

a = Polynomial([9.0, 0.0, 2.3])
b = Polynomial([-2.0, 4.5, 0.0, 2.1])

print(a, "plus", b , "is", a+b)
print(a, "times", b , "is", a*b)
print(a, "times", 3 , "is", a*3)
print(a, "minus", b , "is", a-b)

c = b.derivative()

print("Derivative of", b ,"is", c)

slope = c(3)
print("Value of the derivative at 3 is", slope)

\end{Verbatim}

Run the test code:
\begin{Verbatim}
python3 test_polynomial.py
\end{Verbatim}


