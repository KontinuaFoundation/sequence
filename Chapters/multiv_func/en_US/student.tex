\chapter{Multivariate Functions}

A real-valued multivariate function is a function that takes multiple
real variables as input and produces a single real output.

We generally denote such a function as $f: \mathbb{R}^n \rightarrow
\mathbb{R}$, where $\mathbb{R}^n$ is the domain and $\mathbb{R}$ is
the co-domain.

For example, consider a function $f$ that takes two variables, $x$ and
$y$:

\begin{equation*}
f(x, y) = x^2 + y^2
\end{equation*}

Here, $f: \mathbb{R}^2 \rightarrow \mathbb{R}$ takes an ordered pair
$(x, y)$ from the 2-dimensional real coordinate space, squares each,
and adds them to produce a real number.

In a similar way, a function $g: \mathbb{R}^3 \rightarrow \mathbb{R}$
could take three variables, $x$, $y$, and $z$, and might be defined as:

\begin{equation*}
g(x, y,z) = x^2 + y^2 + z^2
\end{equation*}

Here, the function squares each of the input variables, then adds
them to produce a real number.

These functions are "real-valued" because their outputs are real
numbers, and "multivariate" because they take multiple variables as
inputs.

The concepts of limits, continuity, differentiability, and
integrability can all be extended to multivariate functions, although
they become more complex because we now have to consider different
directions in which we approach a point, not just from the left or
right, as in the univariate case. For example, the partial derivative
is the derivative of the function with respect to one variable,
holding the others constant. It is one of the basic concepts in the
calculus of multivariate functions.

For example, given the function $f(x, y) = x^2 + y^2$, the partial
derivatives of $f$ are computed as:

\begin{equation*}
\frac{\partial f}{\partial x}(x, y) = 2x
\end{equation*}

\begin{equation*}
\frac{\partial f}{\partial y}(x, y) = 2y
\end{equation*}

