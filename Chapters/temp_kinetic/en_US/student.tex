\chapter{Kinetic Energy and Temperature of a Gas}

As mentioned in the previous chapter,   for a particular gas,  the temperature (in Kelvin) is proportional to the average kinetic energy of the individual molecules. 

Perhaps you want to warm 3 moles of helium gas (trapped in a metal cylinder) from 10 degrees Celsius to 30 degrees Celsius.   
How would you compute exactly how many Joules of energy this would require?

The amount of energy necessary to raise one mole of a molecule by one degree is known as \newterm{molar heat capacity}.  
(The molar heat capacity of liquid water, for example, is 75.38 J per mole-degree.)

With gases, are actually two different possible situations:
\begin{enumerate}
\item Constant volume: As you heat the gas,  the pressure and the temperature increase.  This molar heat capacity is usually denoted as $C_{V,m}$.
\item Constant pressure: As you heat the gas,  the temperature and the volume increase.  This molar heat capacity  is usually denoted as $C_{P, m}$.
\end{enumerate}

All gases made up of one atom (Helium, for example, is a monoatomic gas.) have the same values for $C_{V,m}$ and $C_{P,m}$:

$C_{V,m} = \frac{3}{2}R \approx 12.47 \text{ Joules per mole-degree}$

$C_{P,m} = \frac{5}{2}R \approx 20.8  \text{ Joules per mole-degree}$

(Remember from last chapter that $R$ is the ideal gas constant $\approx 8.31446$ Joules per mole-degree.)

\begin{Exercise}[title={Warming Helium},  label=warming_helium]

You have 3 moles of helium.  
  
\begin{enumerate}

\item How many Joules would be required to warm 3 moles of helium gas by 20 degrees Celsius at constant volume?  

\item How many Joules would be required to warm 3 moles of helium gas by 20 degrees Celsius at constant pressure?  

\end{enumerate}


\end{Exercise}
\begin{Answer}[ref=warming_helium]

$E = C_{V,m} (3 \text{ moles }) (20 { degrees Celsius }) = (12.47)(3)(20) = 748 \text{ Joules }$

$E = C_{V,m} (3 \text{ moles }) (20 { degrees Celsius }) = (20.8)(3)(20) = 1247 \text{ Joules }$
\end{Answer}

\section{Molecule Shape and Molar Heat Capacity}

We told you that gases made up of one atom have the same values for $C_{V,m}$ and $C_{P,m}$:

$C_{V,m} = \frac{3}{2}R \approx 12.47 \text{ Joules per mole-degree}$

$C_{P,m} = \frac{5}{2}R \approx 20.8  \text{ Joules per mole-degree}$

For any molecule, it is generally true that 

$$C_{P,m} \approx C_{V,m} + R$$

It is also true that for any molecule,  there is some integer $d$ such that

$$C_{V, m} \approx \frac{d}{2}R$$

For example, for all monoatomic gases,  $d = 3$.  For diatomic gases (like $N_2$ and $O_2$,  $d$ is 5.

$d$ is known as the \newterm{degree of freedom} of the molecule.  When you study chemistry, they will teach you to predict $d$ based on the shape of the molecule. 

Here are the relevant numbers for some gases you are likely to work with:

\begin{tabular}{r|c|c| c| c}
Gas & type & $C_{V,m}$ & $C_{P,m}$ & $d$\\
\hline
$He$ &  monoatomic & 12.4717 & 20.7862 & 3 \\
$Ar$ & monoatomic & 12.4717 & 20.7862 & 3 \\
$O_2$ & diatomic & 21.0 & 29.38 & 5\\
$N_2$ & diatomic & 20.8 & 29.12 & 5\\
$HO_2$ (water vapor) & 3 atoms &  28.03 & 37.47 & 7 \\
$CO_2$ & 3 atoms & 28.46 & 36.94 & 7\\
\end{tabular}

\section{Kinetic Energy and Temperature}

For a sample of a gas, we can calculate its kinetic energy based on its molar heat capacity,  the number of molecules, and the temperature:

$$E_K = C_{V,m} n T$$

where

\begin{itemize}
\item $E_K$ is the kinetic energy in Joules
\item $C_{V,m}$ is the molar heat capacity of the gas at constant volume
\item $n$ is the number of molecules in moles
\item $T$ is the temperature in Kelvin
\end{itemize}


\begin{Exercise}[title={Warming Helium Revisited},  label=warming_helium2]

How much kinetic energy does 3 moles of helium have at 10 degrees Celsius?

How much kinetic energy does 3 moles of helium have at 30 degrees Celsius?

What is the difference?

\end{Exercise}
\begin{Answer}[ref=warming_helium2]

10 degrees Celsius is 283.15 degrees Kelvin.  30 degrees Celsius is 303.15.

For any gas:

$$E_K =C_{V,m} n T$$

And $C_{V,m} = 12.47$ for all monoatomic gases.

So the energy at 10 degrees Celsius:

$$E_1 = (12.47)(3)(283.15) = 10,594 \text{ Joules}$$

The energy at 30 degrees Celsius:

$$E_2 = (12. 47)(3)(303.15) = 11,342 \text{ Joules}$$

The difference?

$$E_2 - E_1 = 11,342 - 10,594   = 748 \text{ Joules }$$

Which is consistent with your earlier exercise.

\end{Answer}

\section{Why is $C_{V,m}$ different from $C_{P,m}$?}

What if, instead of keeping the volume constant while we heat the molecules in the helium tank,  we keep the pressure constant and let the gas expand?  
The change in kinetic energy is the same: 748 Joules.

However,  we know that the molar heat capacity if we keep pressure constant is $\frac{5}{2}R$,  so  heating will require $\frac{5}{2}R(3)(20) = 1247$ Joules.  

What happened to the 499 missing Joules!?  Thermodynamics tells us energy is neither created nor destroyed.  So it must have gone somewhere.

That energy was used pushing against the pressure as the gas expanded.  For example,  maybe the sample was in a balloon in space -- the extra energy stretched the surface of the balloon.  

The 499 Joules were converted into potential energy.   

\section{Work of Creating Volume Against Constant Pressure}

Let's imagine that you had a total vacuum (zero pressure) with a piston.  As you pulled the piston out,  you would be pulling against  the atmospheric pressure.  How much energy would that require?

If you increased the volume of the vacuum by $V$ against a pressure of $P$,   you would do $VP$ work.

Let's check to make sure the 499 Joules mentioned above makes sense with this in mind.

No initial pressure was given in the problem, so let's just make one up and see how things work out: 100 kPa.  Using the ideal gas law,  the initial volume would be:

$$V_1 = \frac{n R T}{P} = \frac{(3)(8.31446)(283.15)}{100,000} = 0.07063\text{ cubic meters}$$

The volume after we heated the gas and let it expand against 100 kPa would be:

$$V_2 = \frac{n R T}{P} = \frac{(3)(8.31446)(303.15)}{100,000} = 0.07562\text{ cubic meters}$$

So the volume increased by $0.07562 - 0.07063 = 0.00499$ cubic meters.   
Multiplying that by 100,000 pa, we get 499 Joules as we expected!

\section{Why does a gas get hotter when you compress it?}

Sticking with this example, let's say that you heated the 3 moles of helium at constant pressure with 1,247 Joules.  Once it reached $30^\circ$ Celsius, you stopped heating it, and started to increase pressure on it until it was its original volume.   What is going to happen?

The extra 499 Joules are going to be converted into kinetic energy.  If you don't let the heat escape, the gas is going to get even hotter!  How much hotter?

The result will be exactly the same is if you never let the volume increase.  Thus the change in temperature (from the original $283.15^\circ$ Kelvin) will be given by:

$$1,247 = C_{V,m} n \Delta T$$

so 

$$\Delta T = \frac{1,247}{(12.47)(3)} = 33.33^\circ \text{ Kelvin}$$

Thus, the new temperature of the gas would be $283.15.15 + 33.33 = 316.48^\circ$ Kelvin.  Significantly warmer than before you compressed it.

When you compress any gas it gets hotter.  You can experience this when you pump up a bike tire.  If you increase the pressure rapidly,  the air inside the tire will get noticeable hotter.

And the reverse is true: If you decompress a gas,  it will get colder.  If you let the air out of your bike tire,  it will be noticeable colder as it comes out.

If you compress or decompress a gas without letting any heat enter or depart,  we say the compression or decompression was \newterm{adiabatic} (pronounced "ay-die-uh-bat-ick").

\section{How much hotter?}

In the example above,   you could measure the work done when you compressed the gas back to its original volume.  And that would tell you how much kinetic energy had to be absorbed by the gas.   Here's is what makes that a little tricky:  As you compress the gas,  the pressure you are fighting goes up with the decrease in volume \emph{and} the increase in temperature.

Fortunately,  there is two handy rules:

\begin{mdframed}[style=important, frametitle={Adiabatic Compression and Decompression}]

Let 

$$\gamma = \frac{C_{P,m}}{C_{V,m}}$$

In an adiabatic compression or decompression,  $P$ and $V$ change,  but

$$P \left(V^\gamma \right)$$

stays constant.

Also 

$$T \left(V^{\left( \gamma - 1 \right)} \right)$$

stays constant

\end{mdframed}

Let's say you have a accordion-like container filled with helium at 100 kPa (about 1 atmosphere) and 300 degrees Kelvin.  It holds 2 cubic meters.  
How many molecules are in the container?

$$n = \frac{PV}{RT} = \frac{(100000)(2)}{(8.31446)(300)} = 80.2 \text{ moles}$$

Now you tighten a vise around the accordion until the 80.2 moles of helium inside has been crushed into 0.5 cubic meters,  without letting any heat escape.

How can we predict the new temperature of the gas inside?

$$\gamma - 1 = \frac{C_{P,m}}{C_{V,m}} - 1= \frac{2}{3}$$

Before the compression

$$T \left(V^{\left( \gamma - 1 \right)} \right) = 300 \left( 2^{0.6667} \right) = 476.22$$

After the compression

$$T \left(V^{\left( \gamma - 1 \right)} \right) = T \left( 0.5^{0.6667} \right) = 476.22$$

Thus 

$$T = 756^\circ \text{ Kelvin}$$

That's hot!  As you let it cool back down to 300 degrees Kelvin,  how much heat would be released?  

$$E = C_{V,m} n \Delta T = (12.47)(80.2)(476.22 - 300) =  455,953 \text{ Joules}$$

\section{How an Air Conditioner Works}

Once again, imagine the accordion-like container filled with helium.  Let's say you walked it outside and compressed it from 2 cubic meters to 0.5 cubic meters in a vise.  The container would get to 476 degrees Kelvin.  You keep it compressed, in the vise  but let it cool down outside.  When it gets back to 300 degrees Kelvin,  you walk it back inside.

Now,  without letting any molecules in or out of the container,  you release the vise.  The gas is decompressed and gets very cold -- how cold?  Cold enough to accept 455,953 Joules of kinetic energy from your house.  That is,  it would absorb heat from your house.

Now you walk outside with your accordion and your vise and repeat:
\begin{enumerate}
\item Compress the gas outside.
\item Let the hot gas cool down outside.
\item Walk the room-temperature compressed gas inside.
\item Decompress the gas inside.
\item Let the cold gas warm up inside.
\end{enumerate}

You could keep your house cool on a hot day this way.  And this is not unlike how an air conditioner works.

There is a hose filled with refrigerant that does a loop:  
\begin{itemize}
\item Outside,  the refrigerant is compressed and allowed to cool to the outside temperature.  (Usually there is a big fan blowing on a coil of refrigerant to speed the process.)
Inside,  the refrigerant is decompressed and allowed to warm to the inside temperature.  (Usually there is a big fan blowing the air of the home past a coil of refrigerant to speed the process.)
\end{itemize}

In each pass of the loop,  the refrigerant absorbs some of the kinetic energy from inside the house, and releases it on the outside.

This same mechanism can be used to heat your house.  (Units that both heat and cool are known as \newterm{heat pumps}.)  
The heat pump does the process backwards:  The hot compressed refrigerant cools down inside.  The cold decompressed refrigerant warms up outside.

