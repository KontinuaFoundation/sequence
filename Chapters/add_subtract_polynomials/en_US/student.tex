\chapter{Adding and Subtracting Polynomials}

Watch Khan Academy's \textbf{Adding polynomias} video at \url{https://youtu.be/ahdKdxsTj8E}

When adding two monomials of the same degree, you sum their coefficients:
\begin{equation*}
  7x^3 + 4x^3 = 11x^3
\end{equation*}\index{monomials!adding}

Using this idea, when adding two polynomials, you convert them into one long
polynomial, then simplify by combining terms with the same degree. For example:
\begin{multline*}
  (10x^3 - 2x + 13) + (-5x^2 + 7x -12) \\
  = 10x^3 + (-2)x + 13 + (-5)x^2 + 7x + (-12) \\
  = 10x^3 + (-5)x^2 + (-2 + 7)x + (13 - 12) \\
  = 10x^3 - 5x^2 + 5x + 1
\end{multline*}\index{polynomials!adding}

\begin{Exercise}[title=Adding Polynomials Practice, label=addpns]
  Add the following polynomials:
  \Question{$2x^3 - 5x^2 + 3x - 9$ and $x^3 - 2x^2 - 2x - 9$}
  \vspace{20mm}
  \Question{$3x^5 - 5x^3 + 3x^2 - x - 3$ and $2x^4 - 2x^3 - 2x^2 + x - 9$}
  \vspace{20mm}
\end{Exercise}
\begin{Answer}[ref=addpns]$3x^3 - 7x^2 + x - 18$ and $3x^5 - 7x^3 + x^2 - 12$\end{Answer}

Notice that in the second question, the degree 1 term disappears completely: $(-x) + x = 0$

One more tricky thing can happen --- sometimes the coefficients don't add nicely.  For example:
\begin{equation*}
  \pi x^2 - 3 x^2 = (\pi - 3) x^2
\end{equation*}
That is as far as you can simplify it.
    
\section{Subtraction}
\index{polynomials!subtracting}
Now, watch Khan Academy's \textbf{Subtracting polynomials} at \url{https://youtu.be/5ZdxnFspyP8}.

When subtracting one polynomial from the other, it is a lot like
adding two polynomials. The difference: when make the two polynomials
into one long polynomial, we multiply each monomial that is being
subtracted by -1. For example:
\begin{multline*}
  (2x^2 - 3x + 9) - (5x^2 - 7x + 4) \\
  = 2x^2 + (-3)x + 9 + (-5)x^2 + 7x + (-4) \\
  = (2 - 5)x^2 + (-3 + 7)x + (9-4) \\
  = -3x^2 + 4x + 5
\end{multline*}

\begin{Exercise}[title=Subtracting Polynomials Practice, label=subtractpns]
  Add the following polynomials:
  \Question{$(2x^3 - 5x^2 + 3x - 9) - (x^3 - 2x^2 - 2x - 9)$}
  \vspace{20mm}
  \Question{$(3x^5 - 5x^3 + 3x^2 - x - 3) - (2x^4 - 2x^3 - 2x^2 + x - 9)$}
  \vspace{20mm}
\end{Exercise}
\begin{Answer}[ref=subtractpns]$x^3 - 3x^2 + 5x$ and $x^5 - 3x^3 + 5x^2 - 2x + 6$\end{Answer}

\section{Adding Polynomials in Python}

As a reminder, in our Python code, we are representing a polynomial
with a list of coefficients.  The first coefficient is the constant
term. The last coefficient is the leading coefficient. So, we can
imagine $-5x^3 + 3x^2 - 4x + 9$ and $2x^3 +4x^2 - 9$ would look
like this: \textit{FIXME: Diagram here}

To add the two polynomials then, we sum the coefficients for each degree.
\textit{FIXME: Diagram here}

Create a file called \filename{add\_polynomials.py}, and type in the following: 
\begin{Verbatim}
def add_polynomials(a, b):
    degree_of_result = len(a)
    result = []
    for i in range(degree_of_result):
        coefficient_a = a[i]
        coefficient_b = b[i]
        result.append(coefficient_a + coefficient_b)
    return result

polynomial1 = [9.0, -4.0, 3.0, -5.0]
polynomial2 = [-9.0, 0.0, 4.0, 2.0]
polynomial3 = add_polynomials(polynomial1, polynomial2)

print('Sum =', polynomial3)
\end{Verbatim}

Run the program.

Unfortunately, this code only works if the polynomails are the same length. For
example, try making \pyvar{polynomial1} have a larger degree than
\pyvar{polynomial2}:
\begin{Verbatim}
# x**4 - 5x**3 + 3x**2 - 4x + 9
polynomial1 = [9.0, -4.0, 3.0, -5.0, 1.0]
  
# 2x**3 + 4x**2 - 9  
polynomial2 = [-9.0, 0.0, 4.0, 2.0]
polynomial3 = add_polynomials(polynomial1, polynomial2)
print('Sum =', polynomial3)
\end{Verbatim}

See the problem?

\begin{Exercise}[title=Dealing with polynomials of different degrees, label=pyaddpolys]
  
Can you fix the function \pyfunction{add\_polynomials} to handle polynomials of different degrees?

Here is a hint: In Python, there is a \pyfunction{max} function that returns the largest of the numbers it is passed.
\begin{Verbatim}
biggest = max(5,7)
\end{Verbatim}
Here \pyvar{biggest} would be set to 7.

Here is another hint: If you have an array \pyvar{mylist}, \pyvar{i},
a non-negative integer, is only a legit index if \texttt{i <
  len(mylist)}.
\end{Exercise}
\begin{Answer}[ref=pyaddpolys]
\begin{Verbatim}
def add_polynomials(a, b):
    degree_of_result = max(len(a), len(b))
    result = []
    for i in range(degree_of_result):
        if i < len(a):
            coefficient_a = a[i]
        else:
            coefficient_a = 0.0   

        if i < len(b):
            coefficient_b = b[i]
        else:
            coefficient_b = 0.0
            
        result.append(coefficient_a + coefficient_b)
    return result
\end{Verbatim}
\end{Answer}

\section{Scalar multiplication of  polynomials}

If you multiply a polynomial with a number, the distributive property applies:
\begin{equation*}
  (3.1)(2x^2 + 3x + 1) = (6.2)x^2 + (9.3)x + 3.1
\end{equation*}
(When we are talking about things that are more complicated than a number, we use the word \emph{scalar} to mean ``Just a number''. This is the product of a scalar and a polynomial.)

In \filename{add\_polynomials.py}, add a function to that multiplies a scalar and a polynomial:
\begin{Verbatim}
def scalar_polynomial_multiply(s, pn):
    result = []
    for coefficient in pn:
        result.append(s * coefficient)
    return result
\end{Verbatim}

Somewhere near the end of the program, test this function:
\begin{Verbatim}
polynomial4 = scalar_polynomial_multiply(5.0, polynomial1)
print('Scalar product =', polynomial_to_string(polynomial4))
\end{Verbatim}

\begin{Exercise}[title=Subtract polynomials in Python, label=pysubpoly]
Now, implement a function that does subtraction using
\pyfunction{scalar\_polynomial\_multiply} and
\pyfunction{add\_polynomials}.

It should look like this:
\begin{Verbatim}
def subtract_polynomial(a, b):
    ...Your code here...

polynomial5 = [9.0, -4.0, 3.0, -5.0]
polynomial6 = [-9.0, 0.0, 4.0, 2.0, 1.0]
polynomial7 = subtract_polynomial(polynomial5, polynomial6)
print('Difference =', polynomial_to_string(polynomial7))
\end{Verbatim}
\end{Exercise}
\begin{Answer}[ref=pysubpoly]
\begin{Verbatim}
def subtract_polynomial(a, b):
    neg_b = scalar_polynomial_multiply(-1.0, b)
    return add_polynomials(a, neg_b)
\end{Verbatim}
\end{Answer}


    
