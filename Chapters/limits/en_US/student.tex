\chapter{Limits}

Here is a function:

$$f(x) = \frac{x^2}{x} + 1$$

This $f$ is defined for any real number \emph{except 0}. (You can't
divide anything, including zero, by zero.)

Let's plot $f$:

\begin{tikzpicture}[
tl/.style = {% tick labels
    fill=white, inner sep=1pt, font=\scriptsize,
            },                        ]
% grid
\draw[sdkblue, very thin] (-3,-3) grid (3,3);


    \draw[<->,thick,dashed] (-3.2,0) -- (3.2,0) node[right] {$x$};
    \draw[<->,thick,dashed] (0,-3.2) -- (0, 3.2) node[above] {$y$};
% curve
\draw[<-,draw=black,thick,domain=-3:-0.1,samples=300,variable=\x] plot (\x,{\x + 1});
\draw[thick] (0,1) circle (0.1);
\draw[->,draw=black,thick,domain=0.1:2,samples=300,variable=\x] plot (\x,{\x + 1});
\end{tikzpicture}

You can see that the function is the same as $x + 1$ everywhere except
$x = 0$.  You can see that as the function approaches $x=0$ from the
left, the value of the function approaches 1.  You can see that as the
function approaches $x=1$ from the right, the value of the function
approaches 1.

Mathematicians say ``The \newterm{limit} of $f$ as $x$ approaches 0, is 1.''  We have a notation for this:

$$\lim_{x \rightarrow 0} f(x) = 1$$

We generally use limit whenever we mean ``We are getting arbitrarily
close, but we can never really get there.''  For example, you might
say ``The limit of $1/t$ as $t$ goes to infinity is 0.''

\begin{tikzpicture}[
tl/.style = {% tick labels
    fill=white, inner sep=1pt, font=\scriptsize,
            },                        ]
% grid
\draw[sdkblue, very thin] (-5,-5) grid (5,5);
    \draw[<->,thick,dashed] (-5.2,0) -- (5.2,0) node[below] {$t$};
    \draw[<->,thick,dashed] (0,-5.2) -- (0, 5.2);
% curve
\draw[<->,draw=black,thick,domain=0.2:5,samples=300,variable=\x] plot (\x,{1/\x});
\draw[<->,draw=black,thick,domain=-5:-0.2,samples=300,variable=\x] plot (\x,{1/\x});
\draw (5.0, 0.2) node[above] {$t \rightarrow \infty$, $1/t \rightarrow 0$};
\draw (0.0, 5.2) node[above] {$t \rightarrow 0$ from the right, $1/t \rightarrow \infty$};
\draw (0.0, -5.2) node[below] {$t \rightarrow 0$ from the left, $1/t \rightarrow -\infty$};
\draw (-5.0, -0.2) node[below] {$t \rightarrow -\infty$, $1/t \rightarrow 0$};

\end{tikzpicture}

What is the limit of $1/t$ as $t$ approaches zero? The limit isn't
defined because if you approach from the right, $1/t$ goes to
infinity, but if you approach from the left, $1/t$ goes to negative
infinity.
