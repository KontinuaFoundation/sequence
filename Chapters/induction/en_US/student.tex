\chapter{Electromagnetic Induction}
\index{Electromagnetic Induction}

Electromagnetic induction is a fundamental concept in physics that describes how a changing magnetic field can produce an electric current. The phenomenon was discovered by Michael Faraday in 1831 and is the principle behind many electrical devices, including generators and transformers.

\section{Faraday's Law of Induction}
It is important to note that induction is the process of generating a current in a conductor by \textbf{changing} the magnetic field around it. In other words, if a conductor is moving through a \textbf{constant} magnetic field, no current will be induced. However, if the magnetic field is changing or if the conductor is moving through a gradient, a current will be induced.

The equation that describes electromagnetic induction is known as \newterm{Faraday's Law of Induction}:
\[
\mathcal{E} = -\frac{d\Phi_B}{dt}
\]
where $\mathcal{E}$ is the induced electromotive force (emf) in volts, and $\Phi_B$ is the magnetic 
flux through the circuit in webers (Wb). The negative sign indicates the direction of the induced emf, as described by Lenz's Law.
Note that the induced emf is proportional to the \textbf{rate of change} of the magnetic flux. This means that a faster change in the magnetic 
field will result in a larger induced emf.

\subsection{Magnetic Flux}
Magnetic flux ($\Phi_B$) through a surface is defined as:
\[
\Phi_B = \int \vec{B} \cdot d\vec{A}
\]  
where $\vec{B}$ is the magnetic field and $d\vec{A}$ is an infinitesimal area vector perpendicular to the surface. You'll learn more about vector calculus in later workbooks.

\section{Lenz's Law}
Lenz's Law gives the direction of the induced current. It states that the induced current will flow in such a way that its magnetic field opposes the change in magnetic flux that produced it. This is reflected in the negative sign in Faraday's Law.

\section{Induction in a Coil}
When a coil of wire is placed in a region where the magnetic field changes over time, an emf is induced in the coil according to Faraday's Law. If the coil has $N$ turns, the total induced emf is given by:
\[
\mathcal{E} = -N \frac{d\Phi_B}{dt}
\]  
This means that the induced emf is proportional to both the number of turns in the coil and the rate of change of the magnetic flux through each turn.

If the magnetic field changes uniformly and the area of the coil is constant, the change in flux can be written as:
\[
\Delta \Phi_B = \Delta B \times A
\]
where $A$ is the area of the coil and $\Delta B$ is the change in magnetic field.

Therefore, the induced emf for a coil experiencing a uniform change in magnetic field is:
\[  
\mathcal{E} = -N \frac{A \Delta B }{\Delta t}
\]

This principle is widely used in devices such as electric generators, where coils rotate in a magnetic field to produce electricity.

\begin{Exercise}[title={Induced EMF in a Coil}, label=induced_emf_coil]
A coil with $100$ turns and area $0.2\,\mathrm{m}^2$ experiences a change in magnetic field from $0$ to $5\,\mathrm{T}$ over $0.1\,\mathrm{s}$. Calculate the induced emf.
\end{Exercise}
\begin{Answer}[ref=induced_emf_coil]
The change in magnetic flux is:
\[
\Delta \Phi_B = \Delta B \cdot A = (5\,\mathrm{T} \times 0.2\,\mathrm{m}^2) = 1\,\mathrm{Wb}
\]
The induced emf is:
\[
\mathcal{E} = -N \cdot \frac{\Delta \Phi_B}{\Delta t} = -100 \cdot \frac{1\,\mathrm{Wb}}{0.1\,\mathrm{s}} = -1000\,\mathrm{V}
\]
The magnitude of the induced emf is $1000\,\mathrm{V}$.
\end{Answer}

\section{Applications of Electromagnetic Induction}
\begin{itemize}
    \item \textbf{Electric Generators}: Convert mechanical energy into electrical energy using electromagnetic induction.
    \item \textbf{Transformers}: Change the voltage of alternating current (AC) electricity using induction between coils.
    \item \textbf{Induction Cooktops}: Use rapidly changing magnetic fields to heat cookware directly.
\end{itemize}


